
\section{Mode Matching}
As will be discussed in section
\ref{sec:expimps}, we went through a revision in our overall layout.
This changed the beam size and waist location in reference to the cavity.
We did not want to open the chamber again for alignment and mode matching.
It was useful to know the power coupling into the cavity with the wrong
mode matching to give us an idea of how much transmitted light we should
expect when the cavity is properly aligned since mode matching is a difficult
process of moving the lenses along the beam path and realigning the beam.

We want to know how much power couples into the cavity from a
perfectly gaussian beam of the wrong size and location. We start with
the equations for the Hermite-Gaussian beam decomposition as discussed in
Siegman \cite{Siegman86}. 

\newcommand{\intinfxy}[1]{\int^{\infty}_{- \infty} \int^{\infty}_{- \infty} #1 \,dx \,dy}
\newcommand{\intinfrt}[1]{\int^{2 \pi}_{0} \int^{\infty}_{0} #1 r \,dr \,d\theta}


\begin{align}
    c_{nm} &= \intinfxy{
    E(x,y,z) u_n^*(x,z)u_m^*(y,z)} \nonumber
\\  &= \intinfxy{
    E(x,y,z) u_0^*(x,z)u_0^*(y,z)} \nonumber
\\  &= \intinfxy{
    u_0(x,z-z_0)u_0(y,z-z_0) u_0^*(x,z)u_0^*(y,z)}
\end{align}

We now assign the labels $b$ for the beam and $c$ for the cavity,

\begin{align}
    c_{00} &= \intinfxy{
    u_{b0}(x,z-z_0)u_{b0}(y,z-z_0) u_{c0}^*(x,z)u_{c0}^*(y,z)} \nonumber
\\  &= \intinfxy{ \left( \frac{2}{\pi w^2_0} \right) \left( \frac{q_{b0}}{q_{b}(z)} \right)
    \left( \frac{q^*_{c0}}{q^*_{c}(z)} \right) \exp \left[-ik \left( x^2 + y^2  \right)
    \left( \frac{1}{2q_b(z)} + \frac{1}{2q^*_c(z)} \right) \right]}
\end{align}

Since $q(z) = q_0 + z - z_0$, and $q_0$ is purely imaginary,


\begin{multline}
    c_{00} = \intinfxy{ \left( \frac{2}{\pi w_{b0} w_{c0}} \right)
    \left( \frac{- q_{b0} q_{c0}}{(q_{b0}+z-z_0)(-q_{c0}+z)} \right) \\
    \exp \left[
    \frac{-ik \left( x^2 + y^2  \right)}{2} \left( \frac{(q_{b0}+z-z_0)-(-q_{c0}+z)}{(q_{b0}+z-z_0)(-q_{c0}+z)} \right)
    \right]}
\end{multline}

We start by changing to cylindrical coordinates,

\begin{multline}
    c_{00} = \intinfrt{ \left( \frac{2}{\pi w_{b0} w_{c0}} \right)
    \left( \frac{- q_{b0} q_{c0}}{(q_{b0}+z-z_0)(-q_{c0}+z)} \right) \\
    \exp \left[
    \frac{-ik \left( r^2  \right)}{2} \left( \frac{(q_{b0}+z-z_0)-(-q_{c0}+z)}{(q_{b0}+z-z_0)(-q_{c0}+z)} \right)
    \right]}
\end{multline}

A careful analysis of the exponent will reveal that the real part must be less than $0$.
We can therefore solve the Gaussian integral, setting $s =
\frac{-ik \left( r^2  \right)}{2} \left( \frac{(q_{b0}+z-z_0)-(-q_{c0}+z)}{(q_{b0}+z-z_0)(-q_{c0}+z)} \right)
$,

\begin{align}
    c_{00} &= \intinfrt{ \left( \frac{2}{\pi w_{b0} w_{c0}} \right)
    \left( \frac{- q_{b0} q_{c0}}{(q_{b0}+z-z_0)(-q_{c0}+z)} \right)
    \exp \left[ s
    \right]} \nonumber
\\  &= \int^{\infty}_{0} \left( \frac{4}{w_{b0} w_{c0}} \right)
    \left( \frac{- q_{b0} q_{c0}}{(q_{b0}+z-z_0)(-q_{c0}+z)} \right)
    \exp \left[ s
    \right] r \,dr
\end{align}

Now we transform the differential and the limits of integration, remembering
that the real part of $s$ is less than $0$, $ds =
-ik \left( \frac{(q_{b0}+z-z_0)-(-q_{c0}+z)}{(q_{b0}+z-z_0)(-q_{c0}+z)} \right) r \,dr
$,

\begin{align}
    c_{00} &= \left( \frac{4i}{k w_{b0} w_{c0}} \right)
    \left( \frac{- q_{b0} q_{c0}}{(q_{b0}+z-z_0)-(-q_{c0}+z)} \right)
    \int^{0}_{-\infty} e^s \,ds \nonumber
\\  &= \left( \frac{4i}{k w_{b0} w_{c0}} \right)
    \left( \frac{- q_{b0} q_{c0}}{(q_{b0}+z-z_0)-(-q_{c0}+z)} \right) \nonumber
\\  &= \left( \frac{4i}{k w_{b0} w_{c0}} \right)
    \left( \frac{- q_{b0} q_{c0}}{(q_{b0}+q_{c0}-z_0)} \right)
\end{align}

Now, we rewrite the coefficient in terms of waist sizes and distance between waists, using,
\begin{align*}
    q_0 &= \frac{i \pi w_0^2}{\lambda}
\\  k &= 2 \pi / \lambda
\end{align*}

\begin{align}
    c_{00} &= \frac{2 w_{b0} w_{c0}}{w_{b0}^2 + w_{c0}^2 + i z_0 \lambda / \pi} \nonumber
\\  &= \frac{2 w_{b0} / w_{c0}}{1 + \left( w_{b0} /w_{c0} \right)^2 + i z_0 / z_R}
\end{align}

Power coupling into cavity is then (assuming no loss and $r_1 = r_2$),

%\begin{align}
%    P_{mathrm{trans}} &= P_{mathrm{incident}} \frac{}{}
%\end{align}

\begin{align}
    P_{\mathrm{trans}}  &= P_{\mathrm{incident}} \frac{4 w_{b0}^2 / w_{c0}^2}{1 +
    2 \left( w_{b0} /w_{c0} \right)^2 + \left( w_{b0} /w_{c0} \right)^4 + \left(
    z_0 / z_R \right)^2} \nonumber
\\  P_{\mathrm{trans}}  &= P_{\mathrm{incident}} \frac{4}{ 2 + \left( w_{c0} /
    w_{b0} \right)^2 + \left( w_{b0} /w_{c0} \right)^2 + \left(
    z_0 / z_R \right)^2}
\\  P_{\mathrm{trans}}  &= P_{\mathrm{incident}} \frac{4}{ \left( w_{c0} /
    w_{b0} + w_{b0} / w_{c0} \right)^2 + \left( z_0 / z_R \right)^2}
    \label{eq:symetricmatch}
\end{align}

In equation \ref{eq:symetricmatch} we can see readily the symmetry between
$w_{b0}$ and $w_{c0}$, and the symmetry of $z_0$ about $0$, as expected.
