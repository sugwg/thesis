
\section{Experimental Layout}

\section{High-Q Payload Suspension}

\section{Procedures}

\subsection{Measure the optical losses up to the cavity}

\subsection{Calibration of carrier and subcarrier photodiodes}
We will calibrate the photodiodes while the cavity is not locked, knowing the incident beam power and the round trip losses.

\begin{itemize}
    \item Block carrier light in Mach-Zehnder path.
    \item Turn the reflection monitor beam half-wave plate to minimize subcarrier light on the photodiode.
    \item Block the reflection monitor beam to measure dark voltage at DCPD ($A$).
    \item Unblock the reflection monitor beam and measure voltage at DCPD ($B$).
    \item Unblock the carrier light and measure again ($C$)
    \item Block subcarrier light and measure the carrier power using power meter after last steering mirror on table 1 ($D$).
    \item Using the measured optical loss to the cavity ($L$) compute the mW/mV calibration:
    \begin{itemize}
        \item $\frac{D\cdot L}{C - B}$
    \end{itemize}
\end{itemize}

\section{Mode Matching}
We want to know how much power couples into the cavity from a
perfectly gaussian beam of the wrong size and location. We start with
the equations for the Hermite-Gaussian beam expansion as discussed in
\cite{Siegman86}.

\newcommand{\intinfxy}[1]{\int^{\infty}_{- \infty} \int^{\infty}_{- \infty} #1 \,dx \,dy}
\newcommand{\intinfrt}[1]{\int^{2 \pi}_{0} \int^{\infty}_{0} #1 r \,dr \,d\theta}


\begin{align}
    c_{nm} &= \intinfxy{
    E(x,y,z) u_n^*(x,z)u_m^*(y,z)}
\\  &= \intinfxy{
    E(x,y,z) u_0^*(x,z)u_0^*(y,z)}
\\  &= \intinfxy{
    u_0(x,z-z_0)u_0(y,z-z_0) u_0^*(x,z)u_0^*(y,z)}
\end{align}

We now assign the labels $b$ for the beam and $c$ for the cavity,

\begin{align}
    c_{00} &= \intinfxy{
    u_{b0}(x,z-z_0)u_{b0}(y,z-z_0) u_{c0}^*(x,z)u_{c0}^*(y,z)}
\\  &= \intinfxy{ \left( \frac{2}{\pi w^2_0} \right) \left( \frac{q_{b0}}{q_{b}(z)} \right)
    \left( \frac{q^*_{c0}}{q^*_{c}(z)} \right) \exp \left[-ik \left( x^2 + y^2  \right)
    \left( \frac{1}{2q_b(z)} + \frac{1}{2q^*_c(z)} \right) \right]}
\end{align}

Since $q(z) = q_0 + z - z_0$, and $q_0$ is purely imaginary,


\begin{multline}
    c_{00} = \intinfxy{ \left( \frac{2}{\pi w_{b0} w_{c0}} \right)
    \left( \frac{- q_{b0} q_{c0}}{(q_{b0}+z-z_0)(-q_{c0}+z)} \right) \\
    \exp \left[
    \frac{-ik \left( x^2 + y^2  \right)}{2} \left( \frac{(q_{b0}+z-z_0)-(-q_{c0}+z)}{(q_{b0}+z-z_0)(-q_{c0}+z)} \right)
    \right]}
\end{multline}

We start by changing to cylindrical coordinates,

\begin{multline}
    c_{00} = \intinfrt{ \left( \frac{2}{\pi w_{b0} w_{c0}} \right)
    \left( \frac{- q_{b0} q_{c0}}{(q_{b0}+z-z_0)(-q_{c0}+z)} \right) \\
    \exp \left[
    \frac{-ik \left( r^2  \right)}{2} \left( \frac{(q_{b0}+z-z_0)-(-q_{c0}+z)}{(q_{b0}+z-z_0)(-q_{c0}+z)} \right)
    \right]}
\end{multline}

A careful analysis of the exponent will reveal that the real part must be less than $0$.
We can therefore solve the Gaussian integral, setting $s =
\frac{-ik \left( r^2  \right)}{2} \left( \frac{(q_{b0}+z-z_0)-(-q_{c0}+z)}{(q_{b0}+z-z_0)(-q_{c0}+z)} \right)
$,

\begin{align}
    c_{00} &= \intinfrt{ \left( \frac{2}{\pi w_{b0} w_{c0}} \right)
    \left( \frac{- q_{b0} q_{c0}}{(q_{b0}+z-z_0)(-q_{c0}+z)} \right)
    \exp \left[ s
    \right]}
\\  &= \int^{\infty}_{0} \left( \frac{4}{w_{b0} w_{c0}} \right)
    \left( \frac{- q_{b0} q_{c0}}{(q_{b0}+z-z_0)(-q_{c0}+z)} \right)
    \exp \left[ s
    \right] r \,dr
\end{align}

Now transform the Jacobian, $ds =
-ik \left( \frac{(q_{b0}+z-z_0)-(-q_{c0}+z)}{(q_{b0}+z-z_0)(-q_{c0}+z)} \right) r \,dr
$,

\begin{align}
    c_{00} &= \left( \frac{4i}{k w_{b0} w_{c0}} \right)
    \left( \frac{- q_{b0} q_{c0}}{(q_{b0}+z-z_0)-(-q_{c0}+z)} \right)
    \int^{0}_{-\infty} e^s \,ds
\\  c_{00} &= \left( \frac{4i}{k w_{b0} w_{c0}} \right)
    \left( \frac{- q_{b0} q_{c0}}{(q_{b0}+z-z_0)-(-q_{c0}+z)} \right)
\end{align}

Now, we write in terms of waist sizes and distance between waists, using:
\begin{align*}
    q_0 &= \frac{i \pi w_0^2}{\lambda}
\\  k &= 2 \pi / \lambda
\end{align*}

\begin{align}
    &= \frac{w_b^2}{\omega^2_0 + i z_0 \lambda /2 \pi}
\\  &= \frac{(w_b/w_0)^2}{1 + i z_0 /2 z_R}
\end{align}

Power coupling into cavity is then (assuming no loss and $r_1 = r_2$),
\begin{align}
P_{\mathrm{trans}} &= P_{\mathrm{incident}} \frac{(w_b/w_0)^4}{1+(z_0/2z_R)^2}
\end{align}

