\appendix

\section{Kerr-NUT parameters}
\label{appendix:KerrNUTParameters}


In this appendix, we provide a review of the parameters of the Kerr-NUT solution. The Kerr family of vacuum solutions is unique when one imposes axisymmetry, stationarity and regularity on the BH horizon along with asymptotic flatness. However, if one allows for generalization by relaxing the asymptotic flatness condition, one arrives at a family of solutions called Kerr-NUT. This solution is a part of the broader family of Einstein-Maxwell type D solutions. This generalized family of spacetimes is parameterized by 6 parameters (potentially 7 if one includes the cosmological constant $\Lambda$). In Table~\ref{tab:Parameters}, we summarize the parameters, as well as their physical meaning and symbols used in various texts.

The general Einstein-Maxwell Type D solution (including cosmological constant $\Lambda$) has the form given in Eq.~21.11 of~\cite{stephani2009exact}, with parameters $m$, $l$, $\gamma$, $\varepsilon$, $e$, and $g$. $m$ refers to the mass parameter (closely related to the mass of the BH), $\gamma$ is related to the angular momentum parameter $a$ (closely related to the spin of the BH), $\varepsilon$ is related to the acceleration $b$, $e$ is the electric charge, $g$ is the magnetic charge, and $l$ is known as the NUT parameter. As outlined in~\cite{PLEBANSKI197698}, the mass and the NUT parameter form a complex quantity, as do the angular momentum and the acceleration, similarly to the electric and magnetic charges. In~\cite{PLEBANSKI197698}, $\varepsilon$ and $\gamma$ do not appear in the curvature quantities, and are called kinematical parameters, while the others are dynamical parameters.

As shown in Table~21.1 of~\cite{stephani2009exact}, setting all of the parameters to zero except for $m$, $a$ (and hence $\gamma$ and $\varepsilon$), and $e$ yields the Kerr-Newman solution, while also setting $a = 0$ yields the Reissner-Nordstrom solution. Kerr-Taub-NUT metrics, meanwhile, are parametrized by mass, spin, and $l$, with $l \neq 0$, and are thought to be unphysical~\cite{kaluzaklein}. The vacuum BBH case considered in this study, meanwhile, sets $e = 0$ and $g = 0$, since there are no electric or magnetic charges at the start of the simulation, and no sourcing of them during the simulation.

An accelerating and rotating BH with a NUT charge will have non-zero $m$, $l$, $a$, and $b$, with $a > l$. A Kerr solution with a NUT charge will then have $b = 0$. An accelerating and rotating BH, meanwhile, will have $l = 0$. Finally, the Kerr solution has both $l = 0$ and $b = 0$. An illustration of this is provided in Fig.~1 of~\cite{Griffiths:2005se}. The condition $l = 0$ gives the Kerr 2 condition considered in this paper, given in Eq.~\eqref{eq:Kerr2}.

After setting $l=0$, the parameters $m$, $\varepsilon$ and $\gamma$ are related to the mass and spin of a BH are as follows,
\begin{align}
\label{eq:massAndspin}
\mathrm{mass} = \frac{m}{\varepsilon^{\frac{3}{2}}} \; \; \; \; \mathrm{and} \; \; \; \;
\mathrm{spin} = \frac{2 \sqrt{|\gamma|}}{\varepsilon} \,.
\end{align}

Since, $\varepsilon > 0$ and $m>0$ for a Kerr BH, the condition that $b = 0$ gives $\varepsilon > 0$, which corresponds to the Kerr 3 condition given in Eq.~\eqref{eq:Kerr3}.

\newcommand*\rot{\rotatebox{90}}
n{table}[h]
%\begin{ruledtabular}
  \begin{tabular}{ l | c  c  c  c}
     & \rot{Stephani~\cite{stephani2009exact}} & \rot{Garc\'{i}a-Parrado~\cite{lobo16}} & \rot{Plebanski~\cite{PLEBANSKI197698}} & \rot{Griffiths~\cite{Griffiths:2005se}}\\
    \hline \hline
    Cosmological constant & $\Lambda$ & & $\lambda$ & \\ \hline
Mass parameter & $m$ & $\mu$ & $m$ & $m$  \\ \hline
NUT parameter & $l$ & $\lambda$ & $n$ & $n$ \\ \hline
Angular momentum parameter& $\gamma$ & $\gamma$ & $\gamma$ & $k$  \\ \hline
Acceleration parameter& $\varepsilon$ & $ \epsilon$ & $\epsilon$ & $\epsilon$  \\ \hline
Electric charge & $e$ & & $e$ & $e$ \\ \hline
Magnetic charge & $g$ & & $g$ & $g$
  \end{tabular}
%\end{ruledtabular}
\caption{%
    Parameters of the family of the Einstein-Maxwell type D solutions, presented with physical meanings in the rows and naming conventions in various literature in the columns. These parameters do not measure the physical quantities directly but are intimately connected to the physical quantities they describe. For instance, Eq.~\eqref{eq:massAndspin} shows how the mass and spin of a BH are related to the mass parameter and the angular momentum parameter. 
 }
  \label{tab:Parameters}
\end{table}

