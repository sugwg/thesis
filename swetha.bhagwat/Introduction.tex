%problems with gravity
Many flagship challenges of 21st-century physics revolve around our understanding of Gravity- the quantization of gravity, the black hole information paradox, building a unified theory for the fundamental forces, the fine-tuning problem in cosmology and understanding the nature of dark energy, are just a few puzzles that top the list. The general theory of relativity (GR) serves as a conventional theory for describing the classical behaviour of gravitation. However, developing a theory to describe gravity at a quantum scale, is still an avid problem and currently, is anything but explicable. Furthermore, while on one hand, our understanding of gravity seems fundamentally inconsistent with quantum mechanics, on the other hand at large scales, we encounter the problem of fine-tuning of the cosmological constant. As a consequence, a plethora of alternative descriptions of gravitation are being developed that aim at resolving one or more of these inconsistencies [c.f. for a review of these theories see \cite{ReviewAlternativeTheories1,ReviewAlternativeTheories2}].. Given the circumstance, experimental and observational lead seems imperative to further our understanding of Gravity.

Metric theories of gravity like GR, predict propagating disturbances in space-time curvature known as gravitational waves (GW). Violent cosmic events, like the supernovae explosion or collision of two black holes, can produce GWs that are loud enough for a GW observatory like the Laser Interferometer Gravitational-Wave Observatory (LIGO) to measure them. Observation of GWs can serve as an unsurpassed probe to understanding the nature of gravity, particularly the dynamics of strong field gravity around compact objects like black holes (BH) and neutron stars (NS) [c.f. see section 5 of \cite{ReviewAlternativeTheories1}]. These compact binary systems are especially suitable for testing the validity of GR because it is in these strong field regimes that one expects to perceive the full complexity of the underlying theory of gravity. 

Assuming that the underlying theory of gravity is GR, the Compact binary coalescences (CBC) emit GWs with a characteristic structure- when the two bodies are far apart and are slowly (compared to the speed of light, c) spiraling in, they produce the long chirping inspiral  signal, where both the frequency and the amplitude of the GW emitted increases with time. This is followed by a violent plunge-merger phase which contains imprints of the complex nonlinear-strong field dynamics predicted by GR. Then, there is a ringdown (RD) phase that corresponds to the distorted merged BH settling down to its final stable state.  

One can device different tests with a GW signal from a CBC event to probe various features that are central to the framework of GR. For example, using the long inspiral signal in a parameterized-post-Newtonian (PPN) scheme [c.f chapter 4 of \cite{CliffWills_book}], One can experimentally constraint  violations of many fundamental features of GR like the amount of spatial curvature given a mass, the time delay effects, the Lorentz dispersion, the predicted nonlinearity in the addition of gravitational fields, diffeomorphism and principle of equivalence, conservation laws etc (See the violin plot in Figure 7 and Table 1 of \cite{TheLIGOScientific:2016src} for details of how this scheme is used). Tests like this can be used to eliminate or constrain classes of alternative theories of gravity. Similarly, the merger-ringdown part of the CBC GW signal can be used to perform tests of GR focused on the strong field dynamics and nature of the compact objects \cite{Mazur:2000pn,misner1973gravitation,Dreyer,Gossan,Kamaretsos}- Eg. the no-hair theorem test, the BH area increase test, direct observation of BH quasinormal mode spectrum, absence of echoes in post-merger singal etc. These tests are invaluable and lets us probe gravity in regimes that were inaccessable by any other expriments.  

The advent of GW astronomy marked with multiple the GW observations \cite{TheLIGOScientific:2016pea} offers an unprecedented prospect of validating GR and probing the nature of gravity in the coming years. Further, with the upgrades to the current LIGO facilities and future ground-based missions like the Cosmic Explorer and the Einstien Telescope complimented with the proposed space-based projects like LISA, the sensitivity with which we can probe strong field gravitational dynamics will allow us to place extremely stringent constraints on the alternative theories. With the existing LIGO's observations, GR has been subjected to various tests and alternative theory are already being constrained [c.f. \cite{Yunes:2016jcc,TheLIGOScientific:2016src,2016arXiv160308955Y} for details on testing GR using GW events]. 

Nonetheless, GW150914, the first GW detection by LIGO, was not loud enough to perform a detailed ringdown based tests of GR. The spread in the estimated frequency of the dominant mode present in the ringdown had a spread of $\sim 40$ Hz, which is too imprecise to perform a stringent test of GR. Moreover, GW150914 signal corresponds to a nearly equal mass progenitor BBH system and the symmetry of the initial system suppresses sub-dominant mode excitations in the ringdown. However, this signal gives us hope that for a slightly louder signal, which is very realistic with upgrades to LIGO or any future proposed detectors, very stringent tests of GR can be carried out using ringdown signals. 

The primary focus of this thesis is on topics that are relevant for designing tests to probe the strong field GR using the ringdown part of a BBH merger. Ringdown is a particularly appealing part of the BBH GW signal; it has an elegant analytical description offered by BH  perturbation theory and also is supplemented by insights offered by full non-linear numerical relativity framework. Although, ringdown signals contain a rich set of information about the strong field dynamics and nature of the compact object, the data analysis of these signals are far from being straightforward. Currently, the GW research community is in the phase of building robust tools to perform the ringdown based tests. In this thesis, we touch upon aspects of ringdown analysis, explored using a conglomeration of techniques developed in analytical perturbation theory, numerical relativity and GW data analysis. In Section \ref{sec:outlineThesis}, we outline the structure and flow of this thesis.


\section{Outline of the thesis}
\label{sec:outlineOfThesis}
In this section we present the outline and organization of this thesis. In this thesis, we first breifly comment on the discovery of GW150914, presenting some preliminary studies perfromed as a member of the LIGO collaboration and summarizing some key results obtained by the LIGO collaboration. Then, we proceed to discuss the studies that contribute towards probing the strong feild gravity with BBH ringdowns. Chapter 3 and 4 foucus on understanding the start time of perturbative region in post-merger signal in GW observed during a CBC event. Chapter 5 presents the prospects of performing a multimodal analysis using the ringdown signal in the current and next gerenation detectors. In Chapter 6, we present an ongoing effort towards developing a fully Bayesian parameter estimation framework that would let us carry out a a multimodal analysis using the ringdown signal from the detector data.

\subsection{On the discovery of GW150914- Chapter 2}
Having witnessed the prestigious first discovery of GW signal from the BBH system as a member of the LIGO collaboration , Chapter 2 of this thesis is dedicated to summarizing the first discovery, GW150914 and the hope it sheds on testing the validity of GR. 

Prior to the observing this golden first event, one would have never expected to discover a GW signal so loud that it can be visually seen with an extremely basic filtering of raw detector data. However, GW150914  happened to be an extraordinary loud signal and stood out in the detector data steam against the noisy background of the detector. As a part of the LIGO collaboration, I along with Prof. Stefan Ballmer performed some preliminary analysis of comparing the raw detector data with the GW waveform corresponding to the best-fit parameters of the BBH system reported by the search pipeline. This preliminary analysis is presented in section \ref{sec:dirrectComparision}. A more refined version of this analysis performed by the LIGO collaboration is presented in Figure 1 of the GW150914 discovery paper \ref{}.  

Furthermore, as a part of the collaboration, I was involved in the review process of implementation of a waveform approximant that uses effective one body formalism and includes precession effects during the inspiral, called "SEOBNRv3" template family. This waveform family was used to improve the parameter estimation results of GW150914 \cite{}. The astrophysical properties of GW150914 obtained by the LIGO collaboration using SEOBNRv3 are summarized in section \ref{PEonGW150915}.   

\subsection{On ringdown and scales of perturbation- Chapter3}
Having seen a signal as loud as GW150914, it is only natural that one develops a reliable framework to extract the rich information present in the ringdown signals. Several ringdown based tests of GR relies on verifying prediction given by BH perturbation theory in the ringdown data of the GW observation. To do this, however, one of the major challenges is to identify where in the post-merger signal one needs to start this analysis (i.e) how long should we wait after the peak of the waveform before it admits a perturbative description. One naive argument could be that one needs to at least wait till the scale of perturbation is smaller than the scale of the event horizon of the final BH formed for perturbation theory to hold. In Chapter 3, we use two toy models to explore this concept - a) a system comprising of two point masses and b) a rotating tri-axial ellipsoid. We retro-engineer the systems such that they produce the observed GW signal and study the dynamics of these toy models. This work is primarily meant to develop an intuition towards the scales of perturbation that occurs in the source frame (close to BH) during the ringdown phase and to motivate the more concrete study presented in Chapter 4. 


\subsection{On choosing the start time of binary black hole ringdown- Chapter4}
In this chapter, we
present an algorithmic method to analyze the choice of ringdown start time in
the observed waveform. This method is based on determining how close the strong
field is to a Kerr black hole (\textit{Kerrness}). Using numerical relativity
simulations, we characterize the Kerrness of the strong-field region close to
the black hole using a set of local, gauge-invariant geometric and algebraic
conditions that measure local isometry to Kerr. We produce a map that
associates each time in the gravitational waveform with a value of each of
these Kerrness measures; this map is produced by following outgoing null
characteristics from the strong and near-field regions to the wave zone. Furthermore, we
perform this analysis on a numerical relativity simulation with parameters
consistent with GW150914- the first gravitational wave detection. 

A particularly intersting result that  we find in this study is that the choice of ringdown start time of $3\,\mathrm{ms}$ after merger used in the
GW150914 study~\cite{TheLIGOScientific:2016src} to test general relativity corresponds to a high dimensionless
perturbation amplitude of $ \sim 7.5 \times 10^{-3}$ in the strong-field
region. This suggests that in higher signal-to-noise detections, one would need
to start analyzing the signal at a later time for studies that depend on the
validity of black hole perturbation theory.

This chapter contains the work published as "
On choosing the start time of binary black hole ringdown" by 
\textit{Swetha Bhagwat}, Maria Okounkova, Stefan W. Ballmer, Duncan A. Brown, Matthew Giesler, Mark A. Scheel, Saul A. Teukolsky \cite{MeAndMasha}.

\subsection{Spectroscopic analysis of stellar mass black-hole mergers in our local universe with ground-based gravitational wave detectors- Chapter 5}
Next, in Chapter 5, we investigate the prospects of ground-based detectors to perform a spectroscopic analysis of signals emitted during the ringdown of the final Kerr black-hole formed by a stellar mass binary black-hole merger. If we assume an optimistic rate of 240 Gpc$^{-3}$yr$^{-1}$, about 3 events per year can be measured by Advanced LIGO. Further, upgrades to the existing LIGO detectors will increase the odds of measuring multiple ringdown modes significantly. New ground-based facilities such as Einstein Telescope or Cosmic Explorer could measure multiple ringdown modes in about thousand events per year. We perform Monte-Carlo injections of $10^{6}$ binary black-hole mergers in a search volume defined by a sphere of radius 1500 Mpc centered at the detector, for various proposed ground-based detector models. We assume a uniform random distribution in component masses of the progenitor binaries, sky positions and orientations to investigate the fraction of the population that satisfy our criteria for detectability and resolvability of multiple ringdown modes. We investigate the detectability and resolvability of the sub-dominant modes $l=m=3$, $l=m=4$ and $l=2, m=1$. Our results indicate that the modes with $l=m=3$ and $l=2, m=1$ are the most promising candidates for sub-dominant mode measurability. We find that for stellar mass black-hole mergers, resolvability is not a limiting criteria for these modes. We emphasize that the measurability of the $l=2, m=1$ mode is not impeded by the resolvability criterion.

This chapter contains the work published as "
Spectroscopic analysis of stellar mass black-hole mergers in our local universe with ground-based gravitational wave detectors
by \textit{Swetha Bhagwat}, Duncan A. Brown, Stefan W. Ballmer \cite{MySpectroscopy}.

\subsection{Full Bayesian PE on RD- Chapter 6}
The work in chapter 5 relies on a formalism that uses Fisher information matrix. This is formalism is elegant and often timesi in this framework, the statistical spread in the estimation of the parameters can be analytically written down. However, it performs well only in a limit of high signal to noise ratio (SNR) limit (c.f \ref{use-and-abuse} for a detailed discussion on use and abuse of Fisher matrix formalism). However, most often GW signals are not very loud and thus, to estimate parameters of the system one needs to resort to a full Bayesian parameter estimation (PE) set up. 

In this chapter, I summarize my ongoing work in developing a fully Bayesian PE framework, focused particularly on the ringdown and present a proof-of-concept result. Further, we intend to develop a systematic scheme and a robust code setup that can be applied to the detector data containing a BBH event to extract the QNM information from the RD portion in the data. We use the code setup developed in the PyCBC open source code library to perform the PE. Furthermore, we use this set up to investigate the presence of a subdominant mode in ringdown and the results are presented in section \ref{presence-of-subdominatmode}. We find that if one assumes the knowledge of sky location of the source, then with an SNR of $\sim 12$ in ringdown, one might be able to detect the presence of subdominant mode. This is an interesting result because GW150914 had an SNR of 8 in its ringdoen and advanced LIGO at its design sensitivty with be $\sim 3$ times more sensitive than when it observed GW15914. Therefore, in the next observing runs of LIGO, we stand a optimistic chance of detecting the presence of subdominant mode in BBH ringdown. Note, however, that all the results in this section are preliminary. 

This work is being done in collaboration with Miriam Cabero, Collin Capano, Alex Nitz, Duncan Brown and Badri Krishnan.  


%%%%%%%%%%%%%%%%%%%%%%%%%%%

%LIGO and observation

%%%%%%%%%%%%%%%%%%%%%%%%%%%
% GW 150914 and where that takes GR. 
%%%%%This is where you can introduce RD as one of the several test of GR. 

%%%%%%%%%%%%%%%%%%%%%%%%%%%%
% Motivate RD as topic of this thesis

%%%%%%% Structure of the thesis. 


