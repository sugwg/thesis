In this thesis, the primary focus has been on understanding aspects of binary black hole ringdown that are of importance to designing tests that probe the nature of strong field gravity.  The three broad themes that we discuss here are the following: When does the ringdown start in the post-merger GW signal from a CBC event? How realistic is it to expect that we can perform a multimodal ringdown based test in the near future? How should the data analysis be carried out to extract multimodal ringdown information from a GW signal present in the detector data?

From the studies presented in  Chapter 3 and 4, we now have a better understanding of what the different choices of the start time of ringdowns in a GW signal from a BBH coalescence correspond to in terms of source frame dynamics. These studies, therefore, inform the systematic modelling bais that one must expect for the choice of the start time used in the analysis. In practice, while performing ringdown based test, this choice of the start time should be chosen, keeping in mind the trade-off between the systematic modelling errors coming due to a large source frame perturbation amplitudes and the statistical error from the detector noise. For the current sensitivity of the LIGO detector, our study in Chapter 4 indicates that the statistical error from the detector noise dominates over the systematic modelling effect. However, in future, with next-generation detectors, one needs to carefully build a framework to pick the start time of ringdown such that error arising from both these independent sources are minimized.  

Furthermore, it should be highlighted that the techniques presented in Chapter 4, although developed in the context of mapping the source frame perturbation amplitude to the gravitational waves observed at asymptotic future null infinity,  can be used to map the evolution of any source frame quantity to the asymptotic frame. It should also be noted that our procedure relies on the construction of outgoing null characteristics and has a shortcoming that this technique fails in the region of spacetime that forms caustics.  

In Chapter 5, we present a study that demonstrates that prospects of multimodal ringdown analysis are optimistic with upgrades to current LIGO facilities and with any of the proposed future ground-based detectors. Also, we propose that if a frequency dependent sensitivity increase can be achieved in the detectors, then for improvement in the band between 300 - 500 Hz will be optimal for ringdown based analysis.  

In Chapter 6, we present an ongoing work where we develop a framework to carry out multimodal ringdown analysis using a fully Bayesian technique. We find that the SNR in the ringdown that would allow us to perform multimodal ringdown analysis depends on the perturbation condition setup during the merger, thereby on the asymmetry of the progenitor binary system.  From the preliminary result of this ongoing study, we also see that there is an optimistic chance to be able to detect the presence of the sub-dominant mode in ringdown at an SNR of $\sim 18$ for a progenitor mass ratio of 2.  We are currently investigating these arguments further. Finally, in near future, we intend to develop a robust parameter estimation framework focused on the performing multimodal parameter estimation that would allow us to perform various tests on strong field gravity of the final BH form during a CBC event. 


