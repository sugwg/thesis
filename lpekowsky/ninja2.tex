The NINJA-1 project was a huge success in bringing the numerical
relativity and gravitational-wave astronomy communities together.  The
project also resulted in several intriguing qualitative results.
Among these were promising indications that the alterations suggested 
by the studies of chapter~\ref{ch:comparison} could indeed improve
search efficiencies.

However, NINJA-1 only began the process of testing detection and
parameter estimation pipelines against realistic signals.  The
follow-up project, NINJA-2, is ongoing as of the time of writing.
NINJA-2 aims to remove some of the shortcomings of NINJA-1 and allow
quantitative studies of the performance of gravitational-wave searches
in various regions of signal parameter space.  Specifically, NINJA-2
addresses issues with both the waveform submissions and the noise used
to construct the data sets.  This chapter describes the contributed
waveforms and the studies that have been performed to validate them.
The next chapter will discuss the construction of the data sets and
will present preliminary results from the low- and high-mass CBC
pipelines.

\section{Contributed Waveforms}

NINJA-1 had an open policy towards waveform submission in order to
encourage wide participation.  This meant there were no requirements
on either waveform quality or length.  The lack of quality
requirements allowed for the possibility of unphysical features in the
waveforms.  There were also no requirements to perform the kind of
convergence testing reported in Sec.~\ref{sec:PNNRHybridWaveform},
although such validation is typically done by numerical
relativists.  The loose requirements limited the conclusions that
could be drawn, for example it makes it difficult to say whether an
injection was missed due to the parameters of the signal or an
unintended feature of the waveform.

The lack of length requirement limited the available mass range to $M
> 36 \msun$ for reasons that can be seen in
Fig.~\ref{fig:Ninja2StildesAndInitialPSD}.  This plot shows the
$q=2$, non-spinning GA Tech submission to NINJA-2, along with the
initial LIGO and advanced LIGO noise curves (see the discussion of
Fig.~\ref{fig:StildesAndInitialPSD} for an explanation of the choice
of axes).  Frequencies of interest are marked.  Note the triangle in
particular, which indicates the frequency at which the numeric
waveform starts.  If a waveform starts above 40 Hz then we will
underestimate the SNR available from such a signal.  If the
waveform in Fig.~\ref{fig:Ninja2StildesAndInitialPSD} had no 
pN component the lowest mass at which it could be injected would be
about 40 $\msun$.


\begin{figure}
    \includegraphics[width=\linewidth]{figures/ninja2/StildesAndInitialPSD}
  \caption[Hybrid MayaKranc waveform scaled to various total masses]{
  \label{fig:Ninja2StildesAndInitialPSD}
    The hybrid $q=2$, non-spinning MayaKranc waveform scaled to various total
    masses, with sources optimally oriented and placed at
    \unit[100]{Mpc}, shown against the Initial- and Advanced-LIGO
    noise curves.  Markers are placed along the lines at frequencies
    corresponding to various instantaneous frequencies of the
    waveforms.  The triangles represent the frequency at which the
    numeric waveform begins; the circle represents the ISCO frequency;
    and the square the light-ring.  The values given for $\rho$ use
    the Initial-LIGO noise curve.}
\end{figure}%

To address these issues NINJA-2 specifies the following requirements
on the submitted waveforms~\cite{ninja2-wiki}.  The raw numerical
simulation should include at least five orbits of usable data before
merger (i.e., not counting bursts of junk radiation or other
significant noise).  Given the computation cost of extending the NR
waveforms, we instead require ``stitching'' to a post-Newtonian
inspiral approximant, which should be performed at a GW frequency of
$M\omega \leq 0.075$, where $M\omega$ is the frequency of the $(l = 2,
m = \pm 2)$ harmonic. The full waveform should be long enough to be
entirely within the sensitivity bands of LIGO and Virgo down to $10
\msun$ with a lower cutoff frequency of 10 Hz, which corresponds to a
starting GW frequency of $M\omega = 0.003$.  The numerical-waveform
(before any hybridization) amplitude should be accurate to within 5\%,
and the phase (as a function of GW frequency) should have an
accumulated uncertainty over the entire inspiral, merger and ringdown,
of no more than 0.5 radian.  The PN approximants used for
hybridization should ideally use the highest PN orders available, both
in phase and amplitude.  These minimal accuracy requirements are
motivated by the results of the Samurai project~\cite{Hannam:2009hh},
and studies performed in preparation for the NR-AR collaboration
project~\cite{ninja-wiki}.

As noted in Eqn.~\ref{eq:mode_decomposition} the complex function
$h_+-ih_\times$ can be decomposed into modes using spin-weighted
spherical harmonics $\Ytwo_{lm}$ of weight -2.  Although most of the
power is in the $(l,m)=(2,2)$ mode, the NINJA-2 project encourages,
although does not require, the inclusion of higher-order modes.  We
chose to restrict attention to non-spinning waveforms and waveforms
with spins aligned or anti-aligned with the orbital angular momentum.
Systems where the component spins are at other angles to the angular
momentum exhibit precession, which complicates the waveform
generation, the hybridization, and the data analysis significantly.
There are sufficient open questions regarding these restricted cases
to make this analysis interesting, without adding these additional
complications.  However, we plan to include precessing waveforms in a
forthcoming NINJA-3.

A total of 60 waveforms from 8 groups were contributed to NINJA-2.
These are summarized in tables ~\ref{tab:ninja2_bam},
\ref{tab:ninja2_fau}, \ref{tab:ninja2_gatech}, \ref{tab:ninja2_lean},
\ref{tab:ninja2_llama}, \ref{tab:ninja2_rit}, \ref{tab:ninja2_spec},
\ref{tab:ninja2_uiuc} and a map of the parameter values is shown in
Fig.~\ref{f:ninja2_param_map}.

\begin{table}
\begin{center}
\begin{tabular}{|l|r|r|r|l|c|}
\hline
Run & $q$ & Spin1${}_z$ & Spin2${}_z$ & pN Approx. & Refs \\
\hline
BAM\_D10spp85\_80.T4.hyb.n2 & 1 & 0.85 & 0.85 & TaylorT4 & \cite{Hannam:2007wf,Brugmann:2008zz} \\
BAM\_D10spp85\_80.T1.hyb.n2 & 1 & 0.85 & 0.85 & TaylorT1 & \cite{Hannam:2007wf,Brugmann:2008zz} \\
BAM\_D125smm50Nep\_80.T1.hyb.n2 & 1 & -0.50 & -0.50 & TaylorT1 & \cite{Hannam:2007wf,Brugmann:2008zz} \\
BAM\_D125smm50Nep\_80.T4.hyb.n2 & 1 & -0.50 & -0.50 & TaylorT4 & \cite{Hannam:2007wf,Brugmann:2008zz} \\
BAM\_D13smm75Nep\_96.T4.hyb.n2 & 1 & -0.75 & -0.75 & TaylorT4 & \cite{Hannam:2007wf,Brugmann:2008zz} \\
BAM\_D13smm75Nep\_96.T1.hyb.n2 & 1 & -0.75 & -0.75 & TaylorT1 & \cite{Hannam:2007wf,Brugmann:2008zz} \\
BAM\_D13smm85Nep\_88.T4.hyb.n2 & 1 & -0.85 & -0.85 & TaylorT4 & \cite{Hannam:2007wf,Brugmann:2008zz} \\
BAM\_D13smm85Nep\_88.T1.hyb.n2 & 1 & -0.85 & -0.85 & TaylorT1 & \cite{Hannam:2007wf,Brugmann:2008zz} \\
BAM\_D11spp50\_96.T4.hyb.n2 & 1 & 0.50 & 0.50 & TaylorT4 & \cite{Hannam:2007wf,Brugmann:2008zz} \\
BAM\_D11spp50\_96.T1.hyb.n2 & 1 & 0.50 & 0.50 & TaylorT1 & \cite{Hannam:2007wf,Brugmann:2008zz} \\
BAM\_D10spp75\_80.T1.hyb.n2 & 1 & 0.75 & 0.75 & TaylorT1 & \cite{Hannam:2007wf,Brugmann:2008zz} \\
BAM\_D10spp75\_80.T4.hyb.n2 & 1 & 0.75 & 0.75 & TaylorT4 & \cite{Hannam:2007wf,Brugmann:2008zz} \\
BAM\_D12smm25Nep\_80.T4.hyb.n2 & 1 & -0.25 & -0.25 & TaylorT4 & \cite{Hannam:2007wf,Brugmann:2008zz} \\
BAM\_D12smm25Nep\_80.T1.hyb.n2 & 1 & -0.25 & -0.25 & TaylorT1 & \cite{Hannam:2007wf,Brugmann:2008zz} \\
BAM\_EP\_um4\_D10-n96.T4.hyb.n2 & 4 & 0.00 & 0.00 & TaylorT4 & \cite{Hannam:2007wf,Brugmann:2008zz} \\
BAM\_EP\_um4\_D10-n96.T1.hyb.n2 & 4 & 0.00 & 0.00 & TaylorT1 & \cite{Hannam:2007wf,Brugmann:2008zz} \\
BAM\_um3\_88.T4.hyb.n2 & 3 & 0.00 & 0.00 & TaylorT4 & \cite{Hannam:2007wf,Brugmann:2008zz} \\
BAM\_um3\_88.T1.hyb.n2 & 3 & 0.00 & 0.00 & TaylorT1 & \cite{Hannam:2007wf,Brugmann:2008zz} \\
BAM\_um2\_88.T1.hyb.n2 & 2 & 0.00 & 0.00 & TaylorT1 & \cite{Hannam:2007wf,Brugmann:2008zz} \\
BAM\_um2\_88.T4.hyb.n2 & 2 & 0.00 & 0.00 & TaylorT4 & \cite{Hannam:2007wf,Brugmann:2008zz} \\
BAM\_R6\_PN\_80.T1.hyb.n2 & 1 & 0.00 & 0.00 & TaylorT1 & \cite{Hannam:2007wf,Brugmann:2008zz} \\
BAM\_R6\_PN\_80.T4.hyb.n2 & 1 & 0.00 & 0.00 & TaylorT4 & \cite{Hannam:2007wf,Brugmann:2008zz} \\
BAM\_D12spp25\_96.T4.hyb.n2 & 1 & 0.25 & 0.25 & TaylorT4 & \cite{Hannam:2007wf,Brugmann:2008zz} \\
BAM\_D12spp25\_96.T1.hyb.n2 & 1 & 0.25 & 0.25 & TaylorT1 & \cite{Hannam:2007wf,Brugmann:2008zz} \\
BAM\_q2a0a025\_T\_96\_344.T1.hyb.n2.bbh & 2 & 0.25 & 0.00 & {} & \cite{Brugmann:2008zz} \\
BAM\_q2a0a025\_T\_96\_344.T4.hyb.n2.bbh & 2 & 0.25 & 0.00 & {} & \cite{Brugmann:2008zz} \\
\hline
\end{tabular}
\end{center}
\caption[BAM submissions to NINJA-2]{
\label{tab:ninja2_bam}
BAM submissions to NINJA-2}
\end{table}

\begin{table}
\begin{center}
\begin{tabular}{|l|r|r|r|l|c|}
\hline
Run & $q$ & Spin1${}_z$ & Spin2${}_z$ & pN Approx. & Refs \\
\hline
BAM\_hybrid\_om0.025etmq3S0.4- & 3 & 0.40 & 0.60 & TaylorT4 &
\cite{Bruegmann:2003aw,Marronetti:2007wz} \\
0\_0\_S0.6\_0\_0\_72 &  &  &  &  &  \\
\hline
\end{tabular}
\end{center}
\caption[FAU submissions to NINJA-2]{
\label{tab:ninja2_fau}
FAU submissions to NINJA-2}
\end{table}

\begin{table}
\begin{center}
\begin{tabular}{|l|r|r|r|l|c|}
\hline
Run & $q$ & Spin1${}_z$ & Spin2${}_z$ & pN Approx. & Refs \\
\hline
MayaKranc\_D12\_a0.00\_m129\_nj & 1 & 0.00 & 0.00 & TaylorT4 &
\cite{Herrmann2007b,MRS:2010} \\
MayaKranc\_D10\_a0.90\_m129\_nj & 1 & 0.90 & 0.90 & TaylorT4 & \cite{Herrmann2007b,MRS:2010} \\
MayaKranc\_D10\_a0.20\_m77\_nj & 1 & 0.20 & 0.20 & TaylorT4 & \cite{Herrmann2007b,MRS:2010} \\
MayaKranc\_D10\_a0.60\_m77\_nj & 1 & 0.60 & 0.60 & TaylorT4 & \cite{Herrmann2007b,MRS:2010} \\
MayaKranc\_D12\_a0.60\_m103\_nj & 1 & 0.60 & 0.60 & TaylorT4 & \cite{Herrmann2007b,MRS:2010} \\
MayaKranc\_Sp02py0935th90\_gr & 1 & 0.80 & 0.00 & TaylorT4 & \cite{Herrmann2007b,MRS:2010} \\
MayaKranc\_D12\_a0.80\_m103\_nj & 1 & 0.80 & 0.80 & TaylorT4 & \cite{Herrmann2007b,MRS:2010} \\
MayaKranc\_D12\_a0.00\_q2\_m90\_nj & 2 & 0.00 & 0.00 & TaylorT4 &
\cite{Herrmann2007b,MRS:2010} \\
MayaKranc\_D11\_a0.20\_q2\_m90\_nj & 2 & 0.02 & 0.09 & TaylorT4 &
\cite{Herrmann2007b,MRS:2010} \\
MayaKranc\_D10\_a0.40\_m90\_nj & 1 & 0.40 & 0.40 & TaylorT4 & \cite{Herrmann2007b,MRS:2010} \\
MayaKranc\_D10\_a0.80\_m90\_nj & 1 & 0.80 & 0.80 & TaylorT4 & \cite{Herrmann2007b,MRS:2010} \\
MayaKranc\_D12\_a0.40\_m103\_nj & 1 & 0.40 & 0.40 & TaylorT4 & \cite{Herrmann2007b,MRS:2010} \\
MayaKranc\_D12\_a0.20\_m103\_nj & 1 & 0.20 & 0.20 & TaylorT4 & \cite{Herrmann2007b,MRS:2010} \\
\hline
\end{tabular}
\end{center}
\caption[GATech submissions to NINJA-2]{
\label{tab:ninja2_gatech}
GATech submissions to NINJA-2}
\end{table}

\begin{table}
\begin{center}
\begin{tabular}{|l|r|r|r|l|c|}
\hline
Run & $q$ & Spin1${}_z$ & Spin2${}_z$ & pN Approx. & Refs \\
\hline
dq4 & 4 & 0.00 & 0.00 & TaylorT1 & \cite{Sperhake:2006cy} \\
\hline
\end{tabular}
\end{center}
\caption[LEAN submissions to NINJA-2]{
\label{tab:ninja2_lean}
LEAN submissions to NINJA-2}
\end{table}

\begin{table}
\begin{center}
\begin{tabular}{|l|r|r|r|l|c|}
\hline
Run & $q$ & Spin1${}_z$ & Spin2${}_z$ & pN Approx. & Refs \\
\hline
Llama\_d550-h64-Hybrid & 1 & 0.00 & 0.00 & 3.5pNTaylorF2 & \cite{Reisswig:2009rx,Reisswig:2009rx} \\
Llama\_d4d4-q1--D10-h64-r250.T4.hybrid & 1 & -0.40 & -0.40 & TaylorT4 & \cite{Pollney:2010hs,Pollney:2009yz} \\
Llama\_d4d4-q1--D10-h64-r250.T1.hybrid & 1 & -0.40 & -0.40 & TaylorT1 & \cite{Pollney:2010hs,Pollney:2009yz} \\
Llama\_u4u4-q1--D8-h64-r250.T1.hybrid & 1 & 0.40 & 0.40 & TaylorT1 & \cite{Pollney:2010hs,Pollney:2009yz} \\
Llama\_u4u4-q1--D8-h64-r250.T4.hybrid & 1 & 0.40 & 0.40 & TaylorT4 & \cite{Pollney:2010hs,Pollney:2009yz} \\
Llama\_d5q2-h016-Hybrid & 2 & 0.00 & 0.00 & 3.5pNTaylorF2 & \cite{Reisswig:2009rx} \\
Llama\_u2u2-q1--D8-h64-r250.T1.hybrid & 1 & 0.20 & 0.20 & TaylorT1 & \cite{Pollney:2010hs,Pollney:2009yz} \\
Llama\_u2u2-q1--D8-h64-r250.T4.hybrid & 1 & 0.20 & 0.20 & TaylorT4 & \cite{Pollney:2010hs,Pollney:2009yz} \\
Llama\_d2d2-q1--D10-h64-r250.T1.hybrid & 1 & -0.20 & -0.20 & TaylorT1 & \cite{Pollney:2010hs,Pollney:2009yz} \\
Llama\_d2d2-q1--D10-h64-r250.T4.hybrid & 1 & -0.20 & -0.20 & TaylorT4 & \cite{Pollney:2010hs,Pollney:2009yz} \\
\hline
\end{tabular}
\end{center}
\caption[Llama submissions to NINJA-2]{
\label{tab:ninja2_llama}
Llama submissions to NINJA-2}
\end{table}

\begin{table}
\begin{center}
\begin{tabular}{|l|r|r|r|l|c|}
\hline
Run & $q$ & Spin1${}_z$ & Spin2${}_z$ & pN Approx. & Refs \\
\hline
LazEV\_D8.4\_10to1\_nj\_hybrid & 10 & 0.00 & 0.00 & TaylorT4 & \cite{Campanelli:2005dd} \\
\hline
\end{tabular}
\end{center}
\caption[RIT submissions to NINJA-2]{
\label{tab:ninja2_rit}
RIT submissions to NINJA-2}
\end{table}

\begin{table}
\begin{center}
\begin{tabular}{|l|r|r|r|l|c|}
\hline
Run & $q$ & Spin1${}_z$ & Spin2${}_z$ & pN Approx. & Refs \\
\hline
SpEC\_q6s0 & 6 & 0.00 & 0.00 & TaylorT1 & \cite{SpECWebsite} \\
SpEC\_q4s0 & 4 & 0.00 & 0.00 & TaylorT2 & \cite{SpECWebsite} \\
SpEC\_EqualMassAntiAlignedSpins & 1 & -0.44 & -0.44 & NA & \cite{chu-2009,SpECWebsite} \\
SpEC\_q1s-0.95 & 1 & -0.95 & -0.95 & TaylorT1 & \cite{SpECWebsite} \\
SpEC\_q2s0 & 2 & 0.00 & 0.00 & TaylorT2 & \cite{SpECWebsite} \\
SpEC\_EqualMassAlignedSpins & 1 & 0.44 & 0.44 & NA & \cite{chu-2009,SpECWebsite} \\
SpEC\_q3s0 & 3 & 0.00 & 0.00 & TaylorT2 & \cite{SpECWebsite} \\
SpEC\_EqualMassNonspinning & 1 & 0.00 & 0.00 & TaylorT4 & \cite{Scheel:2008rj,SpECWebsite} \\
\hline
\end{tabular}
\end{center}
\caption[SpEC submissions to NINJA-2]{
\label{tab:ninja2_spec}
SpEC submissions to NINJA-2}
\end{table}

\begin{table}
\begin{center}
\begin{tabular}{|l|r|r|r|l|c|}
\hline
Run & $q$ & Spin1${}_z$ & Spin2${}_z$ & pN Approx. & Refs \\
\hline
UIUC\_spin\_-0.25\_om0.0528\_22-HYBRID & 1 & -0.25 & -0.25 & NA &
\cite{Husa-Hannam-etal:2007,Papadopoulos98,Campanelli:2008nk} \\
UIUC\_spin\_0.85\_om0.0536\_22-HYBRID & 1 & 0.85 & 0.85 & NA & 
\cite{Husa-Hannam-etal:2007,Papadopoulos98,Campanelli:2008nk} \\
\hline
\end{tabular}
\end{center}
\caption[UIUC submissions to NINJA-2]{
\label{tab:ninja2_uiuc}
UIUC submissions to NINJA-2}
\end{table}


\begin{figure}
  \includegraphics[width=\linewidth]{figures/ninja2/ninja2_cat.png}
  \caption[Parameters of the NINJA-2 submissions]{
  \label{f:ninja2_param_map}
Parameters of the NINJA-2 hybrid waveform submissions showing the
symmetric mass ratio $\eta=m_1 m_2 /(m_1+m_2)^2$ and dimensionless
spin parameter $\chi=(S_1/m_1 + S_2/m_2)/(m_1+m_2)$ after scaling the
waveforms to a reference total mass of 10 $\msun$.  The numbers indicate 
how many distinct waveforms with the specified parameters were submitted.}
\end{figure}%

\section{Verifying the Hybrid Waveforms}

Each NR group verified that their waveforms met the minimum NINJA-2
requirements before submission.  Once submitted, a series of checks
were performed in order to validate the waveforms against each other.
In the first stage the post-Newtonian expressions and codes were
compared against each other and the literature.  This required several
iterations, but resulted in a set of codes in various languages that
produce waveforms that all agree in both phase and amplitude.   The
results of this investigation have been included in the latest version
of Ref.~\cite{Brown:2007jx}, and are included in this thesis as
Appendix A.


\subsection{Time-domain and Frequency-domain Checks}

In the second stage of validation, the complete hybrid waveforms were
examined.  We first plotted the last 40 cycles of each waveform ---
enough to include the full NR portion, the hybridization region, and
some of the pN portion --- and looked for any anomalies such as those
present in some of the NINJA-1 waveforms in Fig.~\ref{fig:NR-Reh22}.
The amplitudes of the Fourier transform of the complete waveforms were
also plotted.  This analysis revealed unphysical features, primarily
due to errors in the hybridization.  An example is shown in
Fig.~\ref{f:ninja2_freq_hybrids}, which shows a visible ``kink'' in
the waveform at the hybridization frequency.  This feature is no
longer present in the waveform after the hybrid was reconstructed
correctly.

\begin{figure}
  \includegraphics[width=0.5\linewidth]{figures/ninja2/BAM_D13smm85Nep_88_T1_hyb_n2_amp_old}
  \includegraphics[width=0.5\linewidth]{figures/ninja2/BAM_D13smm85Nep_88_T1_hyb_n2_amp_new} \\
  \includegraphics[width=0.5\linewidth]{figures/ninja2/BAM_D13smm85Nep_88_T1_hyb_n2_amp_old_zoom}
  \includegraphics[width=0.5\linewidth]{figures/ninja2/BAM_D13smm85Nep_88_T1_hyb_n2_amp_new_zoom} \\
  \caption[Frequency-domain hybrid NINJA-2 waveforms]{
  \label{f:ninja2_freq_hybrids}
Fourier amplitude of the (2,2) mode of a sample NINJA-2 hybrid
waveform from the BAM/AEI group.  The waveform has been scaled to 10
$\msun$ and placed 1 Mpc from the detector to give it physical units.
The waveform on the top left is the version initially submitted, note
there is a small visible ``kink'' in the waveform at around 100 Hz.
The waveform on the top right has been re-hybridized and
there is no longer a visible kink.  On the bottom, the same waveform
before and after rehybridization, zoomed into the region in question.
This feature did not show up in the time domain view of the waveform.
Recall that the in the post-Newtonian waveform Eq.~\ref{eq:spa_waveform}
the amplitude evolved as $f^{-7/6}$.  This can be seen here as a
straight line (on the log scale) that extends up to $\approx
400$ Hz.  Above this point the slope changes as the system transitions
to merger.  The power then drops and approaches a constant frequency
as the system rings down.  
}
\end{figure}%


\subsection{Overlap Comparisons}
\label{ssec:ninja2_overlap_comparisons}

In this check the waveforms were compared against each other using
standard data-analysis techniques, in particular the overlap defined
in Eqn.~\ref{eq:OverlapDefinition}  using the initial LIGO noise
curve.  The waveforms were grouped into sets with identical
parameters.  For each set one waveform was chosen as the reference and
the overlap with all the other waveforms was calculated over a range of
masses, optimizing over the unknown coalescence time and phase. 
This process was then repeated, taking each of the other
waveforms as the reference in turn.

There is an important subtlety involved in calculating these overlaps.
Recall from Sec.~\ref{sec:ihope_match_filter} that the output of
the matched filter is a time series from which the point with the
largest value is chosen in order to maximize over time.  When
comparing two very similar waveforms, such as two numeric simulations
of the same system, the overlap function becomes very sharply peaked
in time.  It is therefore imperative that the sample rate used in
calculating the overlap be large enough to find the true maximum.

This issue is demonstrated in Fig.~\ref{f:overlap_sample_frequency}.
This figure shows the overlap function resulting from the comparison
of two equal-mass, non-spinning waveforms sampled at four different
rates.  At 4096 Hz the peak of the function is missed and the overlap
is underestimated.  The consequence of this is illustrated in
Fig.~\ref{f:overlap_wiggles} which shows that the overlap as a
function of mass exhibits oscillations.  The mass determines the
length of the waveform, as this length changes the waveforms slide
with respect to the grid of sample points, and the overlap function
traces out this beating.  Consequently, all overlaps in this chapter
were calculated at 32768 Hz.  The LIGO/Virgo matched filter searches
operate on data at 4096 Hz, however this issue is not a problem in
these searches for reasons that can be seen in these two images.  The
loss of overlap by undersampling is no larger than 0.2\%.  In
constructing the template bank (Sec.~\ref{sec:bank_metric}) we have already 
incurred a potential loss of 3\%, this effect is therefore negligible.

\begin{figure}
  \includegraphics[width=\linewidth]{figures/ninja2/overlap_time_series}
  \caption[Sensitivity of the overlaps to sample rate]{
  \label{f:overlap_sample_frequency}
The overlap time series between the Georgia Tech and SpEC equal-mass,
non-spinning waveforms at different sample rates.  At 4096 Hz the
sample point nearest the maximum is sufficiently far that the overlap
is underestimated to an extent which is significant in doing such
comparisons, where we are interested in differences to one part in
$10^{-5}$.
}
\end{figure}%

\begin{figure}
  \includegraphics[width=\linewidth]{figures/ninja2/resolutions}
  \caption[Consequence of undersampling the overlap function]{
  \label{f:overlap_wiggles}
Overlaps between the Georgia Tech and SpEC equal-mass, non-spinning
waveforms, as a function of mass, at different sample rates.  At 4096
Hz the Consequence of undersampling the overlap function.  As the
waveform lengths change with mass they beat against the griding caused
by discrete sampling.  This produces periodic beats, most evident in
the overlaps sampled at 4096 Hz.}
\end{figure}%

For reference we first present comparisons between time-domain
post-Newtonian waveforms of the kind used to construct hybrid
waveforms.  These are shown in Fig.~\ref{f:pn_overlaps}.  Overlaps
between hybrid waveforms constructed using different pN approximants
will not agree better than the values presented here at low mass,
where the pN portion of the waveform extends through the most
sensitive portion of the LIGO band.  A sample set of overlaps,
comparing all equal-mass, non-spinning waveforms submissions is shown
in Fig.~\ref{f:ninja2_overlap_test}.  At the high-mass end, where
the numeric portion of the waveform dominates the overlap, the
overlaps approach 1.   Since all these waveforms model the same
physical system, this is the expected behavior.  This is nevertheless
a significant result, as the waveforms were produced with different
codes.  At the low-mass end, where the overlap is dominated by the
post-Newtonian portions of the waveform, the behavior is qualitatively
as expected from Fig.~\ref{f:pn_overlaps}.  Although some of the
overlaps, such as between the SpEC and BAM T1 waveforms, are notably
lower than would be expected based on pN considerations alone.  This
is because different submissions hybridize at different frequencies,
so in some cases there is a loss in overlap due to comparing pN
waveforms against the hybridization.

This is more pronounced in the region of $\approx 20 \msun$ where all
the overlaps drop, some more significantly than others.  This is the
region where the hybridization is passing through the sensitive band.
The goal of hybridization is to smoothly interpolate between the pN
and NR waveforms, and in cases where the pN and NR separately agree we
would expect this agreement to extend through the hybridization
region.  We see here, however, that this is not the case.  Different
choices of hybridization methods and parameters, as well as the
frequencies over which the hybridization is performed, can lead to
waveforms with significant mismatches.  

\begin{figure}
  \includegraphics[width=\linewidth]{figures/ninja2/pn_figure03}
  \caption[Overlaps between pN waveforms as a function of mass]{
  \label{f:pn_overlaps}
Overlaps between equal-mass, non-spinning, post-Newtonian time-domain 
waveforms.  Note in particular the discrepancy between T1 and T4, as
these were used in the majority of the hybrid waveforms.
}
\end{figure}%


These initial results prompted a number of the NR groups to revise
their hybridization procedures, after which the overlaps were more in
line with the expected values.  The overlap plot produced with the
updated waveforms is shown along with the initial results in
Fig.~\ref{f:ninja2_overlap_test}.  It is worth noting that even
after rehybridizing there are still mismatches.  In particular, the
SpEC and MayaKranc submissions use the same hybridization method, the
same pN approximant and are simulating the same physical system.  The
only differences are the length of the NR waveforms and the
frequencies at which the hybridization is performed.  The question of
how many NR cycles are needed in order to produce a robust hybrid
waveform is an area of active research~\cite{MacDonald:2011}. 

\begin{figure}
  \includegraphics[width=0.5\linewidth]{figures/ninja2/q_1_z_0_figure03}
  \includegraphics[width=0.5\linewidth]{figures/ninja2/figure2_1_0_16}
  \caption[Overlaps between NINJA-2 submissions maximized over time
and phase]{
  \label{f:ninja2_overlap_test}
Overlaps between the equal-mass, non-spinning NINJA-2 contributions,
maximized over time and phase.   For the original submissions (left)
overlaps are as low as 0.70 between waveforms using different pN
approximants and 0.94 for waveforms using the same approximant.  After
rehybridization (right) the waveforms achieve much higher overlaps,
with minima above 0.94 for different approximants and above 0.98 for
identical approximants.  The residual differences between waveforms
using TaylorT4 are due to hybridization details.  The Llama waveform
was accidentally omitted from the original runs.}
\end{figure}%

The full set of comparisons between the final versions of the
submissions appears at the end of the chapter, in figures
\ref{f:figure_1_-0d5} through \ref{f:figure_1_0d85}.

The overlap plots discussed thus far address the ``detection
question.''  By validating the waveforms against each other we ensure
that they are all capture the underlying physics well enough to
provide high overlaps with existing templates, which in turn implies
that these waveforms can be found in searches.  

We also extended these overlap studies by maximizing over the mass of
one of the waveforms, as well as the time and phase.  This addresses
the ``parameter estimation question,'' the bias in the recovered mass
gives an estimate of the minimum error that can be expected in
parameter estimation pipelines.  Example plots using the equal-mass,
non-spinning MayaKranc waveform as the signal and BAM plus two
different approximants as the template are shown in
Fig.~\ref{f:ninja2_max_over_mass_bam}.


\begin{figure}
  \includegraphics[width=0.5\linewidth]{figures/ninja2/maya_bamt4_max_over_m}
  \includegraphics[width=0.5\linewidth]{figures/ninja2/maya_bamt1_max_over_m}
  \caption[Overlaps between NINJA-2 submissions maximized over mass]{
  \label{f:ninja2_max_over_mass_bam}
Overlaps (indicated by color) between the equal-mass, non-spinning
MayaKranc waveform taken as the signal, and the equal-mass,
non-spinning BAM waveform hybridized with TaylorT4 (left) and TaylorT1
(right) taken as templates.  Maximization is done over the mass of the
template, as well as over time and phase.  The x-axis gives the mass
of the signal, the y-axis gives the fractional difference between the
injected mass and the mass of the template that maximizes the overlap
Note the lower overall overlaps and mass bias at the low-mass end of
the figure on the right, where the two different pN waveforms dominate
the overlap.}
\end{figure}%

At the high-mass end the overlap is dominated by NR data, and as in
Fig.~\ref{f:ninja2_overlap_test} the overlaps are high without
needing to move off the signal mass.  At the low-mass end the same
result would be expected in a pure pN/pN comparison although there is
enough of the hybridization in-band to reduce the overlaps.  However,
changing the mass introduces a phase difference that accumulates over
all the cycles in-band, and so higher overlaps can not be achieved.
The result is optimal mass values close to the correct mass value, but
with a low overlap.  In the middle region these factors compete.  At
higher masses the overlap is reduced less by changing the mass and so
the recovered value can stray further from the injected value.
As the hybridization passes out of band this adjustment is no
longer needed.  The same general behavior can be seen in comparisons
between non-spinning, unequal-mass ($q=2$) waveforms, shown in
Fig.~\ref{f:max_over_m_q2}.  However, the overlaps drop slightly
above $\approx 60 \msun$, suggesting a disagreement in the NR portion
of the waveforms.  The comparison between two equal-mass, spinning
($s_{1z} = s_{2z} = 0.4$) waveforms is shown in
Fig.~\ref{f:max_over_m_z4}.  Here there is a bias in recovered mass
at the high-mass end, although the overlaps are high.  This could
suggest a problem with the overall scaling of one of the waveforms,
but more study is needed.


\begin{figure}
  \includegraphics[width=\linewidth]{figures/ninja2/maya_bam_q2_max_over_m} 
  \caption[Overlaps between unequal-mass submissions maximized over mass]{
  \label{f:max_over_m_q2}
Overlaps (indicated by color) between the $q=2$ non-spinning MayaKranc
waveform taken as the signal, and the $q=2$, non-spinning BAM waveform
hybridized with TaylorT4 taken as the templates.  Maximization is done
over the mass of the template, as well as over time and phase.  The
x-axis gives the mass of the signal, the y-axis gives the fractional
difference between the injected mass and the mass of the template that
maximizes the overlap.  The low overlaps above $60 \msun$ are due
to differences in the NR waveforms.}
\end{figure}%


\begin{figure}
  \includegraphics[width=\linewidth]{figures/ninja2/maya_llama_z0d4_max_over_m}
  \caption[Overlaps between spinning submissions maximized over mass]{
  \label{f:max_over_m_z4}
Overlaps (indicated by color) between the equal-mass $s_{1z} = s_{2z}
= 0.4$ MayaKranc waveform taken as the signal, and the equal-mass
$s_{1z} = s_{2z} = 0.4$ Llama waveform taken
as the template.  Maximization is done over the mass of the template,
as well as over time and phase.  The x-axis gives the mass of the
signal, the y-axis gives the fractional difference between the
injected mass and the mass of the template that maximizes the overlap.
The low overlaps above $60 \msun$ are due to differences in the NR
waveforms.  The reason for the systematic bias at the high-mass end is
under investogation.}
\end{figure}%


\section{Conclusion}

We have reviewed the hybrid waveforms submitted to the second NINJA
project.  Although many groups have contributed many waveforms, these
primarily cover only two lines  in the (mass ratio, $\chi$) plane,
leaving large regions of parameter space unexplored.  We also do not
consider precessing signals, which adds several dimensions to
parameter space.  These issues will be further explored in NINJA-3.  A
major feature of NINJA-2, not present in NINJA-1, is the validation
and comparisons of the submissions discussed in this chapter.  These
efforts lead to dramatic improvements in the quality of the waveforms,
which in turn will enable data analysts to draw more quantitative
conclusions on the behavior of their pipelines with respect to the
underlying physics.  There is evidence that hybridization choices and
methods effect both overlaps and parameter estimation.   The degree to
which these effects will bias searches is a question we hope NINJA-2
will be able to answer.  In the next chapter we discuss the
construction of data sets using these waveforms, and present
preliminary results from the CBC low- and high-mass pipelines.


%%%%%%%%%%%%%%%%
\clearpage

\iffalse
\begin{figure}
  \includegraphics[width=0.5\linewidth]{figures/ninja2/pn_figure03.png} 
  \includegraphics[width=0.5\linewidth]{figures/ninja2/pn_figure06.png} \\
  \includegraphics[width=0.5\linewidth]{figures/ninja2/pn_figure09.png} 
  \includegraphics[width=0.5\linewidth]{figures/ninja2/pn_figure12.png} \\
  \caption[Overlap of time-domain pN waveforms, $q=1$ $S_{z1} = S_{z2} = 0$]{
  \label{f:figure_pn}
Overlap of time-domain pN waveforms, $q=1$ $S_{z1} = S_{z2} = 0$}
\end{figure}%
\fi

\begin{figure}
  \includegraphics[width=0.5\linewidth]{figures/ninja2/figure_1_-0d5_01.png}
  \includegraphics[width=0.5\linewidth]{figures/ninja2/figure_1_-0d4_01.png} \\
  \includegraphics[width=0.5\linewidth]{figures/ninja2/figure_1_-0d2_01.png}
  \includegraphics[width=0.5\linewidth]{figures/ninja2/figure_1_0d8_01.png}
  \caption[Multiple NINJA-2 overlap plots]{
  \label{f:figure_1_-0d5}
Top left: overlap plot for $q=1$ $S_{z1} = S_{z2} = -0.5$
Top right: overlap plot for $q=1$ $S_{z1} = S_{z2} = -0.4$
Bottom left overlap plot for $q=1$ $S_{z1} = S_{z2} = -0.2$
Bottom right: overlap plot for $q=1$ $S_{z1} = S_{z2} = 0.8$}
\end{figure}%

\begin{figure}
  \includegraphics[width=0.5\linewidth]{figures/ninja2/figure_3_0_02.png} 
  \includegraphics[width=0.5\linewidth]{figures/ninja2/figure_3_0_04.png} \\
  \includegraphics[width=0.5\linewidth]{figures/ninja2/figure_3_0_06.png} 
  \caption[Overlap plots for $q=3$ $S_{z1} = S_{z2} = 0$]{
  \label{f:figure_3_0}
Overlap plots for $q=3$ $S_{z1} = S_{z2} = 0$}
\end{figure}%


\begin{figure}
  \includegraphics[width=0.5\linewidth]{figures/ninja2/figure_1_0d2_03.png} 
  \includegraphics[width=0.5\linewidth]{figures/ninja2/figure_1_0d2_06.png} \\
  \includegraphics[width=0.5\linewidth]{figures/ninja2/figure_1_0d2_09.png} 
  \includegraphics[width=0.5\linewidth]{figures/ninja2/figure_1_0d2_12.png} \\
  \caption[Overlap plots for $q=1$ $S_{z1} = S_{z2} = 0.2$]{
  \label{f:figure_1_0d2}
Overlap plots for $q=1$ $S_{z1} = S_{z2} = 0.2$}
\end{figure}%


\begin{figure}
  \includegraphics[width=0.5\linewidth]{figures/ninja2/figure_1_-0d25_02.png} 
  \includegraphics[width=0.5\linewidth]{figures/ninja2/figure_1_-0d25_04.png} \\
  \includegraphics[width=0.5\linewidth]{figures/ninja2/figure_1_-0d25_06.png} 
  \caption[Overlap plots for $q=1$ $S_{z1} = S_{z2} = -0.25$]{
  \label{f:figure_1_-0d25}
Overlap plots for $q=1$ $S_{z1} = S_{z2} = -0.25$}
\end{figure}%



\begin{figure}
  \includegraphics[width=0.5\linewidth]{figures/ninja2/figure_2_0_04.png} 
  \includegraphics[width=0.5\linewidth]{figures/ninja2/figure_2_0_08.png} \\
  \includegraphics[width=0.5\linewidth]{figures/ninja2/figure_2_0_12.png} 
  \includegraphics[width=0.5\linewidth]{figures/ninja2/figure_2_0_16.png} \\
  \includegraphics[width=0.5\linewidth]{figures/ninja2/figure_2_0_20.png} 
  \caption[Overlap plots for $q=2$ $S_{z1} = S_{z2} = 0$]{
  \label{f:figure_2_0}
Overlap plots for $q=2$ $S_{z1} = S_{z2} = 0$}
\end{figure}%



\begin{figure}
  \includegraphics[width=0.5\linewidth]{figures/ninja2/figure_1_0_04.png} 
  \includegraphics[width=0.5\linewidth]{figures/ninja2/figure_1_0_08.png} \\
  \includegraphics[width=0.5\linewidth]{figures/ninja2/figure_1_0_12.png} 
  \includegraphics[width=0.5\linewidth]{figures/ninja2/figure_1_0_16.png} \\
  \includegraphics[width=0.5\linewidth]{figures/ninja2/figure_1_0_20.png} 
  \caption[Overlap plots for $q=1$ $S_{z1} = S_{z2} = 0$]{
  \label{f:figure_1_0}
Overlap plots for $q=1$ $S_{z1} = S_{z2} = 0$}
\end{figure}%


\begin{figure}
  \includegraphics[width=0.5\linewidth]{figures/ninja2/figure_1_0d5_01.png}
  \includegraphics[width=0.5\linewidth]{figures/ninja2/figure_1_0d25_01.png} \\
  \includegraphics[width=0.5\linewidth]{figures/ninja2/figure_1_-0d75_01.png}
  \includegraphics[width=0.5\linewidth]{figures/ninja2/figure_1_-0d85_01.png} \\
  \includegraphics[width=0.5\linewidth]{figures/ninja2/figure_1_0d75_01.png}
  \caption[Multiple NINJA-2 overlap plots]{
  \label{f:figure_1_0d5}
Top left: overlap plot for $q=1$ $S_{z1} = S_{z2} = 0.5$
Top right: overlap plot for $q=1$ $S_{z1} = S_{z2} = 0.25$
Middle left: overlap plot for $q=1$ $S_{z1} = S_{z2} = -0.75$
Middle right: overlap plot for $q=1$ $S_{z1} = S_{z2} = -0.85$
Bottom left: overlap plot for $q=1$ $S_{z1} = S_{z2} = 0.75$}
\end{figure}%

\begin{figure}
  \includegraphics[width=0.5\linewidth]{figures/ninja2/figure_1_0d4_03.png} 
  \includegraphics[width=0.5\linewidth]{figures/ninja2/figure_1_0d4_06.png} \\
  \includegraphics[width=0.5\linewidth]{figures/ninja2/figure_1_0d4_09.png} 
  \includegraphics[width=0.5\linewidth]{figures/ninja2/figure_1_0d4_12.png} \\
  \caption[Overlap plots for $q=1$ $S_{z1} = S_{z2} = 0.4$]{
  \label{f:figure_1_0d4}
Overlap plots for $q=1$ $S_{z1} = S_{z2} = 0.4$}
\end{figure}%


\begin{figure}
  \includegraphics[width=0.5\linewidth]{figures/ninja2/figure_4_0_03.png} 
  \includegraphics[width=0.5\linewidth]{figures/ninja2/figure_4_0_06.png} \\
  \includegraphics[width=0.5\linewidth]{figures/ninja2/figure_4_0_09.png} 
  \includegraphics[width=0.5\linewidth]{figures/ninja2/figure_4_0_12.png} \\
  \caption[Overlap plots for $q=4$ $S_{z1} = S_{z2} = 0$]{
  \label{f:figure_4_0}
Overlap plots for $q=4$ $S_{z1} = S_{z2} = 0$}
\end{figure}%


\begin{figure}
  \includegraphics[width=0.5\linewidth]{figures/ninja2/figure_1_0d85_02.png} 
  \includegraphics[width=0.5\linewidth]{figures/ninja2/figure_1_0d85_04.png} \\
  \includegraphics[width=0.5\linewidth]{figures/ninja2/figure_1_0d85_06.png} 
  \caption[Overlap plots for $q=1$ $S_{z1} = S_{z2} = 0.85$]{
  \label{f:figure_1_0d85}
Overlap plots for $q=1$ $S_{z1} = S_{z2} = 0.85$}
\end{figure}%

\clearpage
%%%%%%%%%%%%%%%%

