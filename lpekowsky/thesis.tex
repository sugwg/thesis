%-----------------------------------------------------------------------
%
% File Name: thesis.tex
%
% Author: Pekowsky, L. P.
%
% Revision: $Id$
%
%-----------------------------------------------------------------------

% document class and packages
\documentclass[12pt,notitlepage]{report}
\usepackage{bibunits}
\usepackage{syrthesis}
\usepackage{graphicx}
\usepackage{color}
\usepackage{amsmath}
\usepackage{amssymb}
\usepackage{amsfonts}
\usepackage{rotating}
\usepackage{tensor}
\usepackage{lscape}
\usepackage{units} %  do things like \units[1.234]{sec}i

\makeatletter
\let\protect\relax
{\catcode`\|=\active
  \xdef\InnerProduct{\protect\expandafter\noexpand\csname InnerProduct
\endcsname}
  \expandafter\gdef\csname InnerProduct \endcsname#1{%
    \begingroup
    \ifx\SavedDoubleVert\relax
    \let\SavedDoubleVert\|\let\|\IpDoubleVert
    \fi
    \mathcode`\|32768\let|\IPVert
    \left({#1}\right)
    \endgroup
  }
}
\def\IPVert{\@ifnextchar|{\|\@gobble}% turn || into \|
     {\egroup\,\mid@vertical\,\bgroup}}
\def\IPDoubleVert{\egroup\,\mid@dblvertical\,\bgroup}
\let\SavedDoubleVert\relax
\def\midvert{\egroup\mid\bgroup}
\def\SetVert{\@ifnextchar|{\|\@gobble}% turn || into \|
    {\egroup\;\mid@vertical\;\bgroup}}
\def\SetDoubleVert{\egroup\;\mid@dblvertical\;\bgroup}
\def\mid@vertical{\mskip1mu\vrule\mskip1mu}
\def\mid@dblvertical{\mskip1mu\vrule\mskip2.5mu\vrule\mskip1mu}
\makeatother
\usepackage{braket}
\newcommand{\Overlap}{\Braket}

\usepackage{xspace}
\usepackage{url}
\usepackage{alltt}

\pdfoutput=1
\DeclareGraphicsExtensions{.pdf,.png}

\hbadness=10000

% new command definitions
\newcommand{\half}{\frac{1}{2}}
\newcommand{\ospsd}{\ensuremath{S_n\left(\left|f_{k}\right|\right)}}

\newcommand{\msun}{M_\odot}
\newcommand{\chisq}{\chi^2}
\newcommand{\newsnr}{\rho_{\textrm{new}}}
\newcommand{\erf}{\mathrm{erf}}
\newcommand\fake[1]{\textcolor{red}{#1}}
\newcommand\checkme[1]{\textcolor{blue}{\textbf{#1}}}
\newcommand{\Note}[1]{\textcolor{red}{\textbf{[#1]}}}

\newcommand{\darmerr}{\texttt{DARM\_ERR} }

\newcommand\weakheader[1]{
\vspace*{5mm}
\noindent {\it #1}
\vspace*{5mm}
}


\usepackage{hyperref} 

\begin{document}

\title{
Characterization of Enhanced Interferometric Gravitational Wave
detectors and Studies of Numeric Simulations for Compact-Binary
Coalescences
}
\author{\bf Larne Pekowsky}
\majorprof{Duncan Brown}
\submitdate{August 2011}
\degree{Doctor of Philosophy}
\program{Physics}
\copyrightyear{2011}
\majordept{Physics}
\havededicationtrue
\dedication{to\\ my parents}
\haveminorfalse
\copyrighttrue
\doctoratetrue
\figurespagetrue
\tablespagetrue


\Abstract{
Gravitational waves are a consequence of the general theory of
relativity.  Direct detection of such waves will provide a wealth of
information about physics, astronomy, and cosmology.  A worldwide
effort is currently underway to make the first  direct detection of
gravitational waves.  The global network of detectors includes the
Laser Interferometer Gravitational-wave Observatory (LIGO), which
recently completed its sixth science run.

A particularly promising source of gravitational waves is a binary
system consisting of two neutron stars and/or black holes.  As the
objects orbit each other they emit gravitational radiation, lose
energy, and spiral inwards.  This produces a characteristic ``chirp''
signal for which we can search in the LIGO data.  Currently this is
done using matched-filter techniques, which correlates the detector
data against analytic models of the emitted gravitational waves.
Several choices must be made in searching for for signals from such
binary coalescences. 

Any discrepancy between the signals and the models used will reduce
the effectiveness of the matched filter.  However, the analytic models
are based on approximations which are not valid through the entire
evolution of the binary.  In recent years numerical relativity has had
impressive success in simulating the final phases of the coalescence
of binary black holes.  While numerical relativity is too
computational expensive to use directly in the search, this progress
has made it possible to perform realistic tests of the LIGO searches.
The results of such tests can be used to improve the efficiency of
searches.

Conversely, noise in the LIGO and Virgo instruments can reduce the
efficiency.  This must be addressed by characterizing the quality of
the data from the detectors, and removing from the analysis times that
will be detrimental to the search.

In this thesis we utilize recent results from numerical relativity to
study both the degree to which analytic models match realistic
waveforms and the ability of LIGO searches to make detections.  We
also apply the matched-filter search to the problem of removing times
of excess of noise from the search.
}
\beforepreface
\prefacesection{Preface}
The work presented in this thesis stems from my participation in the LIGO
Scientific Collaboration and the NINJA Collaboration.


\vspace*{0.5cm}

\noindent Chapter \ref{ch:comparison} is based on material from

\vspace*{0.25cm}

\noindent M. Boyle, D. A. Brown and L. Pekowsky, ``Comparison of high-accuracy
numerical simulations of black-hole binaries with stationary phase
post-Newtonian template waveforms for Initial and Advanced LIGO,''
{\it Class. Quant. Grav.} {\bf 26}, 114006 (2009)

\vspace*{0.5cm}

\noindent Chapter \ref{ch:ninja1} is based on material from

\vspace*{0.25cm}

\noindent B. Aylott {\it et al} (The NINJA Collaboration), ``Testing
gravitational-wave searches with numerical relativity waveforms:
Results from the first Numerical INJection Analysis (NINJA) project,''
{\it Class. Quant. Grav.} {\bf 26}, 165008 (2009)


\vspace*{0.5cm}

\noindent Chapter \ref{ch:segdb} is based on material from

\vspace*{0.25cm}

\noindent D. Brown {\it et al}, ``S6 Segment Database
Infrastructure Proposal,'' {\it Technical document
{LIGO}-T0900005-00-Z} (2010)


\prefacesection{Acknowledgments}

I have been fortunate to work with a large number of people through my
membership in the LIGO and NINJA collaborations.  Sadly space makes it
impossible to thank them all individually, and I hope the vast
majority I must omit will forgive me.

I would first and foremost like to thank my advisor, Duncan Brown.  I
have benefited immensely from Duncan's vast knowledge of and passion
for gravitational-wave astronomy.  I could not have asked for a better
mentor, nor did I ever expect to find one with whom I could share so
many laughs.

I would also like to thank Andrew Lundgren, with whom I worked on
detchar.  Andy was great fun to work with, and his brilliance is
matched only by his passion for squashing glitches in the detectors.
Much of this work was also done with Josh Smith, to whom I am also
grateful.

I am indebted to the other members of the Syracuse University
gravitational-wave faculty, Peter Saulson and Stefan Ballmer for their
many excellent questions and suggestions regarding my work, and clear
and detailed explanations of their own.

The NINJA project has introduced me to a great many numerical
relativists from whom I have learned a great deal.  I would like
especially to thank Mike Boyle, with whom I worked on the studies in
chapter~\ref{ch:comparison}, Harald Pfeiffer, and Deirdre Shoemaker.

On the data-analysis side of NINJA, I would like to thank Satya
Mohapatra for his help in testing the data sets and numerous other
contributions as well as many fascinating conversations at
conferences.

It has been a joy to share an office with Collin Capano, Justin
Garofoli, and Matt West, and I would like to thank them for their
innumerable assistance and physics bull sessions.

I would also like to thank Ping Wei, with whom I worked on the segment
database, and Peter Couvares for all his help with Condor and friendly
chat about computer science and other things.

None of this would have been possible without the love and support of
my parents, to whom I am eternally grateful.  Finally, my thanks to my
long-time friends, known collectively as "the canetoads" for their
words of encouragement and all the good times.


\prefacesection{Conventions}

We adopt the Einstein summation convention where repeated indices are
summed over.  Parenthesis are shorthand for the fully symmetric sum
%
\begin{equation*}
A_{(\alpha\beta)}
= \frac{1}{2} \left(A_{\alpha\beta} 
+ A_{\beta\alpha} \right)
\end{equation*}
%
and square brackets are fully antisymmetric
%
\begin{equation*}
A_{[\alpha\beta]}
= \frac{1}{2} \left(A_{\alpha\beta} 
- A_{\beta\alpha} \right)
\end{equation*}
%
We take the signature of spacetime to be $(-1,+1,+1,+1)$.

Except where otherwise noted we work in geometric units where
$G=c=1$.

\vspace{0.5cm}

We define the Fourier transform of a function of time $g(t)$ to be
$\tilde{g}(f)$, where
%
\begin{equation*}
\tilde{g}(f)=\int_{-\infty}^\infty g(t)\, e^{- 2 \pi i f t}\, dt
\end{equation*}
%
and the inverse Fourier transform is
%
\begin{equation*}
g(t)=\int_{-\infty}^\infty \tilde{g}(f)\, e^{2 \pi i f t}\, dt
\end{equation*}
%
This convention differs from that used in some gravitational-wave
literature, but is the adopted convention in the LIGO Scientific
Collaboration.

\vspace{0.5cm}

\noindent The time-stamps of interferometer data are measured in
Global Positioning System (GPS) seconds: seconds since 00:00.00 UTC
January 6, 1980 as measured by an atomic clock.


\afterpreface

\Chapter{Introduction}
\label{ch:introduction}
In 1915, Albert Einstein published his theory of general relativity, a 
geometric theory of gravitation that sought to expand upon Newtonian 
mechanics and provide a complete description of gravity and its 
relationship with space and time. Einstein theorized that space 
and time were deeply related and existed together as a manifold 
called spacetime. Matter with energy and momentum 
existing in this manifold would create 
curvature in spacetime. Gravitational forces were the result of 
matter following geodesic curves in spacetime. This concept can 
be summarized in the Einstein field equation, which is presented 
as,
\begin{equation}
G_{\mu\nu} = 8\pi T_{\mu\nu}
\label{eq:EFE}
\end{equation}
where $G_{\mu\nu}$ is the Einstein tensor, which describes the 
curvature of spacetime, $T_{\mu\nu}$ is the 
stres-energy tensor, which describes the energy and momentum in 
spacetime, and  $G=c=1$. The Einstein tensor is defined as,
\begin{equation}
G_{\mu\nu} = R_{\mu\nu} - \frac{1}{2}Rg_{\mu\nu}
\end{equation}
where $R_{\mu\nu}$ is the Ricci curvature tensor and $g_{\mu\nu}$ is 
the metric tensor for the manifold.

An interesting result that arises from this formalism is the 
existence of gravitational waves, which are perturbations in 
spacetime caused by certain types of time-varying mass distributions. 
To describe gravitational waves, we consider 
a Minkowski metric with a small perturbation. The Minkowski metric 
is a flat spacetime metric defined as
\begin{equation}
\eta_{\mu\nu} = 
  \begin{pmatrix}
   -1 & 0 & 0 & 0 \\
    0 & 1 & 0 & 0 \\
    0 & 0 & 1 & 0 \\
    0 & 0 & 0 & 1
  \end{pmatrix}
\end{equation}
where $\mu = 0$ corresponds to the time coordinate and $\mu = {1,2,3}$ 
correspond to the spatial coordinates. In examples, we will use the coordinate 
convention $(x^0,x^1,x^2,x^3) = (ct,x,y,z)$. 
The full spacetime metric, $g_{\mu\nu}$, is then constructed as a 
linear perturbation on the Minkowski metric,
\begin{equation}
g_{\mu\nu} = \eta_{\mu\nu} + h_{\mu\nu}
\end{equation}
where $h_{\mu\nu}$ is the metric perturbation and $|h_{\mu\nu}| \ll 1$.
From here, we follow the convention of Saulson \cite{Saulson:1994} to arrive at the general 
form of a gravitational wave.
At this point it is very useful to move into the transverse traceless 
gauge where coordinates on the manifold are defined by the geodesic 
motion of freely-falling test masses. In this gauge, the weak field 
vacuum solution of the Einstein field equation becomes a wave equation: 
\begin{equation}
\square h_{\mu\nu} = 0.
\end{equation}
The solutions to this differential equation will be plane waves of 
the form
\begin{equation}
h_{\mu\nu} = C_{\mu\nu}e^{i(2\pi ft - \vec{k}\cdot\vec{x})}
\end{equation}
where $C_{\mu\nu}$ is the wave amplitude, $f$ is the frequency, 
and $\vec{k}$ is the wave vector which indicates the direction of 
propagation \cite{Carroll}.

For example, consider the case of a gravitational 
wave propogating along the $\hat{z}$-axis.
When the conditions of the transverse traceless gauge are applied, 
the resulting form of $h_{\mu\nu}$ is 
\begin{equation}
h_{\mu\nu} = 
  \begin{pmatrix}
    0 & 0 & 0 & 0 \\
    0 & h_+ & h_x & 0 \\
    0 & h_x & -h_+ & 0 \\
    0 & 0 & 0 & 0
  \end{pmatrix}
\end{equation}
where the diagonal and off-diagonal terms represent two polarizations 
of the resulting gravitational wave, called "h-plus" and "h-cross" 
respectively.
We can see the effects of this perturbation by observing the  
spacetime interval on the manifold. The spacetime interval is defined as 
\begin{equation}
ds^2 = dx^\mu g_{\mu\nu}dx^\nu.
\end{equation}
Substituting in our perturbed metric for $g_{\mu\nu}$, we find that 
the spacetime interval can be broken up into a standard Minkowski line 
element and a perturbation due to $h_{\mu\nu}$.
\begin{equation}
ds^2 = dx^\mu (\eta_{\mu\nu} + h_{\mu\nu})dx^\nu \\
\end{equation}
\begin{equation}
ds^2 = dx^\mu \eta_{\mu\nu} dx^\nu + dx^\mu h_{\mu\nu}dx^\nu
\label{eq:spacetime}
\end{equation}

As an example, we present the case of a plus-polarized gravitational wave 
propagating in the $\hat{z}$ direction and observe the effect of the perturbation 
on the spacetime interval. The perturbation will have the form 
\begin{equation}
h_{\mu\nu} = 
  \begin{pmatrix}
    0 & 0 & 0 & 0 \\
    0 & h_+ & 0 & 0 \\
    0 & 0 & -h_+ & 0 \\
    0 & 0 & 0 & 0
  \end{pmatrix}
\end{equation}
Using the coordinate convention of $(ct,x,y,z)$, the unperturbed
spacetime interval is given as: 
\begin{equation}
ds^2 = -c^2 dt^2 + dx^2 + dy^2 + dz^2.
\end{equation}
Since the perturbation is spatially transverse to the direction of 
propagation, the ct- and z-coordinates will not be modulated by the 
gravitational wave. The x- and y-coordinates will be modulated  
according to equation \ref{eq:spacetime}. The resulting spacetime 
interval is
\begin{equation}
ds^2 = -c^2 dt^2 + (1 + h_+)dx^2 + (1 - h_+)dy^2 + dz^2.
\end{equation}
This shows that a gravitational wave propogating along the $\hat{z}$-axis 
will differentially stretch and squeeze spacetime in the transverse 
axes. The exact form of $h_+$ will depend on the source of the 
gravitational waves. A visualization of this stretching and squeezing 
is shown in Figure \ref{fig:polarizations}\cite{Polarization}. The cross polarization  
stretches and squeezes at a 45 degree angle relative to the plus 
polarization.

\begin{figure}[ht!]
\includegraphics[width=\textwidth]{figures/introduction/polarisations2}
\caption[Plus and cross polarizations]{Plus and cross polarizations %
         of a gravitational wave.}
\label{fig:polarizations}
\end{figure}

The Advanced LIGO interferometers are designed to be sensitive 
to this differential stretching and squeezing by constructing orthogonal 
optical cavities. A gravitational wave passing through an aLIGO interferometer 
will differentially 
modulate the lengths of the optical cavities, creating an interference 
pattern at the output of the instrument that can be searched for 
gravitational wave signals. The layout and gravitational wave readout scheme 
of the interferometers is discussed below.

\section{The Advanced LIGO Interferometers}\label{sec:aligo}

The Advanced LIGO (aLIGO) interferometers are a pair of dual-recycled Michelson interferometers 
that employ 4km long Fabry-Perot cavities in their arms to increase the interaction time with a 
gravitational wave signal. 
Figure \ref{fig:aligo} shows a simplified layout of an aLIGO interferometer. 

\begin{figure}[ht!]
\includegraphics[width=\textwidth]{figures/introduction/ALIGO_layout}
\caption[Layout of Advanced LIGO]{Layout of Advanced LIGO}
\label{fig:aligo}
\end{figure}

At the input to an aLIGO interferometer is a solid-state Nd:YAG laser that provides laser light 
at a wavelength of 1064 nm. Not included in Figure \ref{fig:aligo} are frequency and 
intensity stabilization control loops designed to provide as stable a laser source as 
possible for the experiment. This stabilized laser is called the pre-stabilized laser 
(PSL). The laser light is passed through a series of 
electro-optic modulators (EOM) where radio-frequency (RF) sidebands are generated 
and imparted onto the light. These RF sidebands are used to control auxiliary optical 
degrees of freedom in the interferometer. The beam is then passed through the 
input mode cleaner (IMC), which rejects higher order spatial modes of the beam 
and transmits a circular TEM00 mode to be used in the instrument.

Once the beam has been stabilized in frequency and intensity and the higher order 
optical modes have been stripped away, it is transmitted through the power 
recycling mirror and enters the vertex of the interferometer. In the vertex, 
the beam is split 50/50 by the beamsplitter. Half of the light is directed toward  
the input test mass (ITM) of the X-arm and half of the light is directed  
toward the ITM of the Y-arm. As mentioned previously, the aLIGO arms are not 
single bounce cavities; they are comprised of Fabry-Perot cavities that allow the 
light to circulate in the arm cavities multiple times. The light is stored in 
the arm cavities for $\sim$1ms, trapped between the highly reflective surfaces 
of the ITM and the end test mass (ETM), before it is transmitted back through 
the ITM and into the vertex.

When a gravitational wave passes through an aLIGO inteferometer, the distance
between the ITM and ETM of each arm is modulated, causing the light to have a
longer or shorter travel time as it traverses the arm. Since gravitational
waves expand space in one direction while the orthogonal direction contracts,     
the X- and Y-arms will experience differential changes in length. When light
from the arms is recombined at the beamsplitter, there will be a difference
in phase between the two beams as they have traveled different paths. The 
resulting light from this recombination of phase shifted beams is called the 
antisymmetric part of the output. The part of the beam that is recombined 
in phase is called the symmetric part of the output.

The beams returning from each arm are recombined at the beamsplitter. The 
symmetric part of the beam 
will be sent back toward the power recycling mirror. The power recycling mirror 
forms a resonant cavity with the ITMs, allowing for light at the symmetric 
port of the beamsplitter to be added coherently to incoming light from the PSL and 
increasing the effective power in the vertex. This increase in effective power 
is known as the power recycling gain. 

The antisymmetric part of the beam is sent toward the signal recycling mirror. 
The signal recycling cavity is used to tune the frequency response of the 
interferometer by adjusting the effective finesse of the coupled cavity 
formed by the signal recycling cavity and the arm cavities. 
If the light returning from the arms has accumulated some differential amount of 
phase as it traveled 
along the arms, perhaps from a gravitational wave modulating the length of each 
arm differentially, it will be transmitted through the signal recycling cavity 
and into the output mode cleaner (OMC). The OMC behaves similarly to the IMC, 
stripping away higher order optical modes and isolating the TEM00 mode of the 
beam. The transmitted, mode cleaned signal is then read out using a homodyne 
detection scheme on a DC photodiode. 

\subsection{DC Readout}

When a gravitational wave modulates the length of an arm cavity, the light 
traveling in that arm experiences a phase modulation. This phase modulation 
can be visualized by picturing the beam in frequency space. In figure 
\ref{fig:omc-freq}, the carrier beam frequency is designated as $f_0$. 
The phase modulation due to 
a gravitational wave signal introduces a frequency sideband at the 
gravitational wave frequency, which is in the 30-2000 Hz range. 
The 
RF sidebands used for auxiliary optical cavity control are offset from the 
carrier frequency by 9, 24, and 45 MHz. 
The RF sidebands, which in a 
homodyne detection scheme would only contribute shot noise to the output signal, 
are rejected by the OMC. The gravitational wave sidebands, however, are at a 
low enough frequency offset that they are within the cavity pole of the OMC 
and are allowed to transmit through the cavity.

Since the OMC DC photodiode measures power, it measures the square of the 
incident optical field and witnesses beat frequencies between different 
components of the light. If the RF sidebands have been filtered out by 
the OMC, the only remaining beat note will be that of the carrier beam ($f_0$) 
beating against the gravitational wave sideband ($f_0 + f_{GW}$). This beat note will 
appear as the difference in frequency between the two optical fields, 
leaving behind a signal in the 30-2000 Hz range ($f_{GW}$) and providing a 
natural demodulation inherent to the measurement process. 
The process of recovering the gravitational wave sideband using the 
carrier field as a reference is known as homodyne detection. 

\begin{figure}[ht!]
\includegraphics[width=\textwidth]{figures/introduction/omc-freq}
\caption[Sidebands and OMC cavity pole]{Frequency domain visualization of beam %
         at OMC. Grey dotted lines indicate the cavity pole. The gravitational %
         wave sidebands are within the cavity pole and are transmitted through %
         the OMC. The RF sidebands are in the MHz range and are rejected by the %
         OMC.}
\end{figure}\label{fig:omc-freq}

\section{Sources of Gravitational Waves}
Include that box with modeled, unmodeled, transient, and continuous.

CBCs are the bread and butter, expect BNS, NSBH, and BBH sources
Continuous waves from pulsars
Bursts from supernovae
Stochastic background


\section{Searching for Compact Binary Coalescences}

Steal this from O1 CBC DQ paper


\section{The Advanced Detector Network}

The Advanced Laser Interferometer Gravitational-Wave Observatory (aLIGO) is 
part of a worldwide effort to detect gravitational waves from astrophysical 
sources. The two aLIGO interferometers, one in Hanford, WA and one in 
Livingston, LA, are part of a growing network of ground-based interferometric 
gravitational wave detectors. Each aLIGO interferometer has 4km long arms 
arranged in an L-shaped configuration. A gravitational wave passing through 
an aLIGO interferometer will cause the arms to expand and contract, 
creating an interferometric signal at the output of the instrument. 
Section \ref{sec:aligo} contains a more detailed description of the aLIGO 
interferometers. 

There are a number of other interferometric gravitational wave detectors 
being built and commissioned for future use in collaboration with aLIGO.
The Advanced VIRGO detector is being built and commissioned in Cascina, Italy. 
When it is fully commissioned, VIRGO will be joining LIGO in observing runs. 
The VIRGO interferometer has 3km arms, which should provide enough 
sensitivity to allow for triangulation of astrophysical sources.

The GEO600 detector, located in Hanover, Germany is an interferometer built in 
collaboration between Germany and the United Kingdom. 
GEO600 is an extremely valuable test bed for interferometric technologies,
including quantum optics and homodyne detection. However, with 600m arms, GEO600 
is unlikely to be sensitive enough to witness expected astrophysical sources.

The KAGRA detector, located underground in the Kamioka mine in Japan, 
is in its commissioning phase. KAGRA has 3 km long arms and, 
unlike other gravitational wave interferometers, employs cryogenics to 
reduce thermal noise in its optics. When complete, KAGRA should be 
sensitive enough to contribute to the worldwide detector network.

Include that cool picture of the advanced detector network.


\Chapter{Gravitational wave Theory and Source Modeling}
\label{ch:theory}
In this chapter we briefly survey the theoretical issues behind
gravitational-wave astronomy.  We start in
Sec.~\ref{sec:general_relativity} with a review of general
relativity, beginning with the relevant mathematics.  In
Sec.~\ref{sec:gravitational_radiation} we show how general
relativity predicts the existence of gravitational radiation and
discuss some of the properties of this radiation.  Then in
Sec.~\ref{sec:effects_of_waves} we show how gravitational radiation
affects freely-falling particles.  This will motivate the design of
the LIGO experiment to search for gravitational waves, an overview of
which is presented in the next chapter.  We then move to the
generation of gravitational waves and discuss two approaches to
modeling the waves produced by the inspiral and eventual merger of
pairs of compact objects.  These are analytic models, discussed in
Sec.~\ref{sec:PNWaveforms} and numerical models, discussed in
Sec.~\ref{sec:NRWaveforms}.


\section{General Relativity}
\label{sec:general_relativity}

We start with an overview of differential geometry and build to
Einstein's equations.  This is of necessity very brief, readers are
referred to the textbooks by Misner, Thorne and Wheeler~\cite{MTW} and
Carroll~\cite{carrollTextbook} for more complete treatments.

\subsection{Elements of Differential Geometry}

An $n$-dimensional ($C^\infty$) manifold $\mathcal{M}$ is a set of
points plus an \emph{atlas}, a set of \emph{charts} $\{\phi_i\}$ which
are invertible maps from open subsets of $\mathcal{M}$ to open subsets
of $\mathcal{R}^n$ such that


\begin{itemize}
\item For all points $p \in \mathcal{M}$ there exists an $\phi_i$ 
such that $p$ is in the domain of $\phi_i$.
\item The composition $\phi_i \circ \phi_j^{-1}$ on the 
intersections of the domains of $\phi_i$ and $\phi_j$ is a
($C^\infty$) function from $\mathcal{R}^n \to \mathcal{R}^n$.
\end{itemize}
%
Two natural structures on a manifold are curves, maps from
$\mathcal{R}\to\mathcal{M}$, and scalar functions,  maps from
$\mathcal{M}\to\mathcal{R}$.  Compositing a function $f$ with a curve
$\gamma(\lambda)$ gives a map from $\mathcal{R} \rightarrow
\mathcal{R}$ which may be differentiated in the usual way at a point
$p$.

\iffalse
\begin{equation*}
\frac{d}{d \lambda} fi \big|_p = 
  \frac{d}{d\lambda} (f \circ \gamma) \big|_p
\end{equation*}


Expanding this in terms of a chart whose domain includes $p$ and then
applying the chain rule
 
\begin{align*}
\frac{d}{d \lambda} fi \big|_p &= 
 \frac{d}{d\lambda} ( (f \circ \phi^{-1}) \circ (\phi \circ
\lambda) ) \\
&= \frac{d(\phi^{-1} \circ \gamma)^\mu}{d\lambda} 
\frac{\partial (f \circ \phi^{-1}) }{\partial x^\mu} \big|_p \\
&= \frac{dx^\mu}{d\lambda} \partial_\mu f \big|_p
\end{align*}

where $x^\mu$ are the coordinates on $\mathcal{R}^n$.  Finally, since
the function $f$ is arbitrary we can define

\begin{equation}
\label{eq:tangent_vector}
\frac{d}{d\lambda} = \frac{dx^\mu}{d\lambda} \partial_\mu
\end{equation}
\fi

Geometrically, taking the derivative gives the tangent vector to the
curve at a point $p$.  It is possible to associate the set of such
vectors with the set of directional derivatives, taking the partial
derivatives along the coordinates as the basis.  Henceforth this basis
will be denoted both $\partial_\mu$ and $\vec{e}_\mu$.  Note that the
tangent to a curve is defined at the point $p$.  Each point in the
manifold possesses its own space of tangent vectors.  These spaces are
distinct, which will be important in what follows.

We next define \emph{one-forms} as linear maps from vectors to
$\mathcal{R}$.  The set of one-forms at a point can be shown to form a
vector space, a natural basis for which can be obtained by requiring
%
\begin{equation*}
\vec{e}_\mu \tilde{\omega}^\nu = \delta_\mu^\nu
\end{equation*}
%
The components of an arbitrary form $\omega$ in this basis may be
found by applying the form to the basis vectors.

\iffalse
\begin{align*}
\omega(\vec{e}_\nu)
&= \tilde{\omega}^\mu \omega_\mu (\vec{e}_\nu) \\
&= \omega_\mu \tilde{\omega}^\mu (\vec{e}_\nu) \\
&= \omega_\mu \delta^\nu_\mu \\
&= \omega_\nu\\
\end{align*}
\fi

We can then build up arbitrary ${m \choose n}$ tensors as linear maps
from tensor products of $m$ vectors and $n$ one-forms to
$\mathcal{R}$.  The components of a tensor $T$ in a choice of
coordinates may found by applying it to combinations of the basis
vectors and basis 1-forms.  Finally, a ${m \choose n}$ tensor field is
a map that associates to each point $p$ in $\mathcal{M}$ an element in
the space of  ${m \choose n}$ tensors at $p$.

\subsection{The Metric Tensor}

A particularly important tensor in general relativity is the
\emph{metric}, a ${2 \choose 0}$ tensor that is symmetric ($g_{\mu\nu}
= g_{\nu\mu})$ and non-degenerate (the determinant of $g$ taken as a
matrix $|g_{\mu\nu}| \neq 0$.  The latter feature makes it possible to
define the inverse metric $g^{\mu\nu}$ as
%
\begin{equation*} g^{\mu_\rho} g_{\rho_\nu} = \delta^\mu_\nu
\end{equation*}
%
Given a vector $x^\mu$ the object $g_{\mu\nu} x^\mu$ maps another
vector to a real number, and is therefore a one-form.  The metric
therefore maps between the space of one-forms and the space of vectors
at each point.  Most importantly, the metric defines a notion of
distance on the manifold.  Infinitesimally
%
\begin{equation} ds^2 = g_{\mu\nu} dx^\mu dx^\nu \end{equation}
%
In special relativity, in Cartesian coordinates, the metric has
components $(-1,1,1,1)$ along the diagonal, all other components are
zero.   The metric will be denoted $\eta_{\mu\nu}$ and called the
\emph{flat space metric}.

\subsection{Covariant Derivatives}

Since vectors are only defined at a point we need additional structure
to define derivatives of vector fields, as there is no natural way to
compare vectors that live in different spaces.  We seek an operator
$\nabla$ with the following properties
%
\begin{itemize}
\item Maps ${m \choose n}$ tensors to ${m \choose {n+1}}$ tensors.
This is so we may consider the directional derivative of a tensor $T$
along a vector $x$ as $x^\mu \nabla_\mu T$.
\item Reduces to partial differentiation when applies to a scalar
field.
\item Linear.
\item Satisfies the Leibniz rule, $\nabla(a b) = a\nabla b + b \nabla a$.
\end{itemize}
%
Such an operator applied to a vector field gives
%
\begin{equation*}
\nabla_\mu (x^\nu \vec{e}_\nu)
= (\partial_\mu x^\nu) \vec{e}_\nu + x^\nu (\nabla_\mu
\vec{e}_\nu)
\end{equation*}
%
In flat space in Cartesian coordinates the basis vectors do not
change and so the last term is zero.  But in a curved space, or even
flat space in non-Cartesian coordinates, they may.  However, the new
vector must be expressible as a linear combination of the original
basis vectors.  The components are called \emph{connection
coefficients} and are denoted as $\Gamma^\rho_{\nu\mu}$ so
%
\begin{align}
\label{eq:covariant_derivative}
\nabla_\mu (x^\nu \vec{e}_\nu) &= 
(\partial_\mu x^\nu) \vec{e}_\nu + 
x^\nu \Gamma^\rho_{\mu\nu} \vec{e}_\rho \\
&= (\partial_\mu x^\nu + x^\rho \Gamma^\nu_{\mu\rho}) \vec{e}_\nu \\
\nabla_\mu x^\nu &= \partial_\mu x^\nu + x^\rho \Gamma^\nu_{\mu\rho}
\end{align}
%
In general relativity the connection is usually assumed to be
\emph{torsion-free}, that is
%
\begin{equation}
\label{eq:torsion}
 \Gamma^\nu_{\mu\rho} =  \Gamma^\nu_{\rho\mu}
\end{equation}
%
which thus far has been borne out by experiment.  However, it is
possible to construct theories where this condition does not hold.

We will henceforth occasionally use commas and semicolons to denote
partial and covariant differentiation, respectively:
%
\begin{align*}
{x^\mu}_{,\nu} &\equiv \partial_\nu x^{\mu} \\
{x^\mu}_{;\nu} &\equiv \nabla_\nu x^{\mu} \\
\end{align*}

By considering the covariant derivative of a scalar constructed from a
one-form acting on a vector, $\nabla (x^\nu \omega_\nu)$, it can be shown
that
%
\begin{equation*}
\nabla_\mu \omega_\nu = \partial_\mu \omega_\nu - 
\Gamma^\rho_{\mu\nu} \omega_\rho
\end{equation*}
%
The covariant derivative of a ${m \choose n}$ tensor generalizes this
and has a partial derivative term, $m$ positive therms in $\Gamma$ and
$n$ negative terms in $\Gamma$.

\subsection{Parallel Transport}
\label{ssec:parallel}

Covariant differentiation provides a way to ``move a vector without
changing it.''  We can \emph{parallel transport} a vector $v^\mu$
infinitesimally along a curve whose tangent vector is $u^\nu$ by
requiring
%
\begin{equation*}
u^\nu \nabla_\nu v^\mu = 0
\end{equation*}
%
As an example of such transport, consider an arrow on the equator of
the Earth pointing towards the north pole.  This arrow can be carried
halfway around the equator without rotating it, so it ends up on the
other side of the globe, still pointing north.  If the vector is then
parallel transported northward to the pole and then continued until it
returns to its starting point it will return pointing south.  Although
the vector was never rotated locally it has returned rotated.  This is
an indication that the underlying space is curved.

Of particular interest is the case where a vector is
parallel-transported along itself
%
\begin{equation*}
0 = v^\mu \nabla_\mu v^\nu 
= v^\mu (\partial_\mu v^\nu + \Gamma^\nu_{\mu\rho} v^\rho)
\end{equation*}
%
Now consider a curve $x(\lambda)$ such that $v$ is the tangent to this
curve, $v^\mu = d x^\mu/d\lambda$, then
%
\begin{align}
\label{eq:geodesic}
\frac{d x^\mu}{d\lambda}
  \frac{\partial }{\partial x^\mu}
  \frac{d x^\nu}{d\lambda}  
+ \Gamma^\nu_{\mu\rho} 
\frac{d x^\mu}{d\lambda}
\frac{d x^\rho}{d\lambda} &= 0 \nonumber \\
\frac{d^2 x^\nu}{d\lambda^2}
+ \Gamma^\nu_{\mu\rho} 
\frac{d x^\mu}{d\lambda}
\frac{d x^\rho}{d\lambda} &= 0 \nonumber \\
\end{align}
%
This is the \emph{geodesic equation}, solutions to which are
\emph{geodesics}.  The same equation can be derived by extremizing the
path length, $\sqrt{g_{\mu\nu} dx^\mu dx^\nu}$.  In general relativity
test masses acting under the influence of gravity and no other forces
follow geodesics.  

\subsection{The Christoffel Symbols}

If we now require that scalars do not change under parallel transport
we have, for arbitrary vectors fields $u^\alpha, v^\beta$ and $x^\mu$
%
\begin{align*}
0 &= x^\mu \nabla_\mu (g_{\alpha\beta} u^\alpha v^\beta) \\
&= x^\mu (\nabla_\mu g_{\alpha\beta}) u^\alpha v^\beta
+ g^{\alpha\beta} (x^\mu \nabla_\mu u^\alpha) v^\beta
+ g^{\alpha\beta} u^\alpha (x^\mu \nabla_\mu v^\beta)
\end{align*}
%
We can now specialize such that $u^\alpha, v^\beta$ are constant
and so the last two terms vanish, which implies the  \emph{metric
compatibility} condition:
%
\begin{equation}
\label{eq:metric_compatibility}
\nabla_\mu g_{\alpha\beta} = 0
\end{equation}

Equations~\ref{eq:metric_compatibility} and~\ref{eq:torsion} together
fix the connection coefficients in terms of the metric:
%
\begin{equation}
\Gamma^{\rho}i_{\mu\nu}
= \frac{1}{2} g^{\rho\sigma}\left[
\partial_\nu g_{\mu\sigma}
+ \partial_\mu g_{\nu\sigma}
- \partial_\sigma g_{\mu\nu}
\right]
\end{equation}
% 
Combining this with the previous section we see that the motion of
particles are completely specified once we know the metric and their
initial positions and velocities.

\subsection{The Riemann Tensor}

We now generalize the example given in Sec.~\ref{ssec:parallel}, and
ask how a vector $A^\mu$ changes as it is parallel-transported around
an infinitesimal parallelogram with sides defined by the vectors
$B^\mu$ and $C^\nu$.  Recalling that vectors and directional
derivatives are the same thing, it can be shown that this is
equivalent to asking how covariant derivatives fail to commute.  The
result must be linear in $A^\mu$ and so we may write
%
\begin{equation}
\label{eq:riemann_def}
\left[\nabla_\mu \nabla_\nu - \nabla_\nu \nabla_\mu\right] A^\rho
= R^\rho_{\sigma\mu\nu} A^\sigma
\end{equation}
%
which defines the \emph{Riemann tensor} $R^\rho_{\sigma\mu\nu}$.  A
number of properties follow from this definition (which are either
obvious or may be proven by substituting the definition of the
covariant derivative, eqn.~\ref{eq:covariant_derivative}).  First,
the symmetry properties
%
\begin{equation}
\label{eq:symmetries}
R_{\rho\sigma\mu\nu}
= -R_{\sigma\rho\mu\nu}
= -R_{\rho\sigma\nu\mu}
= R_{\mu\nu\rho\sigma}
\end{equation}
%
which in turn imply
%
\begin{align}
R^\rho_{[\sigma\mu\nu]} = 0
\end{align}
%
Second, the Bianchi identity,
%
\begin{equation}
\label{eq:bianchi}
R_{\rho\sigma\mu\nu;\alpha}
+R_{\rho\sigma\nu\alpha;\mu}
+R_{\rho\sigma\alpha\mu;\nu} = 0
\end{equation}

We may now generalize Eqn.~\ref{eq:riemann_def} and ask how an
arbitrary tensor changes after being parallel-transported 
around a loop.  It can be shown that
%
\begin{equation}
\label{eq:higher_order_riemann}
\left[\nabla_\mu \nabla_\nu - \nabla_\nu \nabla_\mu\right] 
B^{\rho_1 \rho_2 \ldots \rho_n}
= - R^{\rho_1}_{\sigma \mu\nu} B^{\sigma \rho_2 \ldots \rho_n}
- R^{\rho_2}_{\sigma \mu\nu} B^{\rho_1 \sigma \ldots \rho_n}
- \ldots -
- R^{\rho_n}_{\sigma \mu\nu} B^{\rho_1 \rho_2 \ldots \sigma }
\end{equation}
%
which may be proved by expanding
%
\begin{equation*}
\left[\nabla_\mu \nabla_\nu - \nabla_\nu \nabla_\mu\right] 
(\vec{e}_\rho \otimes \vec{e}_\sigma)
\end{equation*}
%
and then proceeding by induction.  

The symmetries of the Riemann tensor imply that there is, up to sign,
only one non-trivial contraction
%
\begin{equation}
R_{\mu\nu} = R^\sigma_{\mu\sigma\nu}
\end{equation}
%
which defines the \emph{Ricci tensor}.  This may be contracted again
%
\begin{equation}
R = R^\mu_\mu
\end{equation}
%
to obtain the \emph{Ricci scalar}.

Now, contracting the Bianchi identity twice gives
%
\begin{equation*}
g^{\sigma\nu}
\left(\nabla_\alpha R_{\sigma\nu}
+ \nabla^\rho R_{\rho\sigma\nu\alpha}
+ \nabla_\nu R^\mu_{\sigma\alpha\mu}\right) = 0
\end{equation*}
%
Using the symmetries of the Riemann tensor (eqn.~\ref{eq:symmetries})
this can be written
%
\begin{equation*}
\nabla_\alpha R
- \nabla^\rho R_{\rho\alpha}
- \nabla^\sigma R_{\sigma\alpha} = 0
\end{equation*}
%
Relabeling the dummy indices and using metric compatibility gives
%
\begin{equation*}
\nabla^\rho \left(g_{\rho\alpha} R - 2 R_{\rho\alpha} \right) = 0
\end{equation*}
%
This motivates the definition of the \emph{Einstein Tensor} as
%
\begin{equation}
\label{eq:einstein_tensor}
G_{\mu\nu} = R_{\mu\nu} - \frac{1}{2} g_{\mu\nu} R
\end{equation}
%
The previous result implies this is divergentless
%
\begin{equation*}
\nabla^\nu G_{\mu\nu} = 0
\end{equation*}
%
Note also that $G$ is symmetric, $G_{\mu\nu} = G_{\nu\mu}$.

We now relate this to physics by noting that the matter and energy
content of a region is described by the stress-energy tensor
$T_{\mu\nu}$ where each component is ``the flow of $\mu$ momentum in the
$\nu$ direction.''  For example, the $0,0$ component is energy density
and the $0,i$ components are the $i^\mathrm{th}$ components of
momentum.

Conservation of energy requires that the difference in momentum
($\p^i$) across each face of a cube be balanced by a change of energy
($\rho$), within the cube,
%
\begin{equation*}
\partial_t \rho = \partial_i p^i
\end{equation*}
%
In terms of the stress-energy tensor this becomes
%
\begin{equation*}
0 = - \nabla^0 T_{00} \nabla^i T_{0i} = 0
= \nabla^\nu T_{0 \nu}
\end{equation*}
%
However the time direction is not uniquely specified as a change of
coordinates will mix space and time components, so this must
generalize to 
%
\begin{equation*}
\nabla^\nu T_{\mu\nu} = 0
\end{equation*}
%
That is, $T$ is also divergentless, like $G$, and like $G$ it is also
symmetric.  It is therefore reasonable to suggest the ansatz
%
\begin{equation*}
G_{\mu\nu} \propto T_{\mu\nu}
\end{equation*}
%
Requiring agreement with Newton's law of gravity in the appropriate
low-energy limit ($T_{00} \gg$ all other components) fixes the
constant of proportionality and gives us \emph{Einstein's field
equation}
%
\begin{equation}
\label{eq:einsteins_equation}
G_{\mu\nu} = 8\pi T_{\mu\nu}
\end{equation}
%
Note that $G_{\mu\nu}$ is entirely determined by the metric.
Equation~\ref{eq:einsteins_equation} may therefore be thought of as a
set of coupled, non-linear differential equations for $g_{\mu\nu}$.

\section{Gravitational Radiation}
\label{sec:gravitational_radiation}

We now move to the prediction of gravitational waves.  We begin with
Einstein's equation in empty space,
%
\begin{equation*}
G_{\mu\nu} = R_{\mu\nu} - \frac{1}{2} g_{\mu\nu} R = 0
\end{equation*}
%
By taking the trace and substituting into Eqn.~\ref{eq:einsteins_equation}
it can be shown that this implies that in empty space $R_{\mu\nu} =
0$.  Similary, using the Bianchi identity and symmetries of the Riemann tensor gives,
in empty space,
%
\begin{equation}
\label{eq:divergence_in_empty_space}
R_{\beta\delta;\gamma}  -R_{\beta\gamma;\delta} = 0
\end{equation}

We next consider the application of the wave operator to the Riemann
tensor.  From the Bianchi identity (eqn.~\ref{eq:bianchi}) this becomes
%
\begin{equation*}
\label{eq:wave_expanded}
g^{\mu\nu} R_{\alpha\beta\gamma\delta;\mu\nu}
= - g^{\mu\nu}
\left[R_{\alpha\beta\delta\mu;\gamma\nu}
+ R_{\alpha\beta\mu\gamma;\delta\nu} \right]
\end{equation*}
%
Consider the first term on the right-hand side:
%
\begin{align*}
g^{\mu\nu} R_{\alpha\beta\delta\mu;\gamma\nu}
&= g^{\mu\nu} R_{\alpha\beta\delta\mu;\nu\gamma}
+ g^{\mu\nu} R_{\alpha\beta\delta\mu;\gamma\nu}
- g^{\mu\nu} R_{\alpha\beta\delta\mu;\nu\gamma} \\
&= g^{\mu\nu} R_{\alpha\beta\delta\mu;\nu\gamma}
+ g^{\mu\nu} 
\left[\nabla_\nu,\nabla_\gamma\right] R_{\alpha\beta\delta\mu}
\end{align*}
%
The first term vanishes by
Eqn.~\ref{eq:divergence_in_empty_space}.  The second term involves
products of the Riemman tensor by~\ref{eq:higher_order_riemann}.  The
second term on the right in Eqn.~\ref{eq:wave_expanded} has the
same form.

We now specialize to the case where the Riemann tensor is small, so
that terms involving multiple factors can be neglected.  This is
equivalent to considering the Riemann tensor as a field on a flat
background.  This gives
%
\begin{equation}
\label{eq:riemann_wave}
g^{\mu\nu}
R_{\alpha\beta\gamma\delta;\mu\nu}
=
\Box R_{\alpha\beta\gamma\delta}
= 0
\end{equation}
%
That is, each component of the Riemann tensor independently 
satisfies the vacuum wave equation.  We can immediately write the
solution:
%
\begin{equation}
R^\alpha_{\beta\gamma\delta} = 
\textrm{Re}\, A^\alpha_{\beta\gamma\delta} \exp(i k_\mu x^\mu)
\end{equation}
%
where $A^\alpha_{\beta\gamma\delta}$ is a set of amplitudes and
$k^\mu$ is the wave vector.  In a chosen coordinate system it has
components $(\omega, k_x, k_y, k_z)$ where $\omega$ is the angular
frequency and the spacial $k$ components are wavelengths in each
direction.  It can be shown that 
%
\begin{align*}
\nabla_{\vec k} \vec{k} &= 0 \\
k_\mu k^\mu &= 0 \\
\end{align*}
%
which together imply that gravitational waves travel along geodesics 
at the speed of light.


\section{Effect of Gravitational Waves}
\label{sec:effects_of_waves}

We now derive the effect of gravitational waves on matter.  Consider
two particles moving along world-lines $A^\mu$ and $B^\mu$.  Choose
coordinates so that $A$ remains fixed at the origin, $A^\mu =
(1,0,0,0)$.  We may further specialize our coordinates such that at
the origin $g_{\mu\nu} = \eta_{\mu\nu}$.  It can be shown that we may
also require that the first derivatives of the metric vanish at this
point.  We may not, however, make the second derivatives vanish in
general.  This corresponds to the fact that the Riemann tensor is
defined in terms of second derivatives.  We call the coordinate system
thus constructed a \emph{Local Lorentz Frame}.

We now define the separation between the two particles as 
%
\begin{equation*}
\xi^\mu = B^\mu - A^\mu
\end{equation*}
%
We fix $\xi$ to be perpendicular to $A$, so that $\xi^0 = 0$.  If space
is curved it can readily be seen that $\xi$ will not remain constant.
For example, if the particles are initially at rest some distance from
the surface of the Earth they will both move towards the center of the
Earth and $\xi$ will decrease.  It can be shown that $\xi$ obeys the
equation of \emph{geodesic deviation},
%
\begin{equation}
\label{eq:geodesic_deviation}
\frac{d^2}{dt^2} \xi^\rho = -R^\rho_{\mu\nu\sigma} A^\mu \xi^\nu A^\sigma
=-R^\rho_{0 \nu 0} \xi^\nu 
\end{equation}
%
Using the condition that $\xi$ has no time component reduces this to 
%
\begin{equation}
\frac{d^2}{dt^2} \xi^i = -R^\rho_{\mu\nu\sigma} A^\mu \xi^\nu A^\sigma
=-R^i_{0 j 0} \xi^j
\end{equation}
%
That is, the change in separation between two
infinitesimally-separated test masses at rest with respect to each
other in an arbitrary gravitational field is entirely specified by
$R^i_{0 j 0}$.  From the symmetries of the Reimann tensor this is
symmetric in $i$ and $j$, and hence appears to have 6 independent
components.  However, it can be shown that these can entirely be
specified by two values, which without loss of generality we take to
be $R^x_{0 x 0}$.  $R^x_{0 y 0}$.  We now recall that in empty space
$R_{\mu\nu} = 0$, which implies that $R^y_{0 y 0} = - R^x_{0 x 0}$.
We summarize this by saying that R is \emph{traceless}.  

We now specialize to the case of gravitational waves, so that the
Riemann tensor satisfies Eqn.~\ref{eq:riemann_wave} and we choose
coordinates such that the wave is traveling in the $z$ direction.
Using the fact that the speed of light is 1 in dimensionless units the
solution can then be written
%
\begin{equation*}
R_{i 0 j 0} = A_{ij}(t - z)
\end{equation*}
%
where we have lowered the first index to simplify notation.

Now, using the fact that $\partial_x R_{i0j0} = \partial_y R_{i0j0}
= 0$ and integrating the Bianchi identity we can show that
$R_{x 0 y 0} = R_{y 0 x 0}$ and that all other components vanish.
It can also be shown that in addition to being traceless $R$ is
\emph{transverse}, $k^j R_{i0j0} = 0$.  We denote these two facts by
adding the superscript $TT$, and define the gravitational-wave field as
%
\begin{equation}
\label{eq:wave_field}
-\frac{1}{2} \frac{\partial^2 h_{ij}^{TT}}{\partial t^2}
\equiv R^{TT}_{i0j0}
\end{equation}

We now decompose the separation vector $\xi$ into the initial
separation and a time-dependant perturbation, $\xi = \xi_0 + \delta
\xi$.  In terms of this Eqn.~\ref{eq:geodesic_deviation} becomes
%
\begin{equation}
\label{eq:geodesic_deviation_delta}
\frac{d^2}{dt^2} \delta \xi^i = -R^{0 i 0 j} \xi_0^j
\end{equation}
%
where we drop the initial portion from the left-hand side because it
is constant, and we drop the perturbation from the right hand side
because it is much smaller than the initial portion.  Comparing
eqn.~\ref{eq:wave_field} and eqn.~\ref{eq:geodesic_deviation_delta}
we obtain the equation for the effect of a gravitational wave on
free-falling test masses:
%
\begin{equation}
\label{eq:wave_effect}
\delta \xi^i = \frac{1}{2} h^{TT}_{ij} \xi^j
\end{equation}
%
We note in passing that this is the same result obtained in other
treatments by expanding the metric in terms of the flat-space metric
plus a perturbation, $g_{\mu\nu} = \eta_{\mu\nu} + h_{\mu\nu}$,
substituting into the Einstein equation and expanding to first order
in $h$, and then choosing a gauge in which $h$ is transverse and
traceless.

Now, define
%
\begin{align*}
h_+ &\equiv h_{xx} = - h_{yy} \\
h_\times &\equiv h_{xy} = h_{yx}
\end{align*}
%
which we refer to as the \emph{plus} ($+$) and \emph{cross} ($\times$)
polarizations, respectively.  Consider the case where $h_\times = 0$.
If particle $B$ is initially on the $x$ axis then 
%
\begin{align*}
\delta \xi^x &= \frac{1}{2} h^{TT}_{xx} \xi^x \\
\delta \xi^y &= \frac{1}{2} h^{TT}_{xy} \xi^y \\
&= 0
\end{align*}
%
The particle remains on the $x$ axis.  For an oscillatory wave the
distance between the two particles likewise oscillates.  We can
describe this as an induced \emph{strain}, $\Delta L/L$ where $L$ is
the initial separation.  If $B$ is initially on the $y$ axis
%
\begin{align*}
\delta \xi^x &= \frac{1}{2} h^{TT}_{xx} \xi^x \\
&= 0 \\
\delta \xi^y &= \frac{1}{2} h^{TT}_{yy} \xi^y \\
&= - \frac{1}{2} h^{TT}_{xx} \xi^y
\end{align*}
%
The particle remains on the $y$ axis and oscillates out of phase with
a corresponding particle on the $x$ axis.  The net effect is that,
after a quarter cycle, a set of masses initially in a circle are moved
into an ellipse flattened along one axis and stretched along the other
such that the area remains constant.  After another quarter cycle they
return to a circle, in the next quarter cycle they are in an ellipse
with the axes flipped, and so on. It is similarly straightforward to
show that for a wave cross-polarized wave the eigendirections are on
the lines $x=\pm y$.  The effects are the same as for the $+$
polarization, rotated 45 degrees.

Now consider a thought experiment, originally due to Feynman, where we
place a bead on a stick in the path of a gravitational wave.  The wave
will cause the bead to slide back and forth, heating the stick trough
friction and imparting energy to the system.  This implies that
gravitational waves carry energy.  This argument can be made
precise~\cite{RevModPhys.29.509}.

\section{Modeling Gravitational Waves}

Having demonstrated the predicted existence of gravitational waves and
their effects on matter we now turn to the question of their
generation.  From Eqn.~\ref{eq:wave_field} we can see that, since
the components of the Riemann tensor obey the wave equation, the
components of $h^{TT}$ do as well.  We now consider the solution of
the wave equation when the source, the stress-energy tensor, is not
zero.  By analogy with electromagnetism we can immediately write down
the solution in terms of retarded fields
%
\begin{equation}
\label{eq:h_from_t}
h^{TT}_{ij} = 4 \left[ \int \frac{T_{ij}(x', t-r)}{r}\, d^3 x'
\right]^{TT}
\end{equation}
%
where we integrate over the source distribution $x'$.  When the energy
and momentum densities are small, so that the curvature of spacetime
is likewise small, the coordinates may be taken to have their
conventional flat-space meaning.  We may also replace the covariant
derivative by regular partial differentiation.

Starting from the conservation of energy and momentum written in terms
of the stress-energy tensor
%
\begin{align*}
{T^{00}}_{,0} &= - {T^{0j}}_{,j} \\
{T^{i0}}_{,0} &= - {T^{ij}}_{,j} \\
\end{align*}
%
Differentiating with respect to time and going through some algebra
yields expressions for the first and second moments of the stress
energy tensor, ${T^{lm}}_{,ml} x^j x^k$ and ${T^{jm}}_{,m} x^k$.  These,
plus Stoke's theorem, can be used to simplify
Eqn.~\ref{eq:h_from_t} to give
%
\begin{equation*}
h^{TT}_{ij} = \frac{2}{r} \left[
\int {T^{00}}_{,00}\, x^i x^j d^3 x' \right]^{TT}
\end{equation*}
%
We recognize $T^{00}$ as the mass/energy density, and therefore this
can be written as the \emph{quadrapole formula}
%
\begin{equation}
\label{eq:quadrupole_formula}
h^{TT}_{jk} = \frac{2}{r} \ddot{\mathcal{I}}(t-r)
\end{equation}
%
where $\mathcal{I}$ is the quadrupole moment of the source
%
\begin{equation*}
\mathcal{I}_{ij} = \int \rho(\mathbf{x})x_i x_j\,d^3 x
\end{equation*}

Any system of mass with a time-dependant quadrupole moment will give
rise to gravitational radiation.  However, restoring physical units to
Eqn.~\ref{eq:quadrupole_formula} scales the right-hand side by
$G/c^4$.  We therefore need very large masses and/or rapid changes in
order to generate waves we have any chance of detecting.  There are
many such sources of astrophysical and cosmological interest:
%
\begin{itemize}
\item supernovae
\item rotating neutron stars with axial asymmetry
\item processes in the early universe, which would have produced
\emph{relic} gravitational waves in principle still detectable today
\item topological defects
\item compact bodies, such as neutron stars and black holes, in 
orbit around each other.
\end{itemize}
%
We will focus on the last of these for the remainder of the thesis.

It is straightforward to start from the mass distribution of two point
masses in orbit around their center:
%
\begin{align*}
\rho(\mathbf{x}) &= m_1\delta(x - r_1\cos(\Omega t)) \delta(y-r_1
\sin(\Omega t)) \delta(z) \\
&\quad + m_2\delta(x + r_2\cos(\Omega t)) \delta(y + r_2 \sin(\Omega t))
\delta(z)
\end{align*}
%
and calculate $h^{TT}$. By choosing coordinates centered on a
terrestrial gravitational-wave detector and a basis we can write
the gravitational-wave strains as~\cite{DBrownThesis}
%
\begin{align}
\label{eq:h_plus_cross}
h_+(t)   &= - \frac{2G}{c^4 r} \mu (\pi G M f)^{\frac{2}{3}}
(1+\cos^2(\iota)) \cos(2\pi f t - 2\phi_0) \\ \nonumber
h_\times(t)  &= - \frac{4G}{c^4 r} \mu (\pi G M f)^{\frac{2}{3}}
\cos(\iota) \sin(2\pi f t - 2\phi_0) \nonumber
\end{align}
%
where $M = m_1+m2$, $\mu = m_1 m_2 / M$, $f$ is the
\emph{gravitational-wave frequency} which is twice the orbital
frequency $f = 2\Omega/2\pi$, $\phi_0$ is the orbital phase at a
specified time, and $\iota$ is the \emph{inclination} of the binary
with respect to the detector, the angle between the normal to the
plane of the binary and the line joining the detector to the binary.

We noted above that gravitational waves carry energy.  This energy
must be balanced by a loss of energy by the system.  This energy can
not be localized to any one point of the wave, since any point can be
placed in a Local Lorentz Frame where there is no wave.  However, by
averaging over a cycle it can be shown~\cite{carrollTextbook} that
%
\begin{equation}
t_{\mu\nu} = \frac{1}{32 \pi} \left\langle h^{TT}_{\rho \sigma,\mu}
(h^{TT})^{\rho\sigma}_{,\nu} \right\rangle
\end{equation}
%
where $t_{\mu\nu}$ is the \emph{stress-energy pseudo-tensor}.  The
stress energy tensor itself is zero in a region of spacetime
containing only gravitational radiation.  However, $t$ may be used to
describe the energy content of such radiation, and we find the flux of
energy in the radial direction out of a sphere enclosing a gravitating
system is
%
\begin{equation}
\label{eq:energy_loss}
\frac{dE}{dt} = T_{0r} = \frac{1}{8 \pi r^2} \left\langle \ddot{\mathcal{I}}^{ij}
\ddot{\mathcal{I}}_{ij} \right\rangle
\end{equation}

When the separation between the masses, $a$ is large and the masses are
moving slowly the gravitational energy of the system is approximately
given by the Newtonian expression
%
\begin{equation*}
E = - \frac{1}{2} \frac{G \mu M}{a}
\end{equation*}
%
Therefore, orbiting bodies will emit gravitational radiation and
\emph{inspiral} until they inevitably collide.

\section{Post-{N}ewtonian Approximations}
\label{sec:PNWaveforms}

Substituting Eqn.~\ref{eq:h_plus_cross} into
Eqn.~\ref{eq:energy_loss} shows that the power emitted goes as
$a^{-5}$.  Most of the power from an inspiral, and therefore our best
chance of detecting such systems, occurs when the masses have become
close and are about to merge.  The approximations we made in the
previous section are no longer valid in this regime.  In order to
obtain analytic models of gravitational waves from inspirals to a
precision that will be useful in searches we must therefore consider
higher-order corrections beyond Newtonian mechanics of the binary.
This leads to \emph{post-Newtonian} (pN) waveforms.

The key to doing this is the requirement that the energy flux carried
away by gravitational radiation must be balanced by loss of energy of the
system
%
\begin{equation}
\label{eq:equivilent_exchange}
\frac{dE}{dt} = - \mathcal{F}
\end{equation}
%
It can be shown~\cite{MTW} that the energy of a body of mass $\mu$ moving
on a geodesic in the Schwarzschild metric with mass $M$ is
%
\begin{equation}
\label{eq:hamiltonian}
E = \mu \frac{1-2M}{\sqrt{1-3M/r}} 
\end{equation}
%
We now introduce the parameter $v = (\pi M f)$, which is the velocity
in Newtonian mechanics.  We now consider the \emph{adiabatic
approximation}, where we treat the system as moving through a series
of circular orbits.  On a circular geodesic $v= \sqrt{M/r}$ and the
energy becomes
%
\begin{equation}
E = \mu \frac{1-2v^2}{\sqrt{1-3v^2}} 
\end{equation}

We have written down the flux to first pN order above in
Eqn.~\ref{eq:energy_loss}, in terms of the parameter $v$ it is
%
\begin{equation}
\mathcal{F} = \frac{32}{5} \left( \frac{\mu}{M} \right) v^{10}
\end{equation}
%
Going to higher order requires techniques that are beyond the scope of
this thesis.

Both the energy and flux may now be expanded in powers of $v$.  We
could then obtain $t$ as a function of $v$ by rewriting
Eqn.~\ref{eq:equivilent_exchange} and integrating.  However, we
will instead obtain an expression relating time and the orbital phase
$\phi = \int 2\pi f\, dt$ by writing
%
\begin{equation*}
\mathcal{F} = - \frac{dE}{dv} \frac{dv}{d\phi} \frac{d\phi}{dt}
\end{equation*}
%
which leads to the expression
%
\begin{equation}
\label{eq:expansion_for_phi}
\phi = \phi_0 + \frac{2}{M} \int_v^{v_\mathrm{ref}} \frac{v^3
dE/dv}{\mathcal{F}}\, dv
\end{equation}
%
where $v_\mathrm{ref}$ is the velocity at a given reference velocity,
or equivalently reference frequency.  If $\mathcal{F}$ is given as a
Taylor series in $v$ then $\mathcal{F}^{-1}$ may be expanded to the
same order.  $dE/dv$ may be similarly expanded, and the integrand
becomes the product of polynomials with rational coefficients.

Motivated by Eqn.~\ref{eq:h_plus_cross} we model the waveform as a
time-dependant amplitude and evolving phase:
%
\begin{equation}
\label{eq:general_waveform}
h(t) = A(t) \cos(\phi(t)) = \frac{1}{2} A(t) \left(e^{i\phi(t)} +
e^{-i\phi(t)} \right)
\end{equation}
%
For reasons that will become clear in the next chapter it is often more
convenient to work in the frequency domain, so we take the Fourier
transform of Eqn.~\ref{eq:general_waveform}
%
\begin{align}
\tilde{h}(f) &= \frac{1}{2} \int A(t) \exp(2\pi i f t + i
\phi(t))\, dt \\ \nonumber
&\qquad + \frac{1}{2} \int A(t) \exp(2\pi i f t - i
\phi(t))\, dt \\ \nonumber
\end{align}

We next apply the \emph{stationary-phase approximation}.  Oscillatory
integrands cancel except where the phase is stationary, at extrema of
the exponent.  We therefore expand the exponent around the time $t_0$
of the extremum and discard the linear term
%
\begin{equation}
2 \pi i f t + i \phi(t) \approx (2 \pi i f t_0 + i \phi(t_0))
+ \frac{1}{2} i \ddot{\phi}(t_0) (t-t_0)^2
\end{equation}

Substituting back, the first term leads to a constant factor and the
second leads to a Gaussian integral.  Going through the calculation
yields the approximate waveform
%
\begin{equation}
\label{eq:spa_waveform}
\tilde{f}(f) = \frac{2 G \msun}{c^2 r}
\left(\frac{5 \mu}{96 \msun} \right)^{\frac{1}{2}}
\left( \frac{M}{\pi^2 \msun} \right)^{\frac{1}{3}}
\left( \frac{G \msun}{c^3}\right)^{-\frac{1}{6}}
f^{-\frac{7}{6}} \Theta(f - f_c) e^{i \Psi(f;M,mu)}
\end{equation}
%
where $\msun$ is the mass of the sun and is introduced to so that $M$
is measured in solar masses, a useful unit in astrophysical work.
$\Psi$ is the result of the expansion~\ref{eq:expansion_for_phi}
written as a function of $f$.  The coefficients coming from the
expansion of the flux will depend on the mass $M$ and ration $\mu$ or
equivalently mass and \emph{symmetric mass ratio} $\eta = \mu/M$.  In
general waveforms may also depend on other parameters such as the
spins of the objects.  We introduce the step function to terminate the
waveform at a \emph{cutoff frequency} $f_c$.  This reflects the fact
that we do not trust this approximation all the way through the
evolution of the binary to merger.  The appropriate value for the
cutoff frequency is one of the questions we will address in
chapter~\ref{ch:comparison}.

There are many different ways to perform post-Newtonian expansions, in
both the time and frequency domains.  These lead to different
waveforms which have been assigned standard names, a summary of the
variations we will encounter throughout this thesis is given in
Appendix A.  We note here in particular that the stationary-phase
approach discussed above leads to waveforms denoted \emph{TaylorF2}.
The degree to which these different waveforms agree as been studied in
Ref.~\cite{BuonannoIyerOchsnerYiSathya2009}. 


\subsection{Effective One-Body}
\label{ssec:EOB}

A relatively recent development in post-Newtonian theory is the
\emph{effective one-body} (EOB)
formulation~\cite{BuonannoDamour:1999}.  The basic idea is familiar
from non-relativistic classical mechanics, where it is common to write
the Hamiltonian for the Kepler problem in terms of a particle with
reduced mass $\mu = m_1m_2 /(m_1 + m_2)$ moving in a central potential
due to a fixed body of mass $M = m_1 + m_2$.  

In general relativity we have noted that the energy, or equivalently
the Hamiltonian, takes the form of Eqn.~\ref{eq:hamiltonian} for
circular geodesics in a spherically symmetric spacetime.  The EOB
approach proceeds by seeking a map from the general two-body problem
to an equivalent spherically symmetric metric
%
\begin{equation}
ds^2 = -A(R) c^2 dT^2 + \frac{D(R)}{A(R)} dR^2 + R^2(d\theta^2 + sin^2 \theta
d\varphi^2)
\end{equation}
%
where the functions are written as expansions in $(GM/c^2R)$.  The
Hamiltonian of geodesic motion in this metric captures the
conservative portion of the motion (as can be seen by the fact that
the metric has no $T$ dependence).  Radiation-reaction terms can then
be added to this Hamiltonian to capture energy carried away by
gravitational waves.  This process results in coupled differential
equations which can be evolved numerically and from which the
gravitational waveforms can be obtained.

A key feature of the EOB approach is the use of \emph{Pad\'{e}
resummation}.  Given a function $f(x)$ with Taylor series $\sum a_i
x^i$ one ``resums'' the series into the ratio of polynomials $\sum b_i
x^i/\sum c^i x_i$ by expanding this ratio in a Taylor series and
matching the coefficients to the original series, then solving for the
$b_i$ and $c_i$.  The resulting function can converge to $f$ faster
than the Taylor series.

The expansion of the unknown functions in the metric to useful order
leads to unknown coefficients that can not be determined from the
problem alone.  These must be found by fitting the results to a
waveform obtained through other means.  This is one application of
waveforms from Numerical Relativity, discussed in the next section.
The resulting waveforms are called
\emph{EOBNR}~\cite{Buonanno:2009qa}.


\section{Numerical Relativity}
\label{sec:NRWaveforms}

Analytic methods themselves can not provide complete gravitational
waveforms valid through the merger of the two objects and the
evolution of the resulting single object to a steady state.  Even
where there is hope of being able to capture the full physics, as in
EOB or phenomenological models that have been developed, the
results must be tuned against complete waveforms which must be
calculated using other means.

There is a long history of solving partial differential equations
numerically on computers, and while Einstein's equations are in a
particularly difficult class of such equations, we can hope to obtain
complete solutions numerically.  This proved to be a remarkably
difficult problem, and progress through the 1990s was slow due to both
conceptual difficulties and limited computational power.  However,
following Pretorius' successful simulation of two merging black
holes~\cite{Pretorius:2005gq} and breakthroughs by researchers at
Goddard~\cite{Campanelli:2005dd} and the University of Texas at
Brownville and Florida Atlantic University~\cite{Campanelli:2005dd} in
2005 the field has expanded rapidly.  Readers are referred to the
textbook by Alcubierre~\cite{alcubierreTextbook} and review articles
by Husa~\cite{Husa:2007zz} and Hindler~\cite{Hinder:2010vn} for more
details than can be provided here.

The basis of most approaches in numeric relativity (hence ``NR'') is the
``3+1'' decomposition of Einstein's equation.  There is a similar
situation in electromagnetism.  Maxwell's equations may be expressed
in a geometric, four-dimensional form
%
\begin{align*}
F^{\mu\nu}_{,\nu} &= j^\mu \\
\mathcal{F}^{\mu\nu}_{,\nu} &= 0
\end{align*}
%
where $F$ is the electromagnetic field tensor and $\mathcal{F}$ is its dual.
However, this is not the most usual form in which to do calculations.
Instead we choose a time direction and rewrite in terms of separate
spatial and temporal differential operators,
%
\begin{align}
\label{eq:constraints1}
\nabla \cdot \mathbf{E} &= 4\pi \rho \\
\label{eq:constraints2}
\nabla \cdot \mathbf{B} &= 0 \\
\label{eq:evolution1}
\partial_t \mathbf{E}   &= \nabla \times \mathbf{B} - 4\pi j^\mu \\
\label{eq:evolution2}
\partial_t \mathbf{B}   &= - \nabla \times \mathbf{E} \\ \nonumber
\end{align}
%
Note that equations~\ref{eq:constraints1} and \ref{eq:constraints2}
are \emph{constraint equations} that impose conditions on valid
solutions at any given time, and in particular for initial conditions.
Equations~\ref{eq:evolution1} and \ref{eq:evolution2} are
\emph{evolution equations} which determine how to evolve the system.

Wile much work in numerical relativity includes matter, for the
purpose of this thesis  we restrict our attention to \emph{vacuum
spacetimes}, that is, situations where $T_{\mu\nu} = 0$ everywhere.
It is a remarkable fact that, although black holes are characterized
by a mass parameter, a spacetime containing only black holes and
gravitational waves is a vacuum.  We need therefore only be concerned
with general relativity, without considering the equations governing
any other fields.

In general relativity the split into space+time  is accomplished by
writing the metric as 
%
\begin{equation*}
ds^2 = -(\alpha^{2}-\gamma_{ij}\beta^{i}\beta^{j})dt^{2}
   + 2 \gamma_{ij}\beta^{j}dt\,dx^{i}
   + \gamma_{ij}dx^{i}\,dx^{j}, 
\end{equation*}
%
where $\gamma_{ij}$ is a metric on the slices of constant time $t$,
and the scalar function $\alpha$ and  vector field $\beta^i$ are
commonly used to encode the freedom of coordinate choice.  This
results in a division of the Einstein Eqn.~\ref{eq:einsteins_equation}
into constraint and evolution equations.  However, this split is not
uniquely determined and care must be taken to ensure that the
resulting systems are well-behaved.  In particular, there may be
issues ensuring that the constraints remain satisfied as the system
evolves.  While this is guaranteed mathematically in the continuum
limit it may not be true once the system is discretized.

Once a scheme for performing the 3+1 split has been decided upon it is
then necessary to find initial data that captures the situation of
interest and satisfies the constrain equations.  There are many ways
to do this.  Most approaches assume the spatial metric on the initial
slice is \emph{conformally flat}, meaning each point has a
neighborhood which can be mapped to flat space.  We have seen that at
any one point the metric can be made to be the flat-space metric with
vanishing first derivatives, but this does not extend into a
neighborhood in general.  In particular, NR systems do not include the
history of gravitational waves that would have been produced by the
system prior to the start of the simulation.  This deviation from the
true physics manifests as a burst of spurious ``junk radiation'' at
the start of the system.

The creation of initial conditions is complicated for black holes by
the presence of singularities, which can not be captured in a
simulation.  Most codes adopt the ``moving puncture'' approach.  The
idea is to consider black holes as ``wormholes'' with an internal
asymptotically flat region, and then compactify this region.  A
conceptually simpler approach is to simply excise the points near the
singularity from the computational grid, but care must be taken to
ensure that the excised region is causally disconnected from the
computational domain in order to avoid unphysical results.

There are a number of techniques for performing the evolution.  A
common approach is to evaluate the metric on a grid and replace
derivatives with differences.  An alternative is to use \emph{spectral
methods}, which involve expanding the solution in terms of a set of
basis functions, and then evolving the coefficients.  In this scheme
differentiation can be performed analytically.  In either scheme there
is a tradeoff between accuracy and performance.  A finer grid will
give more accurate results, but at the cost of higher computational
cost.  As simulations typically take weeks to months even on large
computer clusters, improving the performance is desirable.  Most codes
therefore use some form of \emph{adaptive mesh refinement}, employing
a finer grid only in regions close to the holes where the metric is
changing rapidly.

In order to extract gravitational-wave information most approaches use
the \emph{Newman-Penrose scalar}, which in vacuum may be written
%
\begin{equation*}
\Psi_4 = R_{\alpha \beta \gamma \delta} n^\alpha \bar{m}^\beta n^\gamma
\bar{m}^\delta
\end{equation*}
%
where $m$ and $n$ are vectors constructed from the basis vectors,
which in spherical coordinates are
%
\begin{align*}
n &= \frac{1}{\sqrt{2}} \left(\hat{t} - \hat{r}\right) \\
m &= \frac{1}{\sqrt{2}} \left(\hat{\theta} +i  \hat{\phi}\right)
\end{align*}
%
and $\bar{m}$ is the complex conjugate of $m$. The gravitational-wave
strain is related to $\Psi_4$ by two time derivatives,
%
\begin{equation*}
\Psi_4 = \ddot{h}_+ - i \ddot{h}_\times
\end{equation*}


\subsection{Hybrid Waveforms}
\label{sec:HybridWaveforms}

As noted, NR simulations require a great deal of time and resources;
typically a single simulation starting 10 orbits before the merger
will require a few weeks of runtime on approximately 50-100
processors~\cite{Scheel:2008rj}.  These requirements scale with the
length of the waveform extracted.  Long waveforms are therefore
prohibitively expensive; the longest currently available span about 30
cycles before merger.  Systems of astrophysical interest less massive
than about $36 \msun$ will spend many more cycles in the frequency
range to which Initial LIGO is most sensitive.  We therefore desire
much longer waveforms.  Post-Newtonian waveforms can not be extended
upwards into the late inspiral and merger, and NR waveforms can not be
extended downwards to the early inspiral.  However, we can consider
``stitching'' a pN waveform to an NR in order to create a \emph{hybrid
waveforms}.

This turns out to be possible, although as we will see in
chapter~\ref{ch:ninja2} there are subtleties in doing so that have not
yet been fully resolved.  One approach described 
in Ref.~\cite{Boyle2008a} is to work with $\Psi_4$ and match the numerical
waveform to the pN waveform by adjusting the time and phase offsets of
the pN waveform to minimize the quantity
%
\begin{equation}
  \label{eq:MatchingChiSquared}
  \Xi(\Delta t, \Delta \phi) = \int_{t_{1}}^{t_{2}}\, \left[
    \phi_{\mathrm{NR}}(t) - \phi_{\mathrm{pN}}(t - \Delta t) - \Delta \phi
  \right]^{2}\, d t \ .
\end{equation}
%
This technique has been shown~\cite{Boyle2007} to give good results
when the pN waveform is taken to be the time-domain \textit{TaylorT4}
waveform (see Appendix A) with terms up to 3.5-pN in phase and 3.0-pN
in amplitude.


\iffalse
% Taken from Boyle et. al, left here as a reminder of
% material to cover.

\section{NR}
\label{sec:NR}

Optimal searches for gravitational waves use matched filtering, which
requires accurate knowledge of the waveform~\cite{thorne.k:1987}.
Previous searches in LIGO data have used post-Newtonian and
phenomenological templates to search for the coalescence of black-hole
binaries~\cite{Abbott:2005pf,Abbott:2007xi,Abbott:2008}. Over the last
several years numerical relativity has made remarkable breakthroughs
in simulating the late inspiral, merger and ringdown of black-hole
binaries. The computational cost of these simulations is high,
however, making it impractical to use them directly as template
waveforms for use in a matched-filter search. It has been shown that
there is good agreement between the waveforms generated by numerical
relativity with analytic post-Newtonian waveforms to within just a few
orbits of merger~\cite{Buonanno-Cook-Pretorius:2007, Baker2006d,
  Pan2007, Buonanno2007, Hannam2007, Boyle2007, Gopakumar:2007vh,
  Hannam2007c, Boyle2008a, Mroue2008, Hinder2008b}.


\subsection{Hybrid Waveform}
\label{sec:HybridWaveform} %
Numerical simulations cannot simulate a very large portion of the
inspiral of a black-hole binary system.  Indeed, the longest such
simulation currently in the literature is the one used here---which
extends over just 32 gravitational-wave cycles before merger.
Fortunately, this is the only stage in which simulations are needed.
It has been shown previously~\cite{Boyle2007} that the
\textit{TaylorT4} waveform with 3.5-pN phase and 3.0-pN amplitude
matches the early part of this simulation to very high accuracy.  We
generate a \textit{TaylorT4} waveform of over 8000 gravitational-wave
cycles ($t \sim 1.2\times 10^{6}M$, starting at $M f=0.004$), and
transition between the two to create a hybrid.  This long waveform is
sufficient to ensure that---even for the lowest-mass systems we will
consider---the waveform begins well before it enters the frequency
band of interest to LIGO.

We begin with $\Psi_4$ data, which will later be integrated to obtain
$h$.  Following Ref.~\cite{Boyle2008a}, we match the numerical
waveform to the pN waveform by adjusting the time and phase offsets of
the pN waveform to minimize the quantity
\begin{equation}
  \label{eq:MatchingChiSquared}
  \Xi(\Delta t, \Delta \phi) = \int_{t_{1}}^{t_{2}}\, \left[
    \phi_{\NR}(t) - \phi_{\pN}(t - \Delta t) - \Delta \phi
  \right]^{2}\, d t \ .
\end{equation}
Here, we choose $t_{1}=900\,M$ and $t_{2}=1730\,M$, which is closer to
the beginning of the waveform than in the previous paper.  This
particular interval is chosen to begin and end at troughs of the small
oscillations due to the residual eccentricity $e\sim 5\times 10^{-5}$
in our numerical waveform.  Taking a range from trough to trough or
peak to peak---rather than node to node, for example---of the
eccentricity effects minimizes their influence on the matching.  The
eccentricity oscillations can be seen more easily after low-pass
filtering the waveform, though we find filtering to be unnecessary for
this paper.  The junk radiation apparent in the waveform as shown here
has no effect on the resulting match---as we have verified by
filtering, and redoing the match.  Because the final waveform will
incorporate no numerical data before $t_{1}$ and very little
immediately thereafter (as explained below), the junk radiation will
have no effect on any of our results---as we have also explicitly
verified.  In particular, by integrating $\Psi_{4}$ to obtain $h$, we
will effectively smooth the junk radiation.

%%%%%%%%%%%%%%%%%%%%%%%%%%%%%%%%%%%%%%%%%%%%%%%%%%%%%%%%%%%%%%%%%%%%%%
\begin{figure}
  \begin{center}
    \includegraphics[width=0.55\linewidth]{figures/comparison/PlotDifferences}
  \end{center}
  \caption{Amplitude and phase differences between the numerical and
    post-Newtonian waveforms, $\Psi_4$, that are blended to create the
    hybrid waveform.  The vertical lines at $900M$ and $1730M$ denote
    the region over which matching and hybridization occur.  Note that
    the agreement is well within the numerical accuracy of the
    simulation, represented by the horizontal bands, throughout the
    matching region.  Also note that the phase difference is fairly
    flat for a significant period of time after the matching range,
    which indicates that the match is not sensitive to the particular
    range chosen for matching.}
  \label{fig:MatchingPhaseComparison}
\end{figure}%
%%%%%%%%%%%%%%%%%%%%%%%%%%%%%%%%%%%%%%%%%%%%%%%%%%%%%%%%%%%%%%%%%%%%%%
In Fig.~\ref{fig:MatchingPhaseComparison} we compare the phase of the
numerical and pN waveforms.  The quantities plotted are
\begin{eqnarray}
  \delta \phi & \equiv & \phi_{\pN} - \phi_{\NR}\ , \\
  \frac{\delta A}{A} & \equiv & \frac{A_{\pN} - A_{\NR}}{A_{\NR}}\ , 
\end{eqnarray}
shown over the interval on which both data sets exist.  The vertical
bars denote the matching region.  Note that the phase difference is
well within the accuracy of the simulation (about 0.01 radians,
represented by the horizontal band) over a range extending later than
the matching region.  Also, the difference between the two is fairly
flat, which implies that the match is not very sensitive to the region
chosen for matching.  Because of this, we expect that the phase
coherence between the early pN data and the late NR data will be
physically accurate to high precision.

The hybrid waveform is then constructed by blending the two matched
waveforms together according to
\begin{eqnarray}
  \label{eq:HybridWaveform}
  A_{\hyb}(t) &= & \tau(t)\, A_{\NR} + \left[ 1 - \tau(t) \right]\,
  A_{\pN}(t)\ , \\
  \phi_{\hyb}(t) &= & \tau(t)\, \phi_{\NR} + \left[ 1 - \tau(t)
  \right]\, \phi_{\pN}(t)\ .
\end{eqnarray}
The blending function $\tau$ is defined by

\begin{equation}
  \label{eq:BlendingFunction}
  \tau(t) = \left\{\begin{array}{ll}
      0 & \mathrm{if}\quad t<t_{1}  \\
      \frac{t-t_{1}}{t_{2}-t_{1}} & \mathrm{if}\quad t_{1} \leq t < t_{2} \\
      1 & \mathrm{if}\quad t_{2} \leq t
    \end{array} \right.
\end{equation}

The values of $t_{1}$ and $t_{2}$ are the same as those used for the
matching.  The amplitude discrepancy between the pN waveform and the
NR waveform over this interval is within numerical
uncertainty---roughly $0.4\%$.  As with the matching technique
(Eq.~(\ref{eq:MatchingChiSquared})), this method is similar to that of
Ref.~\cite{Ajith-Babak-Chen-etal:2007b}, but distinct, in that we
blend the phase and amplitude, rather than the real and imaginary
parts.  This leads to a smoothly blended waveform, shown in
Fig~\ref{fig:WaveformSnapshot}.
%%%%%%%%%%%%%%%%%%%%%%%%%%%%%%%%%%%%%%%%%%%%%%%%%%%%%%%%%%%%%%%%%%%%%%
\begin{figure}
  \begin{center}
    \includegraphics[width=0.55\linewidth]{figures/comparison/Waveform}
  \end{center}
  \caption{The last $t=5000\MSun$ of the hybrid waveform used in this
    analysis: the $h_{+}$ waveform seen by an observer on the positive
    $z$ axis.  The vertical lines denote the matching and
    hybridization region.  The $0$ on the time axis corresponds to the
    beginning of data from the numerical simulation.}
  \label{fig:WaveformSnapshot}
\end{figure}%
%%%%%%%%%%%%%%%%%%%%%%%%%%%%%%%%%%%%%%%%%%%%%%%%%%%%%%%%%%%%%%%%%%%%%%

Up to this point, the waveform has been $\Psi_{4}$ data.  With the
long waveform in hand, we numerically integrate twice to obtain $h$,
and set the four integration constants so that the final waveform has
zero average and first moment~\cite{Pfeiffer-Brown-etal:2007}.
Because of the very long duration of the waveforms, this gives a
reasonable result, which is only incorrect at very low
frequencies---lower than any frequency of interest to us.  We have
also checked that our results do not change when we effectively
integrate in the frequency domain by taking
\begin{equation}
  \label{eq:PsiFourIntegration}
  \tilde{h} = -\frac{\tilde{\Psi}_{4}}{4\,\pi\, f^{2}}\ ,
\end{equation}
which is the frequency-domain analog of the equation $\Psi_{4} =
\ddot{h}$.
\fi


\section{Conclusions}

In this chapter we reviewed the basic properties of gravitational
waves and methods used to model such waves from the inspiral and
merger of systems of compact binaries.  In the next chapter we discuss
the principles behind the LIGO detectors, which are looking for
gravitational-wave signals.  Then in chapter~\ref{ch:search} we
discuss how pN waveforms are used to detect signals in the LIGO data.


% Todo:
% Discuss the Weyl tensor, if needed for NR


\Chapter{The LIGO Gravitational Wave Detectors}
\label{ch:ligo_detectors}
The fact that matter responds to gravitational waves as described in
Sec.~\ref{sec:effects_of_waves} offers the possibility of making a
direct detection of such waves.  Although several methods of detection
have been proposed, we focus here on that used by LIGO and, with some
minor differences, Virgo and GEO.  Again, this presentation will be of
necessity brief.  We refer readers to the textbook by
Saulson~\cite{Saulson:1994} for a much more comprehensive treatment.

\section{A Toy Model}

We motivate our discussion of the LIGO interferometers with a toy
model.  We wish to detect gravitational waves, and one method is
suggested by the analysis of the preceding chapter; we look for the
strain, $\Delta L/L$, by measuring change in length of some system.
Simply measuring a length with a ruler will not work, as any ruler
will itself be stretched and compressed by the wave.  However, we can
also measure distance by sending a projectile, say a marble, with
known velocity through the length and measuring the travel time.  To
avoid complex issues of synchronizing clocks at different points in
general relativity we add a (hypothetical perfectly elastic) rubber
wall at the far end, and measure how long it takes to return.  If the
velocity of the marble is much larger than the velocity of the wall
induced by the wave (that is, $v \gg h \omega L$ where $h$ is the
strain and $\omega$ the gravitational-wave frequency) then there is a
simple relationship between the round-trip travel time and the
amplitude of the wave.  However, this requires unrealistic precision
in measurement, the uncertainty in the marbles' launch time will swamp
the small changes in length since, as we will see in the next chapter,
$h$ is typically very small.

Therefore, we instead construct a \emph{null experiment} where we try
to determine if a given quantity is exactly zero.  We arrange two
perpendicular paths, fire marbles down each at the same time and
measure the difference in return times.  As gravitational waves are
rare, we ``lock'' the system by shifting one or both of the walls such
that the marbles always collide exactly (say by measuring their recoil
angle).  Once locked, deviations in the length of either or both arms
will cause the difference in arrive times to become non-zero, which
can be determined by a change in the marbles' recoil angles or lack of
collision entirely.

To obtain robust results we want $\delta L$ to be as large as
possible.  Since gravitational waves are week, this means increasing
$L$.  Practical concerns may limit the ability to do this.  For
example, clearly the entire path from source to walls must be in
vacuum in order for the marbles not to lose energy, and building large
vacuum systems is difficult and expensive.  We therefore use a trick
and add a second set of walls between the source and reflectors, and
arrange the paths so that the marbles bounce back and forth several
times before returning to the source.  This effectively extends $L$ by
the distance between the two surfaces multiplied by the number of
bounces.

Our final extension to this toy model is rather unrealistic, but
imagine that marbles vanish after a collision.  Our ability to detect a
gravitational wave with statistical confidence then reduces to our
ability to count marbles.  We can model this as a Poisson process, where
probability of observing $N$ marbles is 
%
\begin{equation*}
p(N) = \frac{\bar{N}^N \exp(-\bar{N})} {N!}
\end{equation*}
%
where $\bar{N}$ is the average expected number of marbles per
observation period.  The error in estimating $N$ from counting goes as
$1/\sqrt{N}$, and it is therefore advantageous to send out as many 
marbles as possible.

\section{Interferometric Gravitational Wave Detectors}

\begin{figure}
  \includegraphics[width=\linewidth]{figures/detectors/LIGO}
  \caption[Block diagram of LIGO]{
  \label{f:ligo}
Block diagram of LIGO, see the text for description.
}
\end{figure}%

The toy model presented above captures the essential principles behind
LIGO, with the significant difference that light is used instead of
marbles.  A cartoon of the LIGO detectors is shown in
Fig.~\ref{f:ligo}.  The laser, beam splitter and two inner mirrors
(labeled \texttt{ITMX} and \texttt{ITMY} for ``inner test masses'')
form a \emph{Michelson interferometer}, and parallel the original toy
model with two marbles and two reflecting surfaces.
The use of light actually simplifies the analysis because light
travels on null geodesics, so
%
\begin{equation*}
ds^2 = 0 = g_{\mu\nu} dx^\mu dx^\nu
\end{equation*}

We now consider a +-polarized gravitational wave travelling in the $z$
direction, and place the arms on the $x$ and $y$ axes.  We again
assume the frequency of the wave is large compared to the travel time
between the arms, which implies that, over the round trip, $h_+$ may be
taken to be constant and the metric becomes
%
\begin{equation*}
g_{\mu\nu} = -dt^2 + (1+h_+) dx^2 + (1-h_+) dy^2 + dz^2
\end{equation*}
%
Then for the $x$ axis, restoring physical units,
%
\begin{equation*}
c^2 dt^2 = (1+h_+) dx^2
\end{equation*}
%
and the round-trip travel time is
%
\begin{equation*}
t_x = \frac{2L}{c} \sqrt{1+h_+} \approx \frac{2L}{c} \left(1+\frac{h_+}{2} \right)
\end{equation*}
%
where we approximate the square root by its Taylor series and ignore
higher-order terms in $h_+$.  Similarly the round-trip light travel
time along the $y$ arm is
%
\begin{equation*}
t_y = \frac{2L}{c} \left(1-\frac{h_+}{2} \right)
\end{equation*}

Considering a single wavefront leaving the beam splitter, the
difference in return time is
%
\begin{equation*}
\Delta t = t_x - t_y = \frac{2L}{c} h_+
\end{equation*}

In interferometry we measure the difference in phase between returning
wavefronts.  This difference will cause an interference pattern that
will serve as the readout.  If we use laser light of frequency $f$ and
wavelength $\lambda = c/f$ then a time difference of $\Delta t$
corresponds to a phase difference of 
%
\begin{equation*}
\Delta \Phi = \frac{2\pi f}{\Delta t} = \frac{4\pi f L}{c} h_+
= \frac{4\pi L}{\lambda} h_+
\end{equation*}

As in the marble example, we can increase the sensitivity of the
detector by increasing $L$, but practical considerations prevent us
from doing so.  One of these considerations is in fact the same for
marbles and light; the travel path must be in vacuum.  We
therefore employ the same trick and add two additional mirrors,
indicated as \texttt{ETMX} and \texttt{ETMY} (for ``end test
masses'').  The addition of these mirrors creates a \emph{Fabry-Perot
cavity} in each arm.  By arranging the mirrors to be an integer
number of wavelengths apart a resonance is built up that can trap the
light for extended periods, approximately 200 bounces.  It can be seen
that if the mirrors are not appropriately spaced there will be
destructive interference between the light moving in different
directions.  As the power in the beam must be conserved, this results
in energy leaking out of the cavity, reducing the efficiency.

The addition of the Fabry-Perot cavities would suggest an increase in
phase difference of two orders of magnitude.  However, it is in fact
better than that.  The light is now in flight sufficiently long that,
for gravitational waves from astrophysical sources, $h_+$ will have
changed during the interval and the above analysis is no longer
valid.  A more careful analysis shows that the improvement is 
three orders of magnitude.

In interferometry it is typically most useful to think of light as a
wave.  However, where in the toy example our ability to detect
gravitational waves was limited by our ability to count marbles, in
real LIGO we are limited by our ability to count photons.  Photon
number is related to laser energy by $E=h\nu$, so we therefore want to
use as powerful a laser as possible.  There are, however, technical
obstacles to doing so.  In the latest LIGO run the laser power was up
to 14 W, although it was not always possible to run at this level.

In lieu of raising the laser power we can at least ensure that no
power is wasted.  LIGO is configured such that the beams interfere
destructively when they recombine, we say the detector sits on a
\emph{dark fringe}~\footnote{Actually if we sat exactly on a dark
fringe then any change in the arm lengths would cause an increase of
light at the readout, and we would be unable to determine in which
direction the mirrors were moving.  We therefore sit a bit off the
dark fringe.  In addition, this condition is necessary for DC readout,
to be discussed shortly.}.  By conservation of energy all the power
emitted by the laser must go back towards the laser (neglecting the
portion lost to scattering, absorbed by the mirrors, etc).  We can
recover this power by adding another mirror, indicated on the diagram
as \texttt{PRC}, or power-recycling cavity.

The final feature on Fig.~\ref{f:ligo} is the \texttt{OMC} (output
mode cleaner), which was an important addition to the latest science
run.  A full description of this element is outside the scope of this
thesis, but we note briefly that the cross-section of a laser beam can
be decomposed into Hermite-Gaussian modes, as in
Fig.~\ref{f:hermite_gauss}.  The higher-order modes do not contribute
to the readout signal, however they do contribute to the shot noise.
It is therefore advantageous to suppress such modes.  In addition the
OMC was necessary to enable \emph{DC readout}, a scheme in which the
power present at the output port of the interferometer is directly
proportional to gravitational wave
strain~\cite{0264-9381-25-11-114030}.  Such a scheme offers reduced
shot noise and other advantages over \emph{RF readout}, where the
length changes are determined by the interaction of light at the
frequency of the laser and light at sideband frequencies.  We draw
attention to the OMC and disregard other mode cleaners because the OMC
tended to produce glitches, especially in the Livingston detector.
%
\begin{figure}
\includegraphics[width=\linewidth]{figures/detectors/TEMmn}
\caption[Laser modes]{ \label{f:hermite_gauss} Basis functions for the
cross-sectional distribution of power in a laser beam.  The lack of
symmetry in the higher-order modes can induce instabilities in the
LIGO optics. (\it{Public-domain image taken from
Wikipedia}~\cite{wikipedia:temmn}) } \end{figure}%

\subsection{Readout}
\label{ssec:readout}

In order to operate correctly the two Fabry-Perot cavities and the
power-recycling mirror must be positioned such that the light is
resonant.  The Michelson must likewise by arranged so that the output
photodetector is on a dark fringe.  When the detector is in this
state we say it is \emph{locked}, at that point a gravitational wave
will perturb the system within limits and produce light out the
output.  However, the resonances must be very finely tuned and left
untouched the system would quickly fall out of lock due to random
motions of the mirrors.  There is therefore a need for continuous,
active corrections implemented by a system of sensors and servos
throughout the detector.

Ignoring the OMC, in Fig.~\ref{f:ligo} the degrees of freedom are
the two initial test masses, the two end masses, and the PRC.
However, it is the lengths betweens elements that we are interested
in, so the there are fewer degrees of freedom than there are optical
elements.  In addition it is convenient to work with linear
combinations of these degrees of freedom:

\begin{itemize}
\item \texttt{DARM}: the differential arm length, (ETMX-ITMX) -
(ETMY-ITMY)
\item \texttt{CARM}: the common arm length, ((ETMX-ITMX) +
(ETMY-ITMY) / 2
\item \texttt{PRC}: The common recycling cavity length
\item \texttt{MICH}: the differential motion of the ``small Michelson'' comprising
the ITMs and the beam splitter
\end{itemize}

One of the sensors used to keep the system on resonance is the output
photodector.  Along with this, note that \texttt{DARM} is the quantity
of interest in the experiment, the change in length produced by a
gravitational wave.    It turns outs that rather than reporting the
signal at the photodiode directly a better output is the extent to
which this degree of freedom is off the resonance condition, which is
recorded as \texttt{DARM\_ERR}.  Henceforth, and especially in
chapter~\ref{ch:detchar} we consider this ``the output of the
detector''.  It is not, however, the data stream in which we will
search for gravitational waves.  The detector output needs to be
calibrated with respect to the detector's frequency response.  This
can be measured by injecting a sine wave of known amplitude into the
system by actuating one of the mirrors, and measuring the amplitude
and phase of \texttt{DARM\_ERR}.  The result is a complicated function of
frequency.  This can then be inverted to map \texttt{DARM\_ERR} back to the true
input, the result is stored as \texttt{LSC-STRAIN} (for \emph{Length
Sensing and Control}), and it is that channel on which
gravitational-wave searches are performed.

\subsection{Noise Sources}
\label{sec:noise_sources}

In addition to gravitational-wave sources the detector is subject to
various other influences collectively known as noise.  One class of
noise can be modeled as a \emph{Gaussian random process}, as we will
discuss in the next chapter.  These are best characterized by their
frequency profiles.  The dominant source are:
%
\begin{itemize} 
\item \emph{Seismic noise} is due to the coupling of the
detector to the ground.  Much work has been been done, and much
research continues to be done, to isolate the mirrors from the
environment.  However, the isolation is not complete.  This noise
source dominates at low frequency, rising sharply below 40 Hz in
Initial LIGO.  In Advanced LIGO we hope to push this so-called
\emph{seismic wall} down to 10 Hz.  This noise source includes the
natural constant vibrations of the Earth, wind blowing over nearby
structures, and anthropogenic sources such as vehicles near the sites,
logging activity, etc.

\item \emph{Thermal noise}. Any system possesses energy proportional
to the product of Boltzmann's constant and the ambient temperature.
This energy manifests as random motion throughout the detector,
although the motion from the test masses and the wires from which they
are hung are the largest contributors to noise.  The noise produced
dominates from 40 Hz to approximately 200 Hz.

\item \emph{Shot noise} is the uncertainty inherent in counting
photons, as discussed above.  It increases with frequency and
dominates above $\sim 1$ kHz.

\end{itemize}
%
In addition to these broad-band sources of noise there are also
\emph{lines}, particular frequencies at which the noise is much
greater than the three sources above would produce.  Two of the most
significant are:
%
\begin{itemize}
\item \emph{Electrical noise}.  Despite shielding, at 60 Hz there is a
sharp increase in the noise level due to the frequency of the US
electrical grid.
\item \emph{Violin modes}.  Although the wires suspending the mirrors
vibrate over a range of frequencies due to thermal noise, the
suspension system has a resonance at about 340 Hz, producing much
more noise here.
\end{itemize}
%
There are also lines at higher harmonics of these frequencies.

In addition to these continuous noise sources there are numerous
\emph{glitches}, short transient events.  These glitches have a wide
variety of different morphologies and originate from many different
sources.  Some glitches are reactions to external environmental
conditions and others are internal to the detector.  As one
straightforward example, a heavy object dropped on a nearby road 
can, through seismic coupling, shake the mirrors and produce a sharp
impulse in \texttt{DARM\_ERR}.

At the most abstract level gravitational-wave searches entail looking
for features in the data that deviate in some way from the continuous
background.  It is therefore not surprising that glitches can
interfere with searches and hence must be removed to the extent
possible.  Chapter~\ref{ch:segdb} will present the infrastructure used
to store information about the state of the detector and environment,
and chapter~\ref{ch:detchar} will discuss part of the effort to
identify glitches and remove them from the analysis.

\section{Conclusions}

In this chapter we reviewed the basic elements of interferometric
gravitational-wave detectors.  In the next chapter we discuss how we
search the data obtained by these detectors for signals of the kind
described in the previous chapter.



\Chapter{Searching for Gravitational Waves from Compact-Binary
Coalescences}
\label{ch:search}

% from Boyle et al.
\subsection{Matched filtering}
\label{sec:MatchedFiltering}

Current searches for gravitational waves from binary black-hole
coalescence use matched filtering to search for a waveform buried in
noise.  The matched filter is the optimal filter for detecting a
signal in stationary Gaussian noise.  Suppose that $n(t)$ is a
stationary Gaussian noise process with one-sided power spectral
density $S_n(f)$ given by $\langle \tilde{n}(f) \tilde{n}^\ast(f')
\rangle=\frac{1}{2} S_n(|f|)\delta(f-f')$.  For long integration
times, the data stream $s(t)$ output by the detector will always be
dominated by the noise.  Thus, we can simply approximate $n \approx s$
to calculate $S_{n}(f)$.

Using this power spectral density (PSD), we can define the inner
product between two real-valued signals---the data stream $s$ and the
filter template $h$---by
\begin{eqnarray}
  \label{eq:InnerProduct}
  \InnerProduct{s|h} &\equiv 2\, \Re \int_{-\infty}^{\infty}\,
  \frac{\tilde{s}(f)\, \tilde{h}^{\ast}(f)}{S_{n}(\lvert f
    \rvert)}\, d f \\ &= 4\, \Re \int_{0}^{\infty}\,
  \frac{\tilde{s}(f)\, \tilde{h}^{\ast}(f)}{S_{n}(f)}\, d f\ .
\end{eqnarray}
Then, given data $s$ which may contain either noise $n$ or noise and a
gravitational wave signal $h$,
\begin{equation}
  s = \left\{\begin{array}{l}
      n  \\
      n+h
    \end{array} \right.\ ,
\end{equation}
the matched-filter signal-to-noise ratio (SNR) is defined as
\begin{equation}
  \label{eq:InnerProductSNR}
  \rho = \frac{1}{\sqrt{\InnerProduct{h|h}}} \InnerProduct{s|h}\ .
\end{equation}


The SNR can then be used to construct a detection statistic (directly
or in combination with other statistics).  It is therefore important
to ensure that the templates used in searches accurately model the
expected waveforms to avoid reduction in the value of $\rho$. The
\emph{overlap} between two templates $h$ and $h'$ is defined as
\begin{equation}
  \label{eq:OverlapDefinition}
  \Overlap{h|h'} \equiv \frac{\InnerProduct{h|h'}}{
    \sqrt{\InnerProduct{h|h} \InnerProduct{h'|h'}}}\ .
\end{equation}
The overlap encodes the fractional loss in SNR that results from using
the template $h'$ rather than the true waveform $h$.  In a search that
uses $\rho$ as a detection statistic this corresponds to the
fractional loss in distance to which the search is sensitive.

The filter template includes arbitrary time and phase offsets, encoded
by the arrival time and phase, $\ta$ and $\phia$.  Under a change of
these quantities, the Fourier transform behaves as
\begin{equation}
  \label{eq:EffectOfTimeAndPhaseOffset}
  \tilde{h}(f) \to \tilde{h}(f)\, \e^{-2\pi i f \ta - i \phia}\ .
\end{equation}
We maximize over these two variables by calculating the inner product
as

\begin{eqnarray}
  \max_{\ta, \phia}\, \InnerProduct{s|h}
  &= \max_{\ta, \phia}\, 4\, \Re \int_{0}^{\infty}\,
  \frac{\tilde{s}(f)\, \tilde{h}^{\ast}(f)}{S_{n}(f)}\, \e^{2\pi i
    f\ta + i \phia}\, d f
  \\
  & = 4 \max_{\ta}\, \left\lvert \int_{0}^{\infty}\,
    \frac{\tilde{s}(f)\, \tilde{h}^{\ast}(f)}{S_{n}(f)}\, \e^{2\pi i
      f\ta}\, d f \right\rvert\ .
\end{eqnarray}
Note that this integral is just the (inverse) Fourier transform of the
quantity $\tilde{s}(f)\, \tilde{h}^{\ast}(f) / S_{n}(f)$ evaluated at
$\ta$.  Thus finding the maximum over $\ta$ involves taking the
Fourier transform and selecting the largest element of the finite set
that results from discretization.


\Chapter{Comparison of Numerical Simulations of
Black-Hole Binaries with Post-Newtonian Waveforms}
\label{ch:comparison}
\newcommand{\Note}[1]{\textcolor{red}{\textbf{[#1]}}}
\newcommand{\Add}[1]{\textcolor{blue}{#1}}
%\renewcommand{\d}{\mathrm{d}}
%\renewcommand{\e}{\mathrm{e}}
%\renewcommand{\i}{\mathrm{i}}
\newcommand{\hyb}{\mathrm{hyb}}
\newcommand{\NR}{\mathrm{NR}}
\newcommand{\pN}{\mathrm{pN}}
\newcommand{\ADMMass}{M_{\mathrm{ADM}}}
\newcommand{\IrrMass}{M_{\mathrm{AH}}}
\newcommand{\MSun}{\ensuremath{M_{\odot}}}
\newcommand{\deff}{\ensuremath{D_{\mathrm{eff}}}} % effective distance
\newcommand{\ta}{\ensuremath{t_{\mathrm{a}}}}
\newcommand{\phia}{\ensuremath{\phi_{\mathrm{a}}}}
\newcommand{\fsamp}{\ensuremath{f_{\mathrm{s}}}}
\newcommand{\fNy}{\ensuremath{f_{\mathrm{Ny}}}}
\newcommand{\discretize}{\rightsquigarrow}
\newcommand{\software}[1]{\textsc{#1}}
\newcommand{\etal}{\textit{et al.}\xspace}
\newcommand{\G}{G}
\renewcommand{\c}{c}
\newcommand{\e}{e}
%\newcommand{\lvert}{\ensuremath{|}}
%\newcommand{\rvert}{\ensuremath{|}}

%%% Define \InnerProduct with extendible middle line
%%% Adapted from braket.sty
\makeatletter
\let\protect\relax
{\catcode`\|=\active
  \xdef\InnerProduct{\protect\expandafter\noexpand\csname InnerProduct \endcsname}
  \expandafter\gdef\csname InnerProduct \endcsname#1{%
    \begingroup
    \ifx\SavedDoubleVert\relax
    \let\SavedDoubleVert\|\let\|\IpDoubleVert
    \fi
    \mathcode`\|32768\let|\IPVert
    \left({#1}\right)
    \endgroup
  }
}
\def\IPVert{\@ifnextchar|{\|\@gobble}% turn || into \|
     {\egroup\,\mid@vertical\,\bgroup}}
\def\IPDoubleVert{\egroup\,\mid@dblvertical\,\bgroup}
\let\SavedDoubleVert\relax
\def\midvert{\egroup\mid\bgroup}
\def\SetVert{\@ifnextchar|{\|\@gobble}% turn || into \|
    {\egroup\;\mid@vertical\;\bgroup}}
\def\SetDoubleVert{\egroup\;\mid@dblvertical\;\bgroup}
\def\mid@vertical{\mskip1mu\vrule\mskip1mu}
\def\mid@dblvertical{\mskip1mu\vrule\mskip2.5mu\vrule\mskip1mu}
\makeatother


\section{Introduction}
\label{sec:Introduction} %

The coalescence of binary black holes is one the most promising
sources of gravitational waves for interferometric gravitational-wave
detectors, such as LIGO, Virgo and GEO600~\cite{thorne.k:1987}. The
first-generation LIGO detectors have achieved their design sensitivity
and recorded over one year of coincident data~\cite{Abbott:2007kva}.
This data, together with data from the Virgo detector, are currently
being searched for gravitational waves from compact binary
coalescence~\cite{Abbott:2003pj,Abbott:2005pe,Abbott:2005pf,%
  Abbott:2007xi,Abbott:2007ai,Abbott:2008}.  Upgrades to improve the sensitivity
of these detectors by a factor of two, and ultimately 10, are
underway.  Optimal searches using the enhanced detectors in 2009 will
be sensitive to black-hole coalescence out to hundreds of
megaparsecs~\cite{LIGOEnhancedLIGO}. The advanced detectors,
operational next decade, could detect black-hole binaries at distances
of over \unit[1]{Gpc}~\cite{Fritschel:2003qw}.

Optimal searches for gravitational waves use matched filtering, which
requires accurate knowledge of the waveform~\cite{thorne.k:1987}.
Previous searches in LIGO data have used post-Newtonian and
phenomenological templates to search for the coalescence of black-hole
binaries~\cite{Abbott:2005pf,Abbott:2007xi,Abbott:2008}. Over the last
several years numerical relativity has made remarkable breakthroughs
in simulating the late inspiral, merger and ringdown of black-hole
binaries. The computational cost of these simulations is high,
however, making it impractical to use them directly as template
waveforms for use in a matched-filter search. It has been shown that
there is good agreement between the waveforms generated by numerical
relativity with analytic post-Newtonian waveforms to within just a few
orbits of merger~\cite{Buonanno-Cook-Pretorius:2007, Baker2006d,
  Pan2007, Buonanno2007, Hannam2007, Boyle2007, Gopakumar:2007vh,
  Hannam2007c, Boyle2008a, Mroue2008, Hinder2008b}.

This paper uses the high-accuracy Caltech--Cornell
numerical-relativity waveforms to suggest improvements to the analytic
waveforms currently used in gravitational-wave searches by LIGO and
Virgo.  A similar study has been performed by Pan~\etal using
numerical data from Pretorius and the Goddard groups~\cite{Pan2007}.
Our main results are in agreement with their conclusion that a simple
extension of the existing stationary-phase approximation to the
adiabatic post-Newtonian waveforms (called \textit{TaylorF2} in
Ref.~\cite{Damour2001}) yields high overlaps with numerical waveforms.

In Sec.~\ref{sec:Searches}, we review the current techniques used for
searching for gravitational waves in gravitational-wave detector data.
We discuss the construction of the waveform---a pN--NR hybrid---in
Sec.~\ref{sec:PNNRHybridWaveform}.  In Sec.~\ref{sec:Efficiency} we
employ the hybrid waveform in a comparison of the detection efficiency
of gravitational-wave templates that may be used in upcoming searches
of LIGO and Virgo data.  Finally, in Sec.~\ref{sec:Recommendations},
we discuss improvements that may be made to the current data-analysis
techniques to optimize overlaps.

Throughout this paper, we use only the $(l,m)=(2,2)$ component of the
waveform $\Psi_{4}^{2,2}$ (as defined, e.g., in~\cite{Boyle2008a}).
For convenience, we drop the superscript.  Whenever possible, we use
dimensionless quantities, like $r\,M\,\lvert \Psi_{4} \rvert$, where
$r$ is the areal radius of the observation sphere, and $M$ is the
total apparent-horizon mass of the holes in the initial data.
However, for any calculation involving the LIGO noise curve, we have a
physical scale, and thus use standard mks units.

% , where
% \begin{eqnarray}
%  \label{eq:units}
%  \G &= \unit[6.67259 \times 10^{-11}]{{m^3}\ {kg^{-1}\ s^{-2}}}\ ,\\
%  \c &= \unit[299792458]{{m}\ {s^{-1}}}\ ,\\
%  \MSun &= \unit[1.98892 \times 10^{30}]{kg}\ ,\\
%  \unit[1]{Mpc} &= \unit[3.08568025 \times 10^{22}]{m}\ .
%\end{eqnarray}


%%%%%%%%%%%%%%%%%%%%%%%%%%%%%%%%%%%%%%%%%%%%%%%%%%%%%%%%%%%%%%%%%%%%%%
%%%%%%%%%%%%%%%%%%%%%%%%%%%%%%%%%%%%%%%%%%%%%%%%%%%%%%%%%%%%%%%%%%%%%%
\section{Searches for gravitational waves from black-hole binaries}
\label{sec:Searches}


In this paper we are concerned with overlaps between pN waveforms and
NR signals; to weight the inner product we use the following PSDs for
Initial and Advanced LIGO: for Initial LIGO we use an analytic
approximation to the LIGO design PSD given by
\begin{eqnarray}
  S_n(f) &= &3.136 \times 10^{-4} \bigg[
  \left(\frac{ 4.49 f}{150.0}\right)^{-56.0} \nonumber \\
  &+ & 0.16 \left(\frac{f}{150}\right)^{-4.52}
  + \left(\frac{f}{150.0}\right)^2 + 0.52
  \bigg]
\end{eqnarray}
All integrals start from 40 Hz.  As shown in
Fig.~\ref{fig:StildesAndInitialPSD}, at this frequency the noise is an
order of magnitude higher than its lowest value, and below this
frequency it rises rapidly as $\sim f^{-56}$.  The region below 40 Hz
therefore contributes very little signal power to the
SNR~\cite{Abbott:2007xi}.  The PSD for Enhanced LIGO, which will
begin operation in mid 2009, has a similar shape to that for Initial
LIGO although it has a factor of $\sim 2$ increase in strain
sensitivity.  Our results using the Initial-LIGO PSD are therefore
valid for Enhanced LIGO, as the sensitivity factor cancels in
Eq.~\eqref{eq:OverlapDefinition}; overlaps depend on the \emph{shape}
of the PSD.

For Advanced LIGO we use the GWINC program~\cite{AdvancedLIGONoise} to
generate the PSD.  Bench reports the PSD in increments of 0.0124 Hz.
When calculating discrete integrals against signals sampled at other
frequencies we obtain values for the PSD by linearly interpolating
between the values provided by Bench.  We start integrals at 10 Hz as
that is the point where the noise has increased by two orders of
magnitude above its minimum, as also shown in
Fig.~\ref{fig:StildesAndInitialPSD}.




% \subsection{Discretization}
% \label{sec:Discretization}

% The direct and inverse Fourier transforms are defined (using the
% standard LIGO conventions~\cite{T010095}) as
% \begin{align}
%   \label{eq:FourierTransform}
%   \tilde{s}(f) &\equiv \int_{-\infty}^{\infty}\, s(t)\, \e^{-2\pi\i
%     f t}\,
%   \d t\ , \\
%   \label{eq:InverseFourierTransform}
%   s(t) &= \int_{-\infty}^{\infty}\, \tilde{s}(f)\, \e^{2\pi\i f t}
%   \, \d f\ .
% \end{align}
% In transferring these and the continuum expressions of preceding
% sections to computer, we need to introduce two changes.

% First, the ranges of integration must be restricted to finite
% intervals.  We need to assume that the physical signal contains
% nothing of interest at frequencies higher than $\fNy$ or,
% considering negative frequencies, lower than $-\fNy$.  For
% notational simplicity, we define the discretized Fourier transform
% to be periodic, with period $2\fNy$.  Similarly, we will assume that
% the signal $s$ is periodic, with period $T$.  Thus, we can restrict
% each of the integrals given above to one period of the relevant
% quantity.

% Second, the quantities must be given on a discrete grid.  We will
% assume that the signal $s$ is sampled at $N$ uniform intervals of
% $\Delta t=T/N$.  This will give rise to a frequency discretization
% of $\Delta f = 1/T = 1/N\Delta t$.  We define the quantities
% \begin{equation}
%   t_{j} = j\Delta t \quad \text{and} \quad f_{k} = k\Delta f\ ,
% \end{equation}
% for \emph{all} integers $j$ and $k$.  It is not hard to see that the
% highest frequency that can be represented on this discrete set is
% bounded by the Nyquist frequency $\fNy=1/2\Delta t$.

% The combined operation of discretizing and restricting to finite
% range will be denoted by $\discretize$, so
% \begin{align}
%   \label{eq:DiscreteFourierTransform}
%   \tilde{s}(f_{k}) &\discretize \sum_{t_{j}>-N\Delta t/2}^{N\Delta
%     t/2}\, s(t_{j})\, \e^{-2\pi\i f_{k} t_{j}}\, \Delta t
%   \\
%   \label{eq:DiscreteFourierTransformTwo}
%   &\discretize \Delta t\, \sum_{j=0}^{N-1}\, s(t_{j})\, \e^{-2\pi\i
%     j k/N}\ ,
%   \\
%   \label{eq:DiscreteInverseFourierTransform}
%   s(t_{k}) &\discretize \sum_{f_{j} > -\fNy} ^{\fNy}\,
%   \tilde{s}(f_{j})\, \e^{2\pi\i f_{k} t_{j}}\, \Delta f
%   \\
%   \label{eq:DiscreteInverseFourierTransformTwo}
%   &\discretize \Delta f\, \sum_{j=0}^{N-1}\, \tilde{s}(f_{k})\,
%   \e^{2\pi\i j k/N}\ .
% \end{align}
% Note that in the second step of each of these expressions, we have
% used the periodic character of $s$ to re-express negative times as
% positive, and the periodic character of $\tilde{s}$ to re-express
% negative frequencies as positive.  We have also used the relation
% $f_{k}t_{j} = j k/N$.

% A notational subtlety arises when using the frequency-domain
% quantity.  The symbol $\tilde{s}_{k}$ is defined as the sum in
% Eq.~\eqref{eq:DiscreteFourierTransformTwo}, without the factor of
% $\Delta t$~\cite{DBrownThesis}.  In particular, we have
% $\tilde{s}(f_{k}) \discretize \Delta t\, \tilde{s}_{k}$.  The
% expressions given above for $\tilde{s}(f_{k})$ and $s(t_{j})$ should
% not depend strongly on the fineness of the discretization (for
% sufficiently fine discretizations).  Clearly, then, $\tilde{s}_{k}$
% \emph{will} depend strongly on the discretization.  Though the
% factor of $\Delta t$ should drop out for calculations using
% normalized quantities, for calculations of the signal-to-noise
% ratio, or demonstrations of $\tilde{s}(f_{k})$, it is an important
% distinction that needs to be kept in mind.  For example, the
% \software{FFTW} and \software{Matlab} software packages use
% $\tilde{s}_{k}$ as their standard frequency-domain quantity.
% Throughout the remainder of this paper, we will use
% $\tilde{s}(f_{k})$ exclusively.

% For completeness, we include the expressions
% \begin{align}
%   \InnerProduct{s|h} &\discretize 2\, \Re \sum_{k=0}^{N-1}\,
%   \frac{\tilde{s}(f_{k})\, \tilde{h}^{\ast}(f_{k})}{S_{n}(\lvert
%     f_{k} \rvert)}\, \Delta f
%   \\
%   &\discretize 4\, \Re \sum_{k=0}^{\lfloor N/2 \rfloor}\,
%   \frac{\tilde{s}(f_{k})\, \tilde{h}^{\ast}(f_{k})}{S_{n}(f_{k})}\,
%   \Delta f\ ,
%   \\
%   \Overlap{s|h} &\discretize 2 \max_{\ta}\, \left\lvert
%     \sum_{k=0}^{N-1}\, \frac{\tilde{s}(f_{k})\,
%       \tilde{h}^{\ast}(f_{k})}{S_{n}(\lvert f_{k} \rvert)}\,
%     \e^{2\pi \i f_{k}\ta}\, \Delta f \right\rvert
%   \\
%   &\discretize 4 \max_{\ta}\, \left\lvert \sum_{k=0}^{\lfloor N/2
%       \rfloor}\, \frac{\tilde{s}(f_{k})\,
%       \tilde{h}^{\ast}(f_{k})}{S_{n}(f_{k})}\, \e^{2\pi \i
%       f_{k}\ta}\, \Delta f \right\rvert\ .
% \end{align}
% The notation $\lfloor N/2 \rfloor$ denotes the greatest integer less
% than or equal to $N/2$.


%%%%%%%%%%%%%%%%%%%%%%%%%%%%%%%%%%%%%%%%%%%%%%%%%%%%%%%%%%%%%%%%%%%%%%
%%%%%%%%%%%%%%%%%%%%%%%%%%%%%%%%%%%%%%%%%%%%%%%%%%%%%%%%%%%%%%%%%%%%%%
\section{PN--NR hybrid waveform}
\label{sec:PNNRHybridWaveform} %


In order to perform our comparison we need to construct a ``true''
black-hole binary waveform, which we might expect to observe with
detectors.  A numerical simulation will provide the data for the
crucial nonlinear merger phase.  We carefully extract the data and
extrapolate it to large radius, and investigate the effects of
numerical error on the final result.  Because this waveform is very
computationally expensive to produce, it covers only about 32 cycles,
which is not sufficient for a thorough investigation of the
possibility of detecting it in searches of data from
gravitational-wave detectors.  Thus, we match the numerical waveform
to a post-Newtonian waveform, producing a hybrid which extends for
many thousands of cycles, covering the entire band of interest.

\subsection{Numerical simulation, extraction, and extrapolation}
\label{sec:WaveformExtractionAndExtrapolation}
The numerical simulation is the same as that described in
Refs.~\cite{Boyle2007, Scheel2008}: an equal-mass, non-spinning,
black-hole binary with reduced
eccentricity~\cite{Pfeiffer-Brown-etal:2007}, beginning roughly 16
orbits before merger, continuing through merger and
ringdown~\cite{Scheel2008}.  It is performed with the Caltech--Cornell
pseudospectral code, using boundary conditions designed to prevent
constraint violations and gravitational radiation from entering the
domain~\cite{Holst2004, Lindblom2006}.

Data is extracted from the simulation in the form of the
Newman--Penrose scalar
\begin{equation}
  \Psi_{4} = -C_{\alpha \beta \gamma \delta} l^{\alpha}
  \bar{m}^{\beta} l^{\gamma} \bar{m}^{\delta}\ ,
\end{equation}
where $l^{\alpha}$ and the complex vector $\bar{m}^{\beta}$ are
constructed with reference to the coordinate basis.  Along the
positive $z$ axis, we have
\begin{eqnarray}
  l^{\alpha} &= & \frac{1}{\sqrt{2}}\, \left( t^{\alpha} - z^{\alpha}
  \right)\ , \\
  \bar{m}^{\beta} &= & \frac{1}{\sqrt{2}}\, \left(
    \frac{\partial}{\partial x} - i\, \frac{\partial}{\partial y}
  \right)^{\beta}\ .
\end{eqnarray}

Here, $t^{\alpha}$ is the timelike unit normal to the spatial
hypersurface, and $z^{\alpha}$ is the unit vector in the positive $z$
direction.  The vectors $\partial/\partial x$ and $\partial/\partial
y$ are the standard coordinate vectors, which are not normalized.
$\Psi_{4}$ is extracted as a function of time, at various radii along
the positive $z$ axis.  This is then extrapolated to large radii, as
described in Ref.~\cite{Boyle2007}, and in greater detail in
Ref.~\cite{Boyle2008}.

% The Nyquist frequency of the data is \Note{bla, as compared to the
%   measured ringdown frequency}.  Note that the LIGO Nyquist
% frequency is \unit[2048]{Hz}, which is lower than the ringdown
% frequency for masses above bla.

The measured (instantaneous) frequency at the beginning of the
simulation is
\begin{equation}
  \label{eq:MeasuredStartingFreq}
  % \frac{G\,M}{c^{3}}\,\omega & = 0.0333 \pm 0.0002\ , \\
  % \frac{G\,M}{c^{3}}\,f & = 0.00530 \pm 0.00003\ , \\
  f_{\mathrm{initial}} = \unit[(1.08 \pm 0.01) \times 10^{3}]{Hz}\,
  \frac{\MSun}{M}\ .
\end{equation}
The measured ringdown frequency is
\begin{equation}
  \label{eq:MeasuredRingdownFreq}
  % \frac{G\,M}{c^{3}}\,\omega & = 0.553 \pm 0.007\ , \\
  % \frac{G\,M}{c^{3}}\,f & = 0.088 \pm 0.001\ , \\
  % \omega & = \unit[(112 \pm 1) \times 10^{4}]{\frac{1}{sec}}\,
  % \frac{\MSun}{M}\ , \\
  f_{\mathrm{ringdown}} = \unit[(1.78 \pm 0.02) \times 10^{4}]{Hz}\,
  \frac{\MSun}{M}\ .
\end{equation}
The measured \emph{Christodoulou} mass and spin of the final black
hole are
\begin{eqnarray}
  \label{eq:MeasuredFinalMass}
  M_{\chi\mathrm{, final}} &= &(0.95162 \pm 0.00002)\, M_{\chi\mathrm{,
      initial}}\ , \\
  \label{eq:MeasuredFinalSpin}
  S_{\mathrm{final}} &= &(0.68646 \pm 0.00004)\, M^{2}_{\chi\mathrm{, final}}\ .
\end{eqnarray}
Using this value for the spin, a quasi-analytic formula due to
Echeverria~\cite{Echeverria1989} predicts a value of
$\unit[1.77\times10^{4}]{Hz}\, \frac{\MSun}{M}$, for the ringdown
frequency, in close agreement with the measured frequency.

\subsection{Accuracy of the numerical simulation}
\label{sec:Accuracy}

The numerical waveform will be the standard against which we will
judge the \textit{TaylorF2} waveforms used in LIGO data analysis.  To
understand how precisely we should trust our final results, we need to
understand the accuracy of the waveform itself.  The most obvious
measure of the error in this fiducial waveform is its convergence with
increasing numerical resolution.  Fig.~\ref{f:accuracy} shows the
overlap (Eq.~\eqref{eq:OverlapDefinition}) between waveforms computed
at different resolutions.  The data used here are the extrapolated
$\Psi_4$ waveforms, integrated in time twice.
%%%%%%%%%%%%%%%%%%%%%%%%%%%%%%%%%%%%%%%%%%%%%%%%%%%%%%%%%%%%%%%%%%%%%%
\begin{figure}
  \begin{center}
    \includegraphics[width=0.55\linewidth]{figures/comparison/Accuracy}
  \end{center}
  \caption[Convergence testing for numerical waveforms ]{
  \label{f:accuracy}
    Convergence testing for numerical waveforms from a
    data-analysis perspective, using the match between waveforms
    computed at different numerical resolutions.  The waveforms are
    scaled to various masses, and the Initial-LIGO noise curve is used
    in the calculation of the match.  The upper panel shows the
    overlap without maximization over arrival time and phase; the
    lower panel shows the overlap after maximization.  In each panel,
    the lower (dashed) line compares the lowest- and
    highest-resolution simulations, while the upper (solid) line
    compares the medium- and highest-resolution simulations.  Note
    that this plot uses only numerical data, with no post-Newtonian
    contribution.}
\end{figure}%
%%%%%%%%%%%%%%%%%%%%%%%%%%%%%%%%%%%%%%%%%%%%%%%%%%%%%%%%%%%%%%%%%%%%%%

Because of the short extent of the numerical waveforms, we need to be
careful when using their Fourier transforms.  The signal can be
corrupted easily by the non-periodicity of the waveforms, and the
discontinuous jumps that result.  For Fig.~\ref{f:accuracy} we
mitigate this problem by increasing the sampling frequency of the
input data, and restricting the Fourier transform to frequencies
corresponding to instantaneous frequencies contained in the data.  The
input data can easily be upsampled in the time domain by interpolating
the phase and amplitude of the complex data to a finer time grid.  We
then perform the transform, and explicitly set the data to zero at
frequencies below $f_{\mathrm{initial}}$ and above
$f_{\mathrm{ringdown}}$, as given in
Eqs.~\ref{eq:MeasuredStartingFreq} and~\ref{eq:MeasuredRingdownFreq}.
While the results do depend on whether or not we impose these cutoffs,
they do not depend sensitively on the actual cutoff frequencies.

The overlap between the lowest- and highest-resolution simulations
(dashed lines) actually passes through zero, as shown in the upper
panel.  Presumably, this is because of loss of phase accuracy over the
course of the simulation.  All three simulations begin with the same
initial data, so the waveforms are most similar at the beginning.
Masses for which this is the most important segment (the lowest
masses) will naturally have the highest overlap between resolutions.
As the simulation progresses, numerical error accumulates---notably in
the phase---so the overlap decreases with masses for which later
segments dominate the overlap (higher masses).  When the overlap is
optimized over arrival time and phase, we can see that the overlap
becomes much better, as shown in the lower panel, indicating
sufficient accuracy within any frequency band for which phase
coherence is required.  In either case, the medium and
highest resolutions are much more nearly the same.  Without
optimization, their overlap is within a few tenths of a percent of 1;
after optimization, the overlap is within $10^{-6}$ of 1.

In the rest of our analysis we use the highest-resolution waveform.
Because we always optimize over arrival time and phase, the lower
panel of Fig.~\ref{f:accuracy} is the most relevant, and shows that
the waveform has converged to very high accuracy.  The overlaps we
quote below will only be given to three decimal places at most,
because this is roughly the accuracy of the single-precision numerical
methods used in the rest of the paper.  This accuracy is also
sufficient for searches of gravitational-wave data.  Thus, the
truncation error of the simulated waveform is irrelevant for those
purposes.

Other sources of error include residual eccentricity and spin, the
influence of the outer boundary of the simulation, extrapolation
errors, and coordinate effects, as discussed in Ref.~\cite{Boyle2007}.
The eccentricity had a disproportionately large effect on the error
quoted in that paper because of the matching technique, which is not
used here.  Restricting attention to the other effects of
eccentricity, the uncertainty falls below that due to numerical error.
Similarly, using the techniques of Ref.~\cite{Lovelace2008}, the
initial spins of the black holes have been measured more reliably, and
found to be more than an order of magnitude smaller than previously
determined, allowing us to reduce the estimate for that error to less
than the numerical truncation error.  The various coordinate effects
were all estimated to be of roughly the same magnitude as the
numerical error.

With the numerical error being many times more accurate than needed
for this analysis, and the other sources of uncertainty being of
roughly the same size, these considerations indicate that the overall
error in our fiducial waveform is substantially less than the
precision needed for this analysis.


%%%%%%%%%%%%%%%%%%%%%%%%%%%%%%%%%%%%%%%%%%%%%%%%%%%%%%%%%%%%%%%%%%%%%%
%%%%%%%%%%%%%%%%%%%%%%%%%%%%%%%%%%%%%%%%%%%%%%%%%%%%%%%%%%%%%%%%%%%%%%
\section{Detection efficiency of gravitational-wave templates}
\label{sec:Efficiency} %

We now compare the signal described in the previous section to
restricted, stationary phase \textit{TaylorF2} post-Newtonian
templates with terms up to order 2.0, order 3.5, and a ``pseudo-4.0
pN-order'' term recommended in Ref.~\cite{Pan2007}.  Overlaps are
calculated using the techniques of Sec.~\ref{sec:MatchedFiltering},
with the signal $s$ being the hybrid waveform described in
Sec.~\ref{sec:PNNRHybridWaveform} scaled to a range of masses.  We
consider both the Initial- and Advanced-LIGO noise curves.
% We consider mass ranges up to the value where the ISCO frequency is
% roughly at the lower limit of sensitivity set by the seismic wall:
% $110 \MSun$ for Initial LIGO; $440 \MSun$ for Advanced LIGO.

Plots of the hybrid waveforms in comparison to the Initial-LIGO noise
curve are shown in Fig.~\ref{fig:StildesAndInitialPSD}.
%%%%%%%%%%%%%%%%%%%%%%%%%%%%%%%%%%%%%%%%%%%%%%%%%%%%%%%%%%%%%%%%%%%%%%
\begin{figure}
  \begin{center}
    \includegraphics[width=0.55\linewidth]{figures/comparison/StildesAndInitialPSD}
  \end{center}
  \caption[Hybrid Caltech--Cornell waveform scaled to various total masses]{
  \label{fig:StildesAndInitialPSD}
    Hybrid Caltech--Cornell waveform scaled to various total
    masses, with sources optimally oriented and placed at
    \unit[100]{Mpc}, shown against the Initial- and Advanced-LIGO
    noise curves.  Markers are placed along the lines at frequencies
    corresponding to various instantaneous frequencies of the
    waveforms.  The triangles represent the beginning and end of the
    blending region; the circle represents the ISCO frequency; the
    square the light-ring; and the diamond the measured ringdown
    frequency.  See the text for discussion of the normalization.  The
    values given for $\rho$ use the Initial-LIGO noise curve, with
    sources at a distance of 100\,Mpc.}
\end{figure}%
%%%%%%%%%%%%%%%%%%%%%%%%%%%%%%%%%%%%%%%%%%%%%%%%%%%%%%%%%%%%%%%%%%%%%%
The masses are chosen so that various frequencies of interest (the
final stitching frequency, the ISCO, and the ringdown) occur at the
``seismic wall'' for Initial LIGO: \unit[40]{Hz}.  The waveforms
$\tilde{s}$ are scaled to depict the detectability of the signal,
typically quantified by the SNR introduced in
~\eqref{eq:InnerProductSNR}, which may be written as
\begin{equation}
  \label{eq:SNR}
  \rho^{2} \equiv \int_0^\infty \frac{4\, \tilde{s}(f)\,
    \tilde{s}^\ast(f)} {S_n(f)}\, d f = \int_0^\infty
  \frac{\left\lvert 2\, \tilde{s}(f)\, \sqrt{f} \right\rvert^{2}}
  {S_n(f)}\, d\ln{f}\ .
\end{equation}
In the final expression, the numerator and denominator have the same
units, and are directly comparable.  Because the square root of the
denominator is familiar, we plot that along with the square root of
the numerator.  Plotting these two quantities together gives a
graphical impression of the detectability of the waveform, and the
relative importance of each part of the waveform, by its height above
the noise curve.  In Ref.~\cite{BradyCreighton2002}, Brady and
Creighton define a slightly different quantity, the characteristic
strain $h_{\mathrm{char}} \equiv f\, \lvert \tilde{s}(f) \rvert\ .$
The relative factor of $\sqrt{f}$ they use is present so that they can
plot $h_{\mathrm{char}}$ against $\sqrt{f\, S_{n}(f)}$.  Cutler and
Thorne~\cite{Cutler2002} define still another quantity, the signal
strength $\tilde{h}_{s}(f)$, which is related to the Fourier transform
by $\tilde{h}(f) = \sqrt{5}\, \frac{T}{N}\, \tilde{h}(s)\ .$ The
factor of $\sqrt{5}$ comes from averaging over the orientation of the
binary, which we do not do.  $T/N$ is the ratio of the threshold to
the rms noise at the endpoint of signal processing.

\iffalse
\begin{figure}
  % \includegraphics[width=\linewidth]{figures/comparison/AmoebaHistogram}
  \begin{center}
    \includegraphics[width=0.55\linewidth]{figures/comparison/Histogram}
  \end{center}
  \caption{Histogram of overlaps found by 300 instances of the Amoeba
    algorithm, optimizing the overlap against a given waveform over
    $M, \eta, f_c$, with randomized initial conditions.  Note the
    logarithmic scale on the vertical axis.  The majority of instances
    produced a lower overlap than the optimum.  We interpret this as
    pointing to the existence of a broad local maximum which did not
    coincide with the global maximum.}
  \label{fig:AmoebaHistogram}
\end{figure}%
\fi

For each template family we initially optimize over signal mass $M$,
symmetric mass ratio $\eta = m_1 m_2 / (m_1 + m_2)^2$, and upper
cutoff frequency $f_c$.  The optimization is performed using a
Nelder--Mead (``amoeba'') algorithm~\cite{numrec_cpp}.  The amoeba
starts with a simplex in the parameter space, and proceeds through a
series of steps, each of which will improve the value of the function
at at least one vertex.  The algorithm terminates when all vertices
have converged to the same point to within a specified tolerance.
This process is deterministic, and amounts to an enhanced
steepest-ascent algorithm.  It is therefore only guaranteed to find a
local maximum, and indeed we find that an amoeba instance started at a
random point in the parameter space is most likely to converge to a
point that does not give the highest possible overlap.  We interpret
this as being due to a large region in parameter space containing a
local maximum and a relatively smaller region containing the global
maximum.  We therefore supplement the basic amoeba by running 300
instances with random starting values, and taking the best match
obtained over all instances.  In repeated runs the same optimal
parameters were found by at least some of the amoebas, which supports
the claim that this is the true maximum.

The results of optimizing over all of $M, \eta$ and $f_c$ for selected
masses for Initial LIGO are given in
Table~\ref{tab:ThreeParamOverlapDetailInitial} and summarized in
Fig.~\ref{fig:ThreeParamOverlapSummaries}.  For Initial LIGO, in the
range covered by the current Compact Binary Coalescence (CBC) low-mass
search $(M < 35 \MSun)$~\cite{Abbott:2008}, the pseudo-4.0 pN
\textit{TaylorF2} waveforms achieve the highest overlaps, exceeding
those obtained with 3.5 pN waveforms by $\sim 1\%$.  Above $35 \MSun$
the 3.5 pN waveforms produce overlaps as much as 4\% greater than
those obtained with pseudo-4.0 pN waveforms over a range from
$40$--$80 \MSun$. With the Advanced-LIGO noise curve, in the CBC
low-mass range, the 3.5 pN and pseudo-4.0 pN waveforms produce
overlaps within 2\% of each other, with 3.5 pN producing higher
overlaps below 20 $\MSun$ and pseudo-4.0 pN producing higher overlaps
in the range $20$--$35 \MSun$.  Pseudo-4.0 pN continues to give the
highest overlaps up to $60 \MSun$, producing overlaps as much as 4\%
greater than those obtained with 3.5 pN waveforms.  Above $60 \MSun$
3.5 pN waveforms again yield the best overlaps, by as much as 6\%
around 90 $\MSun$.

% We see from Tables~\ref{tab:ThreeParamOverlapDetailInitial}
% and~\ref{tab:ThreeParamOverlapDetail} that the parameters of the
% optimal templates are often far from the parameters of the physical
% waveform, especially for high-mass systems, which emphasize portions
% of the waveform for which the pN and SPA assumptions are poor.  For
% Initial LIGO, pseudo-4.0 pN templates find $M$ to within $19\%$ of
% the true value compared to $52\%$ for 2.0 pN and $159\%$ for 3.5 pN.
% Conversely, for Advanced LIGO, 3.5 pN templates estimate the signal
% mass to within $16\%$ for masses less than $\MSun$, compared to
% $25\%$ for 2.0 pN and $20\%$ for pseudo-4.0 pN templates.

% \Note{Should we drop this paragraph?  We never systematically looked
%   at parameter estimation, and the numbers from the table don't tell
%   a compelling story regarding any of them doing better than the
%   others.}



% \addtolength{\tabcolsep}{3.25mm} \renewcommand{\arraystretch}{1.6}
\begin{table*}
  \begin{center}
    \begin{tabular}{@{}lcccc@{}}
      \hline \hline
      & $(10+10) \MSun$ & $(20+20) \MSun$ & $(30+30) \MSun$ & 
      $(50+50) \MSun$ \\
      \hline
      $\Overlap{s^{\textrm{NR-CC}} |
        h^{\textrm{SPA}_c^{\textrm{ext}}(2.0)}}$ &
      0.99 & 0.98 & 0.97 & 0.96 \\
      $M/\MSun$ &
      $23.27^{+0.13}_{-0.12}$  &
      $25.99^{+0.61}_{-0.56}$  &
      $35.22^{+1.84}_{-1.89}$  &
      $47.52^{+6.87}_{-4.73}$  \\
      $\eta$ &
      $0.199^{+0.0030}_{-0.0030}$  &
      $0.771^{+0.0490}_{-0.0420}$  &
      $1.000_{-0.1390}$  &
      $1.000_{-0.2490}$  \\
      $f_{\mathrm{cut}}$ (Hz) &
      $501.18^{+523.00}_{-153.00}$  &
      $431.35^{+358.00}_{-77.00}$  &
      $296.05^{+53.00}_{-31.00}$  &
      $190.56^{+20.00}_{-14.00}$  \\
      \hline
      $\Overlap{s^{\textrm{NR-CC}} |
        h^{\textrm{SPA}_c^{\textrm{ext}}(3.5)}}$ &
      0.98 & 0.99 & 0.99 & 0.99 \\
      $M/\MSun$ &
      $18.75^{+0.10}_{-0.10}$  &
      $31.88^{+0.77}_{-0.71}$  &
      $47.15^{+4.37}_{-3.27}$  &
      $259.89^{+0.00}_{-194.18}$  \\
      $\eta$ &
      $0.290^{+0.0040}_{-0.0040}$  &
      $0.493^{+0.0530}_{-0.0410}$  &
      $0.756^{+0.2440}_{-0.2290}$  &
      $0.954^{+0.0460}_{-0.2090}$  \\
      $f_{\mathrm{cut}}$ (Hz) &
      $506.50^{+518.00}_{-155.00}$  &
      $448.80^{+576.00}_{-83.00}$  &
      $324.74^{+145.00}_{-42.00}$  &
      $197.17^{+24.00}_{-16.00}$  \\
      \hline
      $\Overlap{s^{\textrm{NR-CC}} |
        h^{\textrm{SPA}_c^{\mathcal{Y}}(4)}}$ &
      0.99 & 0.96 & 0.95 & 0.96 \\
      $M/\MSun$ &
      $23.64^{+0.13}_{-0.12}$  &
      $47.90^{+1.28}_{-1.13}$  &
      $61.81^{+8.68}_{-6.19}$  &
      $89.93^{+20.44}_{-16.60}$  \\
      $\eta$ &
      $0.182^{+0.0030}_{-0.0030}$  &
      $0.181^{+0.0160}_{-0.0140}$  &
      $0.523^{+0.4260}_{-0.1820}$  &
      $0.529^{+0.4720}_{-0.3100}$  \\
      $f_{\mathrm{cut}}$ (Hz) &
      $509.47^{+654.00}_{-145.00}$  &
      $352.44^{+73.00}_{-61.00}$  &
      $309.53^{+72.00}_{-47.00}$  &
      $195.63^{+21.00}_{-15.00}$  \\
      \hline \hline
    \end{tabular}
  \end{center}
  \caption{Maximum overlaps between Caltech--Cornell hybrid waveforms
    and restricted stationary-phase pN templates using the
    Initial-LIGO noise curve.  The first number in each block is the
    overlap; subsequent numbers are the template parameters that
    achieve this overlap.  Parameter values within the specified
    ranges keep the overlap within 1\% of the maximum by varying that
    parameter, while leaving others fixed.  We restrict the search to
    $0 \leq \eta \leq 1.000$, so the upper error bounds when $\eta\sim
    1.000$ may be artificially small.}
  \label{tab:ThreeParamOverlapDetailInitial}
\end{table*}
% \addtolength{\tabcolsep}{-3.25mm} \renewcommand{\arraystretch}{1}

%
% Advanced LIGO results
%

% \addtolength{\tabcolsep}{3.25mm} \renewcommand{\arraystretch}{1.6}
\begin{table*}
  \begin{tabular}{@{}lcccc@{}}
    \hline \hline
    & $(10+10) \MSun$ & $(20+20) \MSun$ & $(30+30) \MSun$ & 
    $(50+50) \MSun$ \\
    \hline
    $\Overlap{s^{\textrm{NR-CC}} |
      h^{\textrm{SPA}_c^{\textrm{ext}}(2.0)}}$ &
    0.98 & 0.92 & 0.91 & 0.94 \\
    $M/\MSun$ &
    $25.15^{+0.02}_{-0.02}$ &
    $47.73^{+0.12}_{-0.11}$ &
    $54.39^{+0.51}_{-0.43}$ &
    $60.19^{+1.55}_{-1.29}$ \\
    $\eta$ &
    $0.170^{+0.0010}_{-0.0010}$ &
    $0.188^{+0.0010}_{-0.0010}$ &
    $0.335^{+0.0080}_{-0.0070}$ &
    $0.891^{+0.0660}_{-0.0490}$ \\
    $f_{\mathrm{cut}}$  (Hz) &
    $444.77^{+132.00}_{-115.00}$ &
    $267.64^{+48.00}_{-50.00}$ &
    $262.44^{+34.00}_{-36.00}$ &
    $182.41^{+24.00}_{-18.00}$ \\
    \hline
    $\Overlap{s^{\textrm{NR-CC}} |
      h^{\textrm{SPA}_c^{\textrm{ext}}(3.5)}}$ &
    0.97 & 0.92 & 0.92 & 0.96 \\
    $M/\MSun$ &
    $20.27^{+0.02}_{-0.02}$ &
    $38.11^{+0.11}_{-0.09}$ &
    $50.09^{+0.49}_{-0.42}$ &
    $78.10^{+1.89}_{-1.50}$ \\
    $\eta$ &
    $0.245^{+0.0010}_{-0.0010}$ &
    $0.277^{+0.0020}_{-0.0020}$ &
    $0.386^{+0.0130}_{-0.0100}$ &
    $0.494^{+0.0760}_{-0.0330}$ \\
    $f_{\mathrm{cut}}$   (Hz) &
    $355.85^{+97.00}_{-88.00}$ &
    $262.83^{+47.00}_{-48.00}$ &
    $281.34^{+41.00}_{-37.00}$ &
    $186.31^{+30.00}_{-19.00}$ \\
    \hline
    $\Overlap{s^{\textrm{NR-CC}} |
      h^{\textrm{SPA}_c^{\mathcal{Y}}(4)}}$ &
    0.97 & 0.96 & 0.94 & 0.90 \\
    $M/\MSun$ &
    $22.24^{+0.02}_{-0.02}$ &
    $46.57^{+0.11}_{-0.11}$ &
    $72.06^{+0.35}_{-0.35}$ &
    $118.50^{+1.99}_{-1.63}$ \\
    $\eta$ &
    $0.208^{+0.0010}_{-0.0010}$ &
    $0.190^{+0.0010}_{-0.0010}$ &
    $0.177^{+0.0020}_{-0.0030}$ &
    $0.186^{+0.0100}_{-0.0070}$ \\
    $f_{\mathrm{cut}}$  (Hz) &
    $473.49^{+551.00}_{-136.00}$ &
    $353.18^{+73.00}_{-69.00}$ &
    $242.43^{+37.00}_{-36.00}$ &
    $152.16^{+19.00}_{-19.00}$ \\
    \hline \hline
  \end{tabular}
  \caption{Maximum overlaps between Caltech--Cornell hybrid waveforms
    and restricted stationary-phase pN templates using the
    Advanced-LIGO noise curve.  The first number in each block is the
    overlap; subsequent numbers are the template parameters that
    achieve this overlap.  Parameter values within the specified
    ranges keep the overlap within 1\% of the maximum by varying that
    parameter, while leaving others fixed.  We restrict the search to
    $0 \leq \eta \leq 1.000$, so the upper error bounds when $\eta\sim
    1.000$ may be artificially small.}
  \label{tab:ThreeParamOverlapDetail}
\end{table*}
% \addtolength{\tabcolsep}{-3.25mm} \renewcommand{\arraystretch}{1}


A significant feature of
Tables~\ref{tab:ThreeParamOverlapDetailInitial}
and~\ref{tab:ThreeParamOverlapDetail} is the size of the error bars on
the cutoff frequencies.  For $M=20 \MSun$ the cutoff frequency can
vary as much as 128\% above and 28\% below the optimal value while
losing no more than 1\% of overlap. This leads us to consider the
range of possible template parameters which may give high overlaps.
In the next section we consider the reduction in overlap as the
parameters $f_{c}$ and $\eta$ are independently varied from the
optimal value.

\begin{figure}
  % \includegraphics[width=\linewidth]{figures/comparison/ThreeParamOverlapSummary}
  \includegraphics[width=0.5\linewidth]{figures/comparison/ThreeParamOverlapSummaryInitial}
  \includegraphics[width=0.5\linewidth]{figures/comparison/ThreeParamOverlapSummaryAdvanced}
  \caption[ Overlaps between Caltech--Cornell hybrid waveforms and pN waveforms]{
  \label{fig:ThreeParamOverlapSummaries}
    Left: Overlaps between Caltech--Cornell hybrid waveforms,
    scaled to various masses, and restricted stationary-phase pN
    waveforms for Initial-LIGO PSD. Optimization is over $M$ and
    $\eta$, which the cutoff frequency $f_{c}$ is prescribed by the
    weighted average described below.  The mass ratio $\eta$ is
    allowed to range over unphysical values.  The best-fit values
    found for the pseudo-4.0 pN templates are always physical in this
    case.  See Sec.~\ref{sec:UnrestrictedEta}.  Right: The same, for
    the Advanced-LIGO PSD}
\end{figure}%


\subsection{Effect of upper frequency cutoff}
\label{sec:EffectOfUpperFreqCutoff}

As shown in Fig.~\ref{fig:StildesAndInitialPSD} the amplitude of the
NR waveforms drops sharply at around the lightring frequency, which
depends on the total mass of the binary.  The \textit{TaylorF2}
waveforms do not model the late inspiral, merger or ringdown and hence
will continue to evolve as $f^{-7/6}$ at all frequencies, increasingly
deviating from the NR waveform.  This suggests that the upper
frequency cutoff of the \textit{TaylorF2} waveform should be chosen to
be below the frequency at which the two diverge.  However, the effect
of the divergence is mitigated by the PSD.  The denominator of the
overlap, Eq.~\eqref{eq:OverlapDefinition}, depends on
$\InnerProduct{s|s}$ which is a constant, and $\InnerProduct{h|h}$
which would increase without limit if not for the PSD.
Fig.~\ref{fig:FilterIntegrand} shows $|\tilde{h}(f)|^2/S_n(f)$---the integrand of
$\InnerProduct{h|h}$---for the Initial-LIGO
noise curve for an example \textit{TaylorF2} waveform for an
equal-mass $10\,\MSun$ binary.  We see that above about 450 Hz there
is very little contribution to the integrand, and so extending the
cutoff frequency above this will not impact the overlap.


\begin{figure}
  % \includegraphics[width=\linewidth]{figures/comparison/FilterIntegrand}
  \includegraphics[width=0.50\linewidth]{figures/comparison/Integrand}
  \includegraphics[width=0.50\linewidth]{figures/comparison/Errorbars}
  \caption[Effect of cutoff frequency on overlaps]{
  \label{fig:FilterIntegrand}
    Left: Integrand of Eq.~\eqref{eq:InnerProduct} for a
    \textit{TaylorF2}, 3.5 pN waveform with $M=10$ and $\eta=0.25$, at
    a distance of 100\,Mpc, using the Initial-LIGO noise curve.  Note
    that the shape of this curve does not change as we change $M$ and
    $\eta$; only the vertical scale changes.  Right: Overlap between
    Caltech--Cornell waveform scaled to $M=40\,\MSun$ and restricted
    \textit{TaylorF2}, 3.5 pN waveform using the best-match values for
    $M$ and $\eta$, as a function of the cutoff frequency $f_c$, with
    the Initial-LIGO noise curve.  The vertical bars are meant to
    delineate 1\% loss.  Note that the upper bound extends to higher
    frequencies indefinitely.  }
\end{figure}%


The numerator of the overlap, $\InnerProduct{s|h}$, can only increase
as the cutoff frequency is raised, however frequencies above the
lightring where the waveforms have diverged will contribute very
little.  The effect of including higher frequencies on the overlap is
therefore determined by the $\InnerProduct{h|h}$ term in the
denominator.  For systems with ringdown frequencies well above the
peak of the integrand in Fig.~\ref{fig:FilterIntegrand}, this term
will not significantly reduce the overlap.  For example, binaries of
total mass roughly $40\,\MSun$ have ringdown frequencies at roughly
450\,Hz.  Only a small fraction of the SNR comes from higher
frequencies.  Thus, we expect that systems with lower masses should
not suffer great loss in overlap if the cutoff frequency is higher
than ringdown.  However for higher-mass systems the overlap can be
significantly reduced if the upper frequency cutoff is too large.
This is indeed what we find, as shown by a representative example on
the right in Fig.~\ref{fig:FilterIntegrand}.  For this $40\,\MSun$
system, using the Initial-LIGO noise curve, the optimal cutoff
frequency is around 450\,Hz---roughly the ringdown frequency.
Decreasing the cutoff quickly decreases the overlap.  The cutoff may
be increased almost indefinitely, however, with only 0.5\% loss in
overlap.  This, of course, changes when using the Advanced-LIGO noise
curve.  We revisit this issue in Sec.~\ref{sec:Recommendations}.


\subsection{Unrestricted $\eta$}
\label{sec:UnrestrictedEta}

The physical symmetric mass ratio is restricted to the range $0 < \eta
\leq 0.25$, values above this imply complex-valued masses.  However
the pN waveforms are well-behaved for $0 < \eta < 1.0$, and as seen
from Tables~\ref{tab:ThreeParamOverlapDetailInitial}
and~\ref{tab:ThreeParamOverlapDetail}, the highest overlaps are often
obtained at unphysical values of $\eta$.  In
Fig.~\ref{fig:PhysicalEta} we show the effect of limiting the
optimization to physical $\eta$.  At high masses, the limitation
reduces the optimal overlap by up to 12\%.  \textit{TaylorF2}
waveforms with $\eta \leq 1/4$ would not be expected to accurately
model the late-inspiral and merger part of the waveform, as
non-Newtonian effects are increasingly significant in this region.  We
find that allowing unphysical $\eta$ broadens the space of waveforms
covered by the \textit{TaylorF2} approximation sufficiently to capture
more of the late-inspiral and merger.

\begin{figure}
  % \includegraphics[width=\linewidth]{figures/comparison/PhysicalEta}
  \begin{center}
    \includegraphics[width=0.55\linewidth]{figures/comparison/PhysicalAndUnphysicalEta}
  \end{center}
  \caption[Maximum overlaps obtained by allowing $\eta$ to range over unphysical values]{
  \label{fig:PhysicalEta}
    Maximum overlaps obtained by allowing $\eta$ to range over
    unphysical values, compared to those obtained by restricting the
    range of $\eta$.  These overlaps are generated using 3.5 pN
    TaylorF2 templates, searching over values of the total mass and
    mass ratio.  Extending to unphysical values of $\eta$ improves the
    match by up to 11\%.}
\end{figure}%


%%%%%%%%%%%%%%%%%%%%%%%%%%%%%%%%%%%%%%%%%%%%%%%%%%%%%%%%%%%%%%%%%%%%%%
%%%%%%%%%%%%%%%%%%%%%%%%%%%%%%%%%%%%%%%%%%%%%%%%%%%%%%%%%%%%%%%%%%%%%%
\section{Recommendations for improvements}
\label{sec:Recommendations} %

Based on the analysis of the previous sections we propose a series of
adjustments to searches using \textit{TaylorF2} template waveforms to
enhance the efficiency of those searches.  First, as seen in
Fig.~\ref{fig:ThreeParamOverlapSummaries} for Initial LIGO, adding
terms up to 3.5 pN order produces overlaps as large or larger than the
current 2.0 pN templates over most of the mass range, while the
pseudo-4.0 pN templates recommended in Ref.~\cite{Pan2007} produce
slightly larger overlaps at masses near $20\,\MSun$.  Thus, we
recommend pseudo-4.0 pN templates for the low mass range, $M < 35
\MSun$, and 3.5 pN templates for higher masses.  The improvement due
to 3.5 pN templates over 2.0 pN generally holds for Advanced LIGO as
well.  The 3.5 pN templates produce larger overlaps than 2.0 pN
templates above $50\,\MSun$ without a significant loss (within 1\%) at
lower masses.  However, there is a large region for which the
pseudo-4.0 pN term does significantly better.  When using an
Advanced-LIGO noise curve, we recommend 3.5 pN templates generally,
2.0 pN templates in the range $12$--$21\,\MSun$ and pseudo-4.0 pN
templates for masses in the range $21$--$65\,\MSun$.

As a second improvement, we note from Fig.~\ref{fig:PhysicalEta} that
allowing $\eta$ to range over unphysical values significantly improves
matches with 3.5 pN templates above $30\,\MSun$. In preliminary
studies we have found that extending to $\eta \leq 1$ roughly doubles
the size of the template bank, and the advantages must therefore be
weighed against the increase in false alarm rate.

\begin{figure}
  % \includegraphics[width=\linewidth]{figures/comparison/FcRecommendationInit}
  \includegraphics[width=0.5\linewidth]{figures/comparison/FcRecommendationInitial}
  \includegraphics[width=0.5\linewidth]{figures/comparison/FcRecommendationAdvanced}
  \caption[Recommended cutoff frequencies]{
  \label{fig:FcRecomendations}
    Left: Candidate $f_c$ values for 3.5 pN templates with
    Initial LIGO.  The dark gray band contains cutoff frequencies with
    matches within 1\% of the value at which the best overlap was
    obtained.  The light gray band contains frequencies with matches
    within 3\%.  Right: Candidate $f_c$ values for 3.5 pN templates
    with Advanced LIGO.  The dark gray band contains cutoff
    frequencies with matches within 1\% of the value at which the best
    overlap was obtained.  The light gray band contains frequencies
    with matches within 3\%.  Note that the weighted-average cutoff
    extends past the 1\% error bars for $12 < M/\MSun < 40$.  However,
    in that same region, the 3.5 pN templates do poorly overall, and
    we recommend pseudo-4.0 pN templates.  The optimal cutoff
    frequency for pseudo-4.0 pN templates is much closer to the
    weighted-average cutoff in this mass range.}
\end{figure}%

Our third recommendation involves the cutoff frequency used for the
template waveform.  Optimization over the cutoff frequency is too
computationally intensive to be done in searches.  Currently, the
cutoff frequency is typically taken to be the Schwarzschild ISCO
frequency.  To examine the effect of this choice we vary $f_c$ while
keeping the mass and $\eta$ at their optimal values, for each of the
signal masses in our range.  The result of one such variation is shown
in Fig.~\ref{fig:FilterIntegrand} (right).
Figs.~\ref{fig:FcRecomendations} shows the variations for all masses,
highlighting the regions within which the overlap drops by less than
1\% (dark gray) and 3\% (light gray) of the optimal value.  This
figure also shows the ISCO and ERD frequencies, neither of which stays
within the 1\% band for both Initial and Advanced LIGO.  In
particular, the ISCO is a poor choice for both Initial and Advanced
LIGO except at very low masses, where the precise value of the cutoff
is almost irrelevant.

The ISCO is often pointed to---somewhat arbitrarily---as a good
estimate of the breakdown of post-Newtonian
approximations~\cite{Blanchet2006}.  So, for instance, if we were to
match a pN template to a physical waveform, beginning at some point in
the distant past, we might expect them to separate quite badly near
the ISCO.  Of course, for realistic black-hole binaries, the
gravitational waves will only enter the LIGO band late in the
inspiral---just before the ISCO for low-mass systems, or after the
ISCO for high-mass systems.  We can see from
Fig.~\ref{fig:StildesAndInitialPSD} that, for masses below about
$30\,\MSun$, the ISCO is high enough that lower-frequency parts of the
waveform contribute the most to the SNR.  For very high masses,
however, this basically cuts the waveform down to nothing.  In Initial
LIGO, the ISCO is completely buried in seismic noise for masses above
about $100\,\MSun$.  Thus, we must move the cutoff frequency up.  We
cannot push the cutoff far above ringdown, because the physical
waveform simply ceases to exist (see
Fig.~\ref{fig:StildesAndInitialPSD}).  It has been suggested that an
``effective ringdown'' (ERD) frequency $f_{\mathrm{ERD}} \equiv 1.07\,
f_{\mathrm{Ringdown}}$ is a useful upper limit~\cite{Pan2007}.  For
intermediate masses, we would like to interpolate somehow between
these two extremes of ISCO and ERD.  We suggest setting the cutoff
frequency to a weighted average of the two, where the weights are the
contributions to the SNR below the given frequency.  If we assume
coherent phasing between the template and the physical waveform, we
can simply take the amplitudes of the two waveforms.  Also, note that
the restricted SPA approximation for the amplitude is reasonable.
Thus, define
\begin{eqnarray}
  \label{eq:rhoISCO}
  \rho_{\mathrm{ISCO}}^{2} &\equiv & \int_{0}^{f_{\mathrm{ISCO}}}\,
  \frac{f^{-7/3}}{S_{n}(f)}\, d f\ , \\
  \label{eq:rhoERD}
  \rho_{\mathrm{ERD}}^{2} &\equiv & \int_{f_{\mathrm{ISCO}}}^{f_{\mathrm{ERD}}}\,
  \frac{f^{-7/3}}{S_{n}(f)}\, d f\ , \\
  \label{eq:rhoTOT}
  \rho_{\mathrm{tot}}^{2} &\equiv & \int_{0}^{f_{\mathrm{ERD}}}\,
  \frac{f^{-7/3}}{S_{n}(f)}\, d f\ , \\
  \label{eq:fCut}
  f_{\mathrm{cut}} &\equiv & \frac{f_{\mathrm{ISCO}}\ \rho_{\mathrm{ISCO}} +
    f_{\mathrm{ERD}}\ \rho_{\mathrm{ERD}}} {\rho_{\mathrm{tot}}}\ .
\end{eqnarray}
We have already dropped constant factors in the expressions for $\rho$
that will cancel out.

Note that these expressions only depend on the total mass by way of
the limits of integrations---which are very simple, known functions of
the mass---so these integrals could be done just once for a given
noise curve, storing the intermediate values.  When the cutoff needs
to be calculated, the cumulative integral could be evaluated at the
given ISCO and ringdown frequencies.  Hence, this would be a fast way
of calculating the cutoff, with no need to do the integrals each time
the cutoff is needed.

We can test this recommended frequency by comparing it to the optimal
cutoff frequency found by the amoeba search described in
Sec.~\ref{sec:Efficiency}.  For 3.5 pN templates in Initial LIGO, we
find that it is an excellent match to the optimal frequency.
Fig.~\ref{fig:FcRecomendations} shows these two values, along with
dark and light bands showing the regions in which changing $f_{c}$
results in a loss of overlap of 1\% and 3\%, respectively.  Of course,
the same figure shows that using the ERD recommendation would stay
within the 1\% error bounds.  Nonetheless, the close match between
this recommendation and the true optimum suggests that it is sound.
Thus, our final recommendation is to use the weighted-average
frequency cutoff throughout the entire mass range.  While our analysis
has been restricted to equal-mass systems, the cutoff frequency we
have defined here could be applied to unequal-mass systems as well.
It will be interesting to see how this cutoff fares in those
situations.

Similar results hold for Advanced LIGO, when using our recommended
template for each mass.  That is, in regions where 3.5 pN templates do
poorly (see Fig.~\ref{fig:ThreeParamOverlapSummaries}), the weighted
average is a poor predictor of the optimal cutoff frequency using
those templates, as shown in Fig.~\ref{fig:FcRecomendations}.
However, in those same regions---where pseudo-4.0 pN templates do
well---the weighted average is a good predictor of the optimal cutoff
frequency for 4.0 pN templates.  Thus, again, we recommend using the
weighted-average frequency cutoff throughout the entire mass range
with Advanced LIGO.

By prescribing a cutoff frequency, the search does not need to extend
over that parameter.  Similarly, by prescribing a post-Newtonian
order, we need use only one template for a given total mass.  On the
other hand, if these recommendations decrease the overlap found by too
much when using them compared to the overlap found by an unconstrained
search, it may be better to search the larger parameter space.  We can
evaluate the loss in overlap by comparing the results found using our
recommendations to the results found when searching over the set of
all three template families, and all masses, mass ratios, and cutoff
frequencies.  We have determined that this loss in overlap when using
our recommendations is always less than 0.0025 for Initial LIGO, and
less than 0.007 for Advanced LIGO.

\iffalse
\begin{figure}
  \begin{center}
    \includegraphics[width=0.55\linewidth]{figures/comparison/WeightedAverageLoss}
  \end{center}
  \caption{Loss in overlap when using our recommendations, compared to
    results searching over all template families, masses, mass ratios,
    and cutoff frequencies.  Our recommendations prescribe the
    template family for a given total mass and the cutoff frequency to
    be used.  In that case, the search is performed for the optimal
    mass and mass ratio of the template.  For Initial LIGO, the loss
    in overlap when using our recommendations is always less than
    0.0025; for Advanced LIGO the loss is always less than 0.007.}
  \label{fig:WeightedAverageLoss}
\end{figure}%
\fi

\section{Conclusions}
\label{sec:Conclusions} %

We have compared high-accuracy NR waveforms for equal-mass binary
black holes from the Caltech--Cornell group to stationary phase
post-Newtonian waveforms.  We examined a number of factors that
influence the matches between the two, with the goal of optimizing the
matches and hence improving the efficiency of templated searches in
Initial and Advanced LIGO.  We first considered the effect of the
post-Newtonian order to which the phase evolution is taken, and found
that adding terms up to 3.5 pN or pseudo-4.0 pN to the currently-used
2.0 pN templates significantly improves the matches over a large range
of masses, as shown in Fig.~\ref{fig:ThreeParamOverlapSummaries}.  We
then studied the effect of varying the upper cutoff frequency of the
templates.  The frequency that achieves the optimal match is a
function of mass, and we find this function is well-approximated by an
average between ISCO and ERD, weighted by contribution to the SNR, as
shown in Fig.~\ref{fig:FcRecomendations}.  Finally, we allow the
symmetric mass ratio $\eta$ to range over unphysical values up to
$1.0$, and find that this dramatically improves matches, as shown in
Fig.~\ref{fig:PhysicalEta}.  Based on the results we recommend
adjusting the searches using \textit{TaylorF2} template waveforms by
going up to 3.5 pN or 4.0 pN over most of the mass range, integrating
up to our recommended cutoff, and allowing allowing $\eta$ to extend
up to 1.  For Initial LIGO, the overlaps obtained using these
parameters is always within 0.0025 of overlaps achievable by
optimizing over all three parameters.

In future work we plan to extend this analysis to unequal-mass and
spinning black-hole systems.  We have found that allowing unphysical
values of $\eta$ roughly doubles the size of the template bank, and we
also plan to study the impact of this on the false alarm rate.




%%%%%%%%%%%%%%%%%%%%%%%%%%%%%%%%%%%%%%%%%%%%%%%%%%%%%%%%%%%%%%%%%%%%%%
%%%%%%%%%%%%%%%%%%%%%%%%%%%%%%%%%%%%%%%%%%%%%%%%%%%%%%%%%%%%%%%%%%%%%%
%%%%%%%%%%%%%%%%%%%%%%%%%%%%%%%%%%%%%%%%%%%%%%%%%%%%%%%%%%%%%%%%%%%%%%
\newpage
  We thank Luisa Buchman, Yanbei Chen, Curt Cutler, Lisa Goggin, Larry
  Kidder, Luis Lehner, Lee Lindblom, Harald Pfeiffer, Mark Scheel and
  Kip Thorne for useful discussions.  This work was supported in part
  by grants from the Sherman Fairchild Foundation and by NSF grants
  PHY-0601459, PHY-0652995, and DMS-0553302 to Caltech.



\Chapter{The First NINJA Project}
\label{ch:ninja1}
\newcommand\T{\rule{0pt}{2.6ex}}
\newcommand\B{\rule[-1.2ex]{0pt}{0pt}}
\newcommand\TT{\rule{0pt}{4.2ex}}
\newcommand\BB{\rule[-2.4ex]{0pt}{0pt}}
\newcommand\TTT{\rule{0pt}{3.8ex}}


The workhorse template of the LSC-Virgo search pipeline is based on the
stationary-phase approximation to the Fourier transform of the
non-spinning post-Newtonian inspiral~\cite{Droz:1999qx,Allen:2005fk}.
This waveform (referred to as SPA or TaylorF2) has been used in the
search for binary neutron
stars~\cite{Abbott:2003pj,Abbott:2005pe,Abbott:2007xi,Abbott:2009tt}, 
sub-solar mass black holes~\cite{Abbott:2005pf,Abbott:2007xi,Abbott:2009tt}
and stellar mass black holes~\cite{Abbott:2009tt}. The TaylorF2 waveform
is parametrised by the binary's component masses $m_1$ and $m_2$ (or
equivalently the total mass $M = m_1 + m_2$ and the symmetric mass ratio
$\eta = m_1 m_2 / M^2$) and an upper frequency cutoff $f_\mathrm{c}$.
Amplitude evolution is modelled to leading order and phase evolution is
modelled to a specified post-Newtonian order. In this section we investigate
the performance of TaylorF2-based searches on the three simulated LIGO
detectors. Results which include the simulated Virgo detector are described in
the next section.  Several analyses were performed 
which test the ability of TaylorF2 waveforms to detect numerical relativity
signals. The analyses differed in the way the TaylorF2 waveforms or the
template bank were constructed.  The results of these searches are summarised
in Table~\ref{tab:inspiral_results}, each column giving the results from a
different search with a summary of the chosen parameters.  We first describe
the parameters varied between these analyses and then present a more detailed
discussion of the results.

All TaylorF2 NINJA analyses used restricted templates (i.e.~the amplitude is
calculated to leading order), however the phase was calculated to various
different post-Newtonian orders~\cite{Blanchet:2002av}. Phases were computed to
either two~\cite{Blanchet:1996pi,Blanchet:1995ez} or three point five
post-Newtonian order~\cite{Blanchet:2001ax,PhysRevD.71.129902,Blanchet:2004ek}
since these are, respectively, the order currently used in LSC-Virgo
searches~\cite{Abbott:2009tt} and the highest order at which post-Newtonian
corrections are known. After choosing a post-Newtonian order, one chooses a
region of mass-parameter space to cover with the template bank.
Figure~\ref{f:ninjaBanks} shows the boundaries of the template banks used in
the analyses. One search used the range used by the LSC-Virgo ``low-mass''
search \cite{Abbott:2009tt} ($m_1,m_2 \ge  1 M_\odot, M \le 35 M_{\odot}$) and
all other searches used templates with total masses in the range $20 M_\odot
\le M \le 90 M_\odot$.  These boundaries were chosen since there were no
signals in the NINJA data with mass smaller than $36 M_\odot$ and there is
little, if any, inspiral power in the sensitive band of the NINJA data for
signals with $M \gtrsim 100 M_\odot$.  The standard LSC-Virgo template bank
generation code~\cite{Babak:2006ty} restricts template generation to signals
with $\eta \le 0.25$, since it is not possible to invert $M$ and $\eta$ to
obtain real-valued component masses for $\eta > 0.25$. All but one of the
searches enforced this constraint, with the $0.03 \le \eta \le 0.25$ for the
low-mass CBC search and $0.1 \le \eta \le 0.25$ for the other
``physical-$\eta$'' searches. It is, however, possible to generate TaylorF2
waveforms with ``unphysical'' values of $\eta > 0.25$.  In two separate studies
using Goddard and Pretorius waveforms~\cite{Pan:2007nw}, and Caltech-Cornell
waveforms~\cite{Boyle:2009dg} it was observed that match between numerical
signals and TaylorF2 templates could be increased by relaxing the condition
$\eta \le 0.25$. One NINJA contribution uses a template bank with $0.1 \le \eta
\le 1.0$ to explore this.

Finally, it is necessary to specify a cutoff frequency at which to terminate the
TaylorF2 waveform. In the LSC-Virgo analyses, this is chosen to be the
innermost stable circular orbit (ISCO) frequency for a test mass in a
Schwarzschild spacetime 
%
\begin{equation}
\label{f_ISCO}
f_\mathrm{ISCO} = \frac{c^3}{6\sqrt{6}\pi GM}.
\end{equation}
%
This cutoff was chosen as the point beyond which the TaylorF2 waveforms
diverge significantly from the true evolution of the
binary~\cite{Blanchet:2002av}.  More recently, comparisons with numerical
relativity waveforms have shown that extending the waveforms up to higher
frequencies improves the sensitivity of TaylorF2 templates to higher mass
signals~\cite{Pan:2007nw,Boyle:2009dg}. The NINJA TaylorF2 analyses use
templates terminated at the ISCO frequency and two additional cut-off
frequencies: the effective ringdown (ERD) frequency and a weighted
ringdown ending (WRD) frequency. The ERD frequency was obtained by comparing
post-Newtonian models to the Pretorius and Goddard
waveforms~\cite{Pan:2007nw}. The ERD almost coincides with the fundamental
quasi-normal mode frequency of the black hole formed by the merger of an
equal-mass non-spinning black-hole binary. The weighted ringdown ending (WRD)
frequency lies between ISCO and ERD, and was obtained by comparing
TaylorF2 waveforms to the Caltech-Cornell numerical
signals~\cite{Boyle:2009dg}.

\begin{table}
\begin{tabular}{| l || c | c | c | c | c | c | c |}
\hline
\bf{Analysis} \T \B & $(1)$ & $(2)$ & $(3)$ & $(4)$ & $(5)$ & $(6)$ \\ \hline
\bf{Freq. Cutoff} \T \B & ISCO & ISCO & ERD & ERD &  WRD & WRD  \\ 
\hline
\bf{PN Order} & 2 PN & 2 PN & 2 PN & 3.5 PN &  3.5 PN& 3.5 PN  \\
\hline
%\parbox{2.8cm}{
%\bf{Component\\ Mass $M_{\odot}$}} & 1--34 & 10--60 & 10--60 & 10--60 & 10--60 & 10--60 \\ 
%\hline
\bf{Total Mass $M_{\odot}$} \T \B & 2--35 & 20--90 & 20--90 & 20--90 & 20--90 & 20--90  \\ 
\hline
\bf{$\eta$ range} \T \B & 0.03--0.25 & 0.10--0.25 & 0.10--0.25 & 0.10--0.25 & 0.10--0.25 & 0.10--1  \\ 
\hline
\parbox{2.3cm}{
\bf{Found Single\\ (H1, H2, L1)}}\TT \BB  & 69, 66, 75 & 72, 43, 66 & 83, 51, 81 & 91, 56, 87 & 90, 55, 88 & 90, 56, 88 \\ 
\hline
\parbox{2.3cm}{
\bf{Found \\Coincidence }} \TT \BB & 49 & 59 & 79 & 82 &  82 & 84 \\ 
\hline
\parbox{2.5cm}{
\bf{Found Second\\Coincidence}} \TT \BB & 48 & 59 & 77 & 81 &  81 & 81 \\ 
\hline 
\end{tabular}
\caption{{\bf Results of inspiral searches using TaylorF2 templates.}  There were 126
injections performed into the data.  The table above shows the number of
injections which were recovered from the three simulated LIGO detectors (H1, H2 and L1) using various different waveform families,
termination frequencies $f_\mathrm{ISCO}$, $f_\mathrm{ERD}$ and $f_\mathrm{WRD}$ 
(as described in the text), and post-Newtonian orders.} 
\label {tab:inspiral_results}
\end{table}


\begin{figure}
  \begin{center}
  \includegraphics[width=0.8\textwidth]{figures/ninja1/ninja_banks}
  \end{center}
  \caption{{\bf Boundaries of the template banks used in inspiral searches} as a
  function of total mass $M$ and symmetric mass ratio $\eta$. The crosses show
  the location of the injections in the NINJA data set. The numbers in the
  legend correspond to entries in table~\ref{tab:inspiral_results}. Bank 6
  extends in a rectangle up to $\eta = 1.00$, as indicated by the arrows. NP
  is the bank used in the Neyman-Pearson analysis described in 
  Section~\ref{sssec:neyman}.}
  \label{f:ninjaBanks}
\end{figure}

The results of these searches are reported in
Table~\ref{tab:inspiral_results}.  The principal result is the number of
injected signals detected by the search.  For simplicity, we define a
detected signal as one for which there is a candidate gravitational-wave
signal observed within $50$~ms of the coalescence time of the injection,
determined by the maximum gravitational-wave strain of the injected
signal.  We do not impose any additional threshold on the measured SNR or
effective SNR of the candidate.  For a single detector, this will lead
to a small number of falsely identified injections, but for coincidence
results the false alarm rate is so low that we can be confident that the
triggers are associated with the injection. We now describe these
results in the order that they appear in
Table~\ref{tab:inspiral_results}.

Search~$(1)$ used second order post-Newtonian templates terminated at
$f_\mathrm{ISCO}$ with a maximum mass of $M \le 35 M_{\odot}$.  Despite the
fact that no NINJA injections had a mass within the range of this search, a
significant number of signals were still recovered in coincidence both before
and after signal consistency tests.  Although the templates are not a
particularly good match to the injected signals, they are still similar enough
to produce triggers at the time of the injections.  Search~$(2)$ changed the
boundary of the template bank to $20 M_\odot \le M \le 90 M_{\odot}$, but left
all other parameters unchanged.  The number of detected signals increases
significantly as more signals now lie within the mass range searched. 

Search~$(3)$ extended the upper cutoff frequency of the waveforms to
$f_\mathrm{ERD}$. The number of signals detected increased from 59 to 77, as
expected since these waveforms can detect some of the power contained in the
late inspiral or early merger part of the
signal~\cite{Pan:2007nw,Boyle:2009dg}. Search~$(4)$ extends the post-Newtonian
order to 3.5~PN, slightly increasing the number of detected signals to 81.
With the limited number of simulations performed in this first NINJA analysis,
it is difficult to draw a strong conclusion, although there does seem to be
evidence that the higher post-Newtonian order waveforms perform better,
consistent with previous comparisons of post-Newtonian and numerical
relativity waveforms
\cite{Pan:2007nw,Baker:2006ha,Hannam:2007ik,Boyle:2008ge,Boyle:2009dg}.
Search~$(5)$ uses an upper-frequency cutoff of $f_\mathrm{WRD}$ for the
templates. The number of injections found in coincidence for this search is
the same as the search using $3.5$ order templates with a cutoff of
$f_\mathrm{ERD}$, although there are slight differences in the number of found
injections at the single detector level.

Search~$(6)$ extends the template bank of search~$(5)$ to unphysical values of
the symmetric mass ratio. Extending the bank to $\eta\le 1$ increases the
number of templates in the bank by a factor of $\sim 2$. The original and
modified template banks are shown in Figure~\ref{f:templateBanks}. With the
extended template bank the number of injections found in coincidence remains
the same as search~$(5)$ after signal-based vetoes are applied.  However, many
of the injections are recovered at a higher SNR, particular the low-mass
signals, as shown in Figure~\ref{f:templateBanks}.  Some injections show a
reduction in SNR; more work is needed to understand this effect.

\begin{figure}
  \includegraphics[width=0.50\textwidth]{figures/ninja1/BankBoth}
  \includegraphics[width=0.50\textwidth]{figures/ninja1/HanfordSNR}
  \caption{{\bf Results from the extended template bank.} 
  {\bf Left:} The template bank generated by the LSC-Virgo
  search pipeline (circles) and the bank obtained by extending to
  $\eta \leq 1.00$ (crosses). In this figure the bank is parametrised
  by $\tau_0$ and $\tau_3$ which are related to the binary masses by
  $\tau_0 = 5M/(256\eta v_0^8)$ and $\tau_3 = \pi M/(8\eta v_0^5)$,
  where $v_0 = (\pi M f_0)^{1/3}$ is a fiducial velocity parameter
  corresponding to a fiducial frequency $f_0 = 40.0 Hz$.
  {\bf Right:} The
  signal-to-noise (SNR) ratio at which NINJA injections were recovered using
  the $\eta \le 0.25$ bank (squares) and the $\eta \le 1$ extended bank
  (circles) in the Hanford detectors, given by $\rho =
  (\rho_\mathrm{H1}^2 + \rho_\mathrm{H2}^2)^{1/2}$. The SNR of the signal
  recovered using the extended bank shows with significant ($> 10\%$) 
  increases over the standard bank for certain injections.}
  \label{f:templateBanks}
\end{figure}

Finally, we note that the majority of signals passed the $\chi^2$
signal-based veto with the thresholds used in the LSC-Virgo pipeline.  The
last two lines of Table~\ref{tab:inspiral_results} show the number of
recovered signals before and after these signal-based vetoes are
performed. The post-Newtonian templates and numerical relativity signals are
similar enough that virtually all of the injected signals survive the signal
based vetoes. 

To illustrate the results of these analyses in more detail, 
Figure~\ref{fig:3_5pn_found_missed} shows which signals were detected and which were
missed by the 3.5 order post-Newtonian TaylorF2 templates terminated at
$f_\mathrm{ERD}$, as a function of injected
total mass and effective distance of the binary (a measure of the
amplitude of the signal in the detector), defined by~\cite{Allen:2005fk}
\begin{equation}
D_\mathrm{eff} = d \left/ \sqrt{F_+^2 (1 + \cos^2 \iota)^2/4 + F_\times^2 \cos^2 \iota}\right.,
\label{eq:effdist}
\end{equation}
where $d$ is the luminosity distance of the binary.

One signal, with total mass of $110 M_{\odot}$ and effective distance $\sim
200$ Mpc, was missed while others with similar parameters were found.  This
signal was one of the Princeton waveforms (labelled \verb|PU-e0.5| in
Figure \ref{fig:NR-Reh22}) for which the maximum amplitude occurs at the start
of the waveform rather than at coalescence\footnote{That the maximum 
occurs at the start of the waveform is in part an ``artifact'' of the 
double-time integration from the Newman-Penrose scalar $\psi_4$ to the 
metric perturbation $h$, and in part a coordinate artifact.
The two integration constants were chosen to remove a 
constant and linear-in-time piece for $h$, however, there is still 
a non-negligible quadratic component; we {\em suspect} this is purely gauge, 
though lacking a better understanding of this it was not removed from the 
waveform.}, rendering our simple coincidence
test invalid.  The injection finding algorithm compares the peak time to the
trigger time and, even though triggers are found at the time of the simulation,
there are no triggers within the $50$~ms window used to locate detected
signals.

\begin{figure}
\begin{center}
  \includegraphics[width=0.49\textwidth]{figures/ninja1/spa_erd_3_5pn_found_missed_mchirp}
  \includegraphics[width=0.49\textwidth]{figures/ninja1/spa_erd_3_5pn_found_missed_mchirp_l}
\end{center}
\caption{\textbf{Found and missed injections using TaylorF2 templates terminated at ERD}, plotted as a function
of the injected effective distance in Hanford (left) and Livingston (right) and the total mass of the injection. Since the LIGO Observatories are not exactly aligned, the effective distance of a signal can differ, depending on the sky location of the signal.
The vertical bars mark the limits of the template bank used in the search.  For
the lower masses, we see that the majority of the closer injections 
are found in coincidence in all three of
the detectors.  There is then a band of injections which are found only
in two detectors -- H1 and L1 and not the less sensitive H2 detector.
For higher masses, the results are less meaningful as the template bank
was only taken to a total mass of $90 M_{\odot}$.}
\label{fig:3_5pn_found_missed}
\end{figure}

Figure~\ref{fig:3_5pn_params} shows the accuracy with which the total mass and
coalescence time of the binary are recovered when using the 3.5 post-Newtonian
order Taylor F2 templates. The total mass fraction difference is computed as
$(M_\mathrm{injected} - M_\mathrm{detected})/ M_\mathrm{injected}$. For lower
mass signals, the end time is recovered reasonably accurately, with accuracy
decreasing for the high mass systems. The total mass recovery is poor for the
majority of signals, with good parameter estimation for only a few of the
lowest mass simulations.

\begin{figure}
    \includegraphics[width=0.50\textwidth]{figures/ninja1/spa_erd_3_5pn_mass_estimate}
    \includegraphics[width=0.50\textwidth]{figures/ninja1/spa_erd_3_5pn_time_estimate_vs_mt}
\caption{\textbf{Parameter accuracy using TaylorF2 templates terminated at
ERD}.\textbf{Left:} Accuracy with which the total mass is recovered. The
template bank covers the region $20 M_\odot \le M \le 90 M_\odot$, hence
the mass of injections with $M > 90 M_\odot$ are always underestimated.
Even within the region covered by the bank, the TaylorF2 templates
systematically underestimate the mass of the injected signals and the total
mass is recovered accurately only for a few injections.  The vast majority of
recoverd signals have an error of $40\%$ or greater. \textbf{Right:} Accuracy
of determining the coalescence time of the injections.  The end time is not
recovered accurately, the timing error can become as large as $50
\mathrm{ms}$, the limits of the injection window.  }
\label{fig:3_5pn_params}
\end{figure}








\Chapter{Waveform Analysis for the Second NINJA project}
\label{ch:ninja2}
The NINJA-1 project was a huge success in bringing the numerical
relativity and gravitational-wave astronomy communities together.  The
project also resulted in several intriguing qualitative results.
However, it only began the process of testing detection and parameter
estimation pipelines against realistic signals.  The follow-up
project, NINJA-2, is ongoing as of the time of writing.  NINJA-2 aims
to remove some of the shortcomings of NINJA-1 and allow quantitative
studies of the behaviors of pipelines in varying regions of signal
parameter space.  Specifically, NINJA-2 addresses issues with both the
waveform submissions and the noise used to construct the data sets.

This chapter describes the contributed waveforms, the studies that
have been done to verify them, and the construction of the first round
of data sets.  The next chapter will present preliminary results from
running the CBC pipelines on the longest and most carefully
constructed of these data sets.

\section{Contributed waveforms}

NINJA-1 had an open policy towards waveforms submission in order to
encourage wide participation.  This meant there were no requirements
on either waveform quality or length.  The lack of quality
requirements allowed for the possibility of unphysical features in the
waveforms.  There were also no requirements to perform the kind of
convergence testing reported in section~\ref{sec:PNNRHybridWaveform},
although such validation is typically done by numerical
relativistists.  The loose requirements limited the conclusions that
could be drawn, for example it makes it difficult to say whether an
injection was missed due to the parameters of the signal or an
unintended feature of the waveform.

The lack of length requirement limited the available mass range to $M
< 36 \msun$ for reasons that can be seen in
figure~\ref{fig:StildesAndInitialPSD}.   Had the waveform in that plot
not had a post-Newtonian component, the NR component to the right of
the triangles would have had to be placed below 40 Hz in order to
prevent turning on in-band.  This mass range limited the tests that
could be done, for example it entirely excludes the standard CBC
low-mass pipeline.

To address these issues NINJA-2 specifies the following minimal
requirements~\cite{ninja2-wiki}.  The raw numerical simulation should
include at least five orbits of usable data before merger (i.e., not
counting bursts of junk radiation or other significant noise).  Given
the computation cost of extending the NR waveforms, we instead require
stitching to a post-Newtonian inspiral approximant, which should be
performed at a GW frequency of $M\omega \leq 0.075$, where $M\omega$ is
the frequency of the $(l = 2, m = \pm 2)$ harmonic. The full waveform
should be long enough to be entirely within the sensitivity bands of
LIGO and Virgo down to $10 \msun$ with a lower cutoff frequency of 10
Hz, which corresponds to a starting GW frequency of $M\omega = 0.003$.
The numerical-waveform (before any hybridization) amplitude should be
accurate to within 5\%, and the phase (as a function of GW frequency)
should have an accumulated uncertainty over the entire inspiral,
merger and ringdown, of no more than 0.5 radian. The PN approximants
used for hybridization should ideally use the highest PN orders
available, both in phase and amplitude.  

These minimal accuracy requirements are motivated by the results of
the Samurai project~\cite{Hannam:2009hh}, and studies performed in
preparation for the NR-AR collaboration project~\cite{ninja-wiki}.
The question of how many NR cycles are needed in order to produce a
robust waveform is an area of current research~\cite{MacDonald:2011}. 

The NINJA-2 project encourages although does not require the addition
of higher-order modes.  We chose to restrict attention to non-spinning
waveforms and waveforms with spins aligned or anti-aligned with the
orbital angular momentum.  There are sufficient open questions
regarding these restricted cases to make this analysis interesting,
without adding the additional complications on both the NR and data
analysis sides associated with precession. 

A total of 60 waveforms from 8 groups were contributed, these are
summarized in tables ~\ref{tab:ninja2_bam}, \ref{tab:ninja2_fau},
\ref{tab:ninja2_gatech}, \ref{tab:ninja2_lean},
\ref{tab:ninja2_llama}, \ref{tab:ninja2_rit}, \ref{tab:ninja2_spec},
\ref{tab:ninja2_uiuc} and a map of the parameter values is shown in
figure~\ref{f:ninja2_param_map}.

\begin{table}
\begin{center}
\begin{tabular}{|l|r|r|r|l|c|}
\hline
Run & $q$ & Spin1${}_z$ & Spin2${}_z$ & pN Approx. & Refs \\
\hline
BAM\_D10spp85\_80.T4.hyb.n2 & 1 & 0.85 & 0.85 & TaylorT4 & \cite{Hannam:2007wf,Brugmann:2008zz} \\
BAM\_D10spp85\_80.T1.hyb.n2 & 1 & 0.85 & 0.85 & TaylorT1 & \cite{Hannam:2007wf,Brugmann:2008zz} \\
BAM\_D125smm50Nep\_80.T1.hyb.n2 & 1 & -0.50 & -0.50 & TaylorT1 & \cite{Hannam:2007wf,Brugmann:2008zz} \\
BAM\_D125smm50Nep\_80.T4.hyb.n2 & 1 & -0.50 & -0.50 & TaylorT4 & \cite{Hannam:2007wf,Brugmann:2008zz} \\
BAM\_D13smm75Nep\_96.T4.hyb.n2 & 1 & -0.75 & -0.75 & TaylorT4 & \cite{Hannam:2007wf,Brugmann:2008zz} \\
BAM\_D13smm75Nep\_96.T1.hyb.n2 & 1 & -0.75 & -0.75 & TaylorT1 & \cite{Hannam:2007wf,Brugmann:2008zz} \\
BAM\_D13smm85Nep\_88.T4.hyb.n2 & 1 & -0.85 & -0.85 & TaylorT4 & \cite{Hannam:2007wf,Brugmann:2008zz} \\
BAM\_D13smm85Nep\_88.T1.hyb.n2 & 1 & -0.85 & -0.85 & TaylorT1 & \cite{Hannam:2007wf,Brugmann:2008zz} \\
BAM\_D11spp50\_96.T4.hyb.n2 & 1 & 0.50 & 0.50 & TaylorT4 & \cite{Hannam:2007wf,Brugmann:2008zz} \\
BAM\_D11spp50\_96.T1.hyb.n2 & 1 & 0.50 & 0.50 & TaylorT1 & \cite{Hannam:2007wf,Brugmann:2008zz} \\
BAM\_D10spp75\_80.T1.hyb.n2 & 1 & 0.75 & 0.75 & TaylorT1 & \cite{Hannam:2007wf,Brugmann:2008zz} \\
BAM\_D10spp75\_80.T4.hyb.n2 & 1 & 0.75 & 0.75 & TaylorT4 & \cite{Hannam:2007wf,Brugmann:2008zz} \\
BAM\_D12smm25Nep\_80.T4.hyb.n2 & 1 & -0.25 & -0.25 & TaylorT4 & \cite{Hannam:2007wf,Brugmann:2008zz} \\
BAM\_D12smm25Nep\_80.T1.hyb.n2 & 1 & -0.25 & -0.25 & TaylorT1 & \cite{Hannam:2007wf,Brugmann:2008zz} \\
BAM\_EP\_um4\_D10-n96.T4.hyb.n2 & 4 & 0.00 & 0.00 & TaylorT4 & \cite{Hannam:2007wf,Brugmann:2008zz} \\
BAM\_EP\_um4\_D10-n96.T1.hyb.n2 & 4 & 0.00 & 0.00 & TaylorT1 & \cite{Hannam:2007wf,Brugmann:2008zz} \\
BAM\_um3\_88.T4.hyb.n2 & 3 & 0.00 & 0.00 & TaylorT4 & \cite{Hannam:2007wf,Brugmann:2008zz} \\
BAM\_um3\_88.T1.hyb.n2 & 3 & 0.00 & 0.00 & TaylorT1 & \cite{Hannam:2007wf,Brugmann:2008zz} \\
BAM\_um2\_88.T1.hyb.n2 & 2 & 0.00 & 0.00 & TaylorT1 & \cite{Hannam:2007wf,Brugmann:2008zz} \\
BAM\_um2\_88.T4.hyb.n2 & 2 & 0.00 & 0.00 & TaylorT4 & \cite{Hannam:2007wf,Brugmann:2008zz} \\
BAM\_R6\_PN\_80.T1.hyb.n2 & 1 & 0.00 & 0.00 & TaylorT1 & \cite{Hannam:2007wf,Brugmann:2008zz} \\
BAM\_R6\_PN\_80.T4.hyb.n2 & 1 & 0.00 & 0.00 & TaylorT4 & \cite{Hannam:2007wf,Brugmann:2008zz} \\
BAM\_D12spp25\_96.T4.hyb.n2 & 1 & 0.25 & 0.25 & TaylorT4 & \cite{Hannam:2007wf,Brugmann:2008zz} \\
BAM\_D12spp25\_96.T1.hyb.n2 & 1 & 0.25 & 0.25 & TaylorT1 & \cite{Hannam:2007wf,Brugmann:2008zz} \\
BAM\_q2a0a025\_T\_96\_344.T1.hyb.n2.bbh & 2 & 0.25 & 0.00 & {} & \cite{,Brugmann:2008zz} \\
BAM\_q2a0a025\_T\_96\_344.T4.hyb.n2.bbh & 2 & 0.25 & 0.00 & {} & \cite{,Brugmann:2008zz} \\
\hline
\end{tabular}
\end{center}
\caption[BAM submissions to NINJA-2]{
\label{tab:ninja2_bam}
BAM submissions to NINJA-2}
\end{table}

\begin{table}
\begin{center}
\begin{tabular}{|l|r|r|r|l|c|}
\hline
Run & $q$ & Spin1${}_z$ & Spin2${}_z$ & pN Approx. & Refs \\
\hline
BAM\_hybrid\_om0.025etmq3S0.4- & 3 & 0.40 & 0.60 & TaylorT4 & \cite{none,???} \\
0\_0\_S0.6\_0\_0\_72 &  &  &  &  &  \\
\hline
\end{tabular}
\end{center}
\caption[FAU submissions to NINJA-2]{
\label{tab:ninja2_fau}
FAU submissions to NINJA-2}
\end{table}

\begin{table}
\begin{center}
\begin{tabular}{|l|r|r|r|l|c|}
\hline
Run & $q$ & Spin1${}_z$ & Spin2${}_z$ & pN Approx. & Refs \\
\hline
MayaKranc\_D12\_a0.00\_m129\_nj & 1 & 0.00 & 0.00 & TaylorT4 & \cite{,} \\
MayaKranc\_D10\_a0.90\_m129\_nj & 1 & 0.90 & 0.90 & TaylorT4 & \cite{,} \\
MayaKranc\_D10\_a0.20\_m77\_nj & 1 & 0.20 & 0.20 & TaylorT4 & \cite{,} \\
MayaKranc\_D10\_a0.60\_m77\_nj & 1 & 0.60 & 0.60 & TaylorT4 & \cite{,} \\
MayaKranc\_D12\_a0.60\_m103\_nj & 1 & 0.60 & 0.60 & TaylorT4 & \cite{,} \\
MayaKranc\_Sp02py0935th90\_gr & 1 & 0.80 & 0.00 & TaylorT4 & \cite{,} \\
MayaKranc\_D12\_a0.80\_m103\_nj & 1 & 0.80 & 0.80 & TaylorT4 & \cite{,} \\
MayaKranc\_D12\_a0.00\_q2\_m90\_nj & 2 & 0.00 & 0.00 & TaylorT4 & \cite{,} \\
MayaKranc\_D11\_a0.20\_q2\_m90\_nj & 2 & 0.02 & 0.09 & TaylorT4 & \cite{,} \\
MayaKranc\_D10\_a0.40\_m90\_nj & 1 & 0.40 & 0.40 & TaylorT4 & \cite{,} \\
MayaKranc\_D10\_a0.80\_m90\_nj & 1 & 0.80 & 0.80 & TaylorT4 & \cite{,} \\
MayaKranc\_D12\_a0.40\_m103\_nj & 1 & 0.40 & 0.40 & TaylorT4 & \cite{,} \\
MayaKranc\_D12\_a0.20\_m103\_nj & 1 & 0.20 & 0.20 & TaylorT4 & \cite{,} \\
\hline
\end{tabular}
\end{center}
\caption[GATech submissions to NINJA-2]{
\label{tab:ninja2_gatech}
GATech submissions to NINJA-2}
\end{table}

\begin{table}
\begin{center}
\begin{tabular}{|l|r|r|r|l|c|}
\hline
Run & $q$ & Spin1${}_z$ & Spin2${}_z$ & pN Approx. & Refs \\
\hline
dq4 & 4 & 0.00 & 0.00 & TaylorT1 & \cite{,Sperhake:2006cy} \\
\hline
\end{tabular}
\end{center}
\caption[LEAN submissions to NINJA-2]{
\label{tab:ninja2_lean}
LEAN submissions to NINJA-2}
\end{table}

\begin{table}
\begin{center}
\begin{tabular}{|l|r|r|r|l|c|}
\hline
Run & $q$ & Spin1${}_z$ & Spin2${}_z$ & pN Approx. & Refs \\
\hline
Llama\_d550-h64-Hybrid & 1 & 0.00 & 0.00 & 3.5pNTaylorF2 & \cite{Reisswig:2009rx,Reisswig:2009rx} \\
Llama\_d4d4-q1--D10-h64-r250.T4.hybrid & 1 & -0.40 & -0.40 & TaylorT4 & \cite{Pollney:2010hs,Pollney:2009yz,} \\
Llama\_d4d4-q1--D10-h64-r250.T1.hybrid & 1 & -0.40 & -0.40 & TaylorT1 & \cite{Pollney:2010hs,Pollney:2009yz,} \\
Llama\_u4u4-q1--D8-h64-r250.T1.hybrid & 1 & 0.40 & 0.40 & TaylorT1 & \cite{Pollney:2010hs,Pollney:2009yz,} \\
Llama\_u4u4-q1--D8-h64-r250.T4.hybrid & 1 & 0.40 & 0.40 & TaylorT4 & \cite{Pollney:2010hs,Pollney:2009yz,} \\
Llama\_d5q2-h016-Hybrid & 2 & 0.00 & 0.00 & 3.5pNTaylorF2 & \cite{,Reisswig:2009rx} \\
Llama\_u2u2-q1--D8-h64-r250.T1.hybrid & 1 & 0.20 & 0.20 & TaylorT1 & \cite{Pollney:2010hs,Pollney:2009yz,} \\
Llama\_u2u2-q1--D8-h64-r250.T4.hybrid & 1 & 0.20 & 0.20 & TaylorT4 & \cite{Pollney:2010hs,Pollney:2009yz,} \\
Llama\_d2d2-q1--D10-h64-r250.T1.hybrid & 1 & -0.20 & -0.20 & TaylorT1 & \cite{Pollney:2010hs,Pollney:2009yz,} \\
Llama\_d2d2-q1--D10-h64-r250.T4.hybrid & 1 & -0.20 & -0.20 & TaylorT4 & \cite{Pollney:2010hs,Pollney:2009yz,} \\
\hline
\end{tabular}
\end{center}
\caption[Llama submissions to NINJA-2]{
\label{tab:ninja2_llama}
Llama submissions to NINJA-2}
\end{table}

\begin{table}
\begin{center}
\begin{tabular}{|l|r|r|r|l|c|}
\hline
Run & $q$ & Spin1${}_z$ & Spin2${}_z$ & pN Approx. & Refs \\
\hline
LazEV\_D8.4\_10to1\_nj\_hybrid & 10 & 0.00 & 0.00 & TaylorT4 & \cite{Campanelli:2005dd} \\
\hline
\end{tabular}
\end{center}
\caption[RIT submissions to NINJA-2]{
\label{tab:ninja2_rit}
RIT submissions to NINJA-2}
\end{table}

\begin{table}
\begin{center}
\begin{tabular}{|l|r|r|r|l|c|}
\hline
Run & $q$ & Spin1${}_z$ & Spin2${}_z$ & pN Approx. & Refs \\
\hline
SpEC\_q6s0 & 6 & 0.00 & 0.00 & TaylorT1 & \cite{SpECWebsite} \\
SpEC\_q4s0 & 4 & 0.00 & 0.00 & TaylorT2 & \cite{SpECWebsite} \\
SpEC\_EqualMassAntiAlignedSpins & 1 & -0.44 & -0.44 & NA & \cite{chu-2009,SpECWebsite} \\
SpEC\_q1s-0.95 & 1 & -0.95 & -0.95 & TaylorT1 & \cite{SpECWebsite} \\
SpEC\_q2s0 & 2 & 0.00 & 0.00 & TaylorT2 & \cite{SpECWebsite} \\
SpEC\_EqualMassAlignedSpins & 1 & 0.44 & 0.44 & NA & \cite{chu-2009,SpECWebsite} \\
SpEC\_q3s0 & 3 & 0.00 & 0.00 & TaylorT2 & \cite{SpECWebsite} \\
SpEC\_EqualMassNonspinning & 1 & 0.00 & 0.00 & TaylorT4 & \cite{Scheel:2008rj,SpECWebsite} \\
\hline
\end{tabular}
\end{center}
\caption[SpEC submissions to NINJA-2]{
\label{tab:ninja2_spec}
SpEC submissions to NINJA-2}
\end{table}

\begin{table}
\begin{center}
\begin{tabular}{|l|r|r|r|l|c|}
\hline
Run & $q$ & Spin1${}_z$ & Spin2${}_z$ & pN Approx. & Refs \\
\hline
UIUC\_spin\_-0.25\_om0.0528\_22-HYBRID & 1 & -0.25 & -0.25 & NA & \cite{none} \\
UIUC\_spin\_0.85\_om0.0536\_22-HYBRID & 1 & 0.85 & 0.85 & NA & \cite{none} \\
\hline
\end{tabular}
\end{center}
\caption[UIUC submissions to NINJA-2]{
\label{tab:ninja2_uiuc}
UIUC submissions to NINJA-2}
\end{table}


\begin{figure}
  \includegraphics[width=\linewidth]{figures/ninja2/ninja2_cat.png}
  \caption[Parameters of the NINJA-2 submissions]{
  \label{f:ninja2_param_map}
Parameters of the NINJA-2 hybrid waveform submissions showing the
symmetric mass ratio $\eta=m_1 m_2 /(m_1+m_2)^2$ and dimensionless
spin parameter $\chi=(S_1/m_1 + S_2/m_2)/(m_1+m_2)$ after scaling the
waveforms to a total mass of 10 $\msun$.  The numbers indicate how
many distinct waveforms with the specified parameters were submitted.}
\end{figure}%

\section{Verifying the hybrid waveforms}

Each NR group verified that their waveforms met the minimum NINJA-2
requirements before submission.  Once submitted, a series of checks
were performed in order to validate the waveforms against each other.

In the first stage the post-Newtonian expressions and codes were
compared against each other and the literature.  This required several
iterations, but resulted in a set of codes in various languages that
produce waveforms that all agree in both phase and amplitude. 

\subsection{Time-domain and frequency-domain checks}

In the second stage the complete hybrid waveforms were examined.
We first plotted the last 40 cycles of each waveform -- enough to
include the full NR portion, the hybridization region, and some of the
pN portion -- and looked for any anomalies such as those present in
some of the NINJA-1 waveforms in figure~\ref{fig:NR-Reh22}.  A few such
features were indeed visible, spotting them in this way allowed them
to be corrected.  One example is shown in figure (\Note{show the dq4
waveform before and after, note this was due to a problem integrating
psi4, and discuss this stage a little in the NR section of chapter
3}).

The amplitude of the Fourier transform of the complete waveforms were
also plotted.  This analysis also revealed unphysical features, 
primarily due to hybridization.  An example is shown in
figure~\ref{f:ninja2_freq_hybrids}, which shows a visible ``kink'' in
the waveform at the hybridization frequency, which vanishes after the
waveform was reconstructed.

\begin{figure}
  \includegraphics[width=0.5\linewidth]{figures/ninja2/bam_d125smm50nep_80_t1_hyb_n2_amp.png}
  \includegraphics[width=0.5\linewidth]{figures/ninja2/bam_d125smm50nep_80_t1_hyb_n2_amp_v2.png}
  \caption[Frequency-domain hybrid NINJA-2 waveforms]{
  \label{f:ninja2_freq_hybrids}
Fourier amplitude of the (2,2) mode of a sample NINJA-2 hybrid
waveform from the BAM/AEI group.  The waveform has been scaled to 10
$\msun$ and placed 1 Mpc from the detector to give it physical units.
\Note{I need to dig up the old version of the waveform and remake it
with the new code} The waveform on the left is the version
initially submitted, note there is a visible ``kink'' in the waveform
at the hybridization frequency.  The waveform on the right has been
re-hybridized and there is no longer a visible kink.  This feature did
not show up in the time domain view of the waveform.}
\end{figure}%


\subsection{Overlap comparisons}

\Note{Explain why we have to sample the overlaps at high rate, refer to
figure~\ref{f:overlap_sample_frequency}}

\begin{figure}
  \includegraphics[width=0.5\linewidth]{figures/ninja2/resolutions}
  \includegraphics[width=0.5\linewidth]{figures/ninja2/overlap_time_series}
  \caption[Sensitivity of the overlaps to sample rate]{
  \label{f:overlap_sample_frequency}
\Note{TODO: explain that the maximum is sharply peaked, undersampling
can miss the true maximum, as the waveforms beat against each other we
get cycles}}
\end{figure}%

In this check the waveforms were compared against each other using
standard data-analysis techniques, in particular the overlap defined
in equation~\ref{eq:OverlapDefinition}  using the initial LIGO noise
curve.  The waveforms were grouped into sets with identical
parameters.  For each set one waveform was chosen as the reference and
the overlap with all the other waveforms calculated over a range of
masses, optimizing over the unknown coalescence time and phase as
usual.  This process was then repeated, taking each of the other
waveforms as the reference in turn.

The initial set of contributions showed unexpectedly large mismatches
at masses $\approx 20 \msun$, at the point where the hybridization
frequency for several waveforms passes through the most sensitive
portion of the LIGO band.  This prompted a number of the NR groups to
revise their hybridization procedures, after which the overlaps were
more in line with the expected values.  A sample of these plots before
and after rehybridization is shown in
figure~\ref{f:ninja2_overlap_test}.  It is worth noting that even
after rehybridizing there are still mismatches.  In particular, the
SpEC and MayaKranc submissions use the same hybridization method, the
same pN approximant, and are simulating the same physical system.  The
overlaps approach 1 at higher masses, where the NR portion dominates.
This is an important validation, as the two groups use completely
different codes based on different principles.  The overlap is also
close to 1 at the lowest masses, where pN dominates.  This is expected
following the pN cross-checks done previously.  That the overlap
diminishes slightly at intermediate masses must be due to the
different hybridization frequencies \Note{get these values}, and
indicates the sensitivity of the waveform to the hybridization
details.


\begin{figure}
  \includegraphics[width=0.5\linewidth]{figures/ninja2/q_1_z_0_figure03}
  \includegraphics[width=0.5\linewidth]{figures/ninja2/figure2_1_0_16}
  \caption[Overlaps between NINJA-2 submissions maximized over time
and phase]{
  \label{f:ninja2_overlap_test}
Overlaps between the equal-mass, non-spinning NINJA-2 contributions,
maximized over time and phase.   For the original submissions (left)
overlaps are as low as 0.70 between waveforms using different pN
approximants and 0.94 for waveforms using the same approximant.  After
rehybridization (right) the waveforms achieve much higher overlaps,
with minima above 0.94 for different approximants and above 0.98 for
identical approximants.  The residual differences  between waveforms
using TaylorT4 are due to hybridization details.  The Llama waveform
was accidentally omitted from the original runs.}
\end{figure}%

%%%%%%%%%%%%%%%%
\clearpage

\begin{figure}
  \includegraphics[width=0.5\linewidth]{figures/ninja2/pn_figure03.png} 
  \includegraphics[width=0.5\linewidth]{figures/ninja2/pn_figure06.png} \\
  \includegraphics[width=0.5\linewidth]{figures/ninja2/pn_figure09.png} 
  \includegraphics[width=0.5\linewidth]{figures/ninja2/pn_figure12.png} \\
  \caption[Overlap of time-domain pN waveforms, $q=1$ $S_{z1} = S_{z2} = 0$]{
  \label{f:figure_pn}
Overlap of time-domain pN waveforms, $q=1$ $S_{z1} = S_{z2} = 0$}
\end{figure}%

\begin{figure}
  \includegraphics[width=\linewidth]{figures/ninja2/figure_1_-0d5_01.png}
  \caption[Overlap plots for $q=1$ $S_{z1} = S_{z2} = -0.5$]{
  \label{f:figure_1_-0d5}
Overlap plots for $q=1$ $S_{z1} = S_{z2} = -0.5$}
\end{figure}%


\begin{figure}
  \includegraphics[width=\linewidth]{figures/ninja2/figure_1_-0d4_01.png}
  \caption[Overlap plots for $q=1$ $S_{z1} = S_{z2} = -0.4$]{
  \label{f:figure_1_-0d4}
Overlap plots for $q=1$ $S_{z1} = S_{z2} = -0.4$}
\end{figure}%


\begin{figure}
  \includegraphics[width=0.5\linewidth]{figures/ninja2/figure_3_0_02.png} 
  \includegraphics[width=0.5\linewidth]{figures/ninja2/figure_3_0_04.png} \\
  \includegraphics[width=0.5\linewidth]{figures/ninja2/figure_3_0_06.png} 
  \caption[Overlap plots for $q=3$ $S_{z1} = S_{z2} = 0$]{
  \label{f:figure_3_0}
Overlap plots for $q=3$ $S_{z1} = S_{z2} = 0$}
\end{figure}%


\begin{figure}
  \includegraphics[width=0.5\linewidth]{figures/ninja2/figure_1_0d2_03.png} 
  \includegraphics[width=0.5\linewidth]{figures/ninja2/figure_1_0d2_06.png} \\
  \includegraphics[width=0.5\linewidth]{figures/ninja2/figure_1_0d2_09.png} 
  \includegraphics[width=0.5\linewidth]{figures/ninja2/figure_1_0d2_12.png} \\
  \caption[Overlap plots for $q=1$ $S_{z1} = S_{z2} = 0.2$]{
  \label{f:figure_1_0d2}
Overlap plots for $q=1$ $S_{z1} = S_{z2} = 0.2$}
\end{figure}%


\begin{figure}
  \includegraphics[width=0.5\linewidth]{figures/ninja2/figure_1_-0d25_02.png} 
  \includegraphics[width=0.5\linewidth]{figures/ninja2/figure_1_-0d25_04.png} \\
  \includegraphics[width=0.5\linewidth]{figures/ninja2/figure_1_-0d25_06.png} 
  \caption[Overlap plots for $q=1$ $S_{z1} = S_{z2} = -0.25$]{
  \label{f:figure_1_-0d25}
Overlap plots for $q=1$ $S_{z1} = S_{z2} = -0.25$}
\end{figure}%


\begin{figure}
  \includegraphics[width=\linewidth]{figures/ninja2/figure_1_-0d2_01.png}
  \caption[Overlap plots for $q=1$ $S_{z1} = S_{z2} = -0.2$]{
  \label{f:figure_1_-0d2}
Overlap plots for $q=1$ $S_{z1} = S_{z2} = -0.2$}
\end{figure}%


\begin{figure}
  \includegraphics[width=0.5\linewidth]{figures/ninja2/figure_2_0_04.png} 
  \includegraphics[width=0.5\linewidth]{figures/ninja2/figure_2_0_08.png} \\
  \includegraphics[width=0.5\linewidth]{figures/ninja2/figure_2_0_12.png} 
  \includegraphics[width=0.5\linewidth]{figures/ninja2/figure_2_0_16.png} \\
  \includegraphics[width=0.5\linewidth]{figures/ninja2/figure_2_0_20.png} 
  \caption[Overlap plots for $q=2$ $S_{z1} = S_{z2} = 0$]{
  \label{f:figure_2_0}
Overlap plots for $q=2$ $S_{z1} = S_{z2} = 0$}
\end{figure}%


\begin{figure}
  \includegraphics[width=\linewidth]{figures/ninja2/figure_1_0d8_01.png}
  \caption[Overlap plots for $q=1$ $S_{z1} = S_{z2} = 0.8$]{
  \label{f:figure_1_0d8}
Overlap plots for $q=1$ $S_{z1} = S_{z2} = 0.8$}
\end{figure}%


\begin{figure}
  \includegraphics[width=0.5\linewidth]{figures/ninja2/figure_1_0_04.png} 
  \includegraphics[width=0.5\linewidth]{figures/ninja2/figure_1_0_08.png} \\
  \includegraphics[width=0.5\linewidth]{figures/ninja2/figure_1_0_12.png} 
  \includegraphics[width=0.5\linewidth]{figures/ninja2/figure_1_0_16.png} \\
  \includegraphics[width=0.5\linewidth]{figures/ninja2/figure_1_0_20.png} 
  \caption[Overlap plots for $q=1$ $S_{z1} = S_{z2} = 0$]{
  \label{f:figure_1_0}
Overlap plots for $q=1$ $S_{z1} = S_{z2} = 0$}
\end{figure}%


\begin{figure}
  \includegraphics[width=\linewidth]{figures/ninja2/figure_1_0d5_01.png}
  \caption[Overlap plots for $q=1$ $S_{z1} = S_{z2} = 0.5$]{
  \label{f:figure_1_0d5}
Overlap plots for $q=1$ $S_{z1} = S_{z2} = 0.5$}
\end{figure}%


\begin{figure}
  \includegraphics[width=\linewidth]{figures/ninja2/figure_1_0d25_01.png}
  \caption[Overlap plots for $q=1$ $S_{z1} = S_{z2} = 0.25$]{
  \label{f:figure_1_0d25}
Overlap plots for $q=1$ $S_{z1} = S_{z2} = 0.25$}
\end{figure}%


\begin{figure}
  \includegraphics[width=0.5\linewidth]{figures/ninja2/figure_1_0d4_03.png} 
  \includegraphics[width=0.5\linewidth]{figures/ninja2/figure_1_0d4_06.png} \\
  \includegraphics[width=0.5\linewidth]{figures/ninja2/figure_1_0d4_09.png} 
  \includegraphics[width=0.5\linewidth]{figures/ninja2/figure_1_0d4_12.png} \\
  \caption[Overlap plots for $q=1$ $S_{z1} = S_{z2} = 0.4$]{
  \label{f:figure_1_0d4}
Overlap plots for $q=1$ $S_{z1} = S_{z2} = 0.4$}
\end{figure}%


\begin{figure}
  \includegraphics[width=\linewidth]{figures/ninja2/figure_1_-0d75_01.png}
  \caption[Overlap plots for $q=1$ $S_{z1} = S_{z2} = -0.75$]{
  \label{f:figure_1_-0d75}
Overlap plots for $q=1$ $S_{z1} = S_{z2} = -0.75$}
\end{figure}%


\begin{figure}
  \includegraphics[width=\linewidth]{figures/ninja2/figure_1_-0d85_01.png}
  \caption[Overlap plots for $q=1$ $S_{z1} = S_{z2} = -0.85$]{
  \label{f:figure_1_-0d85}
Overlap plots for $q=1$ $S_{z1} = S_{z2} = -0.85$}
\end{figure}%


\begin{figure}
  \includegraphics[width=0.5\linewidth]{figures/ninja2/figure_4_0_03.png} 
  \includegraphics[width=0.5\linewidth]{figures/ninja2/figure_4_0_06.png} \\
  \includegraphics[width=0.5\linewidth]{figures/ninja2/figure_4_0_09.png} 
  \includegraphics[width=0.5\linewidth]{figures/ninja2/figure_4_0_12.png} \\
  \caption[Overlap plots for $q=4$ $S_{z1} = S_{z2} = 0$]{
  \label{f:figure_4_0}
Overlap plots for $q=4$ $S_{z1} = S_{z2} = 0$}
\end{figure}%


\begin{figure}
  \includegraphics[width=\linewidth]{figures/ninja2/figure_1_0d75_01.png}
  \caption[Overlap plots for $q=1$ $S_{z1} = S_{z2} = 0.75$]{
  \label{f:figure_1_0d75}
Overlap plots for $q=1$ $S_{z1} = S_{z2} = 0.75$}
\end{figure}%


\begin{figure}
  \includegraphics[width=0.5\linewidth]{figures/ninja2/figure_1_0d85_02.png} 
  \includegraphics[width=0.5\linewidth]{figures/ninja2/figure_1_0d85_04.png} \\
  \includegraphics[width=0.5\linewidth]{figures/ninja2/figure_1_0d85_06.png} 
  \caption[Overlap plots for $q=1$ $S_{z1} = S_{z2} = 0.85$]{
  \label{f:figure_1_0d85}
Overlap plots for $q=1$ $S_{z1} = S_{z2} = 0.85$}
\end{figure}%

\clearpage
%%%%%%%%%%%%%%%%

We also calculated the overlaps between waveforms with identical
parameters, optimizing over mass as well as time and phase.  This gives
a sense of the range over which parameter estimation pipelines could 
be biased by unphysical features of the waveform.  Example plots
using the equal-mass, non-spinning MayaKranc waveform as the signal and
BAM plus two different approximants as the template are shown in 
figure~\ref{f:ninja2_max_over_mass_bam}.


\begin{figure}
  \includegraphics[width=0.5\linewidth]{figures/ninja2/maya_bamt4_max_over_m}
  \includegraphics[width=0.5\linewidth]{figures/ninja2/maya_bamt1_max_over_m}
  \caption[Overlaps between NINJA-2 submissions maximized over mass]{
  \label{f:ninja2_max_over_mass_bam}
Overlaps between the equal-mass, non-spinning MayaKranc waveform taken
as the signal, and the equal-mass, non-spinning BAM waveform
hybridized with TaylorT4 (left) and TaylorT1 (right) taken as
templates.  Maximization is done over the mass of the template, as well
as over time and phase.  Note the lower overall overlaps and mass bias
at the low-mass end of the figure on the right, where the two different
pN waveforms dominate the overlap.}
\end{figure}%

At the high-mass end the overlap is dominated by NR data, and as in
figure~\ref{f:ninja2_overlap_test} the overlaps are high without
needing to move off the signal mass.  At the low-mass end the same
result would be expected in a pure pN/pN comparison.  However, there
is enough of the hybridization in-band to reduce the overlaps.  However,
changing the mass introduces a phase difference that accumulates over
all the cycles in-band, and so higher overlaps can not be achieved.
The result is optimal mass values close to the correct mass value, but
with a low overlap.

In the middle region these factors compete.  At higher masses the
overlap is reduced less by changing the mass and so the recovered
value can stray further from the injected value.  However as the
hybridization passes out of band this adjustment is no longer needed.


\noindent \Note{TODO: Fill in more details, discuss the need to sample
overlap time series at 16384, include the rest of the overlap plots after
remaking them for latest submissions}




% TODO:
% make tables of submissions
% plot of paramater space in eta, chi scaled to 10 M
% plot of injection set (use chi instead of sum of magnitudes)
% text
% results
% explain move to 16384 (plot showing aliasing)



\Chapter{Preliminary CBC Results from the Second NINJA project}
\label{ch:ninja2_results}
\section{Construction of the NINJA-2 data set}


In broad terms the plans for the NINJA-2 data sets follow those for
NINJA-1 (ch.~\ref{ch:ninja1}).  Simulated Gaussian noise was generated
to model the initial LIGO and Virgo noise curves, the spectra are
identical to those in figure~\ref{f:ninjapsd}.  Injection parameters,
including choice of waveform, were then selected randomly.  The
injections were then added to the Gaussian noise and distributed to
data analysis groups.  However, several key changes were made in the
details of this process in order to correct shortcomings in NINJA-1.  

The NINJA-2 data was sampled at 16384 Hz rather than the 4096 Hz used
by NINJA-1.  This was done because investigations showed that there is
power above 4096 Hz in the waveforms, which would get aliased down to 
lower frequencies if the sample rate is too low.  This problem is
illustrated in figure~\ref{f:ninja2_aliasing}.

\begin{figure}
  \includegraphics[width=\linewidth]{figures/ninja2/ninja2_aliasing}
  \caption[Aliasing of waveform power]{
  \label{f:ninja2_aliasing}
Frequency-domain amplitudes of a NINJA-2 waveform at different
sampling rates.  At a sample rate of 4096 Hz the late portion of the
waveform are distorted due to aliasing of power to lower frequencies.
}
\end{figure}%

NINJA-1 consisted of only 127 injections in a day of data, which
severely limited the ability to draw statistical conclusions on the
behavior of the pipelines.  To correct this in NINJA-2 we extended the
duration to eight weeks.  The density of injections was varied over
this span:  weeks 1-3 had one injection on average every 2000 seconds,
week 4-6 had one injection on average every 4 hours, and the final two
weeks had one injection every on average every 2.5 days.  The intent
was that data analysts can tune and test their pipelines on the dense
weeks, and then optionally perform a self-blinded test on the final two
weeks.

In NINJA-1 the SNR was not chosen a priori but was determined by the
other parameters.  For NINJA-2 we draw the network SNR ($\sqrt{\sum_i
\rho_i^2}$ where $i$ ranges over the IFOs) from a distribution and
then scale the distance of the injection in order to achieve that SNR.
For the first three weeks the distribution is linear from 6 to 130 in
order to allow pipelines to test and tune out to large SNRs on the
densest set of injections.  For the remaining weeks the distribution
falls as the reciprocal of the network SNR (uniform in
$\log(\mathrm{SNR})$) in order to better model the expected
astrophysical distribution.

The mass and waveform selection were also done slightly differently in
NINJA-2.  For each injection a mass was first selected uniformly over
the specified range; for the full 2-month run this range is from
$10-350 \msun$.  Then waveforms were selected at random until one was
found that could be injected at the chosen mass such that the waveform
turns on below 35 Hz.  In practice this condition never caused any
waveform to be rejected, as all submitted waveforms were long enough
to be injected down to the lowest mass in the range.  The mass ratio
and spins are intrinsic to the waveforms, so choosing a submission
amounts to a choice of these parameters as well.  A plot of the
selected masses, spins, and distances in the two-month run is shown in
figure~\ref{f:ninja2_dataset}.


\begin{figure}
  \includegraphics[width=\linewidth]{figures/ninja2/ninja2_dataset.png}
  \caption[Parameters of the NINJA-2 two-month data set]{
  \label{f:ninja2_dataset}
Distribution of mass, spin and distance parameters in the two-month,
Gaussian-noise data set.
}
\end{figure}%


As in NINJA-1 the sky location and inclination were chosen uniformly
at random.

For NINJA-2, in sharp contrast to NINJA-1, we created several two-week
test data sets in order to verify the waveforms and injection codes.
These tests were broken into smaller mass regions, ``low mass'' from
$10-40 \msun$, ``high-mass'' from $35-100 \msun$ and
``burst/ringdown'' from $80-350 \msun$.  The original intent had been
to create three full 2-month data sets along these same lines, but
that proved to be infeasible due to the file size of the data sets.
However, it was useful in the test sets as it allowed several
pipelines to do sanity checks in the mass region to which they are
most sensitive.  These tests exposed several bugs in the injection
codes which were fixed before generating the two-month data set.

Real detector data is far from Gaussian, and real data analysis is
concerned not only with foreground triggers from signals but also
background triggers from noise.  In order to comprehensively test the
ability of pipelines to detect signals and recover their parameters it
is necessary to perform injections into real detector noise.  As of
this writing a memorandum of understanding (MoU) between the NINJA
collaboration, the LIGO collaboration and the Virgo collaboration has
been signed which will allow subsequent NINJA-2 data sets to use data
from the previous (S5/VSR1) science run as noise.  A key feature of
this agreement is that NINJA is not a gravitational-wave search.  We
will therefore use data from disjoint periods in each instrument.
Details of this plan, such as which times to use, have yet to be
decided.  However it is clear we will need a custom segment database
(see chapter~\ref{ch:detchar}) to mark times where the instruments
were glitching. However, injection times will not use this
information, as it is entirely possible that real signals will land on
or near a glitch, and the ability to detect such signals is an
important test.
 
There are two motivations for constructing and distributing the data
sets as NINJA-1 and NINJA-2 thus far have done.  The first is to
ensure that every group is looking at the same set of injections so
that results can be compared.  The second is due to the terms of the
NINJA agreement, which restricted distribution of the raw NR
waveforms.  However, distribution of such static sets limits the
ability of individual groups to tune their pipelines in optimal ways,
and conceptually distributing a set of parameters would be sufficient
to compare results across pipelines.  In addition the size of the data
sets makes distribution slow and complex.  The NR groups within NINJA
have therefore relaxed the conditions on their use of their waveforms.
Consequently, subsequent NINJA-2 data sets will be distributed as sets
of parameters, and data analysis groups will use the available code to
either create data sets locally, or perform the injections ``on the
fly'' as the analysis is performed.  This will also allow groups to do
special-purpose tuning runs or analyses, the results of which may be
published as short-author papers subject to the conditions of the
NINJA agreement.


\subsection{CBC preliminary results}

The NINJA-2 data was analyzed with the standard CBC low-mass and
high-mass pipelines.  The parameters were exactly as in the S6/VSR2,3
runs, no changes were made to the configurations except for those
relating to the names of the data files.  The low-mass search uses
Taylor F2 templates to 3.5 pN order in phase evolution
(section~\ref{sec:PNWaveforms}) in a mass region defined by minimum component
masses of $1 \msun$ and maximum total mass of $25 \msun$.  The high
mass search uses EOBNR templates (section~\ref{ssec:EOB}) in a region
defined by minimum component masses of $1 \msun$, minimum total mass
of $25 \msun$, and maximum total mass of $100 \msun$.  Plots of found
and missed injections are shown in figure~\ref{f:ninja2_cbc_results}.

\begin{figure}
  \includegraphics[width=0.5\linewidth]{figures/ninja2/H1-plotinspmissed_LOW_FULL_DATA_mchirp-eff_dist-log-H1-871147552-5209912.png}
  \includegraphics[width=0.5\linewidth]{figures/ninja2/H1-plotinspmissed_HIGH_FULL_DATA_mchirp-eff_dist-log-H1-871147552-5209912.png} \\
  \includegraphics[width=0.5\linewidth]{figures/ninja2/L1-plotinspmissed_LOW_FULL_DATA_mchirp-eff_dist-log-L1-871147552-5209912.png}
  \includegraphics[width=0.5\linewidth]{figures/ninja2/L1-plotinspmissed_HIGH_FULL_DATA_mchirp-eff_dist-log-L1-871147552-5209912.png} \\
  \includegraphics[width=0.5\linewidth]{figures/ninja2/V1-plotinspmissed_LOW_FULL_DATA_mchirp-eff_dist-log-V1-871147552-5209912.png}
  \includegraphics[width=0.5\linewidth]{figures/ninja2/V1-plotinspmissed_HIGH_FULL_DATA_mchirp-eff_dist-log-V1-871147552-5209912.png} \\
  \caption[Preliminary NINJA2 CBC results]{
  \label{f:ninja2_cbc_results}
Preliminary results from the CBC low-mass (left) and high-mass (right)
pipelines on the two-month NINJA-2 data set.
}
\end{figure}%


\noindent \Note{TODO: Show efficiency as a function of mass, for all
signals and just for non-spinning.}

\noindent \Note{TODO: explain anomalous Virgo results.}



\Chapter{A Database for Instrumental Data Quality}
\label{ch:segdb}

\subsection{Data Quality Flags}

The state of the instrument at any time is summarized by a set of
\emph{flags}.  Flags are identified by a triple of (ifo id, flag name,
version number), where \emph{ifo id} identifies the instrument,
\emph{flag name} is a unique identifier, and \emph{version number} is
an integer starting from 1.  The version number allows information to
be updated without losing information that may be needed to
reconstruct the results of earlier searches.  The full set of flags is
stored in a database designed at Syracuse and hosted at Caltech.

Flags are stored as a set of \emph{segments}, half-open intervals
aligned on GPS integer second boundaries.  Each flag triple has an
associated set of segments indicating the times during which it is
\emph{defined}.  Such triples also have a set of segments indicating
times during which they are \emph{active}.  The set of active segments
must be a subset of defined segments.  Times during which a flag is
defined and not active are considered \emph{inactive}.

Flags may be entered manually through a web interface.  This can be
used to indicate nonstandard operating conditions, such as heavy
equipment being operated on site.  However, most flags are generated
automatically.

The first line of defense against noise triggers is on-site as the
instrument is running.  At all such times the control room is staffed
by an operator who is an expert in running the instrument employed by
LIGO labs, and a science monitor (``SciMon'') who is a member of the
LIGO Scientific Collaboration.  The two jointly decide when to enable
\emph{science mode}, which marks the data as suitable for analysis.
This declaration is not specific to the CBC group, but extends to all
searches in the collaboration.  Science time is indicated by the flag
\texttt{DMT\_SCIENCE}.

In addition to the readout channel (\texttt{DARM\_ERR}, section
\ref{sec:ligo_detectors}) many other data channels are recorded,
falling into two broad categories.  Physical environmental monitor
(``PEM'') channels record information about the environment such as
seismic activity at the base and end stations, microphones and
magnetometers placed throughout the site, etc.  Instrumental
(``INST'') channels record data from numerous subsystems such as
servos for each mirror and the output of photodiodes at points
throughout the light path.  Software running at the sites called the
data monitoring tool (``DMT'') creates data quality flags based on
these channels.  For example, when a channel's value or standard
deviation over time exceeds a given threshold.  The DMT is also
responsible for recording the state of the science mode flag.  Other
flags are set by programs that run analyses on these auxiliary
channels, for examples see~\cite{Isogai:2010}.


\subsection{The Veto Definer}

Problems in the data may have differing levels of severity, and
consequently we define several \emph{veto categories} to characterize
them.  The categories used by daily ihope differ slightly from those
used by the full analysis.

Category 1 vetoes indicates time that should not have been marked as
science mode.  Typically attempting to analyze this time will
adversely affect the entire 2048-second analysis chunk, for example by
biasing the PSD (section \ref{sec:ihope_psd}).  Note that it is
possible to correct science time by creating a new version of the
\texttt{DMT\_SCIENCE} flag with an incremental version number.
However, doing so is a more complicated process than creating a veto
flag, and removing time by denoting it CAT1 carries additional
information about the reason for the veto.  Note that category 1
vetoes are undesirable, as they may remove time outside the range of
the problem.  For example, a 4080-second span of science-mode data
with a one-second CAT1 veto at 2040 seconds will be completely
unanalyzable by the CBC group, as excising the bad data will leave no
contiguous 2048-second block.

Category 2 vetoes indicates time during which there was a problem,
instrumental or environmental, with well-understood coupling into
\texttt{DARM\_ERR}.  Such time can be analyzed without problem, but
triggers from such time will be discarded. 

Category 3 vetoes remove hardware injections (section
\ref{sec:ihope_hardware_injections}).  In the full analysis hardware
injection vetoes do not have a category number and is denoted
``hardware injections removed.''

Category 4 is for time that appears to be ill-behaved according to 
data quality studies, but where there is no clearly-understood 
cause.  In the full analysis this is denoted category 3.

Finally, a map is needed between data quality flags and veto
categories.  This is achieved through the use of a search-specific
\emph{veto definer file} which associates a flag with a veto level.
Intervals within the range during which that version of the flag are
active will be marked with the corresponding veto level.  The
\texttt{ligolw\_segments\_from\_cats} program (see below) merges the
information in this file with the set of active flag segments to produce
\emph{veto segments} at each veto level.  Analysis is performed on
times marked as science with no CAT1 vetoes.  Triggers from times
marked as CAT2, CAT3 and CAT4 are discarded from both foreground and
background.

Sometimes the threshold on a DQ flag is such that the data is
unsuitable for analysis close to, but outside, the range of the flag.
The veto definer file therefore allows for \emph{padding}, offsets 
which effectively extend the flags.

\iffalse
\subsection{Technical details}

The flag segments are stored in a high-performance relational
database, exposed as a web service which provides secure access to all
members of the collaboration.  Several utilities to interact with the
segment database were written, many of which could run in many modes.
The names of all these utilities start with ligolw, short for ``LIGO
lightweight,'' an XML-based format used throughout the collaboration
in which the output was generated.

Except where noted all programs accept a common subset of arguments
\begin{itemize}
\item \texttt{--gps-start-time} and \texttt{--gps-end-time}: Integer
GPS times specifying the time range over which to run.
\item \texttt{--include-segments}: Takes a comma-separated list of
segment specifiers of the form ifo:flag\_name:version or
ifo:flag\_name and restricts the query to matching flags.  In the
latter case, report on the latest version of the flag defined at each
time within the time range.
\item \texttt{--exclude-segments}: Takes a comma-separated list of
segment specifiers in the same format as \texttt{--include-segments}.
Runs a second query on the excluded segments and returns the set of
segments in included segments - (included segments $\cap$ excluded
segments).
\item \texttt{--database}, \texttt{--segment-url},
\texttt{--dmt-files}: Specifies the data source (segment database or
XML files with segment information).
\end{itemize}


The programs used during S6 were:

\textbf{ligolw\_dq\_query}.  This program reports on the state of
flags at one or more GPS times given on the command line.  This
program does not support \\
\texttt{--gps-start-time} or \texttt{--gps-end-time}.   The available modes are

\begin{itemize}
\item \texttt{--report}:  Returns the active/inactive status of all DQ flags at the
given times.  For active flags reports the start and end times of the
segment within which the given time is contained.  For inactive flags
reports the end time of the nearest preceding segment and the start
time of the nearest subsequent segment.  This mode is used in the
daily ihope ``loudest glitches'' page, see below.
\item \texttt{--defined}: Returns a list of flags defined at the
given times.
\item \texttt{--active}: Returns a list of the flags that were active
at the given times.
\item \texttt{--start-pad}, \texttt{--end-pad}: Extends the
\texttt{--defined} and \texttt{--active} modes to report on the status
of flags within a small range of time.
\end{itemize}


\textbf{ligolw\_segment\_query}.  This program reports on the state of a
set of flags over a span of times.  The available modes are

\begin{itemize}
\item \texttt{--show-types}: Reports the sets of flags that exist over
the time span.
\item \texttt{--query-types}: Reports the segments during which the
flags were defined.
\item \texttt{--query-segments}: Reports the segments during which the
flags were active.
\end{itemize}

Every CBC analysis begins by determining the science mode times with a
call of the form

\vspace*{5mm}
\texttt{ligolw\_segment\_query} \\
\hspace*{0.5in}\texttt{--query-segments --include-segments H1:DMT-SCIENCE:3 ...}
\vspace*{5mm}

\textbf{ligolw\_segments\_from\_cats}.  Given a veto definer file,
report on segments to be vetoed. 

\textbf{ligolw\_dq\_active\_cats}.  Given a veto definer file and GPS
time, report on the active/inactive status and veto category for all
flags defined at the time.

\textbf{ligolw\_segment\_insert}.  Adds new segments to the database,
enforcing various policy decisions:
\begin{itemize}
\item New segments for an existing flag/version number pair must not overlap 
with existing segments.  To update information a new version number must be used.
\item New version numbers must be one greater than the largest
existing version number.
\item New segments must not extend into the future, past the GPS time
at which the program is run.
\end{itemize}
\fi

% https://www.lsc-group.phys.uwm.edu/daswg/wiki/content_of_segment_publication_and_discovery

\Note{The following was taken from T0900005 -- Needs to be cleaned up
and integrated!}


%%%%%%%%%%%%%%%%%%%%%%%%%%%%%%%%%
\section{Introduction}

This document describes the infrastructure used to manage data quality
and veto information in the LIGO S6 Science Run. Data quality
information is generated by a combination of automated Data Monitoring
Tool (DMT) software and scientist investigation. This information must
be stored for archiving and retrieval. A suite of tools should be
provided for users to update and query the stored data. In S6, the
segment database is the central location all DQ data is written in to. 

Two parallel solutions are being pursued:
\begin{enumerate}
\item A database-based solution will allow users to store and query segment
information. This will be based on the infrastrure used in S5, updated with
the lessons learned from the science run.
\item Tools will be provided to allow users/programs to query the online DQ
data files generated by the DMT. This will allow users to perform small
low-latency queries relevant to online searches in the most direct manner
possible.
\end{enumerate}

\section{Implementation plan}
\begin{figure}[h]
  \begin{center}
    \includegraphics[width=0.9\linewidth]{figures/segdb/T0900005_fig1}
  \end{center}
  \caption{Data flow for S6 data quality segment information. Online data
  quality segments and science segment information is generated by the DMT.
  This can be directly queried for low-latency online analysis or inserted
  into a segment database for off-line or higher-latency analyses. Command
  line and web GUI  tools can be used to query and update the segment
  database.} 
\end{figure}

\begin{enumerate}
\item \textbf{DMT trigger manager:} The DMT trigger manager will handle the
creation of all science segments and online data quality segments in S6. Every
60 seconds, the DMT trigger manager writes segment information in XML format
to disk at the Observatories. The DMT calls the \verb|dmtdq_seg_insert|
program to insert the XML data into the segment database.

To prevent confusion with cases sensitivity in processing tools, all DQ flag
names must be in upper case case. Additionally, a 3-letter prefix will be
added to each DQ flag to better identify the source where data comes from. For
example: the S5 DQ flag known as \verb|Wind_Over_30MPH| should be writted
into the XML file as \verb|DMT-WIND_OVER_30MPH| in S6. All DQ flags generated online by the DMT should have version
number 1.

\item \textbf{Archival of DMT segment data:} The 60 second XML files generated
by the DMT will be archived every 3600 seconds onto the LDAS
\verb|/archive/dataprods| filesystem for replication to Caltech and Teir 2
computing centers. Files will be compressed using the gzip algorithm.

\item \textbf Software will be developed to take the XML files from DMT disk
and together with a data quality categorization file managed by the search
groups (e.g. H1-ONLINEALLSKYBURSTCATS-815155200-63072000.xml) will produce XML
files containing the segments for a given DQ category for a given search once
per minute or so. In addition ligolw\_print can be used to convert these XML
files to ASCII format for search groups that want ASCII segment lists. The same
software will be made available for interfacing with the segment database for
offline searches and detector characterization. 

\item \textbf{Expected Latencies:} as described in the implementation plan
diagram, expected latencies are:
\begin{itemize}
\item DMT generates h(t) file: 60 seconds
\item From DMT to raw DMT segment disk: 60 seconds
\item From raw DMT segment disk to segment database: 10 seconds
\item From ligolw\_segment\_insert to segment database: 30 seconds
\end{itemize}
\end{enumerate}



%%%%%%%%%%%%%%%%%%%%%%%%%%%%%%%%%%%%%%%%%%%%%%%%
\section{S6 segment database design}

The S6 segment database schema (and hence the XML files) have been
substantially simplified for S6.  The five tables to be used for S6 are shown
in figure~\ref{f:schema}. All uncessary tables have been eliminated and the
remaining table set simplified. The \verb|process|, \verb|process_params| and
\verb|segment_definer| tables are unchanges from S5, with the exception that
the \verb|domain| column in the \verb|process| table is used to store the
distingished name of the user inserting the data.  In S5, segments were either
stored as on or off allowing users to distinguish between a segment being off
or undefined, due to data not being analyzed. It was foung that the
off/undefined query was used substantially less than the on query, and so in
S6 this column is eliminated from the \verb|segment| table.  To ensure that
off/undefined information is still available, the a \verb|segment_summary| is
introduced in S6 to store time intervals when quality flags are defined. This
further simplified the query ``What versions were defined at what time?''
allowing better use of version information in S6.

\begin{figure}[h]
  \begin{center}
    \includegraphics[width=0.9\linewidth]{figures/segdb/T0900005_fig3}
  \end{center}
  \label{f:schema}
  \caption{S6 Segment Database Schema}
\end{figure}





%%%%%%%%%%%%%%%%%%%%%%%%%%%%%%%%%%%%%%%%%%%%%%%%
\section{Data Replication Between Observatories and Tier 2 Centers}

In S6, Caltech segment database is the official data repository. In
addition to being written into the database, all data inserted in to
the Caltech segment database will be output as xml files to a central
location. These xml files will then be replicated to tier 2 sites at
ldas.ligo-wa.caltech.edu, ldas.ligo-la.caltech.edu. A rsync script
will be implemented at Caltech to periodically replicate xml files
from Caltech to tier 2 sites. At the sites, a publishing script will
be implemented to publish the xml files to the local segment database.

In addition to database replication, the archived DMT segment files will be
replicated between sites using rsync, according to the DASWG
\verb|/archive/dataprods| prototcol.


%%%%%%%%%%%%%%%%%%%%%%%%%%%%%%%%%%%%%%%%%%%%%%%%%
\section{User Interface}

\subsection{GUI Segment Query Tools}

In S6, we will no longer support plain ASCII dumps of the segment database.
Use of segwizard will be depricated in favor of a web-based GUI, similar to
the web GUI used by the Virgo segment database.  Users will be able to
manipulate graphical tools in the web browser interface to retrieve desired
results directly from the segment database. 

A GUI will also be provided for the DetChar chair (and designated people) to
modify segment information stored in the database.

\subsection{Command Line Tools}

\subsubsection{ligolw\_segment\_query}

The glue program \verb|ligolw_segment_query| replaces LSCsegFind as the
command line interface to query for data quality and science segments.
\verb|ligolw_segment_query| can query either the segment database or
directories containing XML segment files generated by the DMT.
\verb|ligolw_segment_query| only provides read access to the segment database.
\verb|ligolw_segment_query| primarily intended for human interaction with the
segment database. Automated generation of DQ files for online searches should
use the program \verb|ligolw\_veto\_segments|.

Below are the questions that ligolw\_segment\_query can answer:
\begin{itemize}
\item What DQ flags exist in the database? ligolw\_segment\_query --show-types
\item When was a given flag inserted? ligolw\_segment\_query --query-types
\item When was a given DQ flag defined? ligolw\_segment\_query --query-types
\item When was a given flag active? ligolw\_segment\_query --query-segments 
\end{itemize}


DESCRIPTION:
{\small
\begin{verbatim}
  --version             show program's version number and exit
  -h, --help            show this help message and exit
  -p, --ping            Ping the target server
  -y, --show-types      Returns a xml table containing segment type
                        information: ifos, name, version,
                        segment_definer.comment, segment_summary.start_time,
                        segment_summary.end_time, segment_summary.comment
  -u, --query-types     Returns a ligolw document whose segment_definer table
                        includes all segment types defined in the given period
                        and included by include-segments and whose
                        segment_summary table indicates the times for which
                        those segments are defined.
  -q, --query-segments  Returns a ligolw document whose segment table contains
                        the times included by the include-segments flag and
                        excluded by exclude-segments
  -s gps_start_time, --gps-start-time=gps_start_time
                        Start of GPS time range
  -e gps_end_time, --gps-end-time=gps_end_time
                        End of GPS time range
  -t segment_url, --segment-url=segment_url
                        Segment URL. Users have to specify either 'https://'
                        for a secure connection or 'http://' for an insecure
                        connection in the segment database url. For example,
                        '--segment-url=https://segdb.ligo.caltech.edu'. No
                        need to specify port number.
  -d, --database        use database specified by environment variable
                        S6_SEGMENT_SERVER. For example,
                        'S6_SEGMENT_SERVER=https://segdb.ligo.caltech.edu'
  -f, --dmt-files       use files in directory specified by environment
                        variable ONLINEDQ, for example,
                        'ONLINEDQ=file:///path_to_dmt'. 'file://' is the
                        prefix, the acutal directory to DMT xml files starts
                        with '/'.
  -a include_segments, --include-segments=include_segments
                        This option expects a comma separated list of a colon
                        separated sublist of interferometer, segment type, and
                        version. The union of segments from all types and
                        versions specified is returned. Use --show-types to
                        see what types are available.   For example:
                        --include-segment-types H1:DMT-SCIENCE:1,H1:DMT-
                        INJECTION:2 will return the segments for which H1 is
                        in either SCIENCE version 1 or INJECTION version 2
                        mode. If version information is not provided, the
                        union of the segments of the latest version of
                        requested segment type(s) will be returned.
  -b exclude_segments, --exclude-segments=exclude_segments
                        This option has to be used in conjunction with
                        --include-segment-types --exclude-segment-types
                        subtracts the union of unwanted segments from the
                        specified types from the results of --include-segment-
                        types. If version information is not provided,
                        --exclude-segment-types subtracts the union of
                        segments from the latest version of the specified
                        segment types. For example, --include-segment-types H1
                        :DMT-SCIENCE:1,H1:DMT-INJECTION:2 --exclude-segment-
                        types H1:DMT-WIND:1,H1:DMT-NOT_LOCKED:2,H2:DMT-
                        NOT_LOCKED:2 will subtract the union of segments which
                        H1 is in version 1 WIND and H1,H2 is version 2
                        NOT_LOCKED from the result of --include-segment-types
                        H1:DMT-SCIENCE:1,H1:DMT-INJECTION:2
  -S, --strict-off      The default behavior is to truncate segments so that
                        returned segments are entirely in the interval [gps-
                        start-time, gps-end-time).  However if this option is
                        given, the entire non-truncated segment is returned if
                        any part of it overlaps the interval.
  -o output_file, --output-file=output_file
                        File to which output should be written.  Defaults to
                        stdout.
\end{verbatim}
}

\subsubsection{ligolw\_dq\_query}
ligolw\_dq\_query can query either the segment database or directories containing XML segment files generated by the DMT. ligolw\_dq\_query only provides read access to the segment database. 

Below are the questions that ligolw\_dq\_query can answer:
\begin{itemize}
\item is a given flag active at a given time? ligolw\_dq\_query --active
\item is a given flag defined at a given time? ligolw\_dq\_query --defined
\item what is the status of all flags at a given time? ligolw\_dq\_query --report
\end{itemize}


DESCRIPTION:
{\small
\begin{verbatim}
  --version             show program's version number and exit
  -h, --help            show this help message and exit
  -p, --ping            Ping the target server
  -y, --defined         Returns a segment summary table containing segments
                        defined at the given time(s).
  -u, --active          Returns a segment table containing segments active at
                        the given time(s).
  -q, --report          Prints which flags are defined/undefined at the given
                        time(s). For the flags which were defined, it
                        determines if the flag was active or inactive at that
                        time. For an active flag, it prints the start and end
                        time of the segment to which the active. For an
                        inactive flag, it prints the end time of the previous
                        adjacent active segment and the start time of the next
                        adjacent active segment
  -s start_pad, --start-pad=start_pad
                        Seconds before given time(s) to include in query
  -e end_pad, --end-pad=end_pad
                        Seconds after given time(s) to include in query
  -t segment_url, --segment-url=segment_url
                        Segment URL
  -d, --database        use database specified by environment variable
                        S6_SEGMENT_SERVER
  -f, --dmt-files       use files in directory specified by environment
                        variable ONLINEDQ
  -a include_segments, --include-segments=include_segments
                        This option expects a comma separated list of a colon
                        separated sublist of interferometer, segment type, and
                        version. The union of segments from all types and
                        versions specified is returned. Use --show-types to
                        see what types are available.   For example:
                        --include-segment-types H1:SCIENCE:1,H1:INJECTION:2
                        will return the segments for which H1 is in either
                        SCIENCE version 1 or INJECTION version 2 mode. If
                        version information is not provided, the union of the
                        segments of the latest version of requested segment
                        type(s) will be returned.
  -o output_file, --output-file=output_file
                        File to which output should be written.  Defaults to
                        stdout.
  -i, --in-segments-only
                        If set, report will only return segments that given
                        times were within
\end{verbatim}
}



\subsubsection{ligolw\_segments\_from\_cats}
ligolw\_segments\_from\_cats reads one or more segment files and a veto file and generates files of veto segments.

DESCRIPTION:
\begin{verbatim}
  --version             show program's version number and exit
  -h, --help            show this help message and exit
  -v veto_file, --veto-file=veto_file
                        veto XML file (required).
  -o output_dir, --output-dir=output_dir
                        Directory to write output (default=cwd).
  -k, --keep-db         Keep sqlite database.
  -t segment_url, --segment-url=segment_url
                        Segment URL
  -d, --database        use database specified by environment variable
                        S6_SEGMENT_SERVER
  -f, --dmt-file        use files in directory specified by environment
                        variable ONLINEDQ
  -c, --cumulative-categories
                        If set the category N files will contain all segments
                        in categories <= N
  -p, --separate-categories
                        If set the category N files will contain only category
                        N
  -s gps_start_time, --gps-start-time=gps_start_time
                        Start of GPS time range
  -e gps_end_time, --gps-end-time=gps_end_time
                        End of GPS time range
\end{verbatim}


For example:
\begin{verbatim}
ligolw\_segments\_from\_cats --gps-start-time 930960015 --gps-end-time 931564887 --segment-url https://segdb.ligo.caltech.edu:30015 --cumulative-categories --veto-file http://www.lsc-group.phys.uwm.edu/ligovirgo/cbc/public/segments/S6/H1L1V1-S6\_CBC\_LOWMASS\_ONLINE-928271454-0.xml
\end{verbatim}




\subsubsection{ligolw\_segment\_insert}
ligolw\_segment\_insert is the replacement of LSCdqInsert. ligolw\_segment\_insert handles two tasks:
\begin{itemize}
\item Insert segments and/or segment types into the segment database.
\item Append segments to the existing segment types.
\end{itemize}
ligolw\_segment\_insert follows the underlying methods of LSCdqInsert to insert segments, but there are several changes made in ligolw\_segment\_insert which includes:
\begin{itemize}
\item - -interval option is removed. In LSCdqInert, - -interval is used to read in a single segment. With ligolw\_segment\_insert, user has to specify a plain text file containing the segment(s), even if there is only one segment in the text file.
\item - -run option is removed.
\item Users have to provide segment file which contains the active segments of a given segment type of the whole run and summary file which contains the validity intervals of this given segment type of the whole run which activ segments belong.
\end{itemize}

DESCRIPTION:
\begin{verbatim}
  -h, --help            show this help message and exit
  -p, --ping            Ping the target server
  -t URL, --segment-url=URL
                        Users have to specify protocol 'https://' for a secure
                        connection in the segment database url. For example,
                        '--segment-url=https://segdb.ligo.caltech.edu'. No
                        need to specify port number'.
  -o FILE, --output=FILE
                        Write segments to FILE rather than the segment
                        database
  -j IDENTITY, --identity=IDENTITY
                        Set the subject line of the server's service
                        certificate to IDENTITY
  -I, --insert          Insert segments to the segment database
  -A, --append          Append segments to an existing segment type
  -i IFOS, --ifos=IFOS  Set the segment interferometer to IFOS (e.g. H1)
  -n NAME, --name=NAME  Set the name of the segment to NAME (e.g. DMT-
                        BADMONTH)
  -v VERSION, --version=VERSION
                        Set the numeric version of the segment to VERSION
                        (e.g. 1)
  -e EXPLAIN, --explain=EXPLAIN
                        Set the segment_definer:comment to COMMENT. This
                        should explaining WHAT this flag mean (e.g. "Light dip
                        10%"). Required when --Insert/-I is specified.
  -c COMMENT, --comment=COMMENT
                        Set the segment_summary:comment to COMMENT. This
                        should explaining WHY these segments were inserted
                        (e.g. "Created from hveto results")
  -S FILE, --summary-file=FILE
                        Read the segment_summary rows from FILE. This should
                        be a file containing the gps start and end times that
                        the flag was defined, deliminated by comma (i.e. the union of on and off)
  -G FILE, --segment-file=FILE
                        Read the segment rows from FILE. This should containin
                        the gps start and end times when the flag was active deliminated by comma
\end{verbatim}

To insert segments of new segment type, the command would look like:
\begin{verbatim}
ligolw\_segment\_insert --segment-url https://segdb.ligo.caltech.edu --ifos 'H1' --name 'DCH-TEST' --version 1 --comment 'testing if insert works' --explain 'test insert' --segment-file segment.txt --summary-file summary.txt --insert
\end{verbatim}

To append segments to an existing segment, the command would look like:
\begin{verbatim}
ligolw\_segment\_insert --segment-url https://segdb.ligo.caltech.edu --ifos 'H1' --name 'DCH-TEST' --version 1 --comment 'testing if append works' --segment-file append_segment.txt --summary-file append_summary.txt --append
\end{verbatim}
%%%%%%%%%%%%%%%%%%%%%%%%%%%%%%%%%%%%%%%%%%%%%%%%%%
\section{Functionality Supported in S6}
With the above described web browser and command line user interface, users will be able to find out:
\begin{itemize}
\item When was a given flag defined?
\item Was a given flag defined at this time?
\item When was a given flag active?
\item What is the dead time for a given flag? 
\item At a given point of time, is the given flag active? If yes, find out the start and end time of the segment to which the specified point of time belongs.
\item At a given point of time, is the given flag active? If no, find out the end time of the previous adjacent active segment and the start time of the next adjacent active segment.
\item Provide the same supports described above for a set of flags.
\item What flags were active at this time, near this time ($\pm 1$~s interval or specified by user)?
\item What is the efficiency for a flag for burst and inspiral and plot it.
\item Export answer to ascii or xml file
\item Handle padding windows for deadtime queries
\item Export config files, start, end times
\end{itemize}


%%%%%%%%%%%%%%%%%%%%%%%%%%%%%%%%%%%%%%%%%%%%%%%%%
\section{Authentication}
In S6, data insertion requires authentication. Retrieving from on-site requires no authentication. Retrieving from off-site requires authentication. 

\section{Exchange of data with Virgo}

\section{Example Veto Configuration File}

The veto configuration file should contain a \verb|process| table describing
how it was created and a \verb|veto_definer| table describing the flags to be
applied at different levels. The comment column of the process table should
contain the version of the file. The columns in the \verb|veto_definer| table
are as follows: \verb|ifo|, \verb|name| and \verb|version| uniquely define a
particular DQ/veto flag. \verb|category| described which veto category it
should be applied at (i.e. $0, 1, 2, 3, \ldots$), \verb|start_time| and
\verb|end_time| denote the GPS time interval for which the DQ/veto flag should
be applied (Note: if \verb|end_time| is zero, then the current GPS time is
assumed). \verb|start_pad| and \verb|end_pad| are the padding time (in
seconds) applied to the start and end of the veto segments. Note that these
are signed: if you want time vetoed to start time to be \emph{earlier} than
the start time listed in the database, then the emph \verb|start_pad| should
be \emph{negative}. Similarly, if the time vetoed should extend \emph{after}
the end time stored in the database, then the value in \verb|end_pad| should
be \emph{positive}. The \verb|comment| column can be used to store an optional
human-readable comment.

{\tiny
\begin{verbatim}
<?xml version='1.0' encoding='utf-8' ?>
<!DOCTYPE LIGO_LW SYSTEM "http://ldas-sw.ligo.caltech.edu/doc/ligolwAPI/html/ligolw_dtd.txt">
<LIGO_LW>
   <Table Name="process:table">
      <Column Name="process:process_id" Type="ilwd:char"/>
      <Column Name="process:program" Type="lstring"/>
      <Column Name="process:version" Type="lstring"/>
      <Column Name="process:cvs_repository" Type="lstring"/>
      <Column Name="process:cvs_entry_time" Type="int_4s"/>
      <Column Name="process:node" Type="lstring"/>
      <Column Name="process:username" Type="lstring"/>
      <Column Name="process:unix_procid" Type="int_4s"/>
      <Column Name="process:start_time" Type="int_4s"/>
      <Column Name="process:end_time" Type="int_4s"/>
      <Column Name="process:ifos" Type="lstring"/>
      <Column Name="process:comment" Type="lstring"/>
      <Stream Name="process:table" Type="Local" Delimiter=",">
      "process:process_id:0","ligolw_veto_file","1.1",
      "/usr/local/cvs/lscsoft/glue/bin/ligolw_veto_file,v",822908378,
      "ldas-grid.ligo.caltech.edu","jrsmith",16830,822879594,822879594,
      "H1","Example file by Josh"
      </Stream>
   </Table>
   <Table Name="veto_definer:table">
      <Column Name="veto_definer:process_id" Type="ilwd:char"/>
      <Column Name="veto_definer:ifo" Type="lstring"/>
      <Column Name="veto_definer:name" Type="lstring"/>
      <Column Name="veto_definer:version" Type="int_4s"/>
      <Column Name="veto_definer:category" Type="int_4s"/>
      <Column Name="veto_definer:start_time" Type="int_4s"/>
      <Column Name="veto_definer:end_time" Type="int_4s"/>
      <Column Name="veto_definer:start_pad" Type="int_4s"/>
      <Column Name="veto_definer:end_pad" Type="int_4s"/>
      <Column Name="veto_definer:comment" Type="lstring"/>
      <Stream Name="veto_definer:table" Type="Local" Delimiter=",">
      "process:process_id:0","H1","OUT_OF_LOCK",0,1,917985615,0,0,0,"",
      "process:process_id:0","H1","BADGAMMA",1,1,917985615,0,0,0,"",
      "process:process_id:0","H1","ASC_Overflow",0,2,917985615,0,-8,8,"ASC saturations are cat2",
      "process:process_id:0","H1","PD_Overflow",0,2,917985615,0,-8,8,"PD saturations are cat2",
      "process:process_id:0","H1","SEVERE_LSC_OVERFLOW",0,2,917985615,0,-8,8,"LSC saturations are cat2",
      "process:process_id:0","H1","INJECTION",1,2,917985615,0,-16,64,"Remove HW injections at cat1",
      "process:process_id:0","H1","ASC_Overflow",0,3,917985615,0,-8,25,"ASC saturations are cat3 with larger pad",
      "process:process_id:0","H1","SEVERE_LSC_OVERFLOW",0,3,917985615,0,-8,25,"LSC saturations are cat3 with larger pad",
      "process:process_id:0","H1","PD_Overflow",0,3,917985615,0,-8,25,"PD saturations are cat3 with larger pad",
      "process:process_id:0","H1","Wind_over_30MPH",0,3,917985615,0,-8,8,"Windy",
      "process:process_id:0","H1","LIGHTDIP_1_PERCENT",0,3,917985615,0,-2,2,"Exclude all lightdip segments",
      "process:process_id:0","H1","PRE_LOCKLOSS_10_SEC",0,3,917985615,0,0,0,"",
      "process:process_id:0","H1","PRE_LOCKLOSS_30_SEC",0,3,917985615,0,0,0,"",
      "process:process_id:0","H1","PRE_LOCKLOSS_60_SEC",0,3,917985615,0,0,0,"",
      "process:process_id:0","H1","PRE_LOCKLOSS_120_SEC",0,3,917985615,0,0,0,"",
      "process:process_id:0","H1","ASI_CORR_OVERFLOW",0,4,917985615,0,-8,25,"Does this make sense for DC readout",
      "process:process_id:0","H1","LSC_OVERFLOW",0,4,917985615,0,-8,25,"LSC saturations",
      "process:process_id:0","H1","PRE_LOCKLOSS_600_SEC",0,4,917985615,0,0,0,"",
      "process:process_id:0","H1","PRE_LOCKLOSS_1800_SEC",0,4,917985615,0,0,0,""
      </Stream>
   </Table>
</LIGO_LW>
\end{verbatim}
}

\section{Segment Data File}

To be ingested into the segment database, segment data to be must be in the
format described in this section. Data should be in a valid LIGO LW XML file
with \verb|process|, \verb|segment_definer|, \verb|segment_summary| and
\verb|segment| tables. An optional \verb|process_params| table can be used to
store extra metadata about segment generation.

The \verb|process| table should contain the columns given in the XML file
below which describe the name, version, cvs repository and revision of the
program used generate the data. The comment column can be used to add
additional human-readable data. The node, username, unix\_procid, start\_time
and end\_time columns should store metadata describing who ran the process and
where it ran. The ifos column should contain an alphabetical list of all ifo
data using as input to the process. The process\_id column is used to link the
defined process to other rows in the file created by that process.

The \verb|segment_definer| table should contain a definition of the segments
included in the file. The ifos, name and version column should contain the
name and version of the segment. The name should be upper case for all
segments. DMT-derived segments should be prefixed with the string \verb|DMT-|,
KleineWelle derived veto segments should be prefixed with the string
\verb|KWV-|, data derived from the detector state vector channels should be
prefixed with \verb|IFO-|, segments created by the detchar group should be
prefixes with \verb|DCH-| and segments from the Virgo database should be
prefixed with \verb|VDB-|.  The comment column can be used to add a
human-readable description of the segment. The segment\_definer columm is used
to link this type of segment to the intervals in the \verb|segment| and
\verb|segment_summary| tables.

The \verb|segment| table should contain the GPS start and end time when the
segment described in the \verb|segment_definer| table was \emph{active}. The 
\verb|segment_summary| table should contain the GPS start time and end time of
the interval for which the segment is \emph{defined}. This will allow users to
distinguish between the cases where a DQ segment is undefined for a particular
time or simply inactive (i.e. it is \emph{not} windy, as opposed to the wind
monitor being down).

{\tiny
\begin{verbatim}
<?xml version="1.0"?>
<!DOCTYPE LIGO_LW SYSTEM "http://ldas-sw.ligo.caltech.edu/doc/ligolwAPI/html/ligolw_dtd.txt">
<LIGO_LW>
  <Table Name="processgroup:process:table">
    <Column Name="processgroup:process:program" Type="lstring"/>
    <Column Name="processgroup:process:version" Type="lstring"/>
    <Column Name="processgroup:process:cvs_repository" Type="lstring"/>
    <Column Name="processgroup:process:cvs_entry_time" Type="int_4s"/>
    <Column Name="processgroup:process:comment" Type="lstring"/>
    <Column Name="processgroup:process:node" Type="lstring"/>
    <Column Name="processgroup:process:username" Type="lstring"/>
    <Column Name="processgroup:process:unix_procid" Type="int_4s"/>
    <Column Name="processgroup:process:start_time" Type="int_4s"/>
    <Column Name="processgroup:process:end_time" Type="int_4s"/>
    <Column Name="processgroup:process:process_id" Type="ilwd:char"/>
    <Column Name="processgroup:process:ifos" Type="lstring"/>
    <Stream Name="processgroup:process:table" Type="Local" Delimiter=",">
      "SegGener","1.17",
      "/ldcg_server/common/repository_gds/gds/Monitors/SegGener/SegGener.cc\,v",
      865755895,"Segment generation from an OSC condition","granite","jzweizig",718,
      918756928,918836992,"process:process_id:0","H0H1H2"
    </Stream>
  </Table>
  <Table Name="segment_definergroup:segment_definer:table">
    <Column Name="segment_definergroup:segment_definer:process_id" Type="ilwd:char"/>
    <Column Name="segment_definergroup:segment_definer:segment_def_id" Type="ilwd:char"/>
    <Column Name="segment_definergroup:segment_definer:ifos" Type="lstring"/>
    <Column Name="segment_definergroup:segment_definer:name" Type="lstring"/>
    <Column Name="segment_definergroup:segment_definer:version" Type="int_4s"/>
    <Column Name="segment_definergroup:segment_definer:comment" Type="lstring"/>
    <Stream Name="segment_definergroup:segment_definer:table" Type="Local" Delimiter=",">
      "process:process_id:0","segment_definer:segment_def_id:35","H2","DMT-LIGHT",1,
      "H2 Light in arms from h(t) DQ flags",
      "process:process_id:0","segment_definer:segment_def_id:36","H2","DMT-SCIENCE",1,
      "H2 Science mode from h(t) DQ flags",
      "process:process_id:0","segment_definer:segment_def_id:37","H2","DMT-INJECTION",1,
      "H2 Injection mode from h(t) DQ flags",
      "process:process_id:0","segment_definer:segment_def_id:38","H2","DMT-UP",1,
      "H2 calibration OK in from h(t) DQ flags",
      "process:process_id:0","segment_definer:segment_def_id:39","H2","DMT-CALIBRATED",1,
      "H2 Calibration OK from h(t) DQ flags",
      "process:process_id:0","segment_definer:segment_def_id:40","H2","DMT-BADGAMMA",1,
      "H2 Bad gamma in h(t) DQ flags"
    </Stream>
  </Table>
  <Table Name="segmentgroup:segment:table">
    <Column Name="segmentgroup:segment:segment_id" Type="ilwd:char"/>
    <Column Name="segmentgroup:segment:start_time" Type="int_4s"/>
    <Column Name="segmentgroup:segment:end_time" Type="int_4s"/>
    <Column Name="segmentgroup:segment:segment_def_id" Type="ilwd:char"/>
    <Column Name="segmentgroup:segment:process_id" Type="ilwd:char"/>
    <Stream Name="segmentgroup:segment:table" Type="Local" Delimiter=",">
      "segment:segment_id:15",918836961,918836977,"segment_definer:segment_def_id:35",
      "process:process_id:0",
      "segment:segment_id:16",918836976,918836992,"segment_definer:segment_def_id:37",
      "process:process_id:0"
    </Stream>
  </Table>
  <Table Name="segment_summarygroup:segment_summary:table">
    <Column Name="segment_summarygroup:segment_summary:segment_sum_id" Type="ilwd:char"/>
    <Column Name="segment_summarygroup:segment_summary:start_time" Type="int_4s"/>
    <Column Name="segment_summarygroup:segment_summary:end_time" Type="int_4s"/>
    <Column Name="segment_summarygroup:segment_summary:comment" Type="lstring"/>
    <Column Name="segment_summarygroup:segment_summary:segment_def_id" Type="ilwd:char"/>
    <Column Name="segment_summarygroup:segment_summary:process_id" Type="ilwd:char"/>
    <Stream Name="segment_summarygroup:segment_summary:table" Type="Local" Delimiter=",">
      "segment_summary:segment_sum_id:5",918836976,918836992,"",
      "segment_definer:segment_def_id:40","process:process_id:0",
      "segment_summary:segment_sum_id:6",918836976,918836992,"",
      "segment_definer:segment_def_id:39","process:process_id:0",
      "segment_summary:segment_sum_id:11",918836976,918836992,"",
      "segment_definer:segment_def_id:37","process:process_id:0",
      "segment_summary:segment_sum_id:12",918836976,918836992,"",
      "segment_definer:segment_def_id:35","process:process_id:0",
      "segment_summary:segment_sum_id:42",918836976,918836992,"",
      "segment_definer:segment_def_id:36","process:process_id:0",
      "segment_summary:segment_sum_id:46",918836976,918836992,"",
      "segment_definer:segment_def_id:38","process:process_id:0"
    </Stream>
  </Table>
</LIGO_LW>
\end{verbatim}
}

\section{Converting XML to plain ASCII}
ligolw\_print can be used to convert XML file to plain ASCII format. ligolw\_print prints the contents of table elements from one or more LIGO Light Weight XML files to stdout in delimited ASCII format.  In addition to regular files, the
program can read from many common URLs such as http:// and ftp://.  Any filename or URL that ends in ".gz" is assumed to be gzip-compressed, and will be decompressed on input.  If no filenames or URLs are given, then input is read from stdin.


DESCRIPTION:
\begin{verbatim}
  --version             show program's version number and exit
  -h, --help            show this help message and exit
  -i name, --input-cache=name
                        Get URLs from the LAL cache file.  Can be provided
                        multiple times to name several caches to iterate over.
  -c name, --column=name
                        Print only the contents of the given column.  Can be
                        provided multiple times to print multiple columns.
                        The default is to print all columns from each table.
  -d string, --delimiter=string
                        Delimit output with the given string.  The default is
                        ",".
  -r rowspec, --rows=rowspec
                        Print rows in the given range(s).  The format is
                        first:last[,first:last,...].  Rows are numbered from
                        0.  A single first:last pair requests rows in the
                        range [first, last).  If the first or last value of a
                        pair is omited it means 0 or infinity respectively.
                        The default is ":", or to print all rows.
  -t name, --table=name
                        Print rows from this table.  Can be provided multiple
                        times to print rows from multiple tables.  The default
                        is to print the contents of all tables.
  -a name, --array=name
                        Print the contents of this array.  Can be provided
                        multiple times to print the elements from multiple
                        arrays.  The default is to print the contents of all
                        arrays.
  -v, --verbose         Be verbose.
  --constrain-lsc-tables
                        Impose additional constraints on official LSC tables.
                        Provides format validation and allows RAM requirements
                        to be reduced.
\end{verbatim}



\Chapter{Characterization of a Gravitational Wave Detector}
\label{ch:detchar}
\newcommand\weakheader[1]{
\vspace*{5mm}
\noindent {\it #1}
\vspace*{5mm}
}

\newcommand{\darmerr}{\texttt{DARM\_ERR} }

As discussed in section \ref{sec:detection_statistics}, the CBC group
measures the significance of candidate events by comparing the
combined new SNR (eqn.~\ref{eq:new_snr}) against the background
distribution.  The background is obtained by sliding the triggers from
each IFO against each other by more than the light travel time between
them.  In the limit of at most one real gravitational wave in each
analysis segment this ensures that the background is composed entirely
of triggers resulting from non- gravitational-wave sources.

Ideally, the instrumental noise would be Gaussian and the only source
of extraneous triggers would be random fluctuations.  In reality
environmental couplings to the instrument, such as electrical and
seismic, as well as glitches within the instrument will also produce
triggers.

It is in the interest of the collaboration to remove as many of these
triggers as possible so that any real events will stand out well above
the background.  This removal must be done in an automated way in
advance of looking at the results from the analysis.  The LIGO and
Virgo collaborations would be justly criticized if the background were
to be adjusted after the analysis in order to make a marginal
candidate appear more significant.

There is therefore a need for a process of detector characterization
(henceforth ``detchar''), a way to remove times from the analysis that
are more likely to produce triggers from noise than from a real event.
One of the most significant features of the S6 run is that the people
working on detchar were in close contact with the people commissioning
the instrument.  This meant that, in addition to cleaning the search,
information about problematic behavior in the instrument could be used
to fix the problems at the source, in turn allowing more time to be
analyzed and the prospects for making a detection to improve.

Detchar was a major undertaking, much more information can be found in
(Smith and Lundgren in progress).  Three primary tools were run
continuously or with low-latencies to provide views into the data,
although these were supplemented with numerous short special-purpose
scripts.  Two of these tools, the klinewelle pipeline~\cite{} and the
Omega pipeline~\cite{} project data onto a wavelet basis and identify
as triggers times and frequencies with amplitudes above a given
threshold.  Both these tools were run on many auxiliary channels in
addition to the detector output.  The third tool, daily ihope, is the
focus of the current chapter.


\section{Daily ihope}

As discussed in (search chapter) the CBC group uses the \emph{ihope}
pipeline to search for gravitational waves produced by the inspiral of
binary systems consisting of neutron stars and/or black holes.  During
LIGO's 6th science run (July 7, 2009 - October 20, 2010) which
overlapped Virgo's second (July 7, 2009 - January 11, 2010), and third
(August 11, 2010 - October 20, 2010) science runs, the goal of the
group was to run the full analysis every two weeks with latencies as
small as possible.  The daily ihope runs were conceived of as a way to
characterize the detector in order to look for potential problems in
advance of the full search, particularly problems specific to the CBC
search.

It is important to stress that daily ihope was not itself a
gravitational wave search, it was a tool to help determine whether the
data in each instrument was suitable to analyze.  This goal determined
the specifics of the daily runs, but more significantly it leaves the
full search unbiased.

\subsection{The Daily Ihope Pipeline}

Recall that ihope is a templated, matched-filter search.  For the
low-mass search, which was the focus of the daily runs, the templates
are restricted, stationary-phase frequency-domain waveforms with phase
evolution taken to 3.5 PN order.  The templates are laid out in a bank
with 97\% overlap between nearest neighbors, and the mass range is
from 2 $\msun - 25 \msun$.  A $\chisq$ discriminator is constructed to
better separate signals from glitches (eqn.\ref{eq:chisq}), and the
information from this discriminator is used along with the SNR to
construct the new SNR detection statistic (eqn.~\ref{eq:new_snr}).

By construction, a great deal of the information provided by
the full bank search is redundant.  Further, the evaluation of each
additional template and the initial layout of the bank incurs
significant computational overhead.  Therefore a number of
simplifications were applied to the daily ihope bank.

First, a static bank was used for each IFO, based on the layout at a
time in each instrument when the inspiral range was high, no
operational problems were noted by the scimon or operator, and the
Omega, KW, and daily ihope pipelines showed no anomalous behavior.
Second, the number of templates was reduced.  Lower-mass systems have
more cycles in the sensitive LIGO band, there is therefore ample
information to distinguish between waveforms with close parameters.
Conversely, this means that the low-mass end of the bank is very dense
(in the sense of number of templates per unit square in parameter
space, they are still equidistant in the sense of the bank metric.
Such fine resolution is not needed when looking for glitches,
therefore the minimal match between templates in the region below a
chirp mass ($\mathcal{M} = M \eta^{3/5}$) of 3.46 $M_\odot$ was set to
0.5.  At higher masses, up to a total mass of $25 M_\odot$, the
minimal match was set to 0.95.  In addition to ensuring coverage in a
mass region which is naturally sparser this ensures that short
glitches, characteristic of many glitch mechanisms, were flagged with
large SNRs.  The resulting hybrid bank for the Hanford detector is
shown in figure \ref{f:daily_ihope_bank}.

\begin{figure}
  \includegraphics[width=\linewidth]{figures/detchar/hybrid_bank.png}
  \caption[The hybrid template bank used by daily ihope]{
  \label{f:daily_ihope_bank}
The hybrid template bank used by daily ihope for the Hanford detector,
the banks for L1 and V1 are similar.
The cut is made at constant chirp mass, which is
a curve in the total mass, $\eta$ plane.}
\end{figure}%

Daily ihope processing ran at 03:00 GMT and examined data spanning the
24 hours ending at 00:00 GMT.  Unlike full ihope, the analysis was
done on all science time, including that which would be vetoed at CAT
1 by the full analysis.  This was done so that we could see the effect
of the CAT 1 vetoes.  It is plausible, for example, that some such
vetoes would be too aggressive and we could decide based on the daily
results to include such times back into the analysis.  For consistency
science time was denoted CAT 0.  In addition CAT 3 vetoes to remove
hardware injections were not applied, so that we could see how many
injections would be lost by CAT 4 vetoes.  The daily pages therefore
displayed results for categories 0, 1, 2, and 4.

For each 2048-second chunk of contiguous data, at each IFO, the data
was run through each template in the bank and triggers selected as
described in section \ref{sec:analysis_trigger_selection}. For each
trigger $\chisq$ and new SNR were calculated and recorded.  Clustering
was then done in order to focus attention on glitches that were most
likely to cause problems in the full search.  Two sets of clustered
triggers were recorded using windows of 30 milliseconds and 16
seconds.  For each set the trigger with the largest value of new SNR
across all templates within the window length was recorded.

The triggers were also filtered.  A version of the veto definer file
designated as ``online'' was maintained and used by daily ihope to
identify times with veto categories.  These vetoed times were then
removed from the original and clustered files.  
% The pipeline is summarized in figure (xx).
The result is, for each 2048-second block in each ifo, 3 cluster
levels times 4 veto levels = 12 sets of triggers.

The structure of the daily pipeline is shown in figure \Note{make
this}.  For comparison we include the structure of the full search,
shown in figure~\ref{f:hipe}~\footnote{Figure by Collin Capano, used
with permission}.  We will not go over it in detail here, but note
that it is a two-stage, coincident pipeline.

\begin{figure}
  \includegraphics[width=\linewidth]{figures/detchar/HIPEDiagram}
  \caption[Structure of the full iHope search]{
  \label{f:hipe}
Structure of the full iHope search.  For details see (\Note{Collin's
thesis}).
}
\end{figure}%

%do analysis
%record science triggers
%cluster 30 ms     determine cat 1
%cluster 16 sec    determine cat 1 + cat 2
%                  ...

%filter and record

\section{The Daily Ihope report pages}

Much of the utility of the daily runs was in the use of the resulting
triggers to identify data quality issues and quantify the value of
proposed vetoes.  Some of these uses will be discussed in
section~\ref{sec:applications_vetoes}.  In addition to the triggers
the daily pipeline also produced a large number of plots and reports,
organized into web pages that were available to the collaboration.
Data analysts could use these pages to spot potential problems for the
full analysis and begin more detailed followup studies.  

We now summarize the contents of the daily pages. Each report and plot
is made for each combination of the following:

\begin{itemize}
\item IFO: H1, L1 and V1
\item Cluster level: Unclustered, 30 ms clustering, 16 second
clustering
\item Veto level: Show all triggers in science time (level ``0''),
only those not removed by category 1 vetos, only those not
removed by categories 1 or 2, or only those note removed by
categories 1,2 and 4.  Hardware injections vetos (category 3) were not 
applied so that we could determine whether category 4 vetoes were 
remove them.
\end{itemize}

Each of these features could be set independently.  The top-level web
interface for a sample day is shown in figure~\ref{f:daily_ihope_top}.

\begin{figure}
  \includegraphics[width=\linewidth]{figures/detchar/daily_ihope_top}
  \caption[Top-level web interface for daily ihope]{
  \label{f:daily_ihope_top}
Top-level web interface for daily ihope.  Options can be selected with
the controls on the left hand side, when a button is clicked the
contents region is immediately replaced.  The default view is the one
shown here; the analysis time and veto efficiency at cat 1 for the
Hanford detector.}
\end{figure}%

The available reports can bee seen at the bottom of the list of
controls in figure~\ref{f:daily_ihope_top}:

\weakheader{1. Analysis time and veto usage.}

This report shows the total time analyzed, the vetoes applied beyond
those applied at the previous level, the \emph{efficiency} (percentage
of triggers removed) and \emph{deadtime} (percentage of time removed)
by each veto.  In addition the ratio of efficiency over deadtime is
reported as a measure of quality of the veto.  A random veto would
result in an efficiency-over-deadtime of approximately 1, a
finely-tuned veto that removes short, loud events that ring off the
entire template bank would have a much higher ratio.   An example is
shown in figure~\ref{f:daily_ihope_vetousage}.


\begin{figure}
  \includegraphics[width=\linewidth]{figures/detchar/vetousage.png}
  \caption[Sample veto usage report from Aug 19, 2010]{
  \label{f:daily_ihope_vetousage}
A sample veto usage report, see text for explanation.  Note that 62.66\% of
all triggers were contained in 14.51\% of time, and the DMT flagged this time
with elevated seismic activity from 3-10 Hz at the LVEA.}
\end{figure}%

\weakheader{2. Loudest triggers}

This report was only run for 16-second-clustered triggers.  Loud
glitches tend to produce families of triggers, without clustering the
loudest triggers from each day would likely result from one underlying
event.  This report considers two classes of triggers; those where no
data quality flag was active and those where at least one flag was
active.  A sample report is shown in
figure~\ref{f:daily_ihope_loudest}.  For the five triggers with
highest new SNR in each category a summary was presented along with a
link to an \emph{omega scan}.  The omega pipeline is described in
(\checkme{omega refs}).  Omega scans are a kind of time-frequency
plot.  They are especially useful as they may be run on all auxiliary
channels recorded by the instruments and therefore provide a visual
aid to detecting coupling between auxiliary channels and the
gravitational wave readout channel.  These can be used to suggest
mechanisms behind glitches, especially those that were not already
marked by a DQ flag.  In addition, over time repeated shapes in the
omega scan can pinpoint underlying problems in the instruments that
need to be addressed.  A few images from one of the glitches in
figure~\ref{f:daily_ihope_loudest} is shown in
figure~\ref{f:daily_ihope_loudest_omega}.

\begin{figure}
  \includegraphics[width=\linewidth]{figures/detchar/loudest.png}
  \caption[Sample loudest trigger report from Aug 19, 2010]{
  \label{f:daily_ihope_loudest}
A sample report on loudest triggers, see text for explanation.  Note the
rightmost column is populated by
the \texttt{ligolw\_dq\_query} program in the \texttt{--report} mode.}
\end{figure}%



\begin{figure}
  \includegraphics[width=0.5\linewidth]{figures/detchar/966222827_940673828_H1_LSC-DARM_ERR_1_00_spectrogram_whitened.png}
  \includegraphics[width=0.5\linewidth]{figures/detchar/966222827_940673828_H0_PEM-ISCT1_ACCX_1_00_spectrogram_whitened.png}
  \caption[Omega scans from the loudest H1 trigger in figure \ref{f:daily_ihope_loudest}]{
  \label{f:daily_ihope_loudest_omega}
Omega scans from the loudest H1 trigger in figure
\ref{f:daily_ihope_loudest}.
Note the trigger closely follows a loud
event in one of the accelerometers on an instrument table.}
\end{figure}%


\weakheader{3. Hardware injections}

This report was generated by code written by John Veitch  at Cardiff
University.  It compared the list of injections, published as an XML
file available from a web site, with the list of analysis times and
triggers.  The results were plotted to indicate whether each injection
was found, missed, or not analyzed.


\weakheader{4. SNR histograms}

These plots show the number of triggers as a function of SNR and new
SNR.  As noted in section \ref{sec:ihope_match_filter}, in Gaussian
noise the number of triggers should be proportional to
$\exp(-\rho^2/2)$.  These plots therefore show the degree of
``non-Gausianity'' in the data.  A common use for these plots was to
flip between veto levels to get a sense of how well the cumulative
vetoes were cleaning the data, as shown in figure
\ref{f:daily_ihope_gaussianity}.

Similarly figure~\ref{f:daily_ihope_gaussianity_newsnr} shows histograms
of the triggers in new SNR.  The total number of triggers is greatly
reduced because many high-SNR triggers have $\newsnr < 5$ , where the
plot cuts off.  While triggers with such low $\newsnr$ values are not
excluded from later stages of analysis in the full pipeline, it is
extremely unlikely that the resulting combined new SNR will be high
enough to stand above background.

In addition to the lower number of triggers overall, the new SNR plot
is closer to a straight line indicating behavior closer to that
expected in Gaussian noise.  However, detchar improves the situation
still further, as the number of triggers is reduced at veto category
4.

However, there is one trigger with $\newsnr = 8$ that is not removed
by any veto.  Such an outlier warrants further investigation, which
here is provided by the loudest events page an example of which is
shown in~\ref{f:daily_ihope_loudest}.  The omega scan from the time of
this event is shown in figure~\ref{f:daily_loudest_glitch}, and it
clearly rules out a gravitational wave as the source of this trigger.
This illustrates the important point that new SNR can be fooled, and
hence continued human participation in detchar is necessary.

\begin{figure}
  \includegraphics[width=0.5\linewidth]{figures/detchar/H1_1_UNCLUSTERED_snr_hist.png}
  \includegraphics[width=0.5\linewidth]{figures/detchar/H1_4_UNCLUSTERED_snr_hist.png}
  \caption[Trigger SNR histograms for H1]{
  \label{f:daily_ihope_gaussianity}
Sample trigger histograms by SNR.  The dashed line shows the
expected values in Gaussian noise.  The dot indicates the cumulative
number of triggers with new SNR greater than 200.  Note that at veto
category 4 (right) the histogram is closer to the expected line than
it is at category 1 (left).  This indicates the degree to which data
quality has removed non-Guassian noise.}
\end{figure}%



\begin{figure}
  \includegraphics[width=0.5\linewidth]{figures/detchar/H1_1_UNCLUSTERED_new_snr_hist.png}
  \includegraphics[width=0.5\linewidth]{figures/detchar/H1_4_UNCLUSTERED_new_snr_hist.png}
  \caption[Trigger new SNR histograms for H1]{
  \label{f:daily_ihope_gaussianity_newsnr}
Sample trigger histograms by new SNR. The data is much cleaner than
the SNR histograms, but there is still an outlier at $\newsnr=8$.}
\end{figure}%



\begin{figure}
  \includegraphics[width=\linewidth]{figures/detchar/966281824_963867187_H1_LSC-DARM_ERR_1_00_spectrogram_whitened}
  \caption[Omega scan of the loudest new SNR trigger]{
  \label{f:daily_loudest_glitch}
The omega scan from the outlier with new SNR=8 that survives all
automated data quality vetoes.  Many auxiliary channels showed the
same behavior at this time.  This is clearly not a gravitational wave,
but was not removed by either signal-based vetoes or data quality
vetoes.}
\end{figure}%


\weakheader{5. The ``glitchgram''}

This was an ``at-a-glance'' summary of the day, showing every trigger
color-coded by SNR.  This highlighted times of loud triggers as well
as dense regions indicating ``grumbly'' times.  A sample is shown in
figure \ref{f:daily_ihope_glitchgram}.

\begin{figure}
  \includegraphics[width=\linewidth]{figures/detchar/H1_1_UNCLUSTERED_glitchgram.png}
  \caption[The Aug 19th daily ihope ``glitchgram'']{
  \label{f:daily_ihope_glitchgram}
A sample ``glitchgram.'' Blue dots have
new snr values below 8, green have values between 8 and 16,
and red have values above 16.  Template length was chosen as the Y
axis in order to capture a feature of the templates that is not
specific to gravitational wave signals, such as chirp mass.
Note the break at 4.3 seconds, corresponding to the
chirp mass at which the bank switches from an overlap of 0.95 to 0.5.
Comparing to figures 3 and 4 shows the same 
excess of triggers after 12:00.  No data is analyzed after 21:00
because ihope requires at least 2048 contiguous seconds to estimate
the PSD and all data after this time was in smaller segments.}
\end{figure}%


\weakheader{6. Rate vs. Time and SNR vs. Time}

These plots complimented the glitchgram by breaking the triggers up
differently.  The rate plot (figure~\ref{f:daily_ihope_rate_v_time})
showed average number of triggers over 1-minute intervals.  The SNR
plot (figure~\ref{f:daily_ihope_snr_v_time}) showed the SNR of every
trigger as a function of time.  These tend to be correlated, as loud
glitches ring off the entire bank and produce large numbers of
triggers.  The plots were accompanied by tables showing the times
where the rate of triggers exceeded 500 Hz for more than one second,
and exceeded 200 Hz for more than 10 seconds.

\begin{figure}
  \includegraphics[width=\linewidth]{figures/detchar/H1_0_UNCLUSTERED_rate_vs_time.png}
  \caption[Daily ihope rate plot]{
  \label{f:daily_ihope_rate_v_time}
Sample rate plot, showing rate in Hz (averaged over
1-minute intervals) as a function of time.  Note that the rates
increase after 12:00.  This was due to increased seismic noise.}
\end{figure}%

\begin{figure}
  \includegraphics[width=\linewidth]{figures/detchar/H1_0_UNCLUSTERED_snr_vs_time.png}
  \caption[Daily ihope SNR plot as a function of time]{
  \label{f:daily_ihope_snr_v_time}
Sample plot of SNR as a function of time.  The density of
triggers increases somewhat after 12:00 and there are more loud
outliers.  However the change in behavior is better seen in 
figure \ref{f:daily_ihope_rate_v_time}.}
\end{figure}%


\weakheader{7. Breakdown by template}

This page showed several histograms of number of triggers as a
function of the length of templates in seconds.  Examples of the
standard histograms are shown in
figure~\ref{f:daily_ihope_trig_histograms}, they show that most of the
triggers come from short templates, and templates shorter than 5
seconds produce up to six times as many triggers as shorter templates.

Figure~\ref{f:count_per_template} shows the same information as a map
of the template bank, color-coded to indicate how many triggers each
template produced.  Figure~\ref{f:daily_ihope_time_mass} breaks this
same information up by time and template mass.

Qualitatively these plots do not change much day-by-day, and so these
plots were not often used.  However, they did prompt a change to
the analysis made early in the S6 run, see
section~\ref{sec:applications_pipeline}.

\begin{figure}
  \includegraphics[width=0.5\linewidth]{figures/detchar/H1_1_UNCLUSTERED_mass_hist}
  \includegraphics[width=0.5\linewidth]{figures/detchar/H1_1_UNCLUSTERED_mass_hist_norm}
  \caption[Histograms of trigger rates by template length]{
  \label{f:daily_ihope_trig_histograms}
Histograms of trigger rates by template length in daily ihope.  The
plot on the left combines all templates, the plot on the right
normalizes by plotting (number of triggers resulting from templates of
length $x$) divided by (number of templates of length $x$).
}
\end{figure}%

\begin{figure}
  \includegraphics[width=\linewidth]{figures/detchar/H1_1_UNCLUSTERED_template_counts}
  \caption[Triggers per template]{
  \label{f:count_per_template}
The daily template bank, color-coded to show how many triggers each
template produced.  Note that the template in the upper-right corner,
which is the template of shortest duration, produced approximately 1.3
times as many triggers as the next most active.
}
\end{figure}%


\begin{figure}
  \includegraphics[width=\linewidth]{figures/detchar/H1_1_UNCLUSTERED_hexmass}
  \caption[Trigger rates as a function of time and template length]{
  \label{f:daily_ihope_time_mass}
Trigger rates as a function of time and template length.  The elevated
trigger rate after 12:00 is visible here as well.  Note that
particularly loud glitches, such as that around 19:30, ring off the
entire bank.
}
\end{figure}%


\weakheader{8.  The $\chisq$ test}

These plots showed all triggers with SNR values on the $x$-axis and
$\chisq / (2p-2)$ (the reduced $\chisq$) on the $y$-axis as a way to
visualize the glitchiness of the data.  These plots are most useful
when compared to a reference plot generated from a day of simulated
Gaussian noise, shown in figure \ref{f:gaussian_snr_chisq}, and
comparing the plots at different veto levels, shown in figure
\ref{f:daily_ihope_snr_chisq}.

\begin{figure}
\includegraphics[width=0.85\linewidth]{figures/detchar/GAUSSIAN_0_UNCLUSTERED_chisq.png}
\caption[SNR/reduced $\chisq$ values in Gaussian noise]{
\label{f:gaussian_snr_chisq}
The SNR/reduced $\chisq$ plane for a reference day of
Gaussian noise.  This is the product of a $\chisq$ distribution on the
$y$ axis and a Gaussian distribution on the $x$ axis.}
\end{figure}%

\begin{figure}
  \includegraphics[width=0.5\linewidth]{figures/detchar/H1_0_UNCLUSTERED_chisq.png}
  \includegraphics[width=0.5\linewidth]{figures/detchar/H1_4_UNCLUSTERED_chisq.png}
  \caption[SNR/reduced $\chisq$ plots of H1 data.]{
  \label{f:daily_ihope_snr_chisq}
SNR/reduced $\chisq$ plots of H1 data.  The expected shape
of figure \ref{f:gaussian_snr_chisq} is discernible, but there are
long tails of non-Gaussian glitches.  The sharp cutoffs arise 
from thresholds within the inspiral code, see section
\ref{sec:analysis_trigger_selection}.  There is a further
population extending to the upper right at category 0 (left) that is
removed by vetoes in category 4 (right).}
\end{figure}%


\section{Applications of daily ihope to pipeline tuning}
\label{sec:applications_pipeline}

Early in S6 there were frequent instances where the full analysis ran
into difficulty.  This was characterized by individual programs taking
abnormally long to complete, consuming far more than the expected
amount of memory, or failing outright.  The problematic jobs tended to
be individual runs of the \texttt{trigbank} program (see
figure~\ref{f:hipe}); this is the step where triggers from the first
stage are examined to determine which templates need to be used at the
second stage (see figure~\ref{f:hipe}).  Comparing the times that
caused problems to the daily pages immediately revealed a correlation
-- times over which \texttt{trigbank} were unable to run were those
where the rates of triggers were abnormally high.

In S5 and the early weeks of S6 triggers were clustered using a method
called \emph{trigscan}~\cite{SenguptaTrigScan:2008} which attempts to
collapse clusters of triggers that are close in time and parameters to
a single most-significant trigger.  Early versions of the daily page
did this clustering as well, and comparison between the unclustered
and trigscan-clustered triggers revealed that even after clustering
periods of high trigger rates remained~\footnote{Trigscan worked well
in S5, it is not known why it did not work as well in S6.  It is
possible that the instruments were simply more glitchy in the early
days of S6.  However, in order to group triggers together trigscan
must use the bank metric, which was correct in S5 when 2.0 pN
templates were used but incorrect in S6 when the analysis moved to 3.5
pN templates (see section~\ref{sec:bank_metric}).  Some preliminary
investigations were performed, but results were inconclusive}.  This
is illustrated in figure~\ref{f:daily_ihope_high_rates}.  There is a
short period of high trigger rate around 05:40 which remained high
after clustering.  A run of \texttt{trigbank} processing this time was
unable to complete.

The possibility of vetoing such glitchy periods was raised, and this
could have been easy to accomplished using the rate information from
daily ihope.  However, such vetoes would have needed to be category 1 to
avoid the problem, which would mean subdividing science segments and
possibly losing short segments.  Instead we replaced trigscan with
fixed 30-millisecond clustering windows, after studies of found/missed
injections determined that this change did not harm the search
efficiency. 

\begin{figure}
  \includegraphics[width=0.5\linewidth]{figures/detchar/20090806_H1_0_UNCLUSTERED_rate_vs_time}
  \includegraphics[width=0.5\linewidth]{figures/detchar/20090806_H1_0_TS_CLUSTERED_rate_vs_time}
  \caption[Problematic times identified by daily ihope]{
  \label{f:daily_ihope_high_rates}
Problematic times identified by daily ihope.  The plot on the left
shows the rate of triggers over the day without any clustering.  Note
the sudden jump in rate at 5:40.  On the right, the same plot after
clustering triggers with the trigscan algorithm.  The overall rate has
been reduced by about a factor of two, indicated by the different
scales on the y axes.  However, the rate remains high enough to cause
problems.
}
\end{figure}%

At the start of S6 the range of the low-mass search extended up to $35
\msun$, as it did in S5.  However, along with times of high trigger
rates, the daily ihope pages indicated that most of the triggers were
coming from the high-mass end of the bank.  This behavior is expected,
as it is easier for short templates to match against glitches, but the
trigger histograms highlighted the extent to which this was a problem.
A sample plot of triggers per template from early S6 is shown on the
left of figure~\ref{f:daily_ihope_el_glitcho}.  It shows that the
shortest template in the bank, with $m_1 = m_2 = 17.5 \msun$ was
producing significantly more triggers than any other template.  In
part this was due to a bug in the template bank code, that caused this
template to appear twice in some banks.  In part the abundance of
triggers from this template is due to it appearing in every bank
throughout the day, whereas other templates tend to get repositioned
as the noise curve changes.  Even taking these effects into account,
most of the triggers come from templates shorter than 5 seconds, as
seen on the right of figure~\ref{f:daily_ihope_el_glitcho}.

The fixed clustering window means that only the loudest trigger in a
30-millisecond window will be passed to subsequent stages of the
analysis.  Given the numbers of triggers from short templates there
was concern that a loud, short glitch could mask a quieter trigger
from a binary neutron star coalescence.  We therefore decided to limit
the low-mass search to $M < 25 \msun$ after comparing found/missed
plots of injections in the same weeks with the different mass ranges,
and determining that the smaller bank did as well or better than the
larger bank.


\begin{figure}
  \includegraphics[width=0.5\linewidth]{figures/detchar/20090806_H1_0_UNCLUSTERED_template_counts}
  \includegraphics[width=0.5\linewidth]{figures/detchar/20090806_H1_0_UNCLUSTERED_mass_hist_norm}
  \caption[Problematic templates seen in daily ihope]{
  \label{f:daily_ihope_el_glitcho}
Problematic templates seen in daily ihope.  This shows the equivalent
of figures~\ref{f:daily_ihope_time_mass} and
\ref{f:count_per_template} from an earlier version of daily ihope and
an earlier version of the CBC search.  Note the excess of triggers
from the high-mass end of the bank, corresponding to shorter
templates.
}
\end{figure}%



\section{Applications of daily ihope to individual vetoes}
\label{sec:applications_vetoes}

In addition to tracking the performance of the pipeline as a whole,
daily ihope was used to flag numerous glitch mechanisms throughout the
course of S6.  Such glitches typically showed up in the rate vs. time
plot, the SNR vs. time plot, and/or the list of loudest triggers.
Often these were spotted either by individuals reviewing the pages on
a daily basis, by an individual in the CBC group assigned  to do
glitch studies preceding a run of the full analysis pipeline, or by
the group during reviews of these studies.

As patterns emerged from these results members of the detector
characterization group would work with the commissioners to identify
the underlying causes, and simultaneously work with data analysts to
develop and test vetoes.  Often these vetoes were verified by running
a script, \texttt{lalapps\_check\_flag} which, given a file
containing veto segments and a time range would report on the
efficiency and deadtime of the veto by applying it to the daily ihope
triggers over the specified time range.  The file could be generated
from an existing flag in the database using {\texttt
ligolw\_dq\_query} (see chapter~\ref{ch:segdb}), or it could be
generated by a script looking at auxiliary channel information
provided by Omega or klinewelle triggers.

We now give examples of the use of daily ihope in constructing vetoes
at each level.

\subsection{Category 1}

\weakheader{Glitches from the Thermal Compensation System}

The design of the LIGO mirrors accounted for the fact that in
operation they would be heated by the laser and the radius of
curvature would change.  However, in S6 as we moved to higher laser
power the mirror's absorption of energy was larger than expected and
the mirror overheated and deformed more than expected~\cite{}.  To
correct for this a compensating laser was added that would heat the
mirror in a ring in order to restore the radius of curvature to an
optimal value.  However, this laser would occasionally glitch, kicking
the mirror and producing loud noise in the detector.

The identification of these glitches was straightforward; they were
often the loudest glitches of the day and were readily visible in the
daily ihope ``loudest trigger'' report, as well as similar reports
generated by KW and Omega.  The cause was similarly easy to identify,
as the omega scans generated by daily ihope showed egregious behavior
in the TCS channels.  An automated veto based on a threshold value in
the TCS channel was added to the DMT at category 2.  However, some
instances were loud enough to interfere with the CBC pipeline in
surrounding times, hence it was decided to veto these times at
category 1.  These features are illustrated in
figure~\ref{f:daily_ihope_tcs}.  


%
% P. Willems, "Thermal Compensation in the LIGO Gravitational-Wave
% Interferometers," in Adaptive Optics: Methods, Analysis and
% Applications, OSA Technical Digest (CD) (Optical Society of America,
% 2009), paper AOThA5.
% http://www.opticsinfobase.org/abstract.cfm?URI=AO-2009-AOThA5
% 

\begin{figure}
  \includegraphics[width=0.5\linewidth]{figures/detchar/20100217_H1_0_UNCLUSTERED_snr_vs_time} 
  \includegraphics[width=0.5\linewidth]{figures/detchar/20100217_H1_0_UNCLUSTERED_rate_vs_time} \\
  \includegraphics[width=0.5\linewidth]{figures/detchar/950432156_482177734_H1_LSC-DARM_ERR_4_00_spectrogram_whitened}
  \includegraphics[width=0.5\linewidth]{figures/detchar/950432156_482177734_H1_TCS-ITMY_PD_ISS_OUT_AC_4_00_spectrogram_whitened}
  \caption[TCS glitch in daily ihope and omega]{
  \label{f:daily_ihope_tcs}
TCS glitches in daily ihope and followup omega scans.  The top left
shows the SNR as a function of time, the loudest triggers are all TCS
glitches.  Note that each loud glitch comes in the form of a ``tower''
containing many triggers.  This phenomena will be discussed further in
section~\ref{ssec:penguins}.  This is further illustrated by the trigger
rate plot (top right), which shows elevated trigger rates at the times
of the glitch.  Note that around the time of the glitch the trigger
rate actually {\it drops} significantly.  This will be discussed in
section~\ref{ssec:sarlacc}, but indicates that the glitches are loud
enough to interfere with the analysis in surrounding times, justifying
the use of a CAT 1 veto.  The lower left plot shows the omega scan
of the gravitational-wave output channel at the time of the loudest
glitch, clearly showing that this is not a gravitational wave.  The
lower right plot shows the omega scan of a channel monitoring the TCS,
identifying it as the source of the glitch.}
\end{figure}%


\weakheader{Grid glitches}

This was a loud, broad-band glitch that produced a grid of triggers in
the KW pipeline's time-frequency plane, where the phenomena was first
seen.  They also showed up in the daily ihope rate and SNR plots as
loud triggers accompanied by elevated trigger rates, as shown in
figure~\ref{f:omega_grid}.  These also often appeared on the
daily list of loudest glitches, the omega scans generated by daily
ihope showed problems in magnetometers and the output mode cleaner
photodiodes, this is shown in figure~\ref{f:omega_grid}.  As with the
TCS glitches, instances that showed up particularly loudly in daily
ihope were hand-vetoed at category 1.  The problem was eventually
traced to a power supply, once this was resoldered the problem
disappeared.

% https://wiki.ligo.org/foswiki/bin/view/DetChar/CurrentUnvetoedGlitchClasses
% https://wiki.ligo.org/DetChar/H1GridGlitchStory

% Example: 
% tconvert 963852915
% Jul 22 2010 16:55:00 UTC

\begin{figure}
  \includegraphics[width=0.5\linewidth]{figures/detchar/20100722_H1_0_UNCLUSTERED_newsnr_vs_time}
  \includegraphics[width=0.5\linewidth]{figures/detchar/20100722_H1_0_UNCLUSTERED_rate_vs_time}
  \caption[Grid glitches in daily ihope] {
  \label{f:daily_ihope_grid}
Grid glitches as seen by daily ihope.  The plot on the left shows
$\newsnr$, with the grid glitches indicated by arrows pointing above
the range at which the plot cuts off (auto-scaling of this plot was
removed from daily ihope to prevent a single loud glitch from
squeezing all the other data to a thin region at the bottom of the
plot).  Note that the high value indicates that this glitch is fooling 
the $\chisq$ signal-based veto.  The plot on the right shows the
associated increase in trigger rate.
}
\end{figure}%

\begin{figure}
  \includegraphics[width=\linewidth]{figures/detchar/963853263_154296875_H1_LSC-DARM_ERR_1_00_spectrogram_whitened}
  \includegraphics[width=0.5\linewidth]{figures/detchar/963853263_154296875_H0_PEM-BSC1_MAG1X_1_00_spectrogram_whitened}
  \includegraphics[width=0.5\linewidth]{figures/detchar/963853263_154296875_H1_OMC-QPD3_SUM_IN1_DAQ_1_00_spectrogram_whitened}
  \caption[Grid glitches in omega]{
  \label{f:omega_grid}
Omega scans of the grid glitch flagged as the loudest trigger of the
day and visible as an arrow in figure~\ref{f:daily_ihope_grid}.  At
the top, the gravitational-wave output channel, and at the bottom two
channels that were associated with this class of glitch.
}
\end{figure}%


\subsection{Category 2}

\weakheader{The spike glitch}

The spike glitch was a very loud, very short ``bang'' in the
Livingston detector, characterized by a sudden drop in \darmerr
immediately followed by a sharp increase.  An example is shown in
figure~\ref{f:spike_glitch_example}.  In some cases there were a few
cycles of ringing before the channel settled back down.  These often
produced SNR values of several thousand.  The $\chisq$ values were
typically large for the events themselves, resulting in negligible
$\newsnr$ values.  However, the triggers in the tails (see
section~\ref{ssec:penguins}) could have large new SNR values.  A
sample daily ihope plot with several spike glitches and representative
omega scans are shown in figure~\ref{f:daily_spikes}.

\begin{figure}
  \includegraphics[width=\linewidth]{figures/detchar/spike_glitch_example}
  \caption[The spike glitch]{
  \label{f:spike_glitch_example}
A sample spike glitches in \darmerr.  Note the characteristic
down-up-down pattern, and short timespan.}
\end{figure}%

Despite a great deal of effort by lots of people, the cause of these
glitches was never identified.  However, the distinctive shape allowed
the creation of an automated veto.  \texttt{LSC\_STRAIN} was sampled
every half-millisecond, giving a time series $x_i$.  For each sample
$i$ the quantity 

\begin{equation*}
x_i - 2 x_{i+1} + x_{i+2} 
\end{equation*}

was calculated, values exceeding a threshold indicated a possible
spike and the time was flagged in the segment database.  Not every
time so-flagged was a spike, but every instance indicated a
potentially problematic rapid fluctuation in the data.



% https://wiki.ligo.org/foswiki/bin/view/DetChar/S6L1OMCGlitchVeto

% DCH-SPIKE_GLITCH (0,4)

% Example 947204646.611816406
% Jan 11 2010 00:23:51.611816406 UTC == SNR 8849


\begin{figure}
  \includegraphics[width=\linewidth]{figures/detchar/20100111_L1_0_UNCLUSTERED_snr_vs_time} \\
  \includegraphics[width=0.5\linewidth]{figures/detchar/947204646_611816406_L1_LSC-DARM_ERR_1_00_spectrogram_whitened}
  \includegraphics[width=0.5\linewidth]{figures/detchar/947204646_611816406_L1_LSC-DARM_ERR_1_00_timeseries_raw}
  \caption[Spike glitch in daily ihope and omega]{
  \label{f:daily_spikes}
  Spike glitches in daily ihope and a typical omega scan.  At the top
a daily ihope SNR-vs-time plot, showing several spike glitches with
SNRs of several thousand.  On the bottom, the loudest of these spike
glitches in an omega scan, showing the time-frequency plot (left) and
unfiltered time series (right).  The sharp drop-rise-drop of the time
series behavior was reasonable well-captured by the filtered used to
create data quality segments.
}
\end{figure}%

\weakheader{HEPI glitches}

Livingston is subject to seismic activity, both natural and
anthropogenic, that does not effect the Hanford detector.  In order
for the L1 detector to achieve the same low-frequency response it is
therefore necessary to take additional steps.  The Hydraulic External
Pre-Isolation (\emph{HEPI}) system sits between the ground and
chambers containing optical components and provides a layer of active
seismic isolation.  This can reduce noise in the 1-3 Hz band by a few
percent~\cite{Wen:thesis}.

In early 2010 it was noticed that the daily ihope triggers contained
several instances with SNRs above 600 even after applying all known
vetoes through category 4.  This included removal of spike glitches
and application of the use-percentage vetoes (discussed in the next
section).  Followup of these triggers in the daily loudest-glitch
reports showed that they were all accompanied by loud glitches in the
HEPI channels.  Samples are shown in figure~\ref{f:daily_hepi}.

A trial veto was created by scanning the auxiliary channel time series
for values exceeding 25,000.  This veto was tested by applying it to
the remaining daily ihope triggers, and it was found to be very
effective.  Histograms of triggers before and after application of
this veto are shown in figure~\ref{f:hepi_veto_effectiveness}.  After
this confirmation the veto was applied at category 2 to the full
search.

A great deal of work was done on HEPI during this period, and after
Jan 10, 2010 the problem ceased.

% DCH-SEI_ITMY_Y_OVER_25000
% example 946834715.499755859 == Jan 06 2010 17:38:20 UTC
% https://ldas-jobs.ligo.caltech.edu/~cbc/ihope_daily/201001/20100106/
% 5th loudest in day
%
% https://wiki.ligo.org/foswiki/bin/view/DetChar/CBC_S6B_L1_SingleDetectorLoudest
% After applying CAT 2, 3, and 4, including UPV and the Spike glitch
% veto, there are still many loud triggers in the CBC single-detector
% background. This does not include the effect of chi squared or
% coincidence. The triggers above SNR 600 are: 
%
% https://ldas-jobs.ligo.caltech.edu/~lundgren/DataQuality/Issues/HEPI/L1-S6B2-snrhist_Cat1234_Spike_HEPI.png
%
% http://www.ligo.caltech.edu/LIGO_web/0409news/0409liv.html


\begin{figure}
  \includegraphics[width=0.5\linewidth]{figures/detchar/20100106_L1_0_UNCLUSTERED_snr_vs_time}
  \includegraphics[width=0.5\linewidth]{figures/detchar/946834715_499755859_L1_LSC-DARM_ERR_1_00_spectrogram_whitened} \\
  \includegraphics[width=0.5\linewidth]{figures/detchar/946834715_499755859_L1_SEI-ITMY_Y_1_00_timeseries_raw}
  \includegraphics[width=0.5\linewidth]{figures/detchar/946834715_499755859_L1_SEI-ITMY_Y_1_00_spectrogram_whitened}
  \caption[The HEPI glitch in daily ihope and Omega]{
  \label{f:daily_hepi}
The HEPI glitch in daily ihope and Omega.  On the top left, the SNR
values for the day including a HEPI glitch at 17:38, as identified by
the ``loudest glitches'' report.  On the top right, this event in
\darmerr.  On the bottom left the time-frequency plot of a HEPI
auxiliary channel, and on the bottom right the unfiltered time series
of this same channel, showing that it has exceeded the threshold of
25,000}
\end{figure}%

\begin{figure}
  \includegraphics[width=\linewidth]{figures/detchar/L1-S6B2-snrhist_Cat1234_Spike_HEPI}
  \caption[Effectiveness of the HEPI veto] {
  \label{f:hepi_veto_effectiveness}
Effectiveness of the HEPI veto. The green line shows the number of
daily triggers after applying all known vetoes, including the removal
of spike glitches.  The blue line show counts after the additional
HEPI veto is applied.  Note that the highest SNR is reduced from 
4,000 to 600, along with a reduction of triggers at SNRs down to 
$~ 10$.}
\end{figure}


\subsection{Category 4}

% https://wiki.ligo.org/foswiki/bin/view/DetChar/H1LVEASEISZS6Cflag

\weakheader{Automated vetoes}

There were three uses of the daily triggers through S6.  The first
two, the daily pages and ad-hoc veto studies, have already been
discussed.  The third was as input to systematic automated veto
studies.

\emph{SeisVeto}~\cite{} compared low-frequency triggers from Omega
running on seismic PEM channels to daily ihope triggers using the
\emph{HVeto}~\cite{} algorithm.  Times of high correlation were
flagged with category 4 vetoes, which could achieve up to 62\%
efficiency with only 6\% deadtime.

The \emph{Used Percentage Veto} (UPV) program looked for correlations
between triggers from \darmerr and auxiliary channels using a figure
of merit defined as

\begin{equation*}
\textrm{Used Percentage}(\rho) \equiv 100 \times
N_\textrm{coinc}(\rho) / N_\textrm{total}(\rho)
\end{equation*}

where $N_\textrm{total}(\rho)$ is the total number of triggers from
the auxiliary channel above significance threshold $\rho$ in the
analysis time, and $N_\textrm{coinc}(\rho)$ is the number of trigger
from the auxiliary channel that lie within a second of a trigger from
\darmerr.  Intervals where this percentage exceeded 50\% were entered
into the segment database and applied as category 4 vetoes.  Although
UPV was originally developed using KW triggers for both \darmerr and
auxiliary channels, it was modified during S6 to accept triggers from
daily ihope.  The resulting analysis was rerun weekly.


% UPV-IHOPE_H1_ASC_WFS2_IY_8_256_100 (in C) and more
% (tconvert 957191412, May 06 2010 14:29:57 UTC)
% Accompanied by rate > 500 Hz (=866) for 4 seconds, slight elevation in SNR

An example is shown in figure~\ref{f:daily_upv}.  During the week
including this day UPV determined that the instrumental channel
\texttt{ASC-WFS2\_IY} \Note{brief description} had a high correlation
with daily triggers.  This is a quiet glitch, with an SNR of 6.5, and
it would have taken considerable effort to track down manually.
However, it is accompanied by 4 seconds of elevated trigger rates
which could have impacted the search and should therefore be flagged
at category 4.

\begin{figure}
  \includegraphics[width=0.5\linewidth]{figures/detchar/20100506_H1_0_UNCLUSTERED_rate_vs_time}
  \includegraphics[width=0.5\linewidth]{figures/detchar/20100506_H1_0_UNCLUSTERED_snr_vs_time} \\
  \includegraphics[width=0.5\linewidth]{figures/detchar/957191413_957191417_H1_LSC-DARM_ERR_1_00_spectrogram_whitened}
  \includegraphics[width=0.5\linewidth]{figures/detchar/957191413_957191417_H1_ASC-WFS2_IY_1_00_spectrogram_whitened}
  \caption[Glitch found by UPV using the daily triggers]{
  \label{f:daily_upv}
A glitch found by UPV using the daily triggers.  On the top the daily
rate and SNR plots, including the glitch in question at 14:29.  On the
bottom the omega scan of \darmerr and the auxiliary channel whose KW
triggers correlate with daily ihope darmerr triggers.}
\end{figure}%

\weakheader{Loud SNR}

Through the S6 run the \emph{horizon distance}, the distance at which
the coalescence of an optimally-oriented binary system consisting of
two $1.4 \msun$ neutron stars would have an SNR of 8, was roughly 30
Mpc.  The SNR scales inversely with distance, hence the distance at
which we would expect to see such a system at, say, SNR 250 is 0.12
Mpc.  Assuming uniform volume distribution, this makes an SNR 250
event $6.4 \times 10^{-8}$ times less likely than an SNR 8 event.  
However, such loud glitches do occur in the data fairly often,
in particular the spike glitch at L1.

Such loud glitches tend to have high $\chisq$ values which suppress
them.  However, around SNR 250 glitches tend to be accompanied by
additional triggers spanning $\pm 8$ seconds resulting from the
interaction of the filter with the inverse spectrum truncation, see
section \ref{sec:daily_ihope_open_questions}.  Some of these auxiliary
triggers can, by chance, have low $\chisq$ values and hence high new
SNR values, and can potentially interfere with the search.  This
suggests a CAT 4 veto centered on times of triggers with SNRs
exceeding 250 with 8 seconds of padding in both directions.  Such a
flag must be in place before the full run, making daily ihope the
obvious choice for generating the flags.

This scheme was implemented starting on June 26, 2010, coinciding with
the portion of the run designated S6D.  An example of its
effectiveness is shown in table \ref{tab:daily_ihope_loud_flag} which
shows the efficiency and deadtime off the flag applied to triggers
from the full analysis after CAT 1, for the two weeks containing the
blind injection.  The efficiency to deadtime ratio is greater than 1,
although still relatively small.  Still, the flag was deemed useful as
it removed triggers we could not have easily claimed were due to a
gravitational wave.

\begin{table*}
\begin{center}
\begin{tabular}{lrrcrrcc}
\hline
ifo & Triggers & Vetoed & Efficiency & Time
& Vetoed & Dead-  & Ratio \\
 & (Count) & (Count) &  & (sec) & (sec) & time (sec) &  \\
\hline
L1  & 2890507 & 12578 & 0.43 & 798720 & 880 & 0.11 & 3.95 \\
H1  & 1692904 &  6452 & 0.38 & 647168 & 416 & 0.06 & 5.92 \\
\end{tabular}
  \caption[Effectiveness of the ``SNR $>$ 250'' flag]{
  \label{tab:daily_ihope_loud_flag}
Effectiveness of the ``SNR $>$ 250'' flag over the weeks
09/04/2010  to 09/17/2010.  Note that the flag was used almost twice
as often in L1 as H1, although there is only 23\% more analysis time.
This is another indication that L1 was glitchier overall.}
\end{center}
\end{table*}



\subsection{Non-vetoed time}

It is at least as important to preserve time in which a detection
could be made as it is to veto problematic time.  On occasion there
were potential problems seen in results from another pipeline or
reported by scimons or operators that daily ihope confirmed were not a
problem for the CBC low-mass search (though they might have been
problematic for other searches).  One example is shown in
figure~\ref{f:non_vetoed_time}.  The omega pipeline reports excess
noise at 200Hz, which was traced to runoff from a nearby dam.  The
corresponding daily ihope report shows no issues with excess or loud
triggers, and the day was able to be analyzed.

\begin{figure}
  \includegraphics[width=0.5\linewidth]{figures/detchar/20100604_H1_0_100MILLISEC_CLUSTERED_snr_vs_time}
  \includegraphics[width=0.5\linewidth]{figures/detchar/S6-H1-omega-959644815-959731215-GlitchTS}
  \caption[Use of daily ihope to indicate no veto needed] {
  \label{f:non_vetoed_time}
Use of daily ihope to indicate that time did not need to be vetoed.
The omega pipeline (left) shows a distinct feature at 200 Hz, within
the sensitive band of LIGO.  Daily ihope shows no excess of triggers
or SNR, and hence the time did not need to be vetoed}
\end{figure}%


\section{Applications of daily ihope to the blind injection challenge}
\label{sec:applications_applications_dog}


In addition to the hardware injections discussed in section
\ref{sec:ihope_hardware_injections} it was known at the start of S6
that there would be any from zero to ``a few'' unannounced, blind
hardware injections performed in order to provide an unbiased test of
the search pipelines.  One such injection was performed on Sep 16
2010 at 06:42 UTC, and showed up in multiple searches as a strong
gravitational wave candidate.  This candidate was followed up to the
point of writing a detection paper and submitting it to the
collaboration for publication approval.  Once approval had been
granted the fact that it had been an injection was revealed.  For more
details on the event and how it was followed up, see (the S6 low mass
paper).

Although daily ihope was not a search, the injection showed up on the
page of loudest triggers for H1 and L1, with parameters shown in table
\ref{tab:daily_ihope_dog} and omega scans shown in figure
\ref{f:daily_ihope_dog_omega}.  The injection was not visible in V1 in
daily ihope.

\begin{landscape}
\begin{table*}
\begin{center}
\begin{tabular}{lllllllllll}
\hline
ifo & end\_time & end\_time\_ns & SNR & $\chisq$ & New SNR & Mchirp & DQ flags \\
\hline
H1  & 968654557 & 997314453 & 15 & 87 & 9.8 & 4.6 & DMT-INSPIRAL\_RANGE\_STDEV\_GT\_0P50\_MPC \\
    &           &           &    &     &    &     & \hspace*{0.5 in} [968654544 968654560) \\
    &           &           &    &     &    &     & DMT-INSPIRAL\_RANGE\_STDEV\_GT\_0P75\_MPC \\
    &           &           &    &     &    &     & \hspace*{0.5 in} [968654544 968654560) \\
    &           &           &    &     &    &     & Light,Up,Calibrated,Science \\
\hline
L1 & 968654557 & 978027343 & 9.9 & 44 & 8.7 & 4.1 & SCI-OTHER\_ELOG [967120215 977875215) \\
    &           &           &    &     &    &     & DMT-INSPIRAL\_RANGE\_STDEV\_GT\_1\_MPC \\
    &           &           &    &     &    &     & \hspace*{0.5 in} [968654544 968654560) \\
    &           &           &    &     &    &     & DMT-INSPIRAL\_RANGE\_STDEV\_GT\_0P50\_MPC \\
    &           &           &    &     &    &     & \hspace*{0.5 in} [968654544 968654560) \\
    &           &           &    &     &    &     & DMT-INSPIRAL\_RANGE\_STDEV\_GT\_0P75\_MPC \\
    &           &           &    &     &    &     & \hspace*{0.5 in} [968654544 968654560) \\
    &           &           &    &     &    &     & SCI-FLAG\_ERROR \\
    &           &           &    &     &    &     & \hspace*{0.5 in} [967137346 977875215) \\
    &           &           &    &     &    &     & Light,Up,Calibrated,Science \\
\hline
\end{tabular}
\end{center}
  \caption[Recovered blind injection parameters]{
  \label{tab:daily_ihope_dog}
The blind injection as reported by daily ihope's ``loudest triggers'' page.}
\end{table*}
\end{landscape}


\begin{figure}
  \includegraphics[width=0.5\linewidth]{figures/detchar/968654557_997314453_H1_LSC-DARM_ERR_1_00_spectrogram_whitened.png}
  \includegraphics[width=0.5\linewidth]{figures/detchar/968654557_978027343_L1_LSC-DARM_ERR_1_00_spectrogram_whitened.png}
  \caption[Omega scans of the injection]{
  \label{f:daily_ihope_dog_omega}
H1 (left) and L1 (right) Omega scans of the injection as
generated by daily ihope.  Note the ``chirp'' shape which is the
expected pattern from a compact binary inspiral.}
\end{figure}%


\subsection{False Alarm Rate Estimate}

The potential detection occurred close to the end of S6 as the
collaboration was preparing to take the LIGO detectors apart 
to install advanced LIGO.  The shutdown could potentially have been
delayed if the existing configuration were necessary to vet
the candidate.  An ad-hoc committee was formed to determine whether
this would be required, and while they found no immediate reason to
delay the advanced LIGO plans the report did include several
recommendations, including:

% https://dcc.ligo.org/cgi-bin/private/DocDB/ShowDocument?docid=22752

\begin{quote}
The committee urges a search in the S6 and S5 data for events with
similar waveforms but lower SNR than the September 16 event in single
detectors as well as in dual detector coincidence. Such an
investigation has a dual purpose. First, it will establish some limits
on the astrophysical source population producing the September 16
event and second, it will help in estimating the false alarm rate,
although this will be better accomplished with more time
slides~\cite{Weiss:injection}.
\end{quote}

The existing daily ihope triggers were ideal for this purpose, as they
spanned all of S6 but with a reduced set of templates.  This sampled the
parameter space but produced a smaller set of triggers that made
rapid significance studies computationally feasible.

% https://www.lsc-group.phys.uwm.edu/ligovirgo/cbcnote/DailyDogHistograms

\iffalse
In order to determine the significance of this candidate event it was
necessary to compare it against the background.  The first such
comparison ranked the event against background triggers from the
two-week analysis in which it occurred as part of the standard ihope
pipeline.  However, the event had a larger combined new SNR value than
all background triggers, and hence had a false alarm probability of
zero.

In order to provide a more meaningful bound it was necessary to
increase the analysis time and/or number of slides to probe the
background more deeply. This is a complex and time-consuming process,
see (s6 paper) for details.  While this was underway we could begin to
bound the significance from daily ihope results.  This was done by
plotting histograms of all triggers throughout S6 and locating the
candidate triggers in the resulting distribution.  This analysis
differs from the full ihope pipeline in several respects.  However,
the goal was not a publishable result but only a rapid estimate.
\fi

The significance was estimated by plotting histograms of all triggers
throughout S6 and locating the candidate triggers in the resulting
distribution.  To parallel the full analysis the results were broken
into mass bins.  The low mass bin spans chirp masses up to
$3.48\msun$, the medium mass bin from $3.48-7.40 \msun$.  Likewise,
category 1,2 and 3 vetoes were applied to parallel the results of the
full search.  30-millisecond clustering was chosen to parallel the
clustering used in the full search.  There are several ways of
reporting the new SNR of the injection; the largest values reported by
daily ihope, the largest single-detector values reported by the full
search, and the component values of the largest combined new SNR
reported by the full search.  All of these options are included on the
plot.

The results are shown in figure \ref{f:daily_histogram_low} for the
low-mass bin and \ref{f:daily_histogram_medium} for the medium-mass
bin.  The result in both bins is qualitatively the same.  The
injection is close to the loudest event in H1 for all measures of new
SNR.  The injection does not stand out as far in L1, which was known
to be glitchier over the course of S6.


\begin{figure}
  \includegraphics[width=0.5\linewidth]{figures/detchar/LM_H1_30MILLISEC_3_hist.png}
  \includegraphics[width=0.5\linewidth]{figures/detchar/LM_L1_30MILLISEC_3_hist.png}
  \caption[Significance of the injection in the low-mass bin]{
  \label{f:daily_histogram_low}
Significance of the blind injection in the low-mass bin in H1 (left)
and L1 (right). Results are shown as cumulative histograms.  The plots
flatten out at low SNRs due the selection of triggers from the SNR
time series, discussed in
section~\ref{sec:analysis_trigger_selection}.  The red lines are the
new SNR reported by the full ihope run (after coincidence).  The green
lines are the loudest trigger in new SNR found at the first stage of
the ihope analysis (before coincidence).  The blue lines are the new
SNR values of the injection found by daily iHope (no coincidence).}
\end{figure}%



\begin{figure}
  \includegraphics[width=0.5\linewidth]{figures/detchar/MM_H1_30MILLISEC_3_hist.png}
  \includegraphics[width=0.5\linewidth]{figures/detchar/MM_L1_30MILLISEC_3_hist.png}
  \caption[Significance of the injection in the medium-mass bin]{
  \label{f:daily_histogram_medium}
Significance of the blind injection in the medium-mass bin in 
H1 (left) and L1 (right).  Note the cumulative counts levels off
around between 5 and 5.5, indicating that there are few triggers with
smaller values.}
\end{figure}%

\iffalse
From these results we can attempt to estimate a false alarm rate for
the injection as follows.   Model coincident triggers as a Bernoulli
trial where ``success'' is obtaining a coincidence with combined new
SNR greater than or equal to that of the injection.  Denote the
probability of success in a single trial as $P$.  Then the probability
of obtaining the first success after $k$ trials is a geometric
distribution, $Prob(k) = P(1-P)^{k-1}$, and the expected number of
trials before success is $1/P$.  Dividing this by the estimated number
of coincident triggers in a year gives the estimated number of years
required to obtain such a trigger by chance.

The rate of coincident triggers, $R$, in the full search was estimated
by choosing a few analysis chunks and dividing the number of H1L1
triggers by the analysis time, both of which are reported after the
coincidence step.  The average rate is approximately 0.004
coincidences per second of analysis time, or $N=126,144$ per year.
This is combined across all mass bins, the result will therefore be an
upper limit for any particular bin.

% CIT:
% /archive/home/sprivite/S6/lowmass/s6d_weeks33_34/968803143-970012887/full_data
% for i in `/bin/ls | grep H1L1-COIRE_FIRST_FULL_DATA `
% do
%  grep 'amount of time analysed' $i | cut -f7 -d' '
% done | awk '{s+=$1} END {print s}'
% 135254
%
% for i in `/bin/ls | grep H1L1-COIRE_FIRST_FULL_DATA `
% do
%  grep 'reconstructed' $i | cut -f7 -d' '
% done | awk '{s+=$1} END {print s}'
%
% 523
% so 523 / 135254 = 0.00387 coinc/sec
%
%
% LHO (dog)
% /archive/home/mtwest/CBC-s6d/weeks_31-32/lowmass_run/967593543-968803287/full_data
% Gives 
% 886 / 244515 = 0.00362 coinc/sec


We estimate $P$ by assuming a probability density function in each
detector of the form

\begin{equation}
P(\rho_\textrm{new}) = \left\{
  \begin{array}{lr}
    0  & \rho < 5.5 \\
    \exp\left(-\frac{\rho^2}{2\sigma^2}\right) & \rho_\textrm{new} \geq 5.5 \\
  \end{array} \right.
\end{equation}

The lower cutoff is approximate.  We do not threshold on new SNR, and
while we do threshold on$\rho > 5.5$ it is possible for $\chisq$ to
push the resulting new SNR down.  In addition, due to clustering, the
probability of obtaining low new SNR triggers is suppressed.  The fit
to the Gaussian portion of the curves is show on  figures
\ref{f:daily_histogram_low} and \ref{f:daily_histogram_medium}, and
the obtained values are shown in table \ref{tab:daily_ihope_sigmas}.

\begin{table*}
\begin{center}
\begin{tabular}{l | l l}
   & $\sigma_H$ & $\sigma_L$ \\
\hline
Low mass bin    & 0.98 & 0.96 \\
Medium mass bin & 0.99 & 0.97 \\
\end{tabular}
\end{center}
  \caption[Fit values for SNR histograms]{
  \label{tab:daily_ihope_sigmas}
$\sigma$ values obtained by fitting Gaussians to daily
ihope trigger counts.}
\end{table*}

The joint PDF is then

\begin{equation}
P(\rho_H,\rho_L) = \frac{1}{A}
\int_{\rho_L = 5.5}^{\sqrt{\rho_i^2 - \rho_L^2}}
\int_{\rho_H = 5.5}^{\sqrt{\rho_i^2 - 5.5^2}}
\exp\left(-\frac{\rho_H^2}{2\sigma^2}\right)
\exp\left(-\frac{\rho_L^2}{2\sigma^2}\right)
\,d\rho_H
\,d\rho_L
\end{equation}

where $A$ is the normalization obtained by taking the upper limits of
both integrals to infinity.  This may be simplified by means of the
substitutions

\begin{align*}
s      &= \frac{\rho_H}{\sqrt{2}\sigma_H} \\
s_{\mathrm{low}}  &= \frac{5.5}{\sqrt{2}\sigma_H} \\
s_{\mathrm{high}} &= \frac{\sqrt{\rho_i^2 - 5.5^2}}{\sqrt{2}\sigma_H} \\
t      &= \frac{\rho_L}{\sqrt{2}\sigma_L} \\
t_{\mathrm{low}}  &= \frac{5.5}{\sqrt{2}\sigma_L} \\
t_{\mathrm{high}}(s) &= \frac{\sqrt{\rho_i^2 - 2 \sigma_H^2 s^2}}{\sqrt{2}\sigma_L} \\
\end{align*}

in terms of which the normalization is

\begin{equation}
A = \frac{\pi}{2} \sigma_x \sigma_y \left[
1 - \erf(s_{\mathrm{low}}) - \erf(t_{\mathrm{low}}) + \erf(s_{\mathrm{low}})\erf(t_{\mathrm{low}})
\right]
\end{equation}

where $\erf$ is the error function.  The probability of obtaining a
trigger with combined new SNR larger than the injection is then

\begin{align*}
P &= 1 - \frac{\pi\sigma_L \sigma_H}{2 A} \bigg[
\frac{2}{\pi} \int_{s_{\mathrm{low}}}^{s_{\mathrm{high}}} e^{-s^2}
\erf(t_{\mathrm{high}}(s))\,ds \nonumber \\
&\quad - \erf(t_{\mathrm{low}}) \erf(s_{\mathrm{high}})  
+ \erf(s_{\mathrm{low}}) \erf (t_{\mathrm{low}}) \bigg] \\
\end{align*}


\iffalse
\begin{equation}
P(\rho_H,\rho_L) = 
\frac{4}{2\pi \sigma_H \sigma_L} 
\exp\left(
-\frac{\rho_H^2}{2\sigma_H^2} -\frac{\rho_L^2}{2\sigma_L^2}
\right)
\end{equation}

where the factor $4$ comes from considering only the quadrant where
both SNRs are positive.

The probability of obtaining a trigger with new SNR greater than
that of the injection, $\rho_i$, is then

\begin{equation}
P = 
\frac{4}{2\pi \sigma_H \sigma_L} 
\int_{\rho_H^2 + \rho_L^2 > \rho_i^2}
\exp\left(
-\frac{\rho_H^2}{2\sigma_H^2} -\frac{\rho_L^2}{2\sigma_L^2}
\right)
\,d\rho_H\,d\rho_L
\end{equation}

Introducing a temporary variable $M = \rho_L \sigma_H/\sigma_L$, going to polar
coordinates, evaluating the integral over $r$ and simplifying gives


\begin{equation}
P(\rho_c^2 > \rho_i^2|\textrm{coincidence}) = 
\frac{1}{2\pi \sigma_H \sigma_L} 
\frac{\sigma_L}{\sigma_H}
\int_{\rho_H^2 + (\sigma_L/\sigma_H)^2 M^2 \leq \rho_i^2}
\exp\left(
-\frac{\rho_H^2}{2\sigma_H^2} -\frac{M^2}{2\sigma_H^2}
\right)
\,d\rho_H\,dM
\end{equation}



\begin{equation}
P = 
\frac{1}{2\pi \sigma_H \sigma_L} 
\frac{\sigma_L}{\sigma_H}
\int_{r^2\cos^2\theta + (\sigma_L/\sigma_H)^2 r^2\sin^2\theta \leq \rho_i^2}
\exp\left(
-\frac{r^2\cos^2\theta}{2\sigma_H^2} -\frac{r^2\sin^2\theta}{2\sigma_H^2}
\right)
r\, dr\, d\theta
\end{equation}


\begin{equation}
P = 
\frac{1}{2\pi \sigma_H \sigma_L} 
\frac{\sigma_L}{\sigma_H}
\int_{r^2\cos^2\theta + (\sigma_L/\sigma_H)^2 r^2\sin^2\theta \leq \rho_i^2}
\exp\left( -\frac{r^2}{2\sigma_H^2} \right)
r\, dr\, d\theta
\end{equation}


\begin{equation}
P = 
\frac{1}{2\pi \sigma_H \sigma_L} 
\frac{\sigma_L}{\sigma_H}
\sigma_H^2
\int_0^{2\pi}
\left( 1 - 
\exp\left(
  -\frac{1}{2\sigma_H^2} 
   \frac{\rho_i^2}
        {\cos^2\theta + (\sigma_L/\sigma_H)^2 \sin^2\theta} 
\right)
\right)
d\theta
\end{equation}

\begin{equation}
P = \frac{4}{2\pi}
\int_0^{\pi/2}
\exp\left(
  -\frac{1}{2} 
   \frac{\rho_i^2}
        {\sigma_H^2 \cos^2\theta + \sigma_L^2 \sin^2\theta} 
\right)
d\theta
\end{equation}

\fi

This integral can be evaluated numerically.  Henceforth we focus on
the medium mass bin, as that was the bin with the most significant
trigger with a combined new SNR $\rho_i = 12.5$.  The result is
$P = 1.4 \times 10^{-20}$, which gives a false alarm rate of
1 event in 

\begin{equation}
\frac{1.0}{1.4 \times 10^{-26}} \times \frac{1}{126,144}
= 8.6\times 10^{24}\quad\textrm{years}
\end{equation}

This grossly underestimates the FAR calculated using time slides based
on the full analysis, which gives 1 in 7,000 years.  There is no
simple factor that explains this discrepancy; the two analyses are
significantly different that it is difficult to reason about the
results of the full analysis based on the output of single-stage
single-ifo triggers.  In particular the two-stage nature of the full
analysis introduces several complication.  However, this analysis does
suggest an alternative method to estimate FARs for a single-stage
pipeline which is currently in development.
\fi


\subsection{Front-end code verification}

% http://www.gravity.phy.syr.edu/dokuwiki/doku.php?id=larne:frontendcode
% https://dcc.ligo.org/cgi-bin/private/DocDB/ShowDocument?docid=39122
% https://dcc.ligo.org/cgi-bin/private/DocDB/ListBy?authorid=272

Before the collaboration could claim a detection it was necessary to
perform extensive checks to remove, or at least reduce, the
possibility that the trigger was due to any source other than a
gravitational wave.  Consequently many components of the
interferometer were subject to scrutiny.  One such component was the
\emph{front-end control code}, which is responsible for \Note{FILL
IN DETAILS}.  This code is updated occasionally as new systems are
added or bugs are found and fixed.   To verify that the most recent
change preceding the event did not significantly change the behavior
of the instruments we compared histograms of triggers from daily ihope
before and after these changes.

Two weeks prior to and following the most recent code changes at each
site were selected.  SNR histograms are shown in figure
\ref{f:code_changes}.  There is a slight variation in H1, somewhat
larger in L1.  More rigorous testing could have been done, such as
estimating the standard deviation in each SNR bin from several sample
times before the code change and then checking whether the rates after
the change fall within one sigma.  However, the detection committee
did not feel this level of analysis was necessary, and based on the
plots in figure \ref{f:code_changes} concluded:

\begin{figure}
  \includegraphics[width=0.5\linewidth]{figures/detchar/frontendtest_h1_log_2.png}
  \includegraphics[width=0.5\linewidth]{figures/detchar/frontendtest_l1_log_2.png}
  \caption[SNR histograms before and after code changes.] {
  \label{f:code_changes}
SNR histograms comparing periods before and after 
front-end code changes at H1 (left) and L1 (right).}
\end{figure}%

% https://dcc.ligo.org/cgi-bin/private/DocDB/ShowDocument?docid=39122

\begin{quote}
  Thus, to the extent allowed by the methods we adopted, there is no
  evidence for any malfunction in the front-end code of the
  interferometers~\cite{Whitcomb:injection}. 
\end{quote}



\section{Open Questions}
\label{sec:daily_ihope_open_questions}

Two potentially problematic features of the analysis were noticed in
daily ihope over the course of S6, both related to the effect of loud
glitches on the match filter.  Of course the ultimate goal is to
remove such glitches at the source.  However, it is likely that such
glitches will continue to be present in the advanced LIGO era and the
search pipeline must be robust against them.  We note here the
problems and some initial studies, but more research will be needed to
resolve them before advanced LIGO comes on-line around 2015.


\subsection{Excess triggers produced by loud glitches}
\label{ssec:penguins}

From figure~\ref{f:hepi_veto_effectiveness} we see that applying a
veto to loud glitches does not only remove loud triggers, but also
numerous triggers with lower SNRs.  This same effect may be seen by
comparing rate-vs-time and SNR-vs-time plots such
as~\ref{f:daily_ihope_grid}; loud glitches correlate with an increase
in trigger rates.  In part this behavior is expected.  Many glitches,
notably the spike glitch, are sharp enough that they may be roughly
modeled as impulses.  The impulse response of the match filter
(eqn.~\ref{eq:InnerProduct}) is the time-reversed template convolved
with a function of the noise curve.  A loud glitch will therefore
elevate the SNR time series for every template in the bank.

Recall from chapter~\ref{ch:search} that triggers are selected from
the SNR time series by finding the largest value above threshold in a
sliding window.  The length of the window is taken to the length of
the template, defined as the time required for the frequency of the pN
waveform to go from 40 Hz to infinity.  The selection is done using
the following algorithm: 

\begin{alltt}
for each sample point j
  if \(\rho\sb{j}\) > threshold
    if there is no event yet
      event\_start = j
      event\_snr   = \(\rho\sb{j}\)
    else if \(\rho\sb{j}\) > event\_snr
      event\_start = j
      event\_snr = \(\rho\sb{j}\)
    else if (j - event\_start) == template length
      record event
      event\_start = j
      event\_snr   = \(\rho\sb{j}\)
\end{alltt}

Based on this we would expect an impulse in the data to produce one
trigger per template at approximately the time of the impulse.
However this would not account for the number of templates seen or the
length of time for which the trigger rate is elevated. 

To study this in more detail simulated Gaussian noise was produced (as
in the NINJA project) and the value of a single sample was increased to
simulate a sharp glitch.  The triggers produced for two values of the
glitch amplitude are shown in
figure~\ref{f:impulses_original_no_chisq}.

\begin{figure}
  \includegraphics[width=0.5\linewidth]{figures/detchar/raw1_1e-17}
  \includegraphics[width=0.5\linewidth]{figures/detchar/raw1_1e-15}
  \caption[Response of the template bank to an impulse] {
  \label{f:impulses_original_no_chisq}
Trigger SNRs as a function of time as the template bank responds to an
impulse in the data.  Color is the length of the template in seconds.
On the left a single sample has been set to $10^{-17}$, and on the
right $10^{-15}$.  The expected response is visible, but there is a
large number of additional triggers arranged in distinct features.
See the text for discussion.
}
\end{figure}%

The expected behavior is seen in the rainbow ``tower'' at the top of
both plots, successively longer templates have lower SNRs and trigger
slightly later.

Below this in both plots there is a ``plateau'' of triggers from short
templates.  This results from the inverse spectrum truncation,
described more fully in~\cite{findchirp}.  This behavior can be
understood qualitatively as follows.

For simplicity denote the square root of the inverse PSD,
$(S_n(|f|))^{-1/2}$ as $\tilde{S}(f)$, and its inverse Fourier
transform in the time domain as $S(t)$.  Likewise, denote the
frequency-domain template as $\tilde{h}(f)$ as usual, and its inverse
Fourier transform as $h(t)$.  Finally, let $W(t)$ be a windowing
function in the time domain with Fourier transform $\tilde{W}(f)$.  In
addition, denote multiplication of function by $\cdot$ and convolution
by $\star$.

The application of the inverse spectrum truncation then proceeds as
follows 

1. Calculate $\tilde{S}(f)$ and from it $S(t)$.

2. Apply the window in the time domain, giving $S(t) \cdot W(t)$.

3. Return to to the frequency domain, giving $\tilde{S}(f) \star
\tilde{W}(f)$.

4. Square this (and correct the normalization, not shown here) giving 

\begin{equation*}
(\tilde{S}(f) \star \tilde{W}(f)) \cdot (\tilde{S}(f) \star \tilde{W}(f)) 
\end{equation*}

This replaces the $S_n(|f|)$ in the denominator of the match filter.

If the signal is then a delta function with strength $M$, $s(t) = M
\delta(t)$ then $\tilde{s} = M$  and the SNR time series the then
given by the matched filter,


\begin{align*}
\rho^2(t) &= \int df\, e^{-2 i\pi i f t} M \tilde{h}^\star(f) \cdot
(\tilde{S}(f) \star \tilde{W}(f)) \cdot 
(\tilde{S}(f) \star \tilde{W}(f)) \\
&= M h(-t) \star
(S(t) \cdot W(t)) \star
(s(t) \cdot w(t))
\end{align*}

Note that if the window function is zero outside a region then the
elevated SNR from an impulse will likewise be bounded in time.  This
is the motivation for the truncation; without it a loud glitch would
corrupt an entire analysis segment.  However, when $M$ becomes large
the SNR value within the bounded region may be sufficiently large to
exceed the trigger threshold.  If the length used when scanning the
SNR time series for triggers is less than the width of the truncation
window then a loud glitch can produce several triggers.  This explains
both the width of the plateaus in
figure~\ref{f:impulses_original_no_chisq} and why they are composed of
triggers from short templates.  This may also be seen in the SNR time
series shown in figure~\ref{f:short_snr_series}.

\begin{figure}
  \includegraphics[width=0.5\linewidth]{figures/detchar/snrs_17_short}
  \includegraphics[width=0.5\linewidth]{figures/detchar/snrs_15_short}
  \caption[SNR time series of a short template and loud impulse] {
The SNR time series of a short (0.3 s) template responding to an
impulse of strength $10^{-17}$ (left) and $10^{-15}$ (right).  The
elevation within the truncation window is clear, and is long enough to
produce several triggers.
  \label{f:short_snr_series}
}
\end{figure}%

The plateau persists as the strength of the impulse is increased.
Beyond a certain point we also get a second ``rainbow'' of triggers
from templates across the bank, seen to the right in
figure~\ref{f:impulses_original_no_chisq}.  Note that the difference
between triggers of the same color is precisely the length of
template, identified by the same value in the colorbar.  As the
impulse-response of the filter is the time-reversed template, a
sufficiently loud impulse will elevate the SNR for the length of the
template, independently of how the spectrum is truncated.   This can
be seen in the SNR time series of a long template, shown in
figure~\ref{f:long_snr_series}.


\begin{figure}
  \includegraphics[width=0.5\linewidth]{figures/detchar/snrs_17_long}
  \includegraphics[width=0.5\linewidth]{figures/detchar/snrs_15_long}
  \caption[SNR time series of a long template and loud impulse] {
  \label{f:long_snr_series}
The SNR time series of a long (45 s) template responding to an
impulse of strength $10^{-17}$ (left) and $10^{-15}$ (right).  The SNR
is elevated over the length of the template. \Note{TODO: triggers are
not lining up with the time series: check code.}
}
\end{figure}%

This second set of triggers from long templates suggests that the
length used when scanning the time series for is too short.  This,
combined with the excess of short triggers within the truncation
window suggests replacing this length with the larger of the length of
the truncation window or 1.1 times the currently used length (the
exact value to be obtained by further investigation).

So far we have not used the $\chisq$ test.  When $\chisq$ is
enabled the trigger clustering algorithm is modified as follows


\begin{alltt}
for each sample point j
  if \(\rho\sb{j}\) > threshold
    if \(\chisq\sb{j}\) < chisq\_thresh * (1 + \(\rho\sp{2}\) * chisqfac ):
      if there is no event yet
        event\_start = j
        event\_snr   = \(\rho\sb{j}\)
      else if \(\rho\sb{j}\) > event\_snr
        event\_start = j
        event\_snr = \(\rho\sb{j}\)
      else if (j - event\_start) == template length
        record event
        event\_start = j
        event\_snr   = \(\rho\sb{j}\)
\end{alltt}

That is, an additional constraint is placed on triggers even before
new SNR is calculated.  We would expect that this would quash many,
and hopefully all, triggers resulting from the glitch.  The results of
rerunning with $\chisq$ enabled are shown in
figure~\ref{f:impulses_original_chisq}.

\begin{figure}
  \includegraphics[width=0.5\linewidth]{figures/detchar/delta_chisq_1e-17}
  \includegraphics[width=0.5\linewidth]{figures/detchar/delta_chisq_1e-15}
  \caption[Triggers produced by loud impulses with $\chisq$ enabled] {
  \label{f:impulses_original_chisq}
Triggers produced by impulses of strength $10^{-17}$ (left) and
$10^{-15}$ (right) when the $\chisq$ test is enabled.  $\chisq$
removes many triggers, but many remain.
}
\end{figure}%

Turning on $\chisq$ does remove many triggers, in particular those in
the original tower resulting from long templates.  However, the
triggers from short templates remain.  This indicates that the
$\chisq$ test is not as effective on short waveforms, which is to be
expected.  The louder impulse on the right no longer produces the
second rainbow of triggers.  However, there are now long-template
triggers in the plateau, and a set of low-SNR, loud-template triggers
resulting from the quieter impulse.  The reason for these is not
clear, but they are have the potential to raise the background of the
binary neutron-star search and are therefore problematic.

Despite passing the $\chisq$ clustering it is unlikely that these
triggers, especially the ones from long templates, have good $\chisq$
values.  We can attempt to remove them by altering the clustering
algorithm as follows:


\begin{alltt}
for each sample point j
  if \(\rho\sb{j}\) > threshold
    if there is no event yet
      event\_start = j
      event\_snr   = \(\rho\sb{j}\)
    else if \(\rho\sb{j}\) > event\_snr
      event\_start = j
      event\_snr = \(\rho\sb{j}\)
    else if (j - event\_start) == template length
      if \(\chisq\sb{event_start}\) < chisq thresh * ( 1 + \(\rho\sp{2}\sb{event\_start}\) * chisqfac )
        record event
      event\_start = j
      event\_snr   = \(\rho\sb{j}\)
\end{alltt}

Rather than applying the $\chisq$ test at each sample point, we
cluster only on SNR and then use the $\chisq$ test to validate the 
candidate trigger.  This was implemented, along with altering the
clustering window as described above.  The results are shown in
Figure~\ref{f:impulses_new_chisq}, and appear very promising.  The
plateaus have been removed almost entirely, leaving only the last
trailing edge.  Only a few triggers from short templates remain, and
in particualr nothing that would interfere with the BNS search.

\begin{figure}
  \includegraphics[width=0.5\linewidth]{figures/detchar/1e-17_fixed_20100909}
  \includegraphics[width=0.5\linewidth]{figures/detchar/1e-15_fixed_20100909}
  \caption[Triggers produced by modified clustering algorithm] {
  \label{f:impulses_new_chisq}
Triggers produced by the modified clustering algorithm which extends
the clustering window and applies the $\chisq$  threshold after the
candidate trigger has been found.  Results are promising: only a
relatively small number of triggers from short templates remain.
}
\end{figure}%

The revised code was then tested by performing injections of simulated
signals into two weeks of real detector noise and examining the
numbers of injections found and missed.  The new code misses many more
injections, as shown in figure~\ref{f:found_missed_penguins}.  To see
why, consider a glitch with high SNR but high $\chisq$ within a
clustering window of an injection with lower SNR and lower $\chisq$.
In the original code the clustering window would not open at the
glitch, because it would not pass the $\chisq$ test.  The window would
open at the time of the injection, and the trigger would be recorded.
Under the new code, however, the higher SNR of the glitch will cause
it to be the single event found in the window, but it will then be
removed by the $\chisq$ test.  This problem is exacerbated by
extending the length of the window.

\begin{figure}
  \includegraphics[width=0.5\linewidth]{figures/detchar/penguin_foundmissed_orig}
  \includegraphics[width=0.5\linewidth]{figures/detchar/penguin_foundmissed_new}
  \caption[Found/missed plots showing the effect of new code] {
  \label{f:found_missed_penguins}
Found and missed plots for simulated injection in two weeks of real
noise.  Results from the current code are on the left, the modified
code (as described in the text) is on the right.  The feature to focus
on is found injections (blue dots) and missed injections (red
crosses).  There are many more missed injections using the modified
code.
}
\end{figure}%


At present the issue of excess triggers from loud glitches remains
unresolved.  It is mitigated somewhat by the ``loud SNR'' veto at
category 4, which extends $\pm 8$ seconds in order to remove the
excess triggers from the inverse spectrum truncation.  However, as
figure~\ref{f:impulses_original_no_chisq} shows, for very loud
glitches excess triggers can be produced for as much as 45.  An
appealing possibility that has been discussed but not tested would be
to run the current algorithm over a new SNR time series formed by
combining the SNR and $\chisq$ series.

\subsection{Bias of the PSD by loud glitches}
\label{ssec:sarlacc}

Loud glitches are often accompanied by surrounding periods of
decreased trigger rates, paradoxically.  This can be seen, for
example, in figure~\ref{f:daily_ihope_tcs}.  This effect is confined
to the 2048-second analysis segment containing the glitch, as can be
seen from figure~\ref{f:move_glitch}, which shows the trigger rate
around a glitch as the analysis boundaries are changed.

\begin{figure}
  \includegraphics[width=0.5\linewidth]{figures/detchar/H1-endtime_hist_ORIG}
  \includegraphics[width=0.5\linewidth]{figures/detchar/H1-endtime_hist_RESEG}
  \caption[Effect on trigger rates of a large glitch] {
  \label{f:move_glitch}
Effect on trigger rates of a large glitch.  The plots show the number
of triggers in an H1 analysis in 1-second blocks around a large glitch
centered at 2500.  The analysis boundaries are arranged so that on the
plot on the left the glitch falls at the end of the earlier segment,
while on the right the glitch falls at the beginning of the later
segment.  In both cases the glitch has the same number of triggers, 
which is well in excess of the surrounding time for reasons discussed
in the previous section.  However, the segment containing the glitch
shows a marked decrease in the number of triggers.
}
\end{figure}%

This behavior is due the effect of a glitch on the PSD estimation for
the analysis period.  Recall from chapter~\ref{ch:search} that the PSD
is estimated by Welch's method, and the value of the PSD at each
frequency $f$ is the median over 15 256-second intervals, overlapped
to span 2048 seconds.  The median is used instead of the mean
precisely because it is more robust against glitches.  However, a
values well outside the expected distribution can still bias the
results to a significant extent.  To demonstrate this we calculate the
PSD in two ways, once by considering the median of all 15 PSDs and
once by consider the median of 14, removing the one containing the 
glitch.  We plot the fractional difference of each frequency bin in
figure~\ref{f:median_bias}.


\begin{figure}
  \includegraphics[width=\linewidth]{figures/detchar/spectra_diffs}
  \caption[Bias in the PSD caused by a large glitch] {
  \label{f:median_bias}
The bias in the PSD caused by a large glitch.  This plot shows the
difference between two PSDs, one calculated including the
segment with the glitch and one without.  The results are shown as 
the fraction difference in each mass bin, (with - without) / with.
There is a notable bias upwards over much of the frequency range, and
in particular over the most sensitive portion of the LIGO band.
}
\end{figure}%

Some preliminary investigations using different-sized chunks and
overlaps to compute the PSD were performed, but these have so far been
inconclusive.  At present the only way to ensure that a loud glitch
won't suppress a quiet signal is to veto the time containing the
glitch at category 1.  This would be conceptually straightforward
using something like the existing category 4 ``loud SNR'' veto, but it
is far from an ideal solution.  This is especially true in advanced
LIGO, where the low-mass templates will be significantly longer.  A
glitch in the middle of such a signal could cause it to be lost, as
the category 1 veto would split the SNR accumulation into two disjoint
segments.   A better option would be to ``gate'' the data around a
glitch; smoothly window out a second or so with a Tukey window or
similar.  This is an area of ongoing research.

% tot_time, tot_count
% veto_time, veto_count

% $ ./check_flags.py L1
% 798720 2890507
%    880   12578
% $ ./check_flags.py H1
% 647168 1692904
%    416    6452





\Chapter{Conclusions}
\label{ch:conclusions}
%%%%%%%%%%%%%%%%%%%%%%%%%%%%%%%%%%%%%%%%%%%%%%%%%%%%%%%%%%%%%%%%%%%%%%%%%%%%%%%
%%% Describe the wondrous conclusions of the thesis.
%%%%%%%%%%%%%%%%%%%%%%%%%%%%%%%%%%%%%%%%%%%%%%%%%%%%%%%%%%%%%%%%%%%%%%%%%%%%%%%


The first observation runs of Advanced LIGO and Advanced Virgo detectors 
are scheduled for $2015$. By $2018$, these detectors will reach 
their design sensitivity. These second-generation terrestrial detectors
will be able to see up to $10$ times further out in the universe 
than their earlier counterparts. For a compact binary population
uniformly distributed in co-moving volume, this translates to 
a thousandfold increase in the expected detection rate.
% 
Gravitational wave searches make use of theoretical knowledge of
binary dynamics and employ modeled waveforms as filter templates.
With the increase in sensitivity, the resolution of the detectors 
for small errors in modeled waveforms also increases. In this dissertation,
we primarily focus on selecting and developing optimal waveform filters
for Advanced LIGO searches. We also validate gravitational-wave 
search algorithms using accurate numerically simulated signals injected 
into emulated detector noise.

Past binary black hole searches have used post-Newtonian (pN) and 
Effective-One-Body (EOB) waveforms as filters. While the pN waveforms are 
computationally inexpensive, they are restricted to the inspiral
regime of binary coalescence. EOB waveforms include the complete
coalescence process through inspiral, merger and ringdown, and also
the sub-dominant waveform harmonics. However, they are also 
computationally more expensive. For low mass binary black holes 
($m_1,m_2\leq 25M_\odot$),
we explore the region of the parameter space over which pN waveform
templates are sufficiently accurate, in the sense of being able to 
recover more than $97\%$ of the optimal signal-to-noise ratio, 
and where in the parameter space would searches need EOB
waveform templates.
% 
% For binaries with masses $m_1,m_2\leq 25M_\odot$, we compare the 
% inspiral-only post-Newtonian waveforms with the recently proposed
% Effective-One-Body (EOB) model~\cite{BuonannoEOBv2Main}. 
% As this EOB model is calibrated
% against high-accuracy numerical simulations of non-spinning binary 
% black holes, it is demonstrably accurate for {\it comparable}
% mass-ratio binaries. However, it is computationally more expensive
% than the post-Newtonian approximants. 
% We investigate the region of the parameter
% space of non-spinnning binaries where the accuracy of post-Newtonian
% approximants is sufficient and we can win with computational cost, as
% well as the region where EOB waveforms would be required. 
Here we approximate the waveforms with their dominant multipoles. Next,
we study the impact of ignoring sub-dominant waveform multipoles in 
searches. We find that including sub-dominant harmonics could increase
the reach of aLIGO and Virgo for binaries which have their orbital
angular momentum highly inclined to the line of sight connecting them
to the detector.

Numerical Relativity (NR) has seen recent breakthroughs and rapid progress
in simulating the merger of orbiting black holes. These are the most 
accurate solutions to Einstein's field equations available. Still, 
due to their computational cost, numerical relativity simulations 
span only the last stages of the binary inspiral, alongwith the merger
and ringdown. It is possible to join these short but accurate strong-field
simulations with post-Newtonian waveforms that cover the slow-motion
regime, to construct pN-NR {\it hybrids}. We demonstrate that, within 
the limits of current NR technology, it is possible and viable to use 
hybrid waveforms in gravitational wave searches. In addition, we show
that hybrid waveforms can cover the entire region of the binary black
hole parameter space where pN waveforms are insufficient for Advanced 
LIGO searches.


Apart from having applications as search templates, and in enhancing the
accuracy of waveform models, NR simulations
can be used to validate gravitational-wave search algorithms.
We do precisely this within the purview of the NINJA-2 project. 
Several numerical relativity groups contributed
post-Newtonian-hybridized simulations to the project. These were subsequently
injected in emulated advanced detector noise. We demonstrate the ability of
existing search algorithms to successfully {\it detect} these simulations
embedded within realistic noise. This is different from the NINJA-1 project
on a few counts, one of them being the nature of the emulated noise. In the 
NINJA-2 project, initial LIGO data with its non-Gaussian transient noise was
recolored to the expected sensitivity of the Advanced LIGO-Virgo detectors, as
opposed to colored Gaussian noise that was used in NINJA-1. 
Therefore this project provided a more robust test of our search methods, and 
provided a benchmark against which future search developments could be compared.


While the above concerns primarily comparable mass-ratio binaries, we 
also develop a waveform model for intermediate mass-ratio ones with 
$m_1/m_2 \in [10, 100]$. 
Intermediate mass-ratio systems, containing intermediate mass  
and stellar mass black holes will also be relatively more massive than stellar
mass binaries.
This would shift the frequency of the emitted gravitational radiation to 
lower values, and their late-inspiral and merger would occur in the most
sensitive frequency band of the Advanced detectors. This makes the modeling 
of the later portion of their waveforms crucial to their detection. 
%
First-order conservative self-force corrections have been derived for a
test-particle moving in the background of a supermassive Schwarzschild 
black hole. Using the form of these calculations, we formulate a 
prescription to model the early and late inspiral
of such binaries. Then, using the implicit rotation source picture
(due to Baker et al~\cite{Baker:2008}), we develop a model for the plunge and merger,
where the black holes are close and the orbits are no longer quasi-circular.
We then complete the description by stitching the quasi-normal modes emitted
by the black hole formed at merger. 
Therefore, we complete a model that captures the entire coalescence process
for intermediate mass-ratio binaries of non-spinning black holes.

To summarize, for {\it comparable} mass ratio binaries, we show that a combination
of post-Newtonian and post-Newtonian--Numerical-Relativity hybrid waveforms
would be sufficient for gravitational wave searches. This is true for the 
entire stellar-mass non-spinning binary black hole parameter space.
We also successfully validate gravitational wave search algorithms 
that have been used in the most recent LIGO-Virgo searches, using accurate 
numerical simulations injected in emulated detector noise. 
For {\it intermediate} mass ratios, we develop an accurate waveform 
model that captures the binary dynamics from the weak-field slow-motion
regime to the strong-field regime up to the merger of both compact objects. 
Therefore the work presented in this dissertation is an effort towards
arriving at optimal search filters for non-spinning binary black holes 
which are prospective sources detectable by the second-generation terrestrial
gravitational wave detectors; as well as towards validating existing search 
algorithms using an improved testing methodology.
















\clearpage
\bibliographystyle{unsrt}
\bibliography{references}
\addcontentsline{toc}{chapter}{\numberline {Bibliography}}

\end{document}

