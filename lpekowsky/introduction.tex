

The coalescence of binary black holes is one the most promising
sources of gravitational waves for interferometric gravitational wave
detectors, such as LIGO, Virgo and GEO600~\cite{thorne.k:1987}. The
first-generation LIGO detectors have achieved their design sensitivity
and recorded over one year of coincident data~\cite{Abbott:2007kva}.
This data, together with data from the Virgo detector, are currently
being searched for gravitational waves from compact binary
coalescence~\cite{Abbott:2003pj,Abbott:2005pe,Abbott:2005pf,%
  Abbott:2007xi,Abbott:2007ai,Abbott:2008}.  Upgrades to improve the sensitivity
of these detectors by a factor of two, and ultimately 10, are
underway.  Optimal searches using the enhanced detectors in 2009 will
be sensitive to black-hole coalescence out to hundreds of
megaparsecs~\cite{LIGOEnhancedLIGO}. The advanced detectors,
operational next decade, could detect black-hole binaries at distances
of over \unit[1]{Gpc}~\cite{Fritschel:2003qw}.
