
\subsection{Data Quality Flags}

The state of the instrument at any time is summarized by a set of
\emph{flags}.  Flags are identified by a triple of (ifo id, flag name,
version number), where \emph{ifo id} identifies the instrument,
\emph{flag name} is a unique identifier, and \emph{version number} is
an integer starting from 1.  The version number allows information to
be updated without losing information that may be needed to
reconstruct the results of earlier searches.  The full set of flags is
stored in a database designed at Syracuse and hosted at Caltech.

Flags are stored as a set of \emph{segments}, half-open intervals
aligned on GPS integer second boundaries.  Each flag triple has an
associated set of segments indicating the times during which it is
\emph{defined}.  Such triples also have a set of segments indicating
times during which they are \emph{active}.  The set of active segments
must be a subset of defined segments.  Times during which a flag is
defined and not active are considered \emph{inactive}.

Flags may be entered manually through a web interface.  This can be
used to indicate nonstandard operating conditions, such as heavy
equipment being operated on site.  However, most flags are generated
automatically.

The first line of defense against noise triggers is on-site as the
instrument is running.  At all such times the control room is staffed
by an operator who is an expert in running the instrument employed by
LIGO labs, and a science monitor (``SciMon'') who is a member of the
LIGO Scientific Collaboration.  The two jointly decide when to enable
\emph{science mode}, which marks the data as suitable for analysis.
This declaration is not specific to the CBC group, but extends to all
searches in the collaboration.  Science time is indicated by the flag
\texttt{DMT\_SCIENCE}.

In addition to the readout channel (\texttt{DARM\_ERR}, section
\ref{sec:inst_readout} many other data channels are recorded, falling
into two broad categories.  Physical environmental monitor (``PEM'')
channels record information about the environment such as seismic
activity at the base and end stations, microphones and magnetometers
placed throughout the site, etc.  Instrumental (``INST'') channels
record data from numerous subsystems such as servos for each mirror
and the output of photodiodes at points throughout the light path.
Software running at the sites called the data monitoring tool
(``DMT'') creates data quality flags based on these channels.  For
example, when a channel's value or standard deviation over time
exceeds a given threshold.  The DMT is also responsible for recording
the state of the science mode flag.  Other flags are set by programs
that run analyses on these auxiliary channels, for examples see
(\checkme{UPV refs}).


\subsection{The Veto Definer}

Problems in the data may have differing levels of severity, and
consequently we define several \emph{veto categories} to characterize
them.  The categories used by daily ihope differ slightly from those
used by the full analysis.

Category 1 vetoes indicates time that should not have been marked as
science mode.  Typically attempting to analyze this time will
adversely affect the entire 2048-second analysis chunk, for example by
biasing the PSD (section \ref{sec:ihope_psd}).  Note that it is
possible to correct science time by creating a new version of the
\texttt{DMT\_SCIENCE} flag with an incremental version number.
However, doing so is a more complicated process than creating a veto
flag, and removing time by denoting it CAT1 carries additional
information about the reason for the veto.  Note that category 1
vetoes are undesirable, as they may remove time outside the range of
the problem.  For example, a 4080-second span of science-mode data
with a one-second CAT1 veto at 2040 seconds will be completely
unanalyzable by the CBC group, as excising the bad data will leave no
contiguous 2048-second block.

Category 2 vetoes indicates time during which there was a problem,
instrumental or environmental, with well-understood coupling into
\texttt{DARM\_ERR}.  Such time can be analyzed without problem, but
triggers from such time will be discarded. 

Category 3 vetoes remove hardware injections (section
\ref{sec:ihope_hardware_injections}).  In the full analysis hardware
injectiob vetos do not have a category number and is denoted
``hardware injections removed.''

Category 4 is for time that appears to be ill-behaved according to 
data quality studies, but where there is no clearly-understood 
cause.  In the full analysis this is denoted category 3.

Finally, a map is needed between data quality flags and veto
categories.  This is achieved through the use of a search-specific
\emph{veto definer file} which associates a flag with a veto level.
Intervals within the range during which that version of the flag are
active will be marked with the corresponding veto level.  The
\texttt{ligolw\_segments\_from\_cats} program (see below) merges the
information in this file with the set of active flag segments to produce
\emph{veto segments} at each veto level.  Analysis is performed on
times marked as science with no CAT1 vetoes.  Triggers from times
marked as CAT2, CAT3 and CAT4 are discarded from both foreground and
background.

Sometimes the threshold on a DQ flag is such that the data is
unsuitable for analysis close to, but outside, the range of the flag.
The veto definer file therefore allows for \emph{padding}, offsets 
which effectively extend the flags.


\subsection{Technical details}

The flag segments are stored in a high-performance relational
database, exposed as a web service which provides secure access to all
members of the collaboration.  Several utilities to interact with the
segment database were written, many of which could run in many modes.
The names of all these utilities start with ligolw, short for ``LIGO
lightweight,'' an XML-based format used throughout the collaboration
in which the output was generated.

Except where noted all programs accept a common subset of arguments
\begin{itemize}
\item \texttt{--gps-start-time} and \texttt{--gps-end-time}: Integer
GPS times specifying the time range over which to run.
\item \texttt{--include-segments}: Takes a comma-separated list of
segment specifiers of the form ifo:flag\_name:version or
ifo:flag\_name and restricts the query to matching flags.  In the
latter case, report on the latest version of the flag defined at each
time within the time range.
\item \texttt{--exclude-segments}: Takes a comma-separated list of
segment specifiers in the same format as \texttt{--include-segments}.
Runs a second query on the excluded segments and returns the set of
segments in included segments - (included segments $\cap$ excluded
segments).
\item \texttt{--database}, \texttt{--segment-url},
\texttt{--dmt-files}: Specifies the data source (segment database or
XML files with segment information).
\end{itemize}


The programs used during S6 were:

\textbf{ligolw\_dq\_query}.  This program reports on the state of
flags at one or more GPS times given on the command line.  This
program does not support \\
\texttt{--gps-start-time} or \texttt{--gps-end-time}.   The available modes are

\begin{itemize}
\item \texttt{--report}:  Returns the active/inactive status of all DQ flags at the
given times.  For active flags reports the start and end times of the
segment within which the given time is contained.  For inactive flags
reports the end time of the nearest preceding segment and the start
time of the nearest subsequent segment.  This mode is used in the
daily ihope ``loudest glitches'' page, see below.
\item \texttt{--defined}: Returns a list of flags defined at the
given times.
\item \texttt{--active}: Returns a list of the flags that were active
at the given times.
\item \texttt{--start-pad}, \texttt{--end-pad}: Extends the
\texttt{--defined} and \texttt{--active} modes to report on the status
of flags within a small range of time.
\end{itemize}


\textbf{ligolw\_segment\_query}.  This program reports on the state of a
set of flags over a span of times.  The available modes are

\begin{itemize}
\item \texttt{--show-types}: Reports the sets of flags that exist over
the time span.
\item \texttt{--query-types}: Reports the segments during which the
flags were defined.
\item \texttt{--query-segments}: Reports the segments during which the
flags were active.
\end{itemize}

Every CBC analysis begins by determining the science mode times with a
call of the form

\vspace*{5mm}
\texttt{ligolw\_segment\_query} \\
\hspace*{0.5in}\texttt{--query-segments --include-segments H1:DMT-SCIENCE:3 ...}
\vspace*{5mm}

\textbf{ligolw\_segments\_from\_cats}.  Given a veto definer file,
report on segments to be vetoed. 

\textbf{ligolw\_dq\_active\_cats}.  Given a veto definer file and GPS
time, report on the active/inactive status and veto category for all
flags defined at the time.

\textbf{ligolw\_segment\_insert}.  Adds new segments to the database,
enforcing various policy decisions:
\begin{itemize}
\item New segments for an existing flag/version number pair must not overlap 
with existing segments.  To update information a new version number must be used.
\item New version numbers must be one greater than the largest
existing version number.
\item New segments must not extend into the future, past the GPS time
at which the program is run.
\end{itemize}


% https://www.lsc-group.phys.uwm.edu/daswg/wiki/content_of_segment_publication_and_discovery


