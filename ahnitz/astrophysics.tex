\def\Msun{M_\odot}
\acrodef{aLIGO}[aLIGO]{Advanced Laser Interferometer Gravitational-wave Observatory}
\acrodef{AdV}[AdV]{Advanced Virgo}
\acrodef{LIGO}[LIGO]{Laser Interferometer Gravitational-wave Observatory}
\acrodef{CBC}[CBC]{compact binary coalescence}
\acrodef{S6}[S6]{LIGO's sixth science run}
\acrodef{VSR23}[VSR2 and VSR3]{Virgo's second and third science runs}
\acrodef{EM}[EM]{electromagnetic}
\acrodef{NS}[NS]{neutron star}
\acrodef{BH}[BH]{black hole}
\acrodef{BNS}[BNS]{binary neutron star}
\acrodef{NSWD}[NSWD]{neutron star-white dwarf}
\acrodef{NSBH}[NSBH]{neutron star and a black hole}
\acrodef{GRB}[GRB]{gamma-ray burst}
\acrodef{S5}[S5]{LIGO's fifth science run}
\acrodef{S4}[S4]{LIGO's fourth science run}
\acrodef{VSR1}[VSR1]{Virgo's first science run}

\acrodef{PSD}[PSD]{power spectral density}
\acrodef{VSR3}[VSR3]{Virgo's third science run}
\acrodef{BBH}[BBH]{binary black holes}
\acrodef{SNR}[SNR]{signal-to-noise ratio}
\acrodef{SPA}[SPA]{stationary-phase approximation}
\acrodef{LHO}[LHO]{LIGO Hanford Observatory}
\acrodef{LLO}[LLO]{LIGO Livingston Observatory}
\acrodef{LSC}[LSC]{LIGO Scientific Collaboration}
\acrodef{PN}[PN]{post-Newtonian}
\acrodef{DQ}[DQ]{data quality}
\acrodef{IFO}[IFO]{interferometer}
\acrodef{DTF}[DTF]{detection template families}
\acrodef{FAR}[FAR]{false alarm rate}
\acrodef{FAP}[FAP]{false alarm probability}
\acrodef{PTF}[PTF]{physical template family}
\acrodef{ADE}[ADE]{advanced detector era}
\acrodef{FFT}[FFT]{Fast Fourier Transformation}
\acrodef{GPU}[GPU]{graphical processing unit}
\acrodef{ISCO}[ISCO]{inner-most stable circular orbit}
\acrodef{MECO}[MECO]{minimum energy circular orbit}


\section{Introduction}

The second-generation gravitational wave detectors Advanced LIGO (aLIGO) and
Advanced Virgo (AdV)~\cite{Harry:2010zz, aVirgo} are expected to begin
observations in 2015, and to reach full sensitivity by 2018-19. These detectors
will observe a volume of the universe more than a thousand times greater than
first-generation detectors and establish the new field of gravitational-wave
astronomy. Estimated detection rates for aLIGO and AdV suggest that binary
neutron stars (BNS) will be the most numerous source detected, with plausible
rates of $\sim 40/\mathrm{yr}$~\cite{Abadie:2010cf}.
Gravitational wave
observations of BNS systems will allow measurement of the properties of
neutron stars and allow us to explore the processes of stellar evolution. 
Binaries containing a \ac{NSBH} have a predicted 
coalescence rate of $0.2$--$300\ \textrm{yr}^{-1}$ within the sensitive volume
of aLIGO~\cite{Abadie:2010cf}, making them another important source for these
observatories. The observation of a \ac{NSBH} by \ac{aLIGO} would be the first 
conclusive detection of this class of compact-object binary.

\section{Populations of BNS Sources}


The gravitational waves that advanced detectors will observe from inspiralling BNS systems
are well described by post-Newtonian theory~\cite{Blanchet:2006zz}.
As the neutron stars orbit each other, they lose energy to gravitational waves
causing them to spiral together and eventually merge.
If the
angular momentum (spin) of the component neutron stars is zero, the gravitational
waveform emitted depends at leading order on the chirp mass of the binary
$\mathcal{M} = \left(m_1 m_2\right)^{3/5}/\left( m_1 +
m_2\right)^{1/5}$~\cite{Peters:1963ux}, where $m_1,m_2$ are the component masses
of the two neutron stars, and at higher order on the symmetric
mass ratio $\eta = m_1 m_2 /
(m_1+m_2)^2$~\cite{Blanchet:1995fg,Blanchet:1995ez,BIWW96,Wi93,BFIJ02,Blanchet:2004ek}.
If the neutron stars are rotating, 
%the gravitational-waves phasing is affected
%by spin-orbit and spin-spin coupling in the
%binary~\cite{Kidder:1992fr,Kidder:1995zr}.
coupling between the neutron stars' spin $\bm{S}_{1,2}$ and the
orbital angular momentum $\bm{L}$ of the binary will affect the dynamics of BNS
mergers~\cite{Kidder:1992fr,Apostolatos:1994mx,Kidder:1995zr,Blanchet:2006gy}.  
We measure the neutron stars' spin using the dimensionless parameter
$\bm{\chi}_{1,2} = {\bm{S}_{1,2}}/{m_{1,2}^2}$.


Electromagnetic observations suggest that the \ac{NS} mass
distribution in \ac{BNS} peaks at $1.35 \Msun--1.5 \Msun$ with a narrow
width~\cite{Kiziltan:2010ct}, although \acp{NS} in globular clusters seem to
have a considerably wider mass distribution~\cite{Kiziltan:2010ct}.  There is
also evidence that a neutron star in one system has a mass as high as $\sim 3
\Msun$~\cite{Freire:2007jd}.  The dimensionless spin magnitude $\chi = cJ/Gm^2$ for
\acp{NS} is constrained by possible \ac{NS} equations of state to a maximum of
0.7~\cite{Lo:2010bj}.  The fastest observed pulsar has a spin period
of 1.4 ms~\cite{Hessels:2006ze}, corresponding to a $\chi \sim 0.4$, and the
most rapidly spinning observed \ac{NS} in a binary, J0737--3039A, has a spin
of only $\chi \sim 0.05$.

The maximum spin value for a wide class of neutron star equations of state is
$\chi \equiv \left| \bm{\chi} \right| \sim 0.7$~\cite{Lo:2010bj}. However, the spins of neutron stars in BNS
systems is likely to be smaller than this limit. The spin period at the birth
of a neutron star is thought to be in the range
$10$--$140$~ms~\cite{Lorimer:2008se,Mandel:2009nx}. During the evolution of
the binary, accretion may increase the spin of one of the
stars~\cite{Bildsten:1997vw}, however neutron stars are unlikely to have
periods less than 1~ms~\cite{Chakrabarty:2008gz}, corresponding to a
dimensionless spin of $\chi \sim 0.4$.  The period of the fastest known pulsar
in a double neutron star system, J0737--3039A, is
$22.70$~ms~\cite{Burgay:2003jj}, corresponding to a spin of only $\chi \sim
0.05$.

\section{Population of NSBH Sources}

Gravitational-wave observations of \ac{NSBH} binaries will allow us to explore the central engine of short,
hard gamma-ray bursts, shed light on models of stellar evolution and core
collapse, and investigate the dynamics of compact objects in the strong-field regime~\cite{lrr-2009-2, Eichler:1989ve, Narayan:1992iy, Paczynski:1991aq, Berger:2010qx, Fryer:2011cx, Hannam:2013uu}.
Achieving aLIGO's optimal sensitivity to
\ac{NSBH} binaries and exploring their physics 
requires accurate modeling of the gravitational waves emitted 
over many hundreds of orbits as the signal sweeps through the detector's
sensitive band. For \ac{BNS} systems the mass
ratio between the two neutron stars is small and the angular momenta of the
neutron stars (the neutron stars' spins) is low. In this case, the emitted waves are
well modeled by \ac{PN}
theory~\cite{Blanchet:2006zz,Buonanno:2009zt,Brown:2012qf}. 
However, \ac{NSBH} binaries can have significantly larger mass ratios and the spin of
the black hole can be much larger than that of a neutron star. The combined
effects of mass ratio and spin present challenges in constructing accurate gravitational waveform models for
\ac{NSBH} systems, compared to \ac{BNS} systems.  In this paper we
investigate how accurately current theoretical models simulate \ac{NSBH} gravitational waveforms
within the sensitive frequency band of \ac{aLIGO}.

Although no \ac{NSBH} binaries have been directly observed, both \acp{BH} and \acp{NS}
have been observed in other binary systems. Several \ac{BNS} systems and
\ac{NSWD} systems have been observed by detecting their electromagnetic
signatures.  The observational data for \acp{BH} is more limited
than for \acp{NS}.  Studies of \acp{BH} in low-mass X-ray
binaries suggest a mass distribution of $7.8 \pm 1.2 \Msun$~\cite{Ozel:2010su}. This extends to $8-11 \pm 2-4 \Msun$ when 5
high-mass, wind-fed, X-ray binary systems are included~\cite{Farr:2010tu}. For
\acp{BH} there is evidence for a broad distribution of spin
magnitudes~\cite{Miller:2009cw}, although general relativity limits it to be
$\chi < 1$. 

\section{Modelling BNS/NSBH Waveforms}
