Proposed shortly after the discovery of General Relativity by Albert Einstein, 
gravitational-waves are ripples that propagate through the
curvature of spacetime. \cite{einstein1916sb}. They travel at the speed of light 
and carry energy away from their source. However,
since gravitational-waves only weakly couple with matter, their direct detection 
in the laboratory is challenging. 

The earliest indirect evidence for
gravitational waves comes from the discovery
by Hulse and Taylor of a pulsar in a binary system. Through the measurement 
of the pulsar’s radio emissions, it was determined that the binary’s orbit was
decaying, and that the rate of decay was in agreement with energy 
being carried away by the emission of gravitational waves \cite{Hulse:1974eb, Weisberg:1981mt}. 

As a gravitational-wave passes by, the distance measured between free falling
objects changes. The first to suggest measuring this change in 
distance using light signals was Felix Pirani in 1956. By 1971 the first
gravitational antenna using laser interferometry was developed and 
tested by Moss, Miller, and Forward \cite{Forward:1971}. The modern design
of gravitational-wave interferometers is based on the work of Weiss and Drever
in the 1970s~\cite{Weiss:1972, Drever:1980}. 



Searches for gravitational waves from compact object
binaries containing neutron stars and stellar-mass black holes have been
performed using the first-generation LIGO and Virgo detectors in LIGO's six
science runs (S1--S6) and three Virgo science runs
(VSR1--VSR3)~\cite{Abbott:2003pj,Abbott:2005pe,Abbott:2005kq,Abbott:2007xi,Abbott:2007ai,Abbott:2009tt,Abbott:2009qj,Abadie:2010yba,Abadie:2011nz}.
Construction of the Advanced LIGO (aLIGO) detectors~\cite{TheLIGOScientific:2014jea} is now complete and the
first aLIGO observing runs are scheduled for autumn 2015~\cite{Aasi:2013wya}.
The Advanced Virgo (AdV) detector~\cite{Acernese:2015gua} is scheduled to join this network in 2016.
The second-generation gravitational wave detectors Advanced LIGO (aLIGO) and
Advanced Virgo (AdV)~\cite{Harry:2010zz, aVirgo} are expected to reach full 
sensitivity by 2018-19. These detectors
will observe a volume of the universe more than a thousand times greater than
first-generation detectors and establish the new field of gravitational-wave
astronomy. Estimated detection rates for aLIGO and AdV suggest that binary
neutron stars (BNS) will be the most numerous source detected, with plausible
rates of $\sim 40/\mathrm{yr}$~\cite{Abadie:2010cf}.
Gravitational wave
observations of BNS systems will allow measurement of the properties of
neutron stars and allow us to explore the processes of stellar evolution. 
Binaries containing a \ac{NSBH} have a predicted  
coalescence rate of $0.2$--$300\ \textrm{yr}^{-1}$ within the sensitive volume
of aLIGO~\cite{Abadie:2010cf}, making them another important source for these
observatories. The observation of a \ac{NSBH} by \ac{aLIGO} would be the first 
conclusive detection of this class of compact-object binary. Gravitational-wave observations of \ac{NSBH} binaries will allow us to explore the central engine of short,
hard gamma-ray bursts, shed light on models of stellar evolution and core
collapse, and investigate the dynamics of compact 
objects in the strong-field regime~\cite{lrr-2009-2, Eichler:1989ve, Narayan:1992iy, Paczynski:1991aq, Berger:2010qx, Fryer:2011cx, Hannam:2013uu}.

Gravitational waves from compact binary coalescence have three distinct
phases: an \emph{inspiral} consisting of a wave of slowly increasing amplitude
and frequency, a \emph{merger} which can be calculated using numerical
simulations, and a \emph{post-merger} signal as the binary stabilizes into a final
state. If the total mass of the binary is lower than $M \lesssim 12\,
M_\odot$~\cite{Buonanno:2009zt,Brown:2012nn}
and the angular momenta of the compact objects (their \emph{spin}) is
small~\cite{Nitz:2013mxa,Kumar:2015tha}
(as is the case for binary neutron stars), then the inspiral phase can be 
well modeled using the post-Newtonian approximations (see e.g.
Ref.~\cite{Blanchet:2013haa} for a review).  For higher mass and higher-spin
binaries, analytic models tuned to numerical relativity can provide accurate
predictions for the gravitational waves from compact 
binaries~\cite{Buonanno:1998gg,Pan:2009wj,Damour:2012ky,Taracchini:2013rva,Damour:2014sva}. 

The gravitational waves that advanced detectors will observe from inspiralling BNS systems
are well described by post-Newtonian theory~\cite{Blanchet:2006zz}.
As the neutron stars orbit each other, they lose energy to gravitational waves
causing them to spiral together and eventually merge.
If the
angular momentum (spin) of the component neutron stars is zero, the gravitational
waveform emitted depends at leading order on the chirp mass of the binary
$\mathcal{M} = \left(m_1 m_2\right)^{3/5}/\left( m_1 +
m_2\right)^{1/5}$~\cite{Peters:1963ux}, where $m_1,m_2$ are the component masses
of the two neutron stars, and at higher order on the symmetric
mass ratio $\eta = m_1 m_2 /
(m_1+m_2)^2$~\cite{Blanchet:1995fg,Blanchet:1995ez,BIWW96,Wi93,BFIJ02,Blanchet:2004ek}.

Ground-based gravitational-wave detectors produce a calibrated strain signal
$s(t)$, which is sensitive to gravitational waves incident on the detector's
arms~\cite{Abadie:2010px}. In addition to possible signals, the strain data contain two classes of
noise: (i) a primarily stationary, Gaussian noise component from fundamental
processes such as thermal noise, quantum noise, and seismic noise coupling
into the detector; and (ii) non-Gaussian noise transients of instrumental and
environmental origin. Since the gravitational-wave signal from compact
binaries is well-modeled and the expected amplitude of astrophysical signals is
comparable to the amplitude of the noise,
matched filtering is used to search for signals in the detector data~\cite{Allen:2005fk}.  Since
we do not \emph{a priori} know the parameters of the compact
binaries we may detect, a \emph{bank} of template waveforms is constructed that spans the astrophysical
signal space~\cite{Sathyaprakash:1991mt,Dhurandhar:1992mw,Owen:1995tm,Owen:1998dk,Babak:2006ty,Cokelaer:2007kx,Brown:2012qf,Keppel:2013yia,Keppel:2013uma}. These banks are designed so that the loss in event rate caused by
their discrete nature is typically no more than 10\%. The exact placement of the
templates depends on the noise power spectral density of the detector data. To
mitigate the effect of the non-Gaussian noise transients in the search, we
require that any signal be seen with consistent parameters (compact objects'
spins and masses and the signal's time of arrival) in the detector network. Additional
statistical tests are applied to mitigate the effect of non-Gaussian
noise transients~\cite{Allen:2004gu}; these are often called \emph{signal-based vetoes}. The 
matched-filter signal-to-noise ratio and the additional statistical tests are used to
create a numerical detection statistic for candidate signals. To assign a
statistical significance to these detection candidates, the network's false-alarm rate is computed as a function of the detection statistic for the
noise background.

We demonstrate in Ch.~\ref{ch:bns_spin} that neglecting spin in matched-filter searches for binary
neutron star mergers causes advanced detectors at final design sensitivity
to lose more than 3\% of the possible signal-to-noise ratio for 59\% (6\%) of
sources, assuming that neutron star dimensionless spins, $c\mathbf{J}/GM^2$, are uniformly distributed
with magnitudes between $0$ and $0.4$ $(0.05)$ and that the neutron stars
have isotropically distributed spin orientations.
We present a new method for constructing template banks for gravitational
wave searches for systems with spin. We 
show that an aligned-spin BNS search using this bank loses only
3\% of the maximium signal-to-noise for only 9\% (0.2\%)
of BNS sources with dimensionless spins between $0$ and $0.4$ $(0.05)$ 
and isotropic spin orientations. Use of this
template bank will prevent selection bias in gravitational-wave searches and
allow a more accurate exploration of the distribution of spins in binary
neutron stars.

We investigate in Ch.~\ref{ch:nsbh_faith} the ability of currently available
post-Newtonian templates to model the gravitational waves emitted during the
inspiral phase of neutron star--black hole binaries. We restrict to the case where the spin of the
black hole is aligned with the orbital angular momentum and compare
post-Newtonian approximants that differ in the expansion of energy and gravitational-wave flux. We examine
restricted amplitude post-Newtonian waveforms that are accurate to
third-and-a-half post-Newtonian order in the orbital dynamics and complete to second-and-a-half post-Newtonian order
in the spin dynamics. We also consider post-Newtonian waveforms that include the recently derived third-and-a-half
post-Newtonian order spin-orbit correction and the third post-Newtonian order spin-orbit tail correction. 
We compare these post-Newtonian approximants to the effective-one-body waveforms for spin-aligned binaries.
For all of these waveform families, we find that
 there is a large disagreement between
different waveform approximants starting at low to moderate black hole spins,
particularly for binaries where the spin is anti-aligned with the orbital
angular momentum. The match between the TaylorT4 and TaylorF2 approximants is $\sim 0.8$ for a binary with $m_{BH}/m_{NS} \sim 4$ and 
$\chi_{BH} = cJ_{BH}/Gm^2_{BH} \sim 0.4$.
We show that the divergence between the gravitational waveforms begins in the early
inspiral at $v \sim 0.2$ for $\chi_{BH} \sim 0.4$.  Post-Newtonian spin corrections beyond those currently
known will be required for optimal detection searches and to measure the
parameters of neutron star--black hole binaries. The strong dependence of 
the gravitational-wave signal on the spin dynamics will make it possible to extract significant
astrophysical information from detected systems with Advanced LIGO and
Advanced Virgo.

In Ch.~\ref{ch:nsbh_prec} we demonstrate that if the effect of the black
hole's angular momentum is neglected in the waveform models used in
gravitational-wave searches, the detection rate of $(10+1.4)M_{\odot}$
neutron-star--black-hole
systems would be reduced by $33 - 37\%$. The error in this measurement is due
to uncertainty in the Post-Newtonian approximations that are used to model the
gravitational-wave signal of neutron-star--black-hole inspiralling binaries. We
describe a new method for creating a bank of filter waveforms where the black
hole has non-zero angular momentum that is aligned with the orbital angular
momentum. With this bank we find that the detection rate of $(10+1.4)M_{\odot}$
neutron-star--black-hole systems would be reduced by $26-33\%$. Systems that
will not be detected are ones where the precession of the orbital plane causes
the gravitational-wave signal to match poorly with non-precessing filter
waveforms. We identify the regions of parameter space where such systems occur
and suggest methods for searching for highly precessing
neutron-star--black-hole binaries.

In Ch.~\ref{ch:single_stage} we describe improvements
made to the offline analysis pipeline searching for gravitational waves from
stellar-mass compact binary coalescences, and assess how these improvements
affect search sensitivity. Starting with the two-stage \texttt{ihope} pipeline
used in S5, S6 and VSR1-3 and using two weeks of S6 data as test periods,
we first demonstrate a pipeline with a simpler workflow. This
\emph{single-stage pipeline} performs matched filtering and coincidence
testing only once. This simplification allows us to reach much lower
false-alarm rates for loud candidate events. We then describe an optimized
$\chi^2$ test which minimizes computational cost. Next, we compare methods of
generating template banks, demonstrating that a fixed bank may be used for
extended stretches of time. Fixing the bank reduces the cost and complexity,
compared to the previous method of regenerating a template bank every 2048 s
of analyzed data. Creating a fixed bank shared by all detectors also allows us
to apply a more stringent coincidence test, whose performance we quantify.
With these improvements, we find a 10\% increase in sensitive volume
with a negligible change in computational cost. We describe additional
compuatational improvements to the matched-filtering algorithm in chapter~\ref{ch:opt}.

Finally, in Ch.~\ref{ch:bns_dev} we demonstrate an analysis pipeline
that is focused on the detection of binary neutron star mergers. Using the 
improved single stage pipeline, we use three weeks of S6/VSR3 data to test 
further improvements to the pipeline. We describe a method for calculating
the significance of candidate events, and measure propabilities under the
assumption of both including and excluding foreground events from a background.
We investigate alternate configurations of the filtering process, including changes
to the spectrum estimation and signal-consistency test. We find that
the new configuration is able to achieve a 25\% increase in sensivitive volume over the
single stage configuration proped in Ch.~\ref{ch:single_stage}. Lastly,
we investigate using an aligned spin template bank, and show that for conservative 
estimates of BNS populations a non-spinning template bank has marginally superior 
sensitivity.



