\section{Introduction}

Searches for gravitational waves from compact object
binaries containing neutron stars and stellar-mass black holes have been
performed using the first-generation LIGO and Virgo detectors in LIGO's six
science runs (S1--S6) and three Virgo science runs
(VSR1--VSR3)~\cite{Abbott:2003pj,Abbott:2005pe,Abbott:2005kq,Abbott:2007xi,Abbott:2007ai,Abbott:2009tt,Abbott:2009qj,Abadie:2010yba,Abadie:2011nz}.
Construction of the Advanced LIGO (aLIGO) detectors~\cite{TheLIGOScientific:2014jea} is now complete and the
first aLIGO observing runs are scheduled for autumn 2015~\cite{Aasi:2013wya}.
The Advanced Virgo (AdV) detector~\cite{Acernese:2015gua} is scheduled to join this network in 2016.
When these second-generation detectors reach design sensitivity, it is
predicted that they will observe on the order of $10$ coalescence events per
year~\cite{Abadie:2010cf}. Binaries containing neutron stars and stellar-mass
black holes are likely to be the first sources observed by aLIGO and AdV. 

Gravitational waves from compact binary coalescence have three distinct
phases: an \emph{inspiral} consisting of a wave of slowly increasing amplitude
and frequency, a \emph{merger} which can be calculated using numerical
simulations, and a \emph{post-merger} signal as the binary stabilizes into a final
state. If the total mass of the binary is lower than $M \lesssim 12\,
M_\odot$~\cite{Buonanno:2009zt,Brown:2012nn}
and the angular momenta of the compact objects (their \emph{spin}) is
small~\cite{Nitz:2013mxa,Kumar:2015tha}
(as is the case for binary neutron stars), then the inspiral phase can be 
well modeled using the post-Newtonian approximations (see e.g.
Ref.~\cite{Blanchet:2013haa} for a review).  For higher mass and higher-spin
binaries, analytic models tuned to numerical relativity can provide accurate
predictions for the gravitational waves from compact 
binaries~\cite{Buonanno:1998gg,Pan:2009wj,Damour:2012ky,Taracchini:2013rva,Damour:2014sva}. 

Ground-based gravitational-wave detectors produce a calibrated strain signal
$s(t)$, which is sensitive to gravitational waves incident on the detector's
arms~\cite{Abadie:2010px}. In addition to possible signals, the strain data contain two classes of
noise: (i) a primarily stationary, Gaussian noise component from fundamental
processes such as thermal noise, quantum noise, and seismic noise coupling
into the detector; and (ii) non-Gaussian noise transients of instrumental and
environmental origin. Since the gravitational-wave signal from compact
binaries is well-modeled and the expected amplitude of astrophysical signals is
comparable to the amplitude of the noise,
matched filtering is used to search for signals in the detector data~\cite{Allen:2005fk}.  Since
we do not \emph{a priori} know the parameters of the compact
binaries we may detect, a \emph{bank} of template waveforms is constructed that spans the astrophysical
signal space~\cite{Sathyaprakash:1991mt,Dhurandhar:1992mw,Owen:1995tm,Owen:1998dk,Babak:2006ty,Cokelaer:2007kx,Brown:2012qf,Keppel:2013yia,Keppel:2013uma}. These banks are designed so that the loss in event rate caused by
their discrete nature is typically no more than 10\%. The exact placement of the
templates depends on the noise power spectral density of the detector data. To
mitigate the effect of the non-Gaussian noise transients in the search, we
require that any signal be seen with consistent parameters (compact objects'
spins and masses and the signal's time of arrival) in the detector network. Additional
statistical tests are applied to mitigate the effect of non-Gaussian
noise transients~\cite{Allen:2004gu}; these are often called \emph{signal-based vetoes}. The 
matched-filter signal-to-noise ratio and the additional statistical tests are used to
create a numerical detection statistic for candidate signals. To assign a
statistical significance to these detection candidates, the network's false-alarm rate is computed as a function of the detection statistic for the
noise background.

Executing the steps described above is the task of the \emph{search pipeline.}
While the basic steps remain the same, different choices can be made to
create various configurations and topologies for a search pipeline. The search
pipelines used in the last joint LIGO-Virgo science run (S6/VSR2,3) used the
\texttt{ihope} search pipeline to search for compact
binaries~\cite{Babak:2012zx}. The \texttt{ihope} pipeline, as well as the
pipelines used in previous LIGO-Virgo
searches~\cite{Brown:2004pv,Brown:2005zs}, are \emph{offline} search
pipelines. These pipelines analyze the data in a batch mode, processing
of the order of one week of data from the network. Offline batch
processing allows the pipeline to incorporate additional information about the
quality of the detector data or search tuning that is not available in real
time~\cite{Aasi:2012wd,Aasi:2014mqd}, and to produces a systematic false-alarm rate
estimation of candidates by using large samples of the noise background before
and after the time of a signal. Batch processing also allows the pipeline to
take advantage of the computationally efficient Fast Fourier Transform (FFT)
when implementing matched filtering~\cite{Allen:2005fk}, and allows
computational tasks to be parallelized over time and binary parameters for
efficient implementation on large computing clusters~\cite{Brown:workflow}.

\section{Assessing Waveform and Template Bank Effectiveness}
\label{sec:analytic_vol}

In the absense of non-Gaussian noise, we can create analytic measures of the
faithfulness and effectiveness of the waveform models and template banks that 
will be used to search for gravitational waves. In this section we describe 
the methods and the terminology that we will use in the rest of this work.

The ``overlap'' between two waveforms $h_1$ and $h_2$ is defined as
%
\begin{equation}
 \mathcal{O}(h_1,h_2) = (\hat{h}_1|\hat{h}_2) =
\dfrac{(h_1|h_2)}{\sqrt{(h_1|h_1)(h_2|h_2)}},
\end{equation}
%
where $(h_1,h_2)$ denotes the noise-weighted inner product
%
\begin{equation}
(h_1|h_2) = 4 \, \mathrm{Re}
\int^{\infty}_{f_{\mathrm{min}}}\dfrac{\tilde{h}_1(f)\tilde{h}_2^*(f)}{S_n(f)} 
df.
\end{equation}
%
Here, $S_n(f)$ denotes the one sided \ac{PSD} of the noise in the
interferometer.

As gravitational wave searches for binary mergers analytically maximize over an
overall phase and time shift, we define the ``match'' between two waveforms
to be the overlap maximized over a phase and time shift
%
\begin{equation}
\mathcal{M}(h_1,h_2) =
\underset{\phi_c,t_c}{\max}(\hat{h}_1|\hat{h}_2(\phi_c,t_c)).
\end{equation}
%
One can understand this match as the fraction of the optimal \ac{SNR} that would
be recovered if a template $h_1$ was used to search for a signal $h_2$.

We define the ``fitting-factor'' between a waveform $h_s$ with unknown
parameters
and a bank of templates $h_b$ to be the maximum match between $h_s$ and all the
waveforms in the template bank~\cite{Apostolatos:1995pj},
%
\begin{equation}
\mathrm{FF}(h_s) = \max_{h \in \{h_b\}} \mathcal{M}(h_s,h).
\end{equation}
%
The ``mismatch''
%
\begin{equation}
\mathrm{MM} = 1 - \mathrm{FF}(h_s)
\end{equation}
%
describes the fraction of \ac{SNR} that is lost due to the fact that the
template in the bank that best matches $h_s$ will not match it exactly due to
the discreteness of the bank and due to any disagreement between the waveform 
families used to model the templates and the signals. 

\section{Coincident Matched-Filter Search for Compact Binaries}
\label{s:search}

To search for coalescing compact binaries with LIGO and Virgo, the offline search
pipeline implements a coincident matched-filter search.  If the detector noise
was stationary and Gaussian, matched filtering alone would be sufficient to
determine the statistical significance of a signal. For such stationary noise, 
demanding that the signal is present in two or more detectors in the network 
(coincidence) would provide a sufficiently low false-alarm rate to claim a 
detection at a matched-filter network signal-to-noise ratio of 8; the signal 
strength used to estimate aLIGO's event rate in Ref.~\cite{Abadie:2010cf}.
However, the presence of non-stationary and non-Gaussian noise
transients (\emph{glitches})
in the detector noise increases the false-alarm rate at a given
signal-to-noise ratio and additional statistical tests must be used to separate
signals from noise. The output of the matched filter is combined with these
additional tests to create a new detection statistic for coincident detection
candidates. To determine the significance of these candidates, the noise
background must be estimated to create a map between the numeric value of the
detection statistic and the false-alarm rate (or, equivalently, false-alarm
probability). Background noise is estimated by performing coincidence tests on
detector data which has been time-shifted such that coincident candidates no
longer represents a coincident detection. The search pipeline consists of
several stages which are applied to the data to construct coincident detection
candidates and measure their significance. 

To search for gravitational waves from compact binaries, the search pipeline
first locates the data from the detectors, which is stored on disk. Analysis
of the week of data can be parallelized over time and over detector allowing
the search to execute multiple search programs simultaneously that process
small blocks of data for each detector. In the S6/VSR2,3 search, the 
analysis block size is set to $2048$ seconds.
The data is first used to construct the
template bank that will be used to matched filter the
data~\cite{Sathyaprakash:1991mt,Dhurandhar:1992mw,Owen:1995tm,Owen:1998dk,Babak:2006ty}.
The bank is constructed by specifying the boundaries of the target
astrophysical space and the desired \emph{minimal match}, the fractional loss in
matched-filter signal-to-noise ratio caused by the discrete nature of the bank.
The minimal match is chosen so that the bank is dense enough that any
gravitational wave in the target space can be recovered with a loss of
signal-to-noise ratio no greater than a chosen maximum, usually set to 3\%~\cite{Abbott:2011ys}. 
A metric is constructed on the signal space that
locally measures the fractional loss in signal-to-noise ratio for varying mass
parameters of the templates~\cite{Owen:1998dk}.  This metric (and hence
the template placement) depends on the power spectral density of the
detector noise. Since inspiral signals have more cycles at lower frequencies,
a detector with better low-frequency sensitivity relative to high frequencies
will have more discriminating power and thus require a denser bank to maintain
the desired minimal match.

The pipeline then matched filters the template waveforms against the data.
The matched filter consists of a weighted inner product in the frequency
domain used to construct the (squared) signal-to-noise ratio, given by
%
\begin{equation}
\rho^2(t) = \frac{(s|h_c)^2 + (s|h_s)^2}{(h_c|h_c)} \, ,
\label{eq:snr}
\end{equation}
%
where $h_c$ and $h_s$ are the two orthogonal phases of the 
%
\begin{equation}
(s|h)(t) = 4\int_{f_{low}}^{f_{high}} \frac{\tilde{s}(f)\tilde{h}^*(f)}{S_n (f)}e^{2\pi i f t}\, \mathrm{d}f.
\label{eq:ip}
\end{equation}
%
Here $\tilde{s}$ denotes the Fourier-transformed detector data and $\tilde{h}$
denotes the Fourier-transformed template waveform. As in the S6/VSR2,3 search,
each 2048 second block of data is sub-divided into fifteen $256$ second
segments, each overlapped by $128$ seconds. The noise power spectral
density $S_n(f)$ is computed by taking the bin-by-bin median of each of the power
spectral density of each of the fifteen segments. The fifteen segments are then each
matched filtered, with the first and last $64$ seconds of each segment
ignored, due to corruption of the filter by FFT
wrap-around~\cite{Allen:2005fk}

Times when the signal-to-noise ratio exceeds a pre-defined threshold are
considered gravitational-wave candidates, called
\emph{triggers}\cite{Allen:2005fk}. This threshold is set to a
signal-to-noise ratio of 5.5.  Since the signal-to-noise ratio can exceed
this threshold for many sample points around the time of a signal, clustering
is performed on these triggers in time, so that one trigger can be associated
with a signal. Template-length based clustering of 
Ref.~\cite{Allen:2005fk} was used in the S6/VSR2,3 search. 
For a sufficiently loud event, several nearby templates in the
bank may also produce triggers associated with the same signal and so
clustering over the template bank can also be used to limit the number of
triggers produced by the search. The S6/VSR2,3 search used a 30~ms time window
to cluster over the bank.

Since non-Gaussian noise transients in the data can also produce excursions in
the signal-to-noise ratio, an additional signal-based veto is then constructed
to ensure that the matched filter signal-to-noise ratio is
consistent with an inspiral signal. To construct this test, the
template is split into $p$ bins of equal power, and a matched filter $\rho_l$
constructed for each of these bins. Triggers are then subject to the
$\chi^2$ test, given by
%
\begin{equation}
\chi^2 = p\displaystyle\sum_{l=1}^{p}\left[\left(\frac{\rho_c}{p}-\rho_c^l\right)^2 + \left(\frac{\rho_s}{p}-\rho_s^l\right)^2 \right] \, ,
\label{eq:chisqr}
\end{equation}
%
where $\rho_c$ and $\rho_s$ are the two orthogonal filter phases.
Real gravitational-wave signals would return a low number for the $\chi^2$
test, while candidates caused by noise transients will
return a high number for the $\chi^2$ test~\cite{Allen:2004gu}. In the 
S6/VSR2,3 analysis, we set $p = 16$. The value of the $\chi^2$ test is used to 
calculate a new detection statistic, called the \emph{reweighted
signal-to-noise ratio}, given by 
%
\begin{equation}
\hat{\rho} =
  \begin{cases}
    \rho &\textrm{for } \chi^2 \leq n_\mathrm{dof}\\
    \rho[\frac{1}{2}(1+(\frac{\chi^2}{n_{dof}})^3)]^{-\frac{1}{6}} &\textrm{for } \chi^2 > n_\mathrm{dof} ,
  \end{cases}
\label{eq:newSNR}
\end{equation}
where $n_\mathrm{dof} = 2p-2$ is the number of degrees of freedom in the
$\chi^2$ test~\cite{Babak:2012zx}.
%
Since candidates caused by noise transients generally return a high $\chi^2$
statistic, the new detection statistic down weights the signal-to-noise ratio
of candidates by dividing with the $\chi^2$ statistic~\cite{Abadie:2011nz}.

The quality of the data generated by the LIGO and Virgo detectors is
scrutinized to mitigate noise and to improve the reach of the 
detectors~\cite{Aasi:2012wd,Aasi:2014mqd}. Data
quality investigations characterize times of poor detector performance
according to three broad classifications: (i) the data quality is sufficiently
poor that the data should be discarded; (ii) an instrumental artifact with a
known physical coupling to the recorded strain is observed by monitoring
environmental or auxillary control channels; (iii) a statistical correlation
is observed between a high trigger rate from the search and excess noise power
in environmental or auxillary control channels. The first class of data is
removed before searching for signals. For the second two classes, a \emph{data
quality veto} is created. Vetoes are time intervals during
which the pipeline removes all candidate events from the search.
Improved methods for tuning and applying vetoes in compact object binary
searches have been investigated~\cite{Canton:2013joa}, however these methods
were not used in S6/VSR2,3. Investigation of these new approaches is outside
the scope of this work and we apply the same data-quality vetoes as
they were tuned for the S6 search.

A true gravitational-wave signal would be incident on all detectors in the network at
approximately the same time. The maximum time-of-arrival difference between
detectors is given by the light-travel time between observatories. Noise,
however, will be independent between detectors since the interferometers
are far apart. For this reason, we require the candidates to be coincident
between detectors: they must arrive within the light-travel time between detectors,
approximately 11 milliseconds for the two LIGO detectors, with
a few milliseconds of padding to account for timing errors. The pipeline also
requires that the mass parameters of the signal are consistent between all
detectors, as would be expected for a true gravitational wave. 

To claim a detection of gravitational waves, it is necessary to calculate the
false-alarm rate of the candidate and demonstrate that it is very unlikely to
occur due to noise in the detectors. To measure the noise background in the
search, the pipeline shifts the triggers generated by filtering each
detector's strain data by an amount greater than the light-travel time between
detectors, and applies the coincidence test again.  Coincident triggers that
occur in the shifted data cannot be due to gravitational waves and thus represent
background noise. By repeating this test with many different time lags, we can
obtain a good estimate of the rate of background triggers as a function of the
combined reweighted signal-to-noise ratio detection statistic. For a
two-detector network, the combined statistic is given by
%
\begin{equation}
\hat{\rho}^c = \sqrt{\hat{\rho}_\mathrm{L1}^2 + \hat{\rho}_\mathrm{H1}^2},
\end{equation}
%
where H1 is the LIGO Hanford detector and L1 is the LIGO Livingston detector. 
The map between $\hat{\rho}^c$ and false-alarm rate allows us to estimate the
significance of gravitational-wave candidates in the search. Since the rate of
background triggers can depend strongly on the mass of the template,
the search computes different maps between $\hat{\rho}^c$ and
false-alarm rate for different regions of the mass parameter space
independently. 

\section{Measuring the performance of a Search Pipeline}
\label{s:volume}

As the primary metric of search sensitivity in the presense of real detector
data that contains non-Gaussian transient noise sources, we measure the sensitivity of a 
pipeline by finding the \emph{sensitive
volume}, which is proportional to the number of detections a pipeline will
make per unit time at a given false-alarm rate. This is given by:
\begin{equation}
V(F) = \int \epsilon(F; r, \Omega, \mathbf{\Lambda}) p(r, \Omega, \mathbf{\Lambda}) r^2 \mathrm{d}r \mathrm{d}\Omega \mathrm{d}\mathbf{\Lambda}.
\end{equation}
Here, $\mathbf{\Lambda}$ are the physical parameters of a signal (in this
study, $\{m_1, m_2\}$), $p(r, \Omega, \mathbf{\Lambda})$ is the distribution of
signals in the universe, and $\epsilon$ is the efficiency of the pipeline at
detecting signals at a distance $r$, sky location $\Omega$, and false-alarm
rate $F$.

We estimate the sensitive volume by adding to the data a large number of
simulated signals (\emph{injections}) that are randomly drawn from a
distribution $q(r, \Omega, \mathbf{\Lambda})$. We re-analyze the data with 
these simulated signals added and,
for each injection, determine if a coincident trigger is present within 1
second of the time of the injection. If a trigger is present, we use the
combined reweighted signal-to-noise ratio to compute its false-alarm rate. 
If no trigger is present, the injection is
\emph{missed}, i.e., the signal cannot be distinguished from noise at any
false-alarm rate
threshold. At some distance $r_{\max}$ we expect that any signal with $r >
r_{\max}$ will be missed.  Likewise, within some distance $r_{\min}$ we expect
that nearly every signal will be
found, even at an extremely small ($\lesssim 10^{-4} / \mathrm{yr}$)
false-alarm rate
threshold. These bounds depend on the physical parameters of a signal.
Gravitational waves from more massive systems have larger amplitudes, and thus
can be detected at greater distances than less massive systems. To first order,
if a binary with a reference \emph{chirp mass} $\mathcal{M}_0 = (m_1
m_2)^{3/5}/(m_1 + m_2)^{1/5}$ is detected at a distance $r_0$, then a binary
with arbitrary chirp mass $\mathcal{M}$ will be detected with approximately the
same reweighted signal-to-noise ratio at a distance:
\begin{equation}
r = r_{0}(\mathcal{M}/\mathcal{M}_{0})^{5/6}.
\label{eqn:chirp_distance}
\end{equation}

For each injection, we scale these bounds using Eq. \eqref{eqn:chirp_distance},
then draw the distance uniformly between them. The sensitive volume
is then simply an average over the total number of injections $N$:
\begin{equation}
V(F) \approx \frac{1}{N} \sum_{i=1}^N g_i(F) \equiv \left<g(F)\right>,
\end{equation}
where:
\begin{equation}
g_i(F) = \frac{4\pi}{3} \left[ r_{\min,i}^3 + 3r_i^2(r_{\max,i}-r_{\min,i})\hat{\epsilon}(F; F_i, r_i, \Omega_i, \mathbf{\Lambda}_i)\right].
\end{equation}
Here, $\hat{\epsilon} = 1$ if $F_i \leq F$, and
$0$ otherwise. The error in the estimate is:% \cite{ref:num_methods}:
\begin{equation}
\delta V = \sqrt{\frac{\left<g^2\right> - \left<g\right>^2}{N}}.
\end{equation}


