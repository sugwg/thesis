This document describes all the data quality (DQ) vetoes which were applied to
the analysis of GW150914. For each DQ flag the definition of the veto is given,
the interferometer this veto is applicable to, the category the veto was
applied to the Burst and Compact Binary Coalescence (CBC) searches and the
total amount of deadtime associated to each DQ veto. This document has been
created as a supplement to LIGO-P1500238.

\section{Data Quality Vetoes}

\textbf{Missing Data Veto} \\
\textbf{Purpose}: This veto captures any data dropouts at either interferometer.\\
\textbf{Definition}: Customized software indicate when the recalibrated data
frames were unable to be produced either due to missing raw interferometer data
or data in the raw data frames that are marked as invalid.\\
\textbf{Veto Category}: Burst - 1, CBC - 1\\
\textbf{Deadtime:} LIGO-Hanford - 0\%, LIGO-Livingston - 0\%
\\

\textbf{Burst Hardware Injection Veto} \\
\textbf{Purpose}: This veto indicates whenever a burst hardware injection has
been performed.\\
\textbf{Defintion}: The times of transient hardware injections labelled as
burst type are recorded by the online detector characterization (ODC) system by
monitoring the state of the calibration injection model. Deadtime quoted
includes the padding used in the analyses ($\pm$4 seconds).\\
\textbf{Veto Category}: Burst - 4\footnote{Burst veto category 4 is reserved
for transient hardware injections only.}, CBC - 2\\
\textbf{Deadtime:} LIGO-Hanford - 0.003\%, LIGO-Livingston - 0\%
\\

\textbf{CBC Hardware Injection Veto} \\
\textbf{Purpose}: This veto indicates whenever a CBC hardware injection has
been performed.\\
\textbf{Defintion}: The times of transient hardware injections labelled as CBC
type are recorded by the ODC system by monitoring the state of the
calibration injection model. Deadtime quoted includes the padding used in the
analyses ($\pm$8 seconds).\\
\textbf{Veto Category}: Burst - 4, CBC - 3\footnote{CBC veto category 3 is
reserved for CBC hardware injections only.}\\
\textbf{Deadtime:} LIGO-Hanford - 0.052\%, LIGO-Livingston - 0.072\%
\\

\textbf{DetChar Hardware Injection Veto} \\
\textbf{Purpose}: This veto indicates whenever a DetChar hardware injection has
been performed.\\
\textbf{Defintion}: The times of transient hardware injections labelled as
DetChar type are recorded by the ODC system by monitoring the state of the
calibration injection model. Deadtime quoted includes the padding used in the
analyses ($\pm$16 seconds).\\
\textbf{Veto Category}: Burst - 4, CBC - 2\\
\textbf{Deadtime:} LIGO-Hanford - 0\%, LIGO-Livingston - 0\%
\\

\textbf{Stochastic Hardware Injection Veto} \\
\textbf{Purpose}: This veto indicates whenever a stochastic hardware injection
has been performed.\\
\textbf{Defintion}: The times of hardware injections labelled as
stochastic type are recorded by the ODC system by monitoring the state of the
calibration injection model.\\
\textbf{Veto Category}: Burst - 1, CBC - 1\\
\textbf{Deadtime:} LIGO-Hanford - 0\%, LIGO-Livingston - 0\%
\\

\textbf{Beckhoff Hardware Problems} \\
\textbf{Purpose}: To capture times when the Beckhoff system (a slow control
system which is used to control a subset of hardware in the interferometer)
suffered a hardware failure at the LIGO-Hanford Y-end. \\
\textbf{Defintion}: The veto was created by hand, where the start time was
recorded as 4 seconds before excess non-stationary data started due to the
hardware failure and finished 3 seconds after the interferometer dropped out of
observing mode.\\
\textbf{Veto Category}: Burst - 1, CBC - 1\\
\textbf{Deadtime:} LIGO-Hanford - 1.50\%
\\

\textbf{45 MHz Sideband Fluctuations} \\
\textbf{Purpose}: This veto identifies times when the amplitude of the 45 MHz
optical sideband, which is used to generate error signals for optical cavities,
has excess noise. If the amplitude of the 45 MHz optical sideband fluctuates,
excess noise will be injected in to the associated optical cavities which has
been seen to couple to the gravitational wave channel.\\
\textbf{Defintion}: An auxiliary channel which monitors amplitude fluctuations
in the signal used to generate the 45 MHz optical sideband was found to be the
optimum witness of non-stationary behaviour seen in the gravitational wave
channel data. This veto was designed to capture long duration (on the order of
one minute) non-stationary behaviour. Various thresholds on the band limited
root-mean-square of this
witness channel were investigated to see which threshold proved most effective
(in terms of efficiency and deadtime) at removing non-stationary data.
Custom software was implemented to automatically capture this behaviour over
the analysis period (and throughout the first observing run).\\
\textbf{Veto Category}: Burst - 1, CBC - 1\\
\textbf{Deadtime:} LIGO-Hanford - 2.95\%
\\

\textbf{Less severe 45 MHz Sideband Fluctuations} \\
\textbf{Purpose}: See above veto, 45 MHz sideband fluctuations, for
description.\\
\textbf{Defintion}: This veto was designed to capture less severe, short time
scale (on the order of 1 second), non-stationary data. This veto was created
in a similar manner as the previous veto - a study of different thresholds
on the band limited root-mean-square of the witness channel were investigated
to give the optimal efficiency and deadtime. Custom software was implemented to
automatically capture this behaviour over the analysis period (and throughout
the first observing run).\\
\textbf{Veto Category}: Burst - not applied, CBC - 2\\
\textbf{Deadtime:} LIGO-Hanford - 0.014\%
\\

\textbf{Saturations in the SUSETMY model channels} \\
\textbf{Purpose}: This veto captures time when the Y-end test mass actuator
saturates. This is due to a relatively fast transient that is on the main
carrier beam, and therefore directly on/at the readout, which gets amplified
by the differential-arm digital filters sufficiently to cross the
digital-to-analog converter limits.\\
\textbf{Defintion}: This veto was created automatically by monitoring the
interface between the computers and the analog electronics that they control on
the Y-end test mass.\\
\textbf{Veto Category}: Burst - 2, CBC - 2\\
\textbf{Deadtime:} LIGO-Hanford - 0.067\%, LIGO-Livingston - 0.021\%
\\

\textbf{Saturations in the SUSETMY model channels with an SNR $>$ 200} \\
\textbf{Purpose}: See veto above - saturations in the SUSETMY model channels.
This veto however is aimed specifically to identify very loud saturations.\\
\textbf{Defintion}: This veto was created automatically by monitoring the
interface between the computers and the analog electronics that they control on
the Y-end test mass. A subset of these saturations is kept based on their
severity as determined by an algorithm that is designed to witness transient
power in a given signal. This veto is specific to the Burst search where
$\pm$3 seconds of padding is applied.\\
\textbf{Veto Category}: Burst - 2, CBC - not applied\\
\textbf{Deadtime:} LIGO-Hanford - 0.146\%, LIGO-Livingston - 0.047\%
\\

\textbf{Output Mode Cleaner (OMC) Photodiodes Analog to Digital Overflows} \\
\textbf{Purpose}: This veto captures times when the signal on the OMC
photodiodes exceeds the limit of the analog-to-digital converter at the
interface to the computers that control the instrument.\\
\textbf{Defintion}: This veto was created automatically by monitoring the
interface between the OMC photodiodes analog signal and the computers.\\
\textbf{Veto Category}: Burst - 2, CBC - 2\\
\textbf{Deadtime:} LIGO-Hanford - 0.002\%, LIGO-Livingston - 0.003\%
\\

\textbf{Non-Stationary Data prior to Loss of Resonant Power in the Optical Cavities} \\
\textbf{Purpose}: To veto times when the data became non-stationary before the
state of the interferometer reported the end of an observation segment.\\
\textbf{Defintion}: These times were found by hand by monitoring an algorithm,
run over the gravitational wave channel, that is designed to witness transient
power. \\
\textbf{Veto Category}: Burst - 1, CBC - 1\\
\textbf{Deadtime:} LIGO-Hanford - 0.0004\%, LIGO-Livingston 0.001\%
\\

\textbf{Glitches due to DC Power Fluctuations of the Photon Calibrator Laser} \\
\textbf{Purpose}: This veto captures times when the photon calibrator has
power fluctuations which exceed 20\% of the nominal level.\\
\textbf{Defintion}: A threshold placed on a witness channel which monitors the
power levels of the photon calibrator laser was used to flag
times when the power fluctuated beyond the 20\% level. These times were then
padded by -10 and +20 seconds to capture the full behaviour.\\
\textbf{Veto Category}: Burst - 1, CBC -1\\
\textbf{Deadtime:} LIGO-Livingston - 0.058\%
\\

\textbf{Seismic Glitches} \\
\textbf{Purpose}: This veto was created to identify times of strong excess
seismic noise that coupled in to the output of the interferometer.\\
\textbf{Defintion}: The 10-30 Hz band limited root-mean-
square of the ground seismometer, located at the input test mass on the Y-arm, in the vertical degree of freedom was found to correlate with excess noise in
the output of the interferometer. Different thresholds on this witness channel
were tested to find the optimal efficiency and deadtime that captured these
effects.\\
\textbf{Veto Category}: Burst - not applied, CBC - 2\\
\textbf{Deadtime:} LIGO-Hanford - 0.431\%


