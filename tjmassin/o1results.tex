\section{First observing run results}

Advanced LIGO's first observing run (O1) lasted from September 12, 2015 - 
January 19, 2016. In this observing run, the first direct detection of 
gravitational waves was achieved with the discovery of two binary black 
hole mergers, GW150914 and GW151226. 

Figure \ref{fig:GW150914} shows a 
filtered time domain representation of the first detection, GW150914, 
with the best estimated 
waveform overlaid on top. Both the signal and the waveform have been 
bandpass filtered to isolate the frequency range where the signal has 
power. Static lines, such as the 60 Hz power line frequency and 
interferometer calibration lines, have been notched out of the data. 
The signal demonstrates the characteristic ``chirp", 
increasing in frequency and amplitude as a function of time as  
expected from a compact binary coalescence. 

\begin{figure}[ht!]
\includegraphics[width=\textwidth]{figures/O1/GW150914-timeseries}
\caption[GW150914 timeseries]{Time domain representation of L1 %
        gravitational wave strain at the time of GW150914. The blue %
        curve is strain, zero-phase bandpass filtered to isolate the %
        frequencies that contain signal. The red curve is a CBC waveform %
        generated using the best estimated parameters. The CBC waveform %
        has been filtered in the same way as the strain curve. The overlap %
        between the two curves is significant, demonstrating many cycles %
        of clear coherence.}
\label{fig:GW150914}
\end{figure}

It is exceptional that GW150914 is visible in the detector data with 
very conservative filtering. Due to the high total mass of the system, 
which is detailed in Table \ref{table:foreground}, the black holes 
of GW150914 coalesced quickly and at a low frequency, spending about 
0.2 seconds in the frequency range that aLIGO is sensitive to. The 
signal was also tremendously loud due to its total mass and relatively 
close distance. 
As a result, the power in the signal is highly localized in time, 
producing a short, loud waveform that is readily visualized.

The other binary black hole signal, GW151226, has a rather different 
morphology. The system that produced GW151226 was roughly three times 
less massive than that of GW150914 and was estimated to have merged 
at a similar distance (see Table \ref{table:foreground}). Due to the 
lower total mass, GW151226 has an overall lower amplitude than GW150914 
and has its power distributed more broadly in time. GW151226 spent about 
2 seconds in the frequency band that aLIGO is sensitive to, which is a 
factor of 10 longer than the duration of GW150914. For these reasons, 
it is not feasible to generate a time domain visualization of the signal. 
However, this is a fantastic demonstration of the value of a 
matched-filter search for CBC signals. The signal-to-noise ratio 
reported by a matched-filter search is an 
integral in the frequency domain that is designed to identify modeled 
signals regardless of their duration.

The third loudest foreground event in the analysis, LVT151012, stands out 
from the background distribution but is not statistically significant 
enough to be labeled as a gravitational wave detection. Its statistical 
significance is calculated to be just under $2\sigma$. While it is not 
being claimed as a gravitational wave detection, there is no obvious 
reason to believe that it is a noise artifact based on detector 
performance. It's possible that LVT151012 is part of a larger 
population of gravitational waves that is expected to contain 
quiet, threshold signals as well as clear detections.

\section{PyCBC results}

The search for compact binary
coalescences was performed by two parallel search pipelines: PyCBC and 
GstLAL. In addition to these searches for modeled sources, an unmodeled 
burst search, Coherent Wave Burst (CWB), was run to search for coherent 
transient signals in the two Advanced LIGO interferometers. 
All three of these analyses produced consistent results. 
For brevity, we will focus on the results of the PyCBC search pipeline.

Figure \ref{fig:pycbc-hist-gw150914} shows the results of the PyCBC search 
over the whole of the first observing run. The black curve shows the 
number of expected foreground events at a given $\hat{\rho_c}$ based on 
background noise alone for the analysis. For this curve, GW150914 is 
allowed to remain in the data when generating a background from timeslides. 
This answers an interesting question: could a combination of signal in 
one detector and noise in the other detector generate a signal as loud is 
GW150914?

The orange squares indicate the 
number of foreground events that were actually recovered by the search 
pipeline. The false alarm rate for a given foreground 
event is defined as the rate of background events with a $\hat{\rho_c}$ 
greater than or equal to that of the foreground event. GW150914 
was an exceptionally loud signal and is the loudest event in the 
analysis. Since there are no background events as loud as GW150914, 
its statistical significance has a lower limit of $5.3\sigma$ 
but is not exactly calculated. The associated statistical significance 
is listed on the horizontal bars on the top of the plot. The color 
of each bar corresponds to the background from which the statistical 
significance was measured.  

The blue curve shows the search background when GW150914 is removed 
from the analysis and not used when generating a background from 
timeslides. Since we believe that GW150914 is a real gravitational wave 
signal, using it in background calculations no longer provides a 
search background that is a realization of detector noise alone 
when evaluating the significance of quieter signals. 
If GW150914 is allowed to produced background events, the 
significance of GW151226, which is represented by the orange square at 
$\hat{\rho_c} = 12.6$, is highly diminished. This can be seen by comparing 
the blue and black curves. The differences between them, including the 
extension of the background to $\hat{\rho_c} = 21$ , are the result 
of GW150914 combining with background noise.

\begin{figure}[ht!]%
\includegraphics[width=0.8\textwidth]{figures/O1/pycbc_hist_GW150914}
\caption[PyCBC result histograms for GW150914]{PyCBC search results for %
         the first observing run. The black curve is the search background %
         relevant to GW150914. The blue curve is the search background %
         relevant to GW151226 where GW150914 has not been included in the %
         search background calculation. GW150914 was the loudest event in %
         the first observing run and was reported with a significance %
         $> 5.3\sigma$. Figure \ref{fig:pycbc-hist-gw151226} provides % 
         a better visualization of the significance of GW151226.}
\label{fig:pycbc-hist-gw150914}
\end{figure}

With GW150914 removed from the search background, we can correctly evaluate the 
statistical significance of GW151226. 
Figure \ref{fig:pycbc-hist-gw151226} shows a zoomed in version
of the search background with GW150914 removed. 
GW151226 is the loudest event in the analysis once GW150914 and its associated 
background triggers have been removed. Since there are no background events 
as loud as GW151226, its false alarm rate can be bounded to 1 per the entire 
analysis time. The associated statistical significance has a lower limit of 
$5.3\sigma$ but can not be directly calculated. The blue curve in this plot 
shows the search background with GW151226 removed from the analysis. Any 
quieter foreground triggers, such as LVT151012, will have their false alarm 
rate and statistical significance determined by this background distribution.  
LVT151012, which is the second loudest foreground event in Figure 
\ref{fig:pycbc-hist-gw151226}, was recovered at $\hat{\rho_c} = 9.6$ and 
assigned a statistical significance of just under $2\sigma$.  

\begin{figure}[ht!]%
\includegraphics[width=0.8\textwidth]{figures/O1/pycbc_hist_GW151226}
\caption[PyCBC result histograms for GW151226]{Histograms of PyCBC results for GW151226}
\label{fig:pycbc-hist-gw151226}
\end{figure}

\begin{table}[ht!]%
  \begin{center}
    \footnotesize
    \begin{tabular}{ccccccc}
    \hline
    Event & Time(UTC) & FAR ($yr^{-1}$) & $m_1$ ($M_{\odot}$) & $m_2$ ($M_{\odot}$) & $S_{eff}$ & $D_L$ (Mpc) \\
    \hline
    GW150914 & \begin{tabular}{@{}c@{}}14 September \\ 2015 \\ 09:50:45 \end{tabular} & 
    $< 5.8\times10^{-7}$ & $36_{-4}^{+5}$ & $29_{-4}^{+4}$ & $-0.06_{-0.18}^{+0.17}$ & 
    $410_{-180}^{+160}$ \\
    GW151226 & \begin{tabular}{@{}c@{}}26 December \\  2015 \\ 03:38:53 \end{tabular} & 
    $< 5.8\times10^{-7}$ & $14_{-3}^{+9}$ & $8_{-3}^{+2}$ & $0.20_{-0.10}^{+0.21}$ & 
    $490_{-210}^{+180}$ \\
    LVT151012 & \begin{tabular}{@{}c@{}}12 October \\ 2015 \\ 09:54:43 \end{tabular} & 
    0.44 & $23_{-5}^{+18}$ & $13_{-5}^{+4}$ & $0.0_{-0.2}^{+0.3}$ & 
    $1100_{-500}^{+500}$ \\
    \hline
    \end{tabular}
  \end{center}
  \caption[Table of foreground events]{Table of foreground events found in the first %
           observing run. %
           The quoted false alarm rates are calculated by the PyCBC search pipeline. %
           The GstLAL search pipeline reported similar results. % 
           Two binary black hole systems, GW150914 and GW151226, were %
           discovered with a false alarm rate $< 5.8\times10^{-7}$, which is the upper %
           limit on false alarm rate set by the amount of time used in the analysis. %
           A third event, LVT151012, was an interesting foreground event that was not %
           statistically significant to be claimed as a detection, but could be part %
           of a larger gravitational wave population that includes weaker signals.
           } 
  \label{table:foreground}
\end{table}

