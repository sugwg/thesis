\section{The Advanced Detector Network}

The Advanced Laser Interferometer Gravitational-Wave Observatory (aLIGO) is 
part of a worldwide effort to detect gravitational waves from astrophysical 
sources. The two aLIGO interferometers, one in Hanford, WA and one in 
Livingston, LA, are part of a growing network of ground-based interferometric 
gravitational wave detectors. Each aLIGO interferometer has 4km long arms 
arranged in an L-shaped configuration. A gravitational wave passing through 
an aLIGO interferometer will cause the arms to expand and contract, 
creating an interferometric signal at the output of the instrument. 
Section \ref{sec:aligo} contains a more detailed description of the aLIGO 
interferometers. 

There are a number of other interferometric gravitational wave detectors 
being built and commissioned for future use in collaboration with aLIGO.
The Advanced VIRGO detector is being built and commissioned in Cascina, Italy. 
When it is fully commissioned, VIRGO will be joining LIGO in observing runs. 
The VIRGO interferometer has 3km arms, which should provide enough 
sensitivity to allow for triangulation of astrophysical sources.

The GEO600 detector, located in Hanover, Germany is an interferometer built in 
collaboration between Germany and the United Kingdom. 
GEO600 has been a tremendously useful test bed for interferometric technologies,
including quantum optics and homodyne detection. However, with 600m arms, GEO600 
is unlikely to be sensitive enough to witness expected astrophysical sources.

The KAGRA detector, located underground in the Kamoika mine in Japan, 
is in the process of being commissioned. KAGRA has 3 km long arms and, 
unlike other gravitational wave interferometers, employs cryogenics to 
reduce thermal noise in its optics. When complete, KAGRA should be 
sensitive enough to contribute to the worldwide detector network.

Include that cool picture of the advanced detector network.


\section{General Relativity and Sources of Gravitational Waves}

GWs are derived as a linear perturbation on a Minkowski metric
Transverse traceless gauge makes it all way prettier
The quadropole moment does a spin, we see gravitational waves.

Include that box with modeled, unmodeled, transient, and continuous. 

CBCs are the bread and butter, expect BNS, NSBH, and BBH sources
Continuous waves from pulsars
Bursts from supernovae
Stochastic background

\section{The Advanced LIGO Interferometers}\label{sec:aligo}

The Advanced LIGO (aLIGO) Interferometers are a pair of dual-recycled Michelson interferometers 
that employ 4km long Fabry-Perot cavities in their arms to increase the interaction time with a 
gravitational wave signal. 
Figure \ref{fig:aligo} shows a simplified layout of an aLIGO interferometer. 

\begin{figure}[ht!]
\includegraphics[width=\textwidth]{figures/introduction/ALIGO_layout}
\caption[Layout of Advanced LIGO]{Layout of Advanced LIGO}
\end{figure}\label{fig:aligo}

At the input to an aLIGO interferometer is a solid-state Nd:YAG laser that provides laser light 
at a wavelength of 1064 nm. Not included in Figure \ref{fig:aligo} are frequency and 
intensity stabilization control loops designed to provide as stable a laser source as 
possible for the experiment. This stabilized laser is called the pre-stabilized laser 
(PSL). The laser light is passed through a series of 
electro-optic modulators (EOM) where radio-frequency (RF) sidebands are generated 
and imparted onto the light. These RF sidebands are used to control auxiliary optical 
degrees of freedom in the interferometer. The beam is then passed through the 
input mode cleaner (IMC), which rejects higher order spatial modes of the beam 
and transmits a circular TEM00 mode to be used in the instrument.

Once the beam has been stabilized in frequency and intensity and the higher order 
optical modes have been stripped away, it is transmitted through the power 
recycling mirror and enters the vertex of the interferometer. In the vertex, 
the beam is split 50/50 by the beamsplitter. Half of the light is directed toward  
the input test mass (ITM) of the X-arm and half of the light is directed  
toward the ITM of the Y-arm. As mentioned previously, the aLIGO arms are not 
single bounce cavities; they are comprised of Fabry-Perot cavities that allow the 
light to circulate in the arm cavities multiple times. The light is stored in 
the arm cavities for $\sim$1ms, trapped between the highly reflective surfaces 
of the ITM and the end test mass (ETM), before it is transmitted back through 
the ITM and into the vertex.

When a gravitational wave passes through an aLIGO inteferometer, the distance
between the ITM and ETM of each arm is modulated, causing the light to have a
longer or shorter travel time as it traverses the arm. Since gravitational
waves expand space in one direction while the orthogonal direction contracts,     
the X- and Y-arms will experience differential changes in length. When light
from the arms is recombined at the beamsplitter, there will be a difference
in phase between the two beams as they have traveled different paths. The 
resulting light from this recombination of phase shifted beams is called the 
antisymmetric part of the output. The part of the beam that is recombined 
in phase is called the symmetric part of the output.

The beams returning from each arm are recombined at the beamsplitter. The 
symmetric part of the beam 
will be sent back toward the power recycling mirror. The power recycling mirror 
forms a resonant cavity with the ITMs, allowing for light at the symmetric 
port of the beamsplitter to be added coherently to incoming light from the PSL and 
increasing the effective power in the vertex. This increase in effective power 
is known as the power recycling gain. 

The antisymmetric part of the beam is sent toward the signal recycling mirror. 
The signal recycling cavity is used to tune the frequency response of the 
interferometer by adjusting the effective finesse of the coupled cavity 
formed by the signal recycling cavity and the arm cavities. 
If the light returning from the arms has accumulated some differential amount of 
phase as it traveled 
along the arms, perhaps from a gravitational wave modulating the length of each 
arm differentially, it will be transmitted through the signal recycling cavity 
and into the output mode cleaner (OMC). The OMC behaves similarly to the IMC, 
stripping away higher order optical modes and isolating the TEM00 mode of the 
beam. The transmitted, mode cleaned signal is then read out using a homodyne 
detection scheme on a DC photodiode. 

\subsection{DC Readout}

When a gravitational wave modulates the length of an arm cavity, the light 
traveling in that arm experiences a phase modulation. This phase modulation 
can be visualized by picturing the beam in frequency space. In figure 
\ref{fig:omc-freq}, the carrier beam frequency is designated as $f_0$. 
The phase modulation due to 
a gravitational wave signal introduces a frequency sideband at the 
gravitational wave frequency, which is in the 30-2000 Hz range. 
The 
RF sidebands used for auxiliary optical cavity control are offset from the 
carrier frequency by 9, 24, and 45 MHz. 
The RF sidebands, which in a 
homodyne detection scheme would only contribute shot noise to the output signal, 
are rejected by the OMC. The gravitational wave sidebands, however, are at a 
low enough frequency offset that they are within the cavity pole of the OMC 
and are allowed to transmit through the cavity.

Since the OMC DC photodiode measures power, it measures the square of the 
incident optical field and witnesses beat frequencies between different 
components of the light. If the RF sidebands have been filtered out by 
the OMC, the only remaining beat note will be that of the carrier beam ($f_0$) 
beating against the gravitational wave sideband ($f_0 + f_{GW}$). This beat note will 
appear as the difference in frequency between the two optical fields, 
leaving behind a signal in the 30-2000 Hz range ($f_{GW}$) and providing a 
natural demodulation inherent to the measurement process. 
The process of recovering the gravitational wave sideband using the 
carrier field as a reference is known as homodyne detection. 

\begin{figure}[ht!]
\includegraphics[width=\textwidth]{figures/introduction/omc-freq}
\caption[Sidebands and OMC cavity pole]{Frequency domain visualization of beam %
         at OMC. Grey dotted lines indicate the cavity pole. The gravitational %
         wave sidebands are within the cavity pole and are transmitted through %
         the OMC. The RF sidebands are in the MHz range and are rejected by the %
         OMC.}
\end{figure}\label{fig:omc-freq}

\section{Searching for Compact Binary Coalescences}

Steal this from O1 CBC DQ paper

\section{Detector Characterization}
