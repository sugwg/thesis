%-----------------------------------------------------------------------
%
% File Name: thesis.tex
%
% Author: Thomas Massinger
%
% Revision: $Id$
%
%-----------------------------------------------------------------------

% document class and packages
\documentclass[12pt,notitlepage]{report}
\usepackage{bibunits}
\usepackage{suthesis}
\usepackage{graphicx}
\usepackage{color}
\usepackage{amsmath}
\usepackage{amssymb}
\usepackage{amsfonts}
\usepackage[bookmarksnumbered, bookmarksopen, breaklinks, colorlinks, linkcolor=blue, citecolor=magenta]{hyperref}
\usepackage{subfig}
\usepackage{tabularx}
\usepackage{adjustbox}
%\usepackage{caption}
%\usepackage{rotating}
%\usepackage{tensor}

\pdfoutput=1
\DeclareGraphicsExtensions{.pdf,.png}

\hbadness=10000

% new command definitions
\newcommand{\half}{\frac{1}{2}}
\newcommand{\ospsd}{\ensuremath{S_n\left(\left|f_{k}\right|\right)}}

% journal definitions
\newcommand{\apj}{{\it Astrophysical J.}}
\newcommand{\apjl}{{\it Astrophysical J.}}
\newcommand{\aap}{{\it Astron. and Astrophys.}}
\newcommand{\cmp}{{\it Commun. Math. Phys.}}
\newcommand{\grg}{{\it Gen. Rel. Grav.}}
\newcommand{\cqg}{{\it Class. Quant. Grav.}}
\newcommand{\lr}{{\it Living Reviews in Relativity}}
\newcommand{\mnras}{{\it Mon. Not. Roy. Astr. Soc.}}
\newcommand{\pr}{{\it Phys. Rev.}}
\newcommand{\prl}{{\it Phys. Rev. Lett.}}
\newcommand{\prd}{{\it Phys. Rev. D}}
\newcommand{\pra}{{\it Phys. Rev. A}}
\newcommand{\prsl}{{\it Proc. R. Soc. Lond. A}}
\newcommand{\ptrsl}{{\it Phil. Trans. Roy. Soc. London}}
\newcommand{\rmp}{{\it Rev. Mod. Phys.}}

\newcommand{\tcr}{\textcolor{red}}
\newcommand{\tcb}{\textcolor{blue}}
\newcommand{\tcm}{\textcolor{magenta}}
\newcommand{\tcg}{\textcolor{green}}
\newcommand{\tcp}{\textcolor{purple}}

\begin{document}
\title{Detector Characterization for Advanced LIGO}
\author{Thomas J. Massinger}
\majorprof{Peter R. Saulson}
\previousdegree{}{B.S. Physics, Utica College, Utica, NY 13502}
%\previousdegree{M.S. Syracuse University, Syracuse, NY, 2013}{}
\submitdate{June 2016}
\degree{Doctor of Philosophy}
\program{Physics}
\copyrightyear{2016}
\majordept{Physics}
\atitlep
\clearpage
\havededicationtrue
\dedication{Well that was fun.}
\haveminorfalse
\copyrighttrue
\doctoratetrue
\figurespagetrue
\tablespagetrue
\electronicsubmitfalse

\Abstract{
The first observing run of Advanced LIGO spanned 4 months, from September 12, 2015 to January 19, 2016,
during which gravitational waves were directly detected from two binary black hole systems,
namely GW150914 and GW151226. Confident detection of gravitational waves requires an
understanding of instrumental noise transients and artifacts that can reduce the sensitivity
of a search for gravitational waves. Studies of the quality of the detector data yield
insights into the cause of instrumental artifacts and data quality vetoes specific to
a search are produced to mitigate the effects of problematic data. 

This dissertation provides an overview of the methods used to characterize noise in the 
LIGO interferometers and provides examples of successful removal of transient noise. 
The data set used in the first observing run is validated. 
Further, the systematic removal of noisy data from analysis time is shown
to improve the sensitivity of searches for compact binary coalescences.
The output of the PyCBC pipeline is used as a metric for improvement. 

The first direct detection of gravitational waves, GW150914, was a loud enough signal that
removing data with excess noise did not improve its significance. However,
the removal of data with excess noise decreased the false alarm rate of GW151226
by a factor of 567, from 1 in 320 years (3.9 $\sigma$) to 1 in 183000 years ($>5.3 \sigma$).
}

\Acknowledgments{
% $Id$

As a member of the LIGO Scientific Collaboration, I have been fortunate to
have benefited through advice from and discussions with many people. It would
not be possible to thank everyone who I have worked with over the past five
years without making the acknowlegements longest chapter in this dissertation,
so I shall only attempt to thank those who I have interacted with the most and
hope that the others forgive me.

First and foremost, I would like to thank Patrick Brady. I am fortunate to
have an advsior whose as big a drunk as I am.

I would also like to thank Jolien Creighton for his help and enthusiasm over
the past five years. It has been fun working with Jolien and I have learnt a
great deal from him through his patient explanations.

I am greatful to Bruce Allen for suggesting the search for binary inspiral as
a research topic and his assitance with the scientific and computational
obstacles along the way. Thanks also to Gabriela Gonz\'{a}lez for patiently
anserwing my many stupid questions about the LIGO interferometers helping me
understand the data that I have been analyzing.

I would like to thank the members of my committee: Daniel Agterberg, John
Friedman and Leonard Parker for their careful reading of this dissertation and
helpful suggestions for its improvement.

I also would like to thank Warren Anderson, Teviet Creighton, Stephen
Fairhurst, Scott Koranda, Eirini Messaritaki, Ben Owen, Xavier Siemens and
Alan Wiseman for help, advice and pints of beer. I am also indebted to Axel's
for many useful discussions.

Thanks to Steve Nelson, Wyatt Osato and Quiana Robinson for their help in the
preparation of this this and, of course, to Sue Arthur for making everything
run smoothly.

I could not have come this far without the constant love and support of my
parents, to whom this thesis is dedicated. Finally, I would like to thank
Emily Dobbins for all the love and understanding over the past two years.
}

\beforepreface

\prefacesection{Preface}
The work presented in this thesis stems from my participation in the LIGO
Scientific Collaboration (LSC). This work does not reflect the
scientific opinion of the LSC and it was not reviewed by the collaboration.

\afterpreface

\Chapter{Introduction}
\label{ch:introduction}
In 1915, Albert Einstein published his theory of general relativity, a 
geometric theory of gravitation that sought to expand upon Newtonian 
mechanics and provide a complete description of gravity and its 
relationship with space and time. Einstein theorized that space 
and time were deeply related and existed together as a manifold 
called spacetime. Matter with energy and momentum 
existing in this manifold would create 
curvature in spacetime. Gravitational forces were the result of 
matter following geodesic curves in spacetime. This concept can 
be summarized in the Einstein field equation, which is presented 
as,
\begin{equation}
G_{\mu\nu} = 8\pi T_{\mu\nu}
\label{eq:EFE}
\end{equation}
where $G_{\mu\nu}$ is the Einstein tensor, which describes the 
curvature of spacetime, $T_{\mu\nu}$ is the 
stres-energy tensor, which describes the energy and momentum in 
spacetime, and  $G=c=1$. The Einstein tensor is defined as,
\begin{equation}
G_{\mu\nu} = R_{\mu\nu} - \frac{1}{2}Rg_{\mu\nu}
\end{equation}
where $R_{\mu\nu}$ is the Ricci curvature tensor and $g_{\mu\nu}$ is 
the metric tensor for the manifold.

An interesting result that arises from this formalism is the 
existence of gravitational waves, which are perturbations in 
spacetime caused by certain types of time-varying mass distributions. 
To describe gravitational waves, we consider 
a Minkowski metric with a small perturbation. The Minkowski metric 
is a flat spacetime metric defined as
\begin{equation}
\eta_{\mu\nu} = 
  \begin{pmatrix}
   -1 & 0 & 0 & 0 \\
    0 & 1 & 0 & 0 \\
    0 & 0 & 1 & 0 \\
    0 & 0 & 0 & 1
  \end{pmatrix}
\end{equation}
where $\mu = 0$ corresponds to the time coordinate and $\mu = {1,2,3}$ 
correspond to the spatial coordinates. In examples, we will use the coordinate 
convention $(x^0,x^1,x^2,x^3) = (ct,x,y,z)$. 
The full spacetime metric, $g_{\mu\nu}$, is then constructed as a 
linear perturbation on the Minkowski metric,
\begin{equation}
g_{\mu\nu} = \eta_{\mu\nu} + h_{\mu\nu}
\end{equation}
where $h_{\mu\nu}$ is the metric perturbation and $|h_{\mu\nu}| \ll 1$.
From here, we follow the convention of Saulson \cite{Saulson:1994} to arrive at the general 
form of a gravitational wave.
At this point it is very useful to move into the transverse traceless 
gauge where coordinates on the manifold are defined by the geodesic 
motion of freely-falling test masses. In this gauge, the weak field 
vacuum solution of the Einstein field equation becomes a wave equation: 
\begin{equation}
\square h_{\mu\nu} = 0.
\end{equation}
The solutions to this differential equation will be plane waves of 
the form
\begin{equation}
h_{\mu\nu} = C_{\mu\nu}e^{i(2\pi ft - \vec{k}\cdot\vec{x})}
\end{equation}
where $C_{\mu\nu}$ is the wave amplitude, $f$ is the frequency, 
and $\vec{k}$ is the wave vector which indicates the direction of 
propagation \cite{Carroll}.

For example, consider the case of a gravitational 
wave propogating along the $\hat{z}$-axis.
When the conditions of the transverse traceless gauge are applied, 
the resulting form of $h_{\mu\nu}$ is 
\begin{equation}
h_{\mu\nu} = 
  \begin{pmatrix}
    0 & 0 & 0 & 0 \\
    0 & h_+ & h_x & 0 \\
    0 & h_x & -h_+ & 0 \\
    0 & 0 & 0 & 0
  \end{pmatrix}
\end{equation}
where the diagonal and off-diagonal terms represent two polarizations 
of the resulting gravitational wave, called "h-plus" and "h-cross" 
respectively.
We can see the effects of this perturbation by observing the  
spacetime interval on the manifold. The spacetime interval is defined as 
\begin{equation}
ds^2 = dx^\mu g_{\mu\nu}dx^\nu.
\end{equation}
Substituting in our perturbed metric for $g_{\mu\nu}$, we find that 
the spacetime interval can be broken up into a standard Minkowski line 
element and a perturbation due to $h_{\mu\nu}$.
\begin{equation}
ds^2 = dx^\mu (\eta_{\mu\nu} + h_{\mu\nu})dx^\nu \\
\end{equation}
\begin{equation}
ds^2 = dx^\mu \eta_{\mu\nu} dx^\nu + dx^\mu h_{\mu\nu}dx^\nu
\label{eq:spacetime}
\end{equation}

As an example, we present the case of a plus-polarized gravitational wave 
propagating in the $\hat{z}$ direction and observe the effect of the perturbation 
on the spacetime interval. The perturbation will have the form 
\begin{equation}
h_{\mu\nu} = 
  \begin{pmatrix}
    0 & 0 & 0 & 0 \\
    0 & h_+ & 0 & 0 \\
    0 & 0 & -h_+ & 0 \\
    0 & 0 & 0 & 0
  \end{pmatrix}
\end{equation}
Using the coordinate convention of $(ct,x,y,z)$, the unperturbed
spacetime interval is given as: 
\begin{equation}
ds^2 = -c^2 dt^2 + dx^2 + dy^2 + dz^2.
\end{equation}
Since the perturbation is spatially transverse to the direction of 
propagation, the ct- and z-coordinates will not be modulated by the 
gravitational wave. The x- and y-coordinates will be modulated  
according to equation \ref{eq:spacetime}. The resulting spacetime 
interval is
\begin{equation}
ds^2 = -c^2 dt^2 + (1 + h_+)dx^2 + (1 - h_+)dy^2 + dz^2.
\end{equation}
This shows that a gravitational wave propogating along the $\hat{z}$-axis 
will differentially stretch and squeeze spacetime in the transverse 
axes. The exact form of $h_+$ will depend on the source of the 
gravitational waves. A visualization of this stretching and squeezing 
is shown in Figure \ref{fig:polarizations}\cite{Polarization}. The cross polarization  
stretches and squeezes at a 45 degree angle relative to the plus 
polarization.

\begin{figure}[ht!]
\includegraphics[width=\textwidth]{figures/introduction/polarisations2}
\caption[Plus and cross polarizations]{Plus and cross polarizations %
         of a gravitational wave.}
\label{fig:polarizations}
\end{figure}

The Advanced LIGO interferometers are designed to be sensitive 
to this differential stretching and squeezing by constructing orthogonal 
optical cavities. A gravitational wave passing through an aLIGO interferometer 
will differentially 
modulate the lengths of the optical cavities, creating an interference 
pattern at the output of the instrument that can be searched for 
gravitational wave signals. The layout and gravitational wave readout scheme 
of the interferometers is discussed below.

\section{The Advanced LIGO Interferometers}\label{sec:aligo}

The Advanced LIGO (aLIGO) interferometers are a pair of dual-recycled Michelson interferometers 
that employ 4km long Fabry-Perot cavities in their arms to increase the interaction time with a 
gravitational wave signal. 
Figure \ref{fig:aligo} shows a simplified layout of an aLIGO interferometer. 

\begin{figure}[ht!]
\includegraphics[width=\textwidth]{figures/introduction/ALIGO_layout}
\caption[Layout of Advanced LIGO]{Layout of Advanced LIGO}
\label{fig:aligo}
\end{figure}

At the input to an aLIGO interferometer is a solid-state Nd:YAG laser that provides laser light 
at a wavelength of 1064 nm. Not included in Figure \ref{fig:aligo} are frequency and 
intensity stabilization control loops designed to provide as stable a laser source as 
possible for the experiment. This stabilized laser is called the pre-stabilized laser 
(PSL). The laser light is passed through a series of 
electro-optic modulators (EOM) where radio-frequency (RF) sidebands are generated 
and imparted onto the light. These RF sidebands are used to control auxiliary optical 
degrees of freedom in the interferometer. The beam is then passed through the 
input mode cleaner (IMC), which rejects higher order spatial modes of the beam 
and transmits a circular TEM00 mode to be used in the instrument.

Once the beam has been stabilized in frequency and intensity and the higher order 
optical modes have been stripped away, it is transmitted through the power 
recycling mirror and enters the vertex of the interferometer. In the vertex, 
the beam is split 50/50 by the beamsplitter. Half of the light is directed toward  
the input test mass (ITM) of the X-arm and half of the light is directed  
toward the ITM of the Y-arm. As mentioned previously, the aLIGO arms are not 
single bounce cavities; they are comprised of Fabry-Perot cavities that allow the 
light to circulate in the arm cavities multiple times. The light is stored in 
the arm cavities for $\sim$1ms, trapped between the highly reflective surfaces 
of the ITM and the end test mass (ETM), before it is transmitted back through 
the ITM and into the vertex.

When a gravitational wave passes through an aLIGO inteferometer, the distance
between the ITM and ETM of each arm is modulated, causing the light to have a
longer or shorter travel time as it traverses the arm. Since gravitational
waves expand space in one direction while the orthogonal direction contracts,     
the X- and Y-arms will experience differential changes in length. When light
from the arms is recombined at the beamsplitter, there will be a difference
in phase between the two beams as they have traveled different paths. The 
resulting light from this recombination of phase shifted beams is called the 
antisymmetric part of the output. The part of the beam that is recombined 
in phase is called the symmetric part of the output.

The beams returning from each arm are recombined at the beamsplitter. The 
symmetric part of the beam 
will be sent back toward the power recycling mirror. The power recycling mirror 
forms a resonant cavity with the ITMs, allowing for light at the symmetric 
port of the beamsplitter to be added coherently to incoming light from the PSL and 
increasing the effective power in the vertex. This increase in effective power 
is known as the power recycling gain. 

The antisymmetric part of the beam is sent toward the signal recycling mirror. 
The signal recycling cavity is used to tune the frequency response of the 
interferometer by adjusting the effective finesse of the coupled cavity 
formed by the signal recycling cavity and the arm cavities. 
If the light returning from the arms has accumulated some differential amount of 
phase as it traveled 
along the arms, perhaps from a gravitational wave modulating the length of each 
arm differentially, it will be transmitted through the signal recycling cavity 
and into the output mode cleaner (OMC). The OMC behaves similarly to the IMC, 
stripping away higher order optical modes and isolating the TEM00 mode of the 
beam. The transmitted, mode cleaned signal is then read out using a homodyne 
detection scheme on a DC photodiode. 

\subsection{DC Readout}

When a gravitational wave modulates the length of an arm cavity, the light 
traveling in that arm experiences a phase modulation. This phase modulation 
can be visualized by picturing the beam in frequency space. In figure 
\ref{fig:omc-freq}, the carrier beam frequency is designated as $f_0$. 
The phase modulation due to 
a gravitational wave signal introduces a frequency sideband at the 
gravitational wave frequency, which is in the 30-2000 Hz range. 
The 
RF sidebands used for auxiliary optical cavity control are offset from the 
carrier frequency by 9, 24, and 45 MHz. 
The RF sidebands, which in a 
homodyne detection scheme would only contribute shot noise to the output signal, 
are rejected by the OMC. The gravitational wave sidebands, however, are at a 
low enough frequency offset that they are within the cavity pole of the OMC 
and are allowed to transmit through the cavity.

Since the OMC DC photodiode measures power, it measures the square of the 
incident optical field and witnesses beat frequencies between different 
components of the light. If the RF sidebands have been filtered out by 
the OMC, the only remaining beat note will be that of the carrier beam ($f_0$) 
beating against the gravitational wave sideband ($f_0 + f_{GW}$). This beat note will 
appear as the difference in frequency between the two optical fields, 
leaving behind a signal in the 30-2000 Hz range ($f_{GW}$) and providing a 
natural demodulation inherent to the measurement process. 
The process of recovering the gravitational wave sideband using the 
carrier field as a reference is known as homodyne detection. 

\begin{figure}[ht!]
\includegraphics[width=\textwidth]{figures/introduction/omc-freq}
\caption[Sidebands and OMC cavity pole]{Frequency domain visualization of beam %
         at OMC. Grey dotted lines indicate the cavity pole. The gravitational %
         wave sidebands are within the cavity pole and are transmitted through %
         the OMC. The RF sidebands are in the MHz range and are rejected by the %
         OMC.}
\end{figure}\label{fig:omc-freq}

\section{Sources of Gravitational Waves}
Include that box with modeled, unmodeled, transient, and continuous.

CBCs are the bread and butter, expect BNS, NSBH, and BBH sources
Continuous waves from pulsars
Bursts from supernovae
Stochastic background


\section{Searching for Compact Binary Coalescences}

Steal this from O1 CBC DQ paper


\section{The Advanced Detector Network}

The Advanced Laser Interferometer Gravitational-Wave Observatory (aLIGO) is 
part of a worldwide effort to detect gravitational waves from astrophysical 
sources. The two aLIGO interferometers, one in Hanford, WA and one in 
Livingston, LA, are part of a growing network of ground-based interferometric 
gravitational wave detectors. Each aLIGO interferometer has 4km long arms 
arranged in an L-shaped configuration. A gravitational wave passing through 
an aLIGO interferometer will cause the arms to expand and contract, 
creating an interferometric signal at the output of the instrument. 
Section \ref{sec:aligo} contains a more detailed description of the aLIGO 
interferometers. 

There are a number of other interferometric gravitational wave detectors 
being built and commissioned for future use in collaboration with aLIGO.
The Advanced VIRGO detector is being built and commissioned in Cascina, Italy. 
When it is fully commissioned, VIRGO will be joining LIGO in observing runs. 
The VIRGO interferometer has 3km arms, which should provide enough 
sensitivity to allow for triangulation of astrophysical sources.

The GEO600 detector, located in Hanover, Germany is an interferometer built in 
collaboration between Germany and the United Kingdom. 
GEO600 is an extremely valuable test bed for interferometric technologies,
including quantum optics and homodyne detection. However, with 600m arms, GEO600 
is unlikely to be sensitive enough to witness expected astrophysical sources.

The KAGRA detector, located underground in the Kamioka mine in Japan, 
is in its commissioning phase. KAGRA has 3 km long arms and, 
unlike other gravitational wave interferometers, employs cryogenics to 
reduce thermal noise in its optics. When complete, KAGRA should be 
sensitive enough to contribute to the worldwide detector network.

Include that cool picture of the advanced detector network.


\Chapter{Searching for Compact Binary Coalescences}
\label{ch:CBCSearches}
Since a signficant portion of this thesis uses the performance of 
CBC searches as a metric for the quality of the 
data, a more thorough discussion of how a CBC search 
works is necessary. This thesis will focus on the output of the PyCBC 
pipeline, which is a Python-based software packaged used to search for 
gravitational waves \cite{Usman:2015kfa,pycbc-github}

CBC search pipelines are designed to search for gravitational wave transients 
from compact binary coalescences \cite{Usman:2015kfa}. 
The signals expected to be measured in the LIGO interferometers are extremely quiet, 
with gravitational wave strains on the order of $10^{-22}$. On these scales, most 
signals will not be able to be extracted from the background noise with simple filtering. 
Figure \ref{fig:quiet-BNS} shows the gravitational wave strain from a 1.4-1.4$M_{\odot}$ 
binary neutron star system at a distance of 20 Mpc overlaid on real detector noise from 
the Livingston 
interferometer. The signal has a peak strain roughly two orders of magnitude lower than 
the peak strain of the detector noise. 
For this reason, the CBC searches employ a matched filter algorithm, which correlates 
expected CBC
waveforms with detector data and assigns a ranking statistic, the signal-to-noise 
ratio (SNR), to every event that it finds. 

\subsection{The matched filter}

The matched filter calculates the correlation of the detector data with expected CBC 
waveforms in the frequency domain. The detector data and expected waveform are 
multiplied together and their product is divided by the background noise in the detector. 
The fundamental operation of the matched filter is defined as an inner product of the 
detector data and the CBC waveform \cite{Usman:2015kfa},
\begin{equation}
(s|h)(t) = 4\mathrm{Re}\int_{f_\mathrm{low}}^{f_\mathrm{high}} \frac{\tilde{s}(f)\tilde{h}^*(f)}{S_n (f)}e^{2\pi i f t}\, \mathrm{d}f,
\label{eq:inner-product}
\end{equation}
where $\tilde{s}$ is the Fourier transformed detector data, $\tilde{h}$ is the Fourier
transformed gravitational waveform, and $S_n (f)$ is the power spectral
density of the detector data averaged over 2048 seconds, which represents the average 
noise in the detector. The bounds of the integral are set to span the frequency space 
for which the interferometers are sensitive enough to detect gravitational waves, 
typically 30 - 2000 Hz. 

As described in Equation \ref{eq:strain}, a gravitational wave is comprised of two 
polarizations, ``h-plus" and ``h-cross". When the data are searched for a CBC  
signal, a waveform is generated for each polarization and the matched filter is 
computed separately for each polarization. 
The SNR of a CBC waveform at any given time is defined as the weighted quadrature sum 
of the SNR measured for each polarization \cite{Usman:2015kfa}, 
\begin{equation}
\rho^2(t) = \frac{(s|h_\mathrm{p})^2 + (s|h_\mathrm{c})^2}{(h_\mathrm{p}|h_\mathrm{p})},
\label{eq:snr}
\end{equation}
where $h_\mathrm{p}$ and $h_\mathrm{c}$ are the plus and cross polarizations of the 
modeled gravitational waveform respectively and $s$ is the detector data. 
When the SNR time-series defined in equation \ref{eq:snr} crosses a certain threshold, 
the waveform is considered to have significant overlap with the detector data and an 
event is generated at the time of the SNR peak. These events are called ``triggers'' 
and are used to generate populations of potential gravitational wave events for analysis.

\begin{figure}[ht!]
\includegraphics[width=\textwidth]{figures/introduction/quiet_BNS}
\caption[BNS signal in detector noise]{A simulated gravitational wave signal from a %
         binary neutron star system overlaid on real detector noise from L1. %
         The blue curve, labeled $h(t)$, represents the detector data. %
         The red curve represents %
         the gravitational wave strain expected from a 1.4-1.4$M_{\odot}$ %
         binary neutron star system %
         at 20 Mpc. The peak strain of the binary neturon star waveform is %
         $8\times10^{-22}$. %
         The detector data have been high pass filtered with a corner frequency %
         at 20 Hz and show a peak strain of $2.2\times10^{-19}$. The signal is %
         buried in the detector noise and requires a matched filter algorithm %
         to be recovered. At this time, the inspiral range for a 1.4-1.4$M_{\odot}$ %
         BNS system was 60 Mpc, indicating that the same system originating at 60 Mpc %
         would be recovered with SNR = 8. % 
         }
\label{fig:quiet-BNS}
\end{figure}

\subsection{Waveform templates}

To perform a search, the matched filter algorithm needs to know what to search for.
A collection of expected CBC waveforms is generated using
the formalism of general relativity before the analysis \cite{Pan:2009wj,Purrer:2015tud}.
Each of the expected waveforms is called a template and
the full collection of waveforms is referred to as the template bank. This template bank
is constructed to span the astrophysical parameter space included in the search
\cite{GW150914-CBC}. This parameter space is constrained by the noise spectrum of 
the interferometers. As shown in Figure \ref{fig:noise-budget}, the LIGO 
interferometers are sensitive enough to detect gravitational waves in the 
region from roughly 30 - 2000 Hz. This rules out detection of sources that are expected to 
coalesce at very low frequencies, such as supermassive black hole binaries 
\cite{Merloni:2008tj} and 
binary white dwarf systems \cite{PostnovYungelson:2006}. 
The template bank used in Advanced LIGO's first observing 
run consisted waveforms representing binary neutron stars, binary black holes, 
and neutron star-black hole binary systems \cite{GW150914-CBC}. 
The total masses of these systems ranged from 
2-100$M_{\odot}$. This reflects the set of systems that will have merger frequencies 
above 30 Hz and will have detectable power in LIGO's sensitive bandwidth.

Each waveform is defined by the mass and spin of each compact
object in the binary system. 
It is convenient to combine
the component masses into a new variable, chirp mass, which is used to
parameterize gravitational wave signals in general relativity. Chirp mass is defined
as
\begin{equation}
M_{chirp} = \frac{(m_1m_2)^{3/5}}{(m_1 + m_2)^{1/5}}
\end{equation}
where the $m_{i}$ are the component masses of the compact objects in the
binary system. 
It is also convenient to combine the effects of each
object's spin into one parameter called effective spin, $\chi_{eff}$,
which is the
mass-weighted spin of the system \cite{Privitera:2013xza}. $\chi_{eff}$ is defined as
\begin{equation}
\chi_{eff} = \frac{\chi_{1}m_{1} + \chi_{2}m_{2}}{m_{1} + m_{2}}
\end{equation}
where the $\chi_{i}$ are the dimensionless spin parameters \cite{Kidder:1995zr}
and the $m_{i}$ are the masses for each compact object in the binary system. 

\subsection{$\chi^{2}$ signal consistency test}\label{sec:chisq}

If the data produced by the interferometers were Gaussian, the matched filter would 
be sufficient for running a search pipeline and recovering gravitational wave signals. 
Unfortunately, the data are non-Gaussian, containing noise transients of varying 
durations \cite{Nuttall:2015dqa,GW150914-DETCHAR}. These noise transients, 
or ``glitches", can have significant amplitude 
and, when multiplied with a waveform template in the matched filter, can cause 
loud triggers to be generated. 
However, a significant advantage of performaing a modeled search for gravitational 
waves is that 
we know we're looking for. With this information, the SNR can be refined into a more 
robust ranking statistic for significant events in the data. This is done using the 
$\chi^{2}$ signal consistency test \cite{Allen:2004gu}. 

The SNR produced by the matched filter is an integral in the frequency domain which 
reports the total accumulated SNR over a given bandwidth. If a noise transient has 
significant amplitude, it can generate a high SNR trigger by overlapping with the 
waveform template in the matched filter. However, these noise 
transients typically have a duration on the order of 0.1s. This type of transient 
is easily distinguished from a chirp signal that increases monotonically in frequency 
over the span of many seconds. 
The $\chi^{2}$ test divides each CBC waveform into
frequency bins of equal power, checking that the SNR is distributed as a 
function of frequency
as expected from an actual CBC signal.
For a signal divided into $p$ frequency bins, each bin should contain $\frac{1}{p}$ of the power in the 
signal. 
In the $\chi^2$ calculation, the SNR is calculated for each frequency bin and compared to the expected 
amount. The $\chi^2$ statistic is calculated as \cite{Usman:2015kfa}
\begin{equation}
\chi^2 = p\displaystyle\sum_{l=1}^{p}\left[\left(\frac{\rho_\mathrm{p}^2}{p}-\rho_{\mathrm{p},l}^2\right)^2 + \left(\frac{\rho^2_\mathrm{c}}{p}-\rho_{\mathrm{c},l}^2\right)^2 \right] \, ,
\label{eq:chisqr}
\end{equation}
where $\rho^2_\mathrm{p}$ is the SNR of the plus polarization of the waveform, $\rho^2_\mathrm{c}$ 
is the SNR of the cross 
polarization of the waveform, $p$ is the number of frequency bins, and 
$\rho^2_{i,l}$ is the calculated SNR for the $l^\mathrm{th}$ frequency bin.
The $\chi^2$ statistic is then normalized such that a real signal will be reported with a value of 1. 
This normalized $\chi^2$ is called the reduced $\chi^2$ and is denoted by $\chi^2_r$.

In the PyCBC search, each trigger that comes out of the matched filter search is 
weighted based on the results of
the $\chi^{2}$ test. This is folded into a new ranking statistic for CBC triggers,
which is called re-weighted SNR and is denoted by $\hat{\rho}$.
The re-weighted SNR is calculated as \cite{Usman:2015kfa} 
\begin{equation}
\hat{\rho} = \left\{\begin{array}{lr}
\rho\left/\left[(1+(\chi^2_r)^3)/2\right]^\frac{1}{6}\right., & \text{if } \chi_r^2 > 1, \\
\rho, & \text{if } \chi_r^2 \le 1,
\label{eq:reweighted}
\end{array}\right.
\end{equation}
where $\rho$ is the measured SNR and $\chi^2_r$ is the reduced $\chi^2$.
It is important to note that if a real signal has a power
distribution that matches the template waveform, it will not be
down-weighted by the $\chi^{2}$ test.  

This test is extremely powerful, as shown in Figure \ref{fig:cbc-newsnr-histograms}, 
which shows the distribution of single detector PyCBC triggers generated from 
September 12 to October 20, 2015. 
Figure \ref{subfig:l1-snr-hist} shows the distribution of triggers in SNR. 
The extensive tail of
triggers with high SNR, which is generated when high amplitude noise transients 
are processed by the matched filter, extends beyond SNR 100. 
These high SNR triggers are down-weighted in the re-weighted SNR distribution,
leaving behind a tail that extends to $\hat{\rho} \approx$ 11.5 as seen in Figure 
\ref{subfig:l1-newsnr-hist}. Keeping in mind that a real signal will be reported 
at the same value in each plot, the re-weighting of triggers has lowered the 
noise floor, allowing for signals with SNR $>$ 11.5 to stand out as the 
loudest events in their respective interferometers rather than being buried 
beneath a population of high SNR triggers.  

\begin{figure}[!ht]%
\centering
  \subfloat[]{
      \includegraphics[width=.75\textwidth]{figures/introduction/l1-snr-histogram}
      \label{subfig:l1-snr-hist}
  }

  \subfloat[]{
      \includegraphics[width=.75\textwidth]{figures/introduction/l1-newsnr-histogram}
      \label{subfig:l1-newsnr-hist}
  }

  \caption[PyCBC SNR and re-weighted SNR histograms]{Histograms of single interferometer PyCBC %
           triggers from the Livingston (L1) interferometer. %
           These triggers were generated from September 12 to October 20, 2015. These histograms %
           contain triggers from the entire template bank, but %
           exclude any triggers found in coincidence between the two interferometers. %
           (\ref{subfig:l1-snr-hist}) A histogram of single interferometer triggers in SNR. %
           The tail of this distribution extends beyond SNR = 100. %
           (\ref{subfig:l1-newsnr-hist}) A histogram of single interferometer triggers in re-weighted SNR. %
           The chi-squared test down-weights the long tail of SNR triggers %
           in the re-weighted SNR distribution. Note that the x-axis has different %
           limits in each plot.}
  \label{fig:cbc-newsnr-histograms}
\end{figure}

The remaining tail of re-weighted SNR triggers represents the loudest 
background triggers in the CBC search. Investigating this set of
loudest background triggers guides data quality efforts in defining the current 
limiting noise sources to the CBC search. This process is detailed in Chapter 
\ref{ch:instrumentalDetchar}.

\subsection{Searching for signals}

The matched filter algorithm is run separately on each interferometer's data using the 
same bank 
of template waveforms. The output of the matched filter, the SNR time-series, is scanned 
for peaks and a set of single interferometer 
triggers is generated. The two sets of single interferometer triggers are then compared to
search for any events that were recorded within 15ms of each other.
Gravitational waves are predicted by general relativity to travel
at the speed of light. The light travel time between the two interferometers is 10 ms
and 5ms is added to the coincidence window due to uncertainty in signal arrival time
\cite{GW150914-CBC}.
As such, any triggers that are found within 15 ms of each other and
are recovered with the same source parameters are considered to be coincident between 
the two
interferometers. These coincident triggers represent potential gravitational wave signals
and are referred to as ``foreground events". 

The ranking statistic for coincident events in the PyCBC search is the network
re-weighted SNR, $\hat{\rho}_{c}$, which is the quadrature sum of the re-weighted
SNR from each interferometer.  
\begin{equation}
\hat{\rho}_{c} = \sqrt{\hat{\rho}^2_L + \hat{\rho}^2_H}
\end{equation}
where $\hat{\rho}_L$ is the re-weighted SNR measured in the Livingston detector 
and $\hat{\rho}_H$ is the re-weighted SNR measured in the Hanford detector.

Most of these foreground events will be
chance coincidences between noise in each interferometer, which is expected given
the number of events in each data set. A large number of foreground events will be 
generated due to the detector noise, but ideally the distribution of foreground events 
will fall of sharply in $\hat{\rho}_{c}$, allowing for genuine signals to be 
recognized. To understand how stastically significant a foreground event is, a background 
distribution must be generated.

To generate the background distribution, we return to the set of single interferometer 
triggers that were generated when the matched filter was initially run. Since we want 
to understand the distribution of triggers based on detector noise, all of the triggers 
that were found to be coincident between the two interferometers are removed from the 
data sets, effectively removing all
potential gravitational wave signals. The remaining triggers are then due to fluctuations 
in the background noise in each interferometer. These two sets of triggers,
one from each interferometer, are then time shifted by a duration longer than the
light travel time between the interferometers.
Since gravitational waves are predicted to travel at the speed of light,
this time shift ensures that the
two sets of triggers are astrophysically uncorrelated and do not contain any gravitational
wave signals. The coincidence test is then performed again with the time shifted triggers,
resulting in a coincident trigger set which represents background noise. 
A full distribution of background triggers is generated by performing this timeslide
technique every 0.1 seconds and iterating over all of the data used in the search 
\cite{Usman:2015kfa}. This 
distribution tells us how often we should expect to see coincident triggers with the 
same waveform parameters at a given value of SNR.

The statistical significance of any candidate gravitational wave is
evaluated by calculating the rate of background events from detector noise 
that are at least as
loud as the candidate event \cite{GW150914-CBC}. 
The results of the first observing run and details on 
the significance of foreground events is presented in Chapter \ref{ch:o1results}. 
Any loud triggers that appear as the result of instrumental
transients will contribute to the tail of the background distribution and
the influence the statistical significance of a recovered foreground event.
The process of performing data quality investigations and data validation are 
detailed in Chapter \ref{ch:instrumentalDetchar}

\subsection{Gating}\label{sec:gating}

The PyCBC search includes a data conditioning stage that applies preventative  
cuts to remove large transients from the input data stream. 
This is done in a process called gating \cite{Usman:2015kfa}, which uses a
window function to remove times containing large transients from the input data stream. This is done
by applying a smooth windowing function to the problematic section of data, excising the large transient.
The gating process is tuned by modifying the selection criteria for transients to be removed and
by adjusting the time window to remove around each transient.
Chapters \ref{ch:GW150914-DQ} and \ref{ch:GW151226-DQ} contain details about the 
gating thresholds used for the analyses containing GW150914 and GW151226 respectively. 



\Chapter{The First Observing Run}
\label{ch:o1results}
\section{O1 results}

We found GW150914!




\Chapter{Detector Characterization}
\label{ch:instrumentalDetchar}
\section{Methods of Detector Characterization}

The Detector Characterization (DetChar) group works at the interface 
between the instrument science and data analysis groups. The goal of 
the group is to understand the effects of instrumental noise sources on 
the output of astrophysical searches and mitigate them if possible.

The first step in a detector characterization study is identifying 
noisy or problematic data. These studies can be initiated in a number of ways. 
The three most common are the appearance of 
loud background events in an astrophysical search pipeline, a message from 
the commissioning team regarding instrument performance, or excess noise 
flagged by data quality monitoring software. Section \ref{sec:tools} discusses 
the data quality monitoring software further.

There are a large number of recorded signals used to monitor and control the
interferometers that are not used in astrophysical searches. These auxiliary
channels are considered safe to use for noise characterization because they are
not sensitive to gravitational wave signals. Analyses of auxiliary channels
allow for the identification of systematic noise sources \cite{Smith:2011,Isogai:2010},
such as environmental
disturbances \cite{Effler:2014zpa} or excess motion of auxiliary optics in the
interferometer \cite{GW150914-DETECTORS,InstrumentNoisePaper}. 

Once data with excess noise have been identified, they must be characterized 
in order to track down the source of the noise. A number of questions can 
be asked to characterize the noise. Is the noise transient or a slow 
drift? What is the typical frequency and bandwidth of the noise? Does the 
noise follow a power law in frequency? Does the 
noise have a characteristic shape in the time-frequency plane? Are the 
noisy frequecies of the signal coherent with other signals in the instrument 
such as environmental monitors and optical control signals? 
Is the noise source localized to a specific chamber or does it exist at 
multiple physical locations in the interferometer? Does the characteristic 
frequency match any of the known mechanical resonances in the interferometer? 
If the noise is a slow drift, does it correlate with the slow drift of 
other signals in the interferometer? Does the noise seem highly digital 
or discretized? After gathering all available information about the 
character of the noise and its coupling mechanisms, efforts shift 
toward attempting to mitigate the effects of the noise on search 
pipelines. 

There are two primary ways to mitigate the effects of instrumental noise on 
the output of a search pipeline. The first option, which is highly preferred, 
is to track down the source of the noise in the interferometer and fix the 
problem at its origin. If investigations provided enough information that 
a problem can be traced back to a specific piece of electronics or a 
specific control loop, the problem can be fixed at the source. However, 
this is not always possible since instrument noise can be difficult to 
pin down and hardware repairs are often too invasive to perform during 
an observing run.

If the problem cannot be fixed at the source, the second option is to 
remove the problematic data from the astrophysical analyses. 
When a significant noise source has
been identified using auxiliary channels and cannot be repaired immediately, 
a data quality flag can be generated
to indicate times when the output data from the interferometer is not nominal
\cite{Nuttall:2015dqa,S6DetChar,GW150914-DETCHAR,Amaldi}.
Data quality vetoes are discussed further in section (??).  
If possible, it is always preferable to fix a problem at the source. 

\section{Tools and algorithms}\label{sec:tools}

Identifying and characterizing instrument noise is facilitated by a 
suite of software tools and algorithms designed to flag data with 
excess noise and help correlate this noise with other signals in the 
interferometer. The major tools required for understanding the data 
quality investigations in this thesis are discussed below. 

\subsection{Omicron}

One way to quantify the amount of excess noise in $h(t)$ is to look 
for times where the signal contains excess power using 
Omicron, a burst algorithm. 
The first stage of the Omicron pipeline applies a set of signal conditioning 
processes, including a whitening filter, a high pass filter, and a downsampling 
process to $h(t)$. Once the data have been whitened they are projected into 
a sine-Gaussian basis. Each sine-Gaussian basis function is defined 
by a central time, $t_0$, a central frequency, $f_0$, and a Q-factor, 
which is defined as \cite{McIverThesis}  
\begin{equation}
Q = \frac{f_0}{\Delta f} = 4\pi f_0 \Delta t,
\end{equation} 
where $\Delta t$ is the time duration of the sine-Gaussian and 
$\Delta f$ is the bandwidth of the sine-Gaussian. Using these parameters, 
each sine-Gaussian basis function can be represented as a tile in the 
time-frequency plane centered around $t_0$ and $f_0$, where the width of 
the tile is determined by the time duration 
and height of the tile is determined by the frequency bandwidth. 

For each 
of these tiles, the energy is measured and compared to the median tile 
energy. If there is an excess of energy in a given tile relative to the 
median tile energy, a signal-to-noise ratio (SNR) is calculated and a trigger 
is generated to annotate the event. For each trigger, the SNR, central time, 
central frequency, duration, bandwidth, and Q of the tile are recorded. 

Once the data have been decomposed into the full set of basis functions, the 
resulting set of triggers is sent through a clustering algorithm. This is 
necessary because the set of sine-Gaussians is an overcomplete, 
non-orthogonal basis and a single event in the data can generate multiple 
triggers corresponding to different values of $t_0$, $f_0$, and Q. The 
resulting clustered triggers define the peak time, peak frequency, and 
SNR of a cluster as the central time, central frequency, and SNR of the 
most significant tile in the cluster. 

The most useful way to visualize the output of Omicron is in the 
time-frequency-SNR plane, sometimes referred to as a 'glitchgram', 
where each trigger is represented as a point 
in a scatterplot. Figure \ref{fig:glitchgram} shows an example set of 
Omicron triggers in the time-frequency-SNR plane. Each dot represents 
a trigger at a certain peak time and peak frequency. The color of each 
dot represents the SNR of that trigger. 

In Gaussian noise, the SNR of a 
given trigger is not expected to exceed 8. In this example, there are a number 
of triggers with SNR $> 8$, some with noticeable structure and some that seem 
more randomly scattered, that represent noise in the output of the interferometer. 
For example, there are numerous triggers between 10-20 Hz that represent excess noise at 
these frequencies, likely due to scattered light in the 
interferometer. There is a line of triggers at just above 2kHz that indicates 
a noise source with a constant peak frequency whose amplitude is being modulated 
and a high SNR is being reported. There is also a scattering of points with high 
SNR that are not as structured as the previous two examples, each one likely due 
to an individual loud glitch rather than a constant, systematic noise source.

\begin{figure}[ht!]
\includegraphics[width=\textwidth]{figures/detchar/Omicron-Dec19}
\caption[Omicron time-frequency-SNR plot]{Time-frequency-SNR plot of Omicron triggers, %
         often referred to as a 'glitchgram'. Each dot on this plot represents an %
         event in $h(t)$ that was recorded with a peak time, peak frequency, and SNR. %. 
         In Gaussian noise, all triggers on %
         this plot would have an SNR $< 8$. Since this plot is generated using real %
         detector data from O1, there are structures of loud triggers that indicate %
         populations of noise transients. For example, the clusters of triggers %
         between 10-20 Hz that likely represent scattered light in the interferometer. %
         }
\label{fig:glitchgram}
\end{figure}

The results of Omicron are a commonly used and extremely valuable tool 
for characterizing the noise in the instrument. A cursory glance at 
Figure \ref{fig:glitchgram} identifies 3 populations of noise in 
the instrument, each of which can be followed up on individually 
to discover both the source of the noise and its effect on astrophysical 
searches. Omicron triggers can also be used in statistical analyses to 
find correlated noise between auxilary channels and $h(t)$. An often used 
example of this, Hierarchichal Veto, is discussed below. 

\subsection{Hierarchichal Veto}

One tool that we have often used in Detector Characterization to look 
for time coincidence between noise transients, or 'glitches', in auxilary 
channels and the output of the interferometer is Hierarchical Veto (Hveto) 
\cite{Smith:2011}. 
Typically, Hveto is used to compare a channel that potentially contains 
gravitational wave signals, denoted $h(t)$, and an auxiliary channel 
that does not have direct astrophysical implications. Hveto counts 
the number of coincident triggers between two time series using a 
user-defined time window centered around each trigger in the auxiliary 
channel. 
Figure \ref{fig:hveto-aux} shows an illustration of auxiliary channels 
with noise transients that are coincident with noise in $h(t)$. Hveto 
iterates over all auxiliary channels to search for noise that is coincident 
with noise in $h(t)$ in a statistically significant way.

\begin{figure}[ht!]
\includegraphics[width=\textwidth]{figures/detchar/hveto_example}
\caption[Example of coincident noise]{A time-series illustrating coincident noise %
         between auxiliary channels and $h(t)$. The top panel is $h(t)$, which contains %
         multiple noise artifacts of varying duration. The middle panel is a readout of %
         wind speed on site, which shows an elevated period coincident with a longer duration %
         burst of noise in $h(t)$. The third panel is a readout of a microphone on site, %
         which shows two glitches that are coincident with bursts in $h(t)$. If noise in %
         these auxiliary channels are coincident with noise in $h(t)$ in a statistically %
         significant way, the noisy data in $h(t)$ can be removed. Figure reproduced from %
         \cite{Smith:2011}.
         }
\label{fig:hveto-aux}
\end{figure}

The figure of merit returned by Hveto for each auxiliary channel 
after comparison to $h(t)$ is called significance.
Significance answers the following question: how unlikely is it that 
the coincident triggers in these two channels were the result of 
two arbitrary Poisson processes occurring in each channel? 
More specifically, given two arbitrary Poisson processes, how 
unlikely is it that we measure $n$ or more coincident triggers 
given that expected number of coincidences from random chance is $\mu$?

Significance is calculated as \cite{Smith:2011},
\begin{equation}
S = -\log_{10} (\sum\limits_{k = n}^{\infty} P(\mu,k)),
\end{equation}
where $n$ is the number of coincidences found between the two channels 
during the total analysis time and $P(\mu,k)$ is the Poisson probability 
distribution function,
\begin{equation}
P(\mu,k) = \frac{\mu^{k}e^{-\mu}}{k!},
\end{equation}
where $\mu$ is the expected number of coincidences between triggers in 
$h(t)$ and the auxiliary channel based solely on chance, which is estimated as,
\begin{equation}
\mu = \frac{N_{h}N_{aux}T_{win}}{T_{tot}},
\end{equation}
where $N_{h}$ and $N_{aux}$ are the number of triggers in $h(t)$ and a 
given auxiliary channel respectively, 
$T_{tot}$ is the total analysis time, and $T_{win}$ is the length of the 
coincidence window used.

A high value of significance indicates that the triggers in the channels 
were very often coincident in time and that there is a very small probability 
that their intersection is a product of random chance. This is a very useful 
measure when we are searching for auxiliary channels that might have some 
noise coupling into our output channel. A significance value of up to 5 is 
often observed in channels with no causal relationship to $h(t)$ \cite{Smith:2011}, 
which is a useful threshold for identifying effective vetoes.

Another interesting figure of merit used for a given comparison Hveto is 
the ratio of $\frac{efficiency}{deadtime}$. Efficiency is defined as the 
percent of triggers vetoed from $h(t)$ during a round of vetoes. Deadtime 
is defined as the percent of total analysis time removed from $h(t)$ during 
a round of vetoes. A ratio of 1 is what we would expect from vetoing time 
at random, indicating no strong time correlation between triggers in the 
two channels. A high value of this ratio, which is ideal, indicates that 
we are vetoing a large number of triggers while maintaining a high percentage 
of our analysis time. This means that the triggers are often close enough 
in time that we can catch a large number of triggers using a small time window.

The deeper utility of Hveto is made evident when a channel is found to have 
a strong correlation with $h(t)$. 
When Hveto discovers an auxiliary channel that has a strong correlation 
with $h(t)$, which is called the round winner, it removes all of the time 
windows surrounding 
auxiliary channel glitches and recalculates the significance of the list of 
auxiliary channels. If a channel's significance has dropped after this removal 
of time, it must have had a large amount of glitches coincident with the 
round winner. The change in significance of each channel is displayed on a 
figure called a `drop-plot`. This is one of the most powerful features of Hveto
 - the ability to find families of channels that often glitch at the same time. 

Ideally, the list of significant channels displayed on the drop-plot will be 
able to localize the issue to a specific subsystem or area of the IFO. 
For example, if a channel representing the alignment of the input mode 
cleaner has glitches that are strongly correlated to $h(t)$, it would be 
interesting to look at the drop-plot and find out what other channels are 
glitching at the same time (suspensions, laser power, etc.).
From there, the issue can be investigated and brought to the attention of 
commissioners for repair or physical inspection. This is not always possible 
as sometimes the cause of the glitches is unclear, but identifying times of 
poor data quality is still useful.

Using Hveto, we can monitor auxiliary channels to find and remove glitches 
in $h(t)$ that would otherwise pollute a gravitational-wave analysis. Removing 
these glitches serves multiple purposes for the search pipelines. Removing 
high SNR glitches cleans up search backgrounds and allows the search 
pipelines to claim a lower SNR threshold for potential detections. A lower 
SNR threshold implies a larger volume for astrophysical analysis. Removing 
glitches reduces the potential for false alarms in the search pipelines, 
which in turn increases the confidence of eventual detections.

\section{Instrumental Detector Characterization Studies}

\subsection{Analog-to-Digital Conversion}

Advanced LIGO interferometers are controlled in real-time using a digital 
control system installed on a series of computers referred to as front end 
computers.  This system overall is referred to as the Front End Control 
(FEC) subsection of the more expansive Control and Data System (CDS).  
In a control loop, the FE computers must be capable of reading in an 
analog signal from the interferometer (position measurements, error signals, 
coil currents, etc), digitally sampling that analog signal, using these now 
digital values in a series of control algorithms, and outputting an analog 
control signal to send back into the interferometer.

The process of digital sampling is handled by an analog-to-digital 
converter (ADC) and the process of analog output is handled by a 
digital-to-analog converter (DAC).  Since these converters are linearly 
mapping a continuous signal onto a discrete range, they are limited by 
their digital bit depth.  For example, a 16 bit ADC is only capable of 
representing $2^{16}$ discrete values, or a range from zero to 65536.  
This range is often centered around zero, giving the ADC the capability 
to handle a range of $\pm32768$.  An incoming analog signal is mapped 
onto this range and converted into a digital signal.

For example, in sampling an analog signal with a range of $\pm20V$, 
10V would be mapped to 32768 digital counts and -10V would be mapped 
to -32768 digital counts with all 
of the intermediate voltage values being linearly mapped to the range. This 
means our digital system would recognize a discrete step size of 
10V/32768 counts $\approx 305 \mu $V/count.

Looking at the system described above, we must be aware of how our system 
is going to react when our analog input signal exceeds the intended maximum 
value of 10V (e.g., an 11V input). The ADC has already assigned its maximum 
digital value to 10V. This is called a digital overflow. In this case the ADC 
will continuously output its maximum value as it has no way to map 11V into 
a discrete value. The same process can occur in a DAC when a digital signal 
is sent out at the maximum allowed digital value. The resulting analog signal 
will be railed at the maximum output value of the DAC, creating a sharp corner 
in the output signal as it flattens out. 

If the digital system is not able to correctly sample and understand an analog 
error signal, it is easy to imagine a scenario where the reponse of the digital 
system and the output control signal are not able to complete the control loop 
as designed. This may cause glitches or misalignments in the interferometer.
We must also consider the fact that many ADCs are calibrated to reflect the 
intended dynamic range of an optic.  If a saturation is occurring, there is 
a good chance that an optic has moved beyond this intended dynamic range, which 
also may cause glitches or misalignments.

The ADCs and DACs are monitored by a series of auxiliary channels, which are 
automatically generated in the front-end system. These auxiliary channels 
monitor each ADC and DAC channel and note when any of the channels has reached 
its digital limit. These channels can be used to generate flags that mark 
ADC and DAC overflows, which can be compared with glitches in $h(t)$ to 
search for glitch mechanisms driven by overflows. These channels can also 
be used to flag any large glitches that cause digital overflows so that they 
can be removed from astrophysical searches. 

Figure \ref{fig:dac-overflow} shows an example of a large glitch that caused 
a digital overflow and was removed from gravitational wave analyses. Figures 
\ref{subfig:strain-dac-overflow} and \ref{subfig:esd-dac-overflow} show a 
large glitch in $h(t)$ and the response of a drive signal that controls 
the motion of ETMY respectively. The signal in \ref{subfig:esd-dac-overflow}, 
which is supposed to be controlling the motion of ETMY, hits its digital 
limit during this glitch. Figure \ref{subfig:etmy-dac-overflow} shows the 
auxiliary channel that monitors this digital overflow incrementing as 
it witnesses the digital overflow. 

\begin{figure}[ht!]%
\centering
\subfloat[]{
  \includegraphics[width=0.495\textwidth]{figures/detchar/strain-dac-overflow}
  \label{subfig:strain-dac-overflow}
  }
\subfloat[]{
  \includegraphics[width=0.495\textwidth]{figures/detchar/esd-dac-overflow}
  \label{subfig:esd-dac-overflow}
  }

\subfloat[]{
  \includegraphics[width=0.495\textwidth]{figures/detchar/etmy-dac-overflow}
  \label{subfig:etmy-dac-overflow}
  }
\caption[ETMY saturation]{Timeseries of an ETMY drive signal saturation in the 
         H1 detector. Figure \ref{subfig:strain-dac-overflow} shows a glitch in 
         the calibrated $h(t)$ channel. Figure \ref{subfig:esd-dac-overflow} shows 
         the response to this glitch in the drive signal used to control the bottom 
         stage of ETMY and actuate on the DARM degree of freedom. This signal hits 
         its digital overflow point at its peak and has no more dynamic range. 
         Figure \ref{subfig:etmy-dac-overflow} shows the front end channel responsible 
         for monitoring digital overflows of this particular ETMY drive signal. 
         Since the witness channel is cumulative, overflows can be identified by 
         flagging any time in which this witness channel is increasing. }
\label{fig:dac-overflow}
\end{figure}

This method was used throughout O1 to generate data quality vetoes that 
were distributed to the Burst and CBC searches. The first veto that was 
generated this way was used to flag DAC overflows of the ETMY drive signal, 
as demonstrated 
in Figure \ref{fig:dac-overflow}. The other veto generated in this framework 
was used to flag ADC overflows in the OMC DC photodiode used as the error 
point of the DARM control loop. 

\subsection{Suspension DAC calibration glitches}

A common glitch mechanism throughout ER6 was due to calibration errors in 
digital-to-analog converters (DACs) responsible for providing analog signals 
to the aLIGO suspensions. The aLIGO suspension subsystem uses 18-bit DACs 
to interact with the optics in the interferometer. These 18-bit DACs are 
created by combining a 16-bit DAC with a 2-bit DAC inside of the same 
electronics box. The 2-bit DAC is responsible for the two highest order 
bits of the output, while the 16-bit DAC is responsible for the 16 lowest 
order bits of the output. If the 16-bit DAC and 2-bit DAC have not had 
their output voltages carefully calibrated, there will be a voltage discontinuity 
at the output of the DAC when engaging the 2 highest order bits. 

Since these DACs use the two's complement 
representation for signed binary numbers, there are two critical points 
where the two highest order bits of the DAC become necessary. The highest 
order bit is used to indicate negative numbers, so an output discontinuity 
is expected when transitioning from a positive number to a negative number, 
that is, crossing through a value of zero.  
The other bit from the 2-bit DAC is used to represent large output values and 
engages when the DAC needs to express a value which is unable to be 
represented by a 16-bit DAC alone. As such, we also 
expect to see discontinuities when the DAC output crosses $\pm2^{16}$. 

The fact that this discontinuity existed in suspension subsystem was 
particularly problematic, as the suspension DACs are used to directly 
actuate on mirror positions and optical cavity lengths. Any time a 
suspension DAC crossed one of these problematic output values, it would 
actuate on the optics with a step function and cause a glitch in the 
optical cavity length. Figure \ref{fig:DAC-glitch} shows an example of 
this issue where the DAC providing actuation signals to the power recycling 
mirror (PRM) is crossing through zero and there are associated glitches 
visible in the length readout of the power recycling cavity.

\begin{figure}[ht!]%
\includegraphics[width=\textwidth]{figures/detchar/PRCL-DAC-glitch}
\caption[DAC glitches in PRCL]{A timeseries plot showing the effects of %
         DAC calibration glitches. The red trace shows the digital drive %
         signal being sent to the digital-to-analog converter. The blue %
         shows the resulting power recycling cavity motion rescaled by a %
         factor of 100. When the drive signal crosses %
         through a value of zero, the output of the DAC experiences a %
         discontinuity, leading to a glitch in the power recycling cavity %
         length.}
\label{fig:DAC-glitch}
\end{figure}

The effects of this issue were visible in the $h(t)$ channel during ER6. 
The most problematic culprit was the DAC that applied actuation directly 
to the optics of the ETMs, effectively pushing directly on the DARM degree 
of freedom and causing glitches in $h(t)$. These calibration errors manifested 
themselves as a population of glitches in $h(t)$ recovered by Omicron in the 
20-100 Hz range. This is a very damaging frequency range for CBC searches, 
which hope to accumulate significant SNR in the region from 30-500 Hz.  
This population of low frequency glitches was obvious in an Omicron 
time-frequency scatter plot and was considered a significant noise source 
throughout the sixth engineering run.

Figure \ref{fig:vetoed-DAC} shows the result of an Hveto run that looked 
for time correlations between Omicron triggers in $h(t)$ and times when 
the ETMY drive signal crossed through a value of $2^{16}$. The blue dots 
represent all Omicron triggers in $h(t)$. The red crosses indicate those 
that were coincident with the ETMY drive signal crossing $2^16$. The 
population of low frequency glitches with SNR $>$ 8 was shown to be 
coincident with the drive signal transitions. This veto 
was very statistically significant, as shown in Table \ref{table:etmy-dac-hveto}. 
The significance 
of 192.5 indicates that the probability of these coincidences being due 
to noise alone is negligible. The effiency:deadtime ratio of 27 indicates 
that these glitches were removed with very small time windows (0.2s) 
and very little instrument uptime was removed in the process.

\begin{figure}[ht!]%
\includegraphics[width=\textwidth]{figures/detchar/vetoed-DAC-glitches}
\caption[Vetoed DARM triggers from DAC calibration]{A time-frequency %
         visualization of Omicron triggers in the H1 $h(t)$ channel. % 
         The black circles indicate glitches in the DARM degree of freedom, %
         each with a central time and central frequency. The red crosses %
         indicate that a given trigger was vetoed by an auxiliary channel %
         trigger which was found to be statistically significant using Hveto. The %
         auxiliary channel triggers in this case indicate that the drive signal %
         on the bottom stage of ETMY has crossed a value of $2^{16}$. The %
         population of glitches between 20 - 100 Hz is highly coincident %
         with these crossings of $2^{16}$, indicating that they are caused %
         by DAC calibration errors on this optic.}
\label{fig:vetoed-DAC}
\end{figure}

\begin{table}[ht!]%
 \footnotesize 
 \begin{center}
  \begin{tabular}{cccccc}
  \hline
  Channel & \begin{tabular}{@{}c@{}} Time \\ window (s) \end{tabular} & 
            \begin{tabular}{@{}c@{}}SNR \\threshold \end{tabular} & 
            Significance & Efficiency \% & Deadtime \% \\ 
  \hline
  \begin{tabular}{@{}c@{}}ETMY drive signal \\ crosses $2^{16}$ \end{tabular} & 
  0.2 & 8 & 192.5 & 18.3 &  0.674 \\
  \hline
  \end{tabular}
  \end{center}
  \caption[HVeto results for ETMY DAC glitches]{Hveto results for ETMY DAC glitches}
  \label{table:etmy-dac-hveto}
\end{table}

To fully understand the scope of this problem, the Detector Characterization 
group developed software that searched through the output of all suspension 
DAC digital output signals and marked times when they crossed 0 or $\pm2^{16}$. 
These marked times were converted into trigger files and sent through Hveto 
to look for correlations between crossings of critical values and glitches 
in DARM as identified by Omicron. Through this method, we were able to identify 
which optics were experiencing DAC calibration glitches that had a coupling 
mechanism into DARM.

There were two approaches taken in an effort to mitigate these DAC glitches. The 
first was to introduce offsets into the suspension drive signals so that they 
did not cross through a value of zero. This did solve the problem temporarily, 
but at the cost of a significant portion of the dynamic range of the output 
actuation. The more permanent fix was to run a calibration routine 
that resolved the issue between the 16-bit and 2-bit DACs. This was 
successful, though it had to be run on a weekly basis during site maintenance 
because the calibration tended to drift away from its nominal point after 
2-3 weeks of operation.

During the first observing run, the systematic check of all suspension DAC 
digital output signals was performed again and the resulting triggers were 
sent through Hveto. This study revealed that the calibration process was 
successful; there was no evidence of residual DAC calibration glitches that 
had any noticeable coupling into $h(t)$. The only signal that had any 
significant correlation with glitches in $h(t)$ was not causally sensible. 
Large glitches $h(t)$ were driving the ETMX actuation signal through a value of 
$2^{16}$, which resulted in crossings of $2^{16}$ that were coincident in time 
with glitches in $h(t)$, but weren't representative of calibration errors.

\textcolor{red}{Discuss Hveto results}

\subsection{RF beatnote whistles}

During Advanced LIGO's commissioning,
a population of glitches appeared in both the L1 and H1 interferometers which 
came to be known as 'whistles' or 'RF whistles'. These glitches were the 
intermodulation products of two RF oscillators; a nonlinear mixing between two 
RF oscillators produced a beatnote signal whose frequency was equal to the 
difference in frequency between the two oscillator signals. Whistle glitches 
occur when two oscillator signals drift and cross each other in frequency. 
If one oscillator 
is drifting in frequency and another oscillator is at a fixed frequency, 
the beatnote generated between them will decrease in frequency as the 
oscillator signals become closer in frequency and then increase in frequency 
as they cross each other and drift away. 
As such, these glitches 
had a characteristic shape, beginning at high frequency and sweeping down 
in frequency through the detection band before turning around and 
sweeping back up to high frequencies. Figure \ref{fig:whistle-spectrograms} 
shows a time-frequency representation of whistle glitches in both the L1 and H1 
interferometers ?? . These show the characteristic 'V' or 'W' shape produced 
when two oscillators drift past one another and have nonlinear mixing.

\begin{figure}[ht!]%
\centering
\subfloat[]{
  \includegraphics[width=\textwidth]{figures/detchar/Spectrogram_Whistle_LLO}
  \label{subfig:llo-whistle}
  }
  
\subfloat[]{
  \includegraphics[width=\textwidth]{figures/detchar/Spectrogram_Whistle_LHO}
  \label{subfig:lho-whistle}
  }
\caption[Spectrograms of RF whistles]{Time-frequency spectrograms of RF whistles at %
         both LLO and LHO. Figure \ref{subfig:llo-whistle} shows a %
         whistle at LLO sweeping down from the kHz range and into the detection band %
         where it interferes with searches for gravitational waves. Figure %
         \ref{subfig:lho-whistle} shows a double whistle whistle at LHO where the %
         two oscillators drifted back and forth across one another and caused two %
         glitches in the detection band.}
\label{fig:whistle-spectrograms}
\end{figure}

Voltage controlled oscillators (VCOs) are oscillators whose frequency can 
be tuned using an input voltage. These oscillators are used in control 
loops throughout the aLIGO interferometers. One particular example of this 
is the control loop which locks the frequency of the input laser to the 
length of the input mode cleaner to guarantee a resonant optical cavity and 
effective mode cleaning. A signal representing the changing length 
of the input mode cleaner is read out using the Pound-Drever-Hall (PDH) technique ?? . 
This signal is used as the input to a VCO, which produces a 
signal whose frequency is a proxy for the length of the input mode cleaner. 
This signal is used as the set point in the frequency stabilization loop that 
controls the frequency of the input laser light. Through this path, the length 
of the input mode cleaner is used to set the frequency of the input laser light. 

The signal that represents the length of the input mode cleaner was found to 
be a good witness for RF whistle glitches. Figure \ref{fig:darm-whistle-hist} 
shows the rate of Omicron triggers, which represent generic transient noise in 
$h(t)$, as a function of the length of the input mode cleaner. The length of 
the input mode cleaner is in kHz as it is an error signal used to set the 
frequency detuning of the VCO. The red curve represents the distribution in 
the absence of whistle glitches. There is no value for which noise transients in 
$h(t)$ seem more likely to occur; the rate of glitches seems Gaussian distributed 
as the length of the input mode cleaner fluctuates about the set point of the 
control loop. The blue curve represents the rate of transients 
when RF whistle glitches are occurring. In this case, there are three preferred 
frequencies where it seems that noise transients in $h(t)$ have a tendency to 
occur. This indicates that there is a relationship between specific values of 
the length of the input mode cleaner and the presence of whistle glitches in the 
interferometer. 

\begin{figure}[ht!]%
\includegraphics[width=\textwidth]{figures/detchar/Rate_Histogram_Whistles_LLO}
\caption[DARM glitch histograms with and without RF whistles]{The red distribution %
         shows the rate of Omicron in triggers in $h(t)$ when RF whistles are not % 
         present. The blue distribution shows the rate of Omicron triggers in $h(t)$ % 
         when RF whistles are occurring. The x-axis is the value of a channel that %
         represents the length of the input mode cleaner. When there are no whistle %
         glitches, there is no channel value for which Omicron triggers are more %
         likely to occur. When there are whistle glitches in $h(t)$, specific %
         values of the input mode cleaner length seem more likely to be coincident %
         with Omicron triggers in $h(t)$.}
\label{fig:darm-whistle-hist}
\end{figure}

The VCO that acts as a proxy to the length of the input mode cleaner is 
nominally set to 80 MHz. As the length of the input mode cleaner drifts, the 
oscillator frequency can be tuned by $\pm$ 1 MHz to track the length, resulting 
in a signal with a frequency of 79 - 81 MHz. It was found that these whistle 
glitches occurred when the VCO frequency swept through 79.2 MHz, which is the 
same frequency as an oscillator used to drive an acousto-optic modulator. 
The variable oscillator that was tracking the length of the input mode cleaner 
was drifting past the static oscillator at 79.2 MHz and creating whistle glitches 
that were visible in $h(t)$.  
To reduce the number of whistle glitches in $h(t)$, the oscillator frequencies 
were moved away from one another so that the static oscillator was outside 
of the range of the tunable oscillator.

Figure \ref{fig:hveto-whistles} demonstrates how prevalent whistle glitches were 
before the oscillator frequencies were shifted to avoid them. Figure 
\ref{fig:hveto-whistles} is a time-frequency scatter plot of Omicron triggers in 
$h(t)$. The blue dots represent all Omicron triggers generated for $h(t)$ over 
this stretch of time. The red crosses indicate that a given Omicron trigger was found 
to be coincident with an Omicron trigger generated for an auxiliary channel that 
was a capable witness for whistle glitches. Approximately 90\% of the glitches in 
$h(t)$ are vetoed by the witness channel, indicating that whistles were the dominant 
source of transient noise in $h(t)$ in this time period. 


\begin{figure}[ht!]%
\includegraphics[width=\textwidth]{figures/detchar/Hveto_whistles_time_frequency}
\caption[Vetoed whistles from Hveto]{A time-frequency scatter plot of Omicron %
         triggers. The blue dots represent all triggers found for the $h(t)$ channel. %
         Red crosses indicate that a trigger was determined to be coincident with an %
         RF whistle and vetoed. This veto is responsible for removing 90\% of the %
         glitches in this time period. The majority of the high frequency glitches %
         were due to RF beatnote whistles}
\label{fig:hveto-whistles}
\end{figure}

The shape of the whistle 
glitches was also very problematic for CBC searches since the second half of a 
whistle is a sinusoid with a monotonically increasing frequency, not unlike the 
characteristic 'chirp' signal produced by a CBC event. Certain CBC waveforms 
matched the shape of the whistle glitches well enough to fool the $\chi^2$ signal 
consistency test and produced loud background triggers in early CBC searches. 
The whistle glitches 
were fixed before the first observing run, so they were not a limiting noise source 
to CBC searches during observation.

%\subsection{Seismic CPS comb}
%
%Oscillators in the capacitive position sensors had drifted apart and caused a 
%beatnote and a comb. Audio analysis pointed towards amplitude modulation. 
%
%Fixed by slaving all oscillators to a master.
%
%\subsection{DC values of auxiliary channels}
%
%No great correlation at the end of the day 
%
%\subsection{Earthquakes during full lock}
%
%Lots of scattering arches during an earthquake, drove up the noise and biased PSD.
%Caused a sarlacc, removing this data was able to repair data on either side.
%
%\subsection{L1 PMC glitches}
%
%
%Characterization of noise and analysis after repair
%
%\subsection{Data quality shifts}
%Performed and mentored data quality shifts.

\section{Validation of Gravitational Wave Signals}\label{sec:GW150914-validation}

The Detector Characterization group was responsible for characterizing the 
noise in the interferometers in order to validate the gravitational 
wave signals GW150914 and GW151226. The first part of this analysis, which studied the 
stationarity of the background noise, is discussed in Section ?? . As a further 
check, the transient noise in the interferometer was also studied so that a 
confident detection claim could be made regarding GW150914. 

In the engineering runs leading up to the first observing run, a great deal of 
work was done to understand as much as possible about the noise coupling mechanisms 
from auxiliary channels into $h(t)$. Among the most important of these was a set 
of signal injections to test the sensitivity of the physical and environmental 
monitoring (PEM) subsystem. The PEM subsystem is comprised of a series of sensors 
that measure the ambient environmental noise at the interferometers. This 
subsystem is comprised of seismometers, accelerometers, magnetometers, radio 
antennae, microphones, temperature sensors, and voltage monitors for the power 
lines supplying the building. While the aLIGO detectors are extremely sensitive 
to external perturbations, they were built to be shielded against as many 
environmental disturbances as possible. In contrast, the PEM subsystem is comprised 
of extremely sensitive sensors that are more sensitive to environmental disturbances 
than the interferometer. By injecting signals into the interferometer enclosure, 
such as magnetic fields or acoustic vibrations, the relative sensitivity of the 
interferometer and the PEM sensors to environmental disturbances was established.

For both GW150914 and GW151226, a review the PEM subsystem reported that any environmental 
disturbances were at least 1 order of magnitude too weak to produce such an event. 
This includes electromagnetic transients, such as lightning strikes, that have the 
potential to generate coincident electromagnetic transients at L1 and H1. 

In addition to the checks performed in the PEM subsystem, a series of standard 
checks were done to ensure nominal performance in the interferometers. These 
checks are organized in a detection checklist, which gathers all of the 
relevant questions about interferometer performance that may influence 
gravitational wave detection. This list includes checks 
for the DAC calibration glitches mentioned in Section ??, the ADC and DAC 
digital saturation glitches mentioned in Section ??, coincidence with 
generic transients 
as reported by Omicron, time-frequency scans of all auxiliary channels to be 
investigated by DetChar subsystem leads, injections and test signals, and 
GPS or digital system timing errors. Each category was investigated and 
followed up on by the DetChar group and none of them gave significant 
reason to doubt the validity of GW150914 or GW151226.  

For both GW150914 and GW151226 there were small sets of auxiliary channels 
that showed excess power coincident with the 
gravitational wave signals, which is expected given the breadth of the auxiliary 
channel network, but after further investigation none of them had the 
necessary amplitude and frequency to generate an event similar to a CBC 
signal. An example of one such auxiliary channel is shown in Figure \ref{fig:compressor}, 
which 
is a time-frequency spectrogram of the H1 Y-end seismometer signal that showed 
excess noise at the time of GW150914. The excess 
noise is due to an air compressor turning on roughly 75 seconds before GW150914. 
The noise is a 14 Hz line with 28, 42, and 56 Hz harmonics visible. This level of 
ground motion with a 175 second duration and a static frequency distribution was 
not capable of producing a 0.2s chirp signal with the amplitude of GW150914 in 
the interferometer output.

\begin{figure}
\includegraphics[width=\textwidth]{figures/detchar/H1_AIR_COMPRESSOR_GW150914}
\caption[H1 Y-end air compressor]{A time-frequency spectrogram of the H1 %
         Y-end seismometer signal near the time of GW150914. An air compressor %
         turns on at -75 seconds and off at +100 seconds, creating ground motion.}
\label{fig:compressor}
\end{figure}


\Chapter{IMC Upconversion}
\label{ch:IMCUpconversion}
LIGO interferometers use several high finesse optical cavities for gravitational wave 
detection. The lengths of these cavities are controlled using radio frequency 
(RF) modulation-demodulation techniques in a Pound-Drever-Hall (PDH) locking scheme 
\cite{Black01}.  
This scheme provides an error signal that is linear to cavity length over a 
specific range. This study examines the specific case of the triangular ring cavity 
uses in LIGO interferometers for input mode cleaning. When the length of the cavity 
approaches the boundaries of the PDH error signal linear range, our model of the 
input mode cleaner PDH response shows that the resulting error signal contains 
non-linear spectral artifacts. This model and understanding of the non-linear 
cavity responses will be useful in the commissioning phase of the Advanced LIGO 
project for more precisely locating and eliminating systematic noise sources in 
the interfereometers

\section{PDH locking}

Resonance in an optical cavity is achieved when the round-trip length of the 
cavity is equal to an integer number of wavelengths of the input beam,
\begin{equation}
L = N\lambda = \frac{Nc}{\nu}
\end{equation}
where L is the round-trip length of the optical cavity, $\lambda$ is the 
wavelength of the light, $\nu$ is the frequency of the light, and c is the 
speed of light. 
Under these conditions, the light circulating in the cavity
will be in phase and add constructively, resulting in an optical gain that
increases the intracavity power. This is the state in which the LIGO 
optical cavities are intended to operate.
If we invert this equation, the allowed 
frequencies of light for which resonance will occur is then 
\begin{equation}
\nu = N\frac{c}{L}.
\end{equation}
The spacing between these allowed frequencies is called the free spectral 
range, 
\begin{equation}
\nu_\mathrm{FSR} = \frac{c}{L}.
\end{equation}
When the frequency of the light is equal to an integer multiple of 
the free spectral range, the system will be on resonance. This is, 
however, a delicate condition to maintain. If the frequency of the light 
changes while the cavity length is stable, the optical field will no longer 
overlap perfectly within the cavity 
and the incident light will be reflected. If the length of the optical cavity 
changes but the frequency is stable, the geometry of the cavity and the optical 
field will once again be mismatched and resonance will be lost. 
This can be thought of as the free spectral range of the cavity varying with L 
while the carrier beam frequency is constant. 
To solve this problem, LIGO 
employs feedback loops that use a PDH error signal to maintain the resonance 
condition. We will use the LIGO input mode cleaner as an example of PDH locking.

The Advanced LIGO input mode cleaner is a resonant triangular ring cavity used to isolate 
the TEM00 mode of the input beam. The geometry of the cavity is designed such that 
higher order modes of the optical field will be reflected and not transmitted to 
the rest of the interferometer. The carrier beam, however, will be resonant in 
the input mode cleaner and will be transmitted. To control the input mode cleaner, 
the reflected 
light incident on the input mode cleaner is read out on a photodiode. The reflected 
part of the carrier beam, which on resonance should be highly transmitted, 
is compared to the reflected part of an RF sideband which should be highly 
reflected.

The first necessary piece of information to generate the PDH error signal is 
the reflectivity of the optical cavity as a function of frequency. This function 
will have minima at integer multiples of the free spectral range, where the 
cavity is on resonance and light is circulating in the cavity. As the frequency 
of the light drifts, the reflectivity of the cavity will increase, rejecting more 
of the incident light. For the IMC, 
the reflection function is 
\begin{equation}
F(\omega) = \frac{r(1 + e^{-i\phi})}{1+r^2e^{-i\phi}} = \frac{r(1 + e^{-i(\frac{\omega}{\nu_{fsr}})})}{1+r^2e^{-i(\frac{\omega}{\nu_{fsr}})}}
\end{equation}
where $\omega$ is the frequency of the light, $r$ is the reflection coefficient 
of the input mirror, $\phi$ is the round-trip phase 
accumulated as the light propagates through the cavity, 
and $\nu_{\mathrm{FSR}}$ is the free spectral range of the cavity \cite{Mueller}.
This function returns a complex value, the amplitude and phase represent the 
amplitude and phase of the reflected optical field relative to the 
optical field incident to the cavity.

Figure \ref{fig:imc-reflection} shows the reflected amplitude and phase of 
the carrier beam and the 24 MHz RF sideband relative to the incident optical 
fields. For this demonstration, 
we will assume that the frequency of the light is stabilized and the 
x-axis represents a displacement of the optical cavity length from the resonance length. 
When the cavity length matches the input beam, the reflectivity is minimized and the 
carrier is transmitted through the IMC. The sideband, which is at a higher frequency, 
is by design not resonant in the cavity and is fully reflected.
As the cavity length deviates from the resonance length, the reflected amplitude and 
phase of the carrier 
beam will change along with it. However, the reflected amplitude and phase of the 
RF sideband, which is not resonant in the IMC, are not sensitive to changes in 
cavity length around the resonance point.

\begin{figure}[h!]
\centering
  \subfloat{
  \includegraphics[width=0.9\textwidth]{figures/IMCUpconversion/reflectivity}
  \label{fig:reflectivity}
  }

  \subfloat{
  \includegraphics[width=0.9\textwidth]{figures/IMCUpconversion/reflected-phase}
  \label{fig:reflected-phase}
  }
\caption[Reflection at the IMC]{Amplitude and phase of light reflected from %
         the IMC relative to the incident optical field. The zero point of the x-axis %
         represents the resonance point of the cavity. The amplitude reflectivity %
         is at a minimum when the cavity is on resonance, allowing the carrier beam %
         to be transmitted into the interferometer. The amplitude and phase of the %
         carrier beam will change as the cavity length changes. The amplitude %
         and phase of the RF sideband are not sensitive to changes in the cavity %
         length. This information can be used to generate an error signal that %
         represents the length of the input mode cleaner.}
\label{fig:imc-reflection}
\end{figure}

Using the function for the complex reflection coefficient, the reflected light 
can be read out on a photodiode and used to generate an 
error signal that is linear to the length of the cavity within a certain range. 
In a situation where the carrier beam is resonant in the cavity and the RF 
sidebands are high enough in frequency that they are not resonant, the PDH error 
signal is
\begin{equation}
\epsilon(\omega) = -2\sqrt{P_{c}P_{s}}\operatorname{Im}\{F(\omega)F^*(\omega + \Omega) - F^*(\omega)F(\omega - \Omega)\},
\end{equation}
where $P_{c}$ is the the carrier beam power and $P_{s}$ is the sideband power 
\cite{Black01}.
Figure \ref{fig:pdh} shows the resulting PDH error signal as a function of 
the free spectral range with an overlaid straight line as a reference for 
linearity. Looking at a zoomed in view of the error signal around the the linear 
part, we can see that the PDH signal matches the linear reference very well 
up to $\pm.5$ nm, or $\sim\lambda$/1000, of cavity displacement. 

\begin{figure}[h!]
\centering
  \subfloat{
  \includegraphics[width=0.9\textwidth]{figures/IMCUpconversion/linear-pdh}
  \label{fig:regular-pdh}
  }

  \subfloat{
  \includegraphics[width=0.9\textwidth]{figures/IMCUpconversion/zoomed-pdh}
  \label{fig:zoom-pdh}
  }
\caption[Example of a PDH error signal]{Example of a PDH error signal. %
         The x-axis in this plot is linearly related to the length of the %
         input mode cleaner. 
         The red line is a straight line reference to estimate the linearity %
         of the error signal. %
         The error signal is linear to the length of the %
         input mode cleaner up to $\pm.5$ nm of cavity displacement, or %
         $\sim\lambda$/1000. Motion %
         beyond this point will begin to contain non-linear artifacts and %
         eventually reach a turning point where control of the optics is lost.}
\label{fig:pdh}
\end{figure}

If the cavity motion exceeds this linear range, the error signal will 
contain non-linear artifacts which will bleed into the control signal 
used to actuate on the cavity optics.
To explore this non-linearity, we injected a sinusoidal cavity motion into our 
model and observed the resulting error signal.
The frequency of the sine wave was selected in an attempt to model 
noise seen in the output of the interferometers.  
The PDH response of the cavity was modeled using measured values of optical 
reflectivity and free spectral range of the Livingston input mode cleaner. 
The input beam was the nominal LIGO carrier beam with a frequency of 
$\omega = 281.8$ THz ($\lambda = 1064$ nm) and modulation sidebands of 
$\Omega = \pm24$ MHz.

We explored two specific cases. Figure \ref{fig:asymmetric-pdh} shows the 
power spectral density of the injected sinusoidal 
cavity motion (green) and the resulting non-linear error signal (blue). 
This motion was injected asymetrically about the nominal cavity locking point 
($\epsilon = 0$). The effect of this non-linearity is to take the injected 
sine wave and produce an error signal that looks like a sine wave with a 
flattened top, resembling a mixture of a pure sine wave with a square wave. 
Thus, we see both even and odd harmonics of the injection frequency when the 
signal is observed in the frequency domain.

Figure \ref{fig:symmetric-pdh} shows the power spectral density of the injected 
sinusoidal cavity motion (green) and the 
resulting non-linear error signal (blue). However, this time the motion was 
injected symetrically about the nominal cavity locking point. The resulting 
error signal was similar to a square wave and as a result 
we only see odd harmonics of the fundamental frequency.

\begin{figure}[h!]
\includegraphics[height=0.6\textwidth]{figures/IMCUpconversion/PDH_error_signal_harmonics.png}
\caption[PDH response to asymmetric cavity motion]{Sinusoidal cavity motion with frequency 2.78 Hz injected asymmetrically about the locking point of the cavity results in a PDH error signal containing non-linear spectral artifacts at harmonics of the injected cavity motion.}
\label{fig:asymmetric-pdh}
\end{figure}

\begin{figure}[h!]
\includegraphics[height=0.6\textwidth]{figures/IMCUpconversion/symmetric_PDH.png}
\caption[PDH response to symmetric cavity motion]{If the motion is symmetric about the cavity locking point, we see only odd harmonics of the injection frequency.}
\label{fig:symmetric-pdh}
\end{figure}

\section{Upconversion noise in aLIGO}
Each of the three mirrors in the input mode cleaner cavity is staged as the bottom 
mass of a triple suspension in order to passively isolate the mirrors from noise. 
In addition, the chambers holding the IMC mirrors are isolated from ground motion by 
two stages of active seismic isolation. This isolation, however, is not completely 
impervious to external excitations. During periods of time with excess ground motion 
we can see seismic noise coupling into the cavity length and its control signal.

Specifically, when we see excess seismic noise in the 1-5 Hz anthropogenic band, 
believed to be caused by a commercial railroad a few kilometers from the LIGO 
Livingston, we see highly structured noise in the IMC control signal in the 10-100 Hz 
band. This physical mechanism is consistent with the model of a non-linear PDH error 
signal. If excess seismic motion reaches the suspension and the optics begin swinging 
around, it's feasible that they could start to saturate the linear range of the PDH loop.

The noise takes a form very similar in structure to the non-linear PDH signal, displaying 
strong odd harmonics and weaker even harmonics. The IMC control signal has an associated 
noise floor that obscures parts of these peaks. The theoretical model uses sinusoids with 
a highly specified frequency and thus displays very sharp peaks in its spectrum. 
It should be noted that the peaks in the IMC control signal are the manifestation of 
a physical process, not digitally generated, and have some natural width to them.

\begin{figure}[h!]
\includegraphics[height=0.6\textwidth]{figures/IMCUpconversion/upconversion_comb.png}
\caption[Spectral comb in IMC control signal]{Spectral comb with a fundamental frequncy of 2.78 Hz in the IMC control signal. Red arrows indicate odd harmonics, green arrows indicate even harmonics. }
\end{figure}

While we have demonstrated that this mechanism is consistent with IMC upconversion noise, 
it has not yet been fully proven. We are currently looking for a better way to look at 
the IMC error point, which is generated using an analog servo board, during times of 
excess seismic motion instead of the control signal. 
We think the source of the excitation may be a vertical resonance of the triple 
pendulum suspension that houses the IMC optics being rung up by the excess motion.

\section{Conclusions}

We found that injecting sinusoidal cavity motion into our input mode cleaner PDH model 
generates an error signal with non-linear spectral artifacts, specifically harmonics 
of the injection frequency, if the cavity motion exceeds the linear PDH range. 
For cavity motion that is symmetric about the locking point of the error signal, 
we find that the error signal contains only odd harmonics. For asymmetric cavity 
motion we find both even and odd harmonics, where the odd harmonics are typically higher 
in amplitude. In such a case, the amplitude of the even harmonics increases as the 
offset from the nominal locking point increases, that is, as the cavity motion is 
more asymmetric.



\Chapter{Detector Characterization Subsystem Lead}
\label{ch:ODC}
List of a whole bunch of ODC stuff. Descriptions and pictures of the LSC and ASC ODC and MEDM screens.


\Chapter{Data Quality Vetoes}
\label{ch:Vetoes}
Describe the data quality vetoes in general

\section{Veto categories}

\section{Generating data quality flags from instrument channels}

\section{Veto definer file}

\section{Important vetoes in O1}

\section{Comparing detector data to Gaussian noise}

Now that all vetoes have been described, how close are we to Gaussian noise?

\section{Quantifying the effects of data quality vetoes}


\Chapter{Effects of Data Quality Vetoes on the Analysis Containing GW150914}
\label{ch:GW150914-DQ}
Effects of DQ on analysis containing GW150914

\section{BNS bin}

\section{Bulk bin}

\subsection{LVT151012}

\section{Edge bin}

\subsection{GW150914}




\Chapter{Effects of Data Quality Vetoes on the Analysis Containing GW151226}
\label{ch:GW151226-DQ}
Effects of DQ on analysis containing GW151226

\section{BNS bin}

\section{Bulk bin}

\subsection{GW151226}

\section{Edge bin}





\Chapter{Limiting Noise Sources in the PyCBC Search}
\label{ch:CBCDetCharLimits}
What are the current limiting noise sources?

\section{Loud transients}

\subsection{Do loud instrumental transients contribute to the newSNR tail?}

\section{Blip glitches}

\subsection{Time-frequency morphology}

\subsection{Time-domain picture with CBC waveforms}

\subsection{What areas of the CBC parameter space are impacted by blips?}

\section{60-200 Hz noise}

\subsection{Time-frequency morphology}

\subsection{What areas of the CBC parameter space are impacted?}




\Chapter{Conclusion}
\label{ch:Conclusion}
Write conclusion if you think you have anything more to tell


\appendix
\Chapter{Data Quality Vetoes in O1}
\label{ap:o1vetoes}
Paste in the TeX from DCC document T1600011


\clearpage
\bibliographystyle{unsrt}
\bibliography{references,cbc-group,updated_refs,GW150914_refs}

\addcontentsline{toc}{chapter}{\numberline {Bibliography}}

\clearpage
\birthplacedate{Rochester, NY \>\>July 22, 1989}
\collegewherewhen{%
\>Utica College \>\>2007--2011, \>B.S.\\
\>\su	\>\>2011--2016, \>Ph.D.}

\newpage
\null\vskip1in%
\begin{center}
{\Large\bf Curriculum Vitae}
\end{center}
\vskip 2em
\begin{tabbing}
\tabset
Title of Dissertation\\
\>Detector Characterization for Advanced LIGO
\end{tabbing}
\vskip 1em

\begin{startvita}
\end{startvita}

\renewenvironment{thebibliography}[1]%
  {\begin{list}{\labelenumi\hss}%
     {\usecounter{enumi}\setlength{\labelwidth}{3em}%
      \setlength{\leftmargin}{5em}}}%
  {\end{list}}
\renewcommand{\bibitem}[1]{\item\label{#1}\relax}%
\renewcommand{\theenumi}{\arabic{enumi}}%
%\begin{publications}
%\putbib[papers]
%\end{publications}


\finishvita
\end{document}
