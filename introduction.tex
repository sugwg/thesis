% $Id$

In chapter \ref{c:findchirp} we describe in detail the algorithms used to
inspiral signals from binary neutron starts and binary black hole MACHOs in
the LIGO data.  findchirp. We first carefully define the conventions that we
use for analysis quantities in section \ref{s:conventions}; in particular the
definition of the Fourier transform and the power spectral density. Section
\ref{s:waveforms} gives a brief description of the the waveforms used and
section \ref{s:matchedfilter} describes the implementation of the matched
filter.  Spurious noise may cause the output of the matched filter to be large
and so in section \ref{s:chisq} we describe our implementation of the $\chi^2$
time--frequency discriminator proposed in~\cite{allen}. Section
\ref{s:practical} contains additional details of the search particular to our
implementation: the computation of the inverse power spectrum and the trigger
selection algorithm. This is followed by a brief conclusion which summarized
the methods used and outlines some future directions for improvement.

\section{Fourier Transform Conventions}
\label{s:ftconv}

There are two possible sign conventions for the Fourier transform of a time
domain quantity $v(t)$. In this thesis, we define the Fourier transform
$\tilde{v}(f)$ of a $v(t)$ to be
\begin{equation}
\label{eq:ft}
\tilde{v}(f)=\int_{-\infty}^\infty dt\,v(t)\, e^{- 2 \pi i f t}
\end{equation}
and the inverse Fourier transform to be 
\begin{equation}
\label{eq:ift}
v(t)=\int_{-\infty}^\infty df\,\tilde{v}(f)\, e^{2 \pi i f t}.
\end{equation}
This convention differs from that used in some gravitational wave literature,
but is the adopted convention in the LIGO Scientific Collaboration.
