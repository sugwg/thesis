% $Id$

In chapter \ref{c:findchirp} we describe in detail the algorithms used to
inspiral signals from binary neutron starts and binary black hole MACHOs in
the LIGO data.  findchirp. We first carefully define the conventions that we
use for analysis quantities in section \ref{s:conventions}; in particular the
definition of the Fourier transform and the power spectral density. Section
\ref{s:waveforms} gives a brief description of the the waveforms used and
section \ref{s:matchedfilter} describes the implementation of the matched
filter.  Spurious noise may cause the output of the matched filter to be large
and so in section \ref{s:chisq} we describe our implementation of the $\chi^2$
time--frequency discriminator proposed in~\cite{allen}. Section
\ref{s:practical} contains additional details of the search particular to our
implementation: the computation of the inverse power spectrum and the trigger
selection algorithm. This is followed by a brief conclusion which summarized
the methods used and outlines some future directions for improvement.



\section{The Effect of Gravitational Waves on Freely Falling Particles}

In this section we briefly the effect of gravitational waves on a pair of
freely falling particles in order to introduce some of the concepts that we
need to discuss the detection of gravitational waves from binary inspirals.
For a detailed decription of the propagation and effect of gravitational
waves, we refer to \cite{MTW73,Thorne:1982cv}. Consider first a single
particle moving in curved spacetime. If the particle is freely falling, its
4-velocity, $\vec{u}$, satisfues the geodessic equation
\begin{equation}
\label{eq:geodessic}
\nabla_{\vec{u}} \vec{u} = \tensor{u}{^\alpha_{;\mu}}u^{\mu} = 0
\end{equation}
where $;$ denotes the covariant derivative, that is
\begin{equation}
\tensor{u}{^\alpha_{;\mu}}\tensor{u}{^{\mu}} = \left(\tensor{u}{^\alpha_{,\mu}} +
\Gamma^\alpha_{\mu\nu}u^\nu\right)
\end{equation}
where $,$ denotes the ordinary derivative and $\Gamma\alpha_{\mu\nu}$ is the
connection coefficent of the metric, $g_{\alpha\beta}$. Now suppose that we have
two particles $A$ and $B$ as shown in figure \ref{f:particles}. The separation
between the particles is $\vec{\xi}$ and the particles are initally at rest
with respect to each other, so
\begin{align}
\nabla_{\vec{u}} \vec{u} &= 0 \\
\vec{u} \cdot \vec{\xi} & = 0 
\end{align}
If the spacetime is curved, the second derivative of $\vec{\xi}$ along
$\vec{u}$ is non-zero, however. It is given by the equation of geodessic
deviation
\begin{equation}
\nabla_{\vec{u}}\nabla_{\vec{u}} \vec{\xi} = - R(_,\vec{u},\vec{\xi},\vec{u})
\end{equation}
where $R(_,\vec{u},\vec{\xi},\vec{u}$ is the Riemann curvature tensor. We now
introduce a \emph{Local Lorentz Frame} (LLF) for particle $A$, that is a
cartesian coordinate system defined by three orthogonally pointing giroscopes
carried by particle $A$. The curvature of spacetime means that the coordinate
system is not exactly cartesian, but it can be shown that this deviation is
second order in the spatial distance from the particle. This means that along
the worldline of particle $A$ the metric is
\begin{equation}
g_{\alpha\beta} = \eta_{\alpha\beta} + \frac{\mathcal{O}\left(|\vec{x}|^2\right)}{R^2}
\end{equation}
where $\eta_{\alpha\beta}$ is the flat Minkowski metric, $\vec{x}$ is the
distance from the particle and $R \sim |R_{\alpha\beta\gamma\delta}|$ In the
Local Lorentz frame of particle $A$, the equation of geodessic deviation
becomes
\begin{equation}
\frac{\partial^2 u^{j}}{\partial t^2} = -
\tensor{R}{^j_{\alpha\beta\gamma}}u^\alpha\xi^\beta u^\gamma =
-\tensor{R}{^j_{0k0}} \xi^k
\end{equation}
We assume further that the spacetime is globally flat, so $g_{\alpha\beta} =
\eta_{\alpha\beta}$, with weak gravitational waves propagating in it. We
describe the gravitational waves by $R_{\alpha\beta\gamma\delta}$ which
satisfies the wave equation
\begin{equation}
\eta^{\mu\nu}R_{\alpha\beta\gamma\delta,\mu\nu} = 0
\end{equation}
What is the effect of these gravitational waves on our test particles $A$ and
$B$? In the Local Lorentz frame, $\vec{\xi}$ is just the coordinates of $B$.
Let the two particles lie on the $x$-axis of the LLF and write
\begin{equation}
\label{eq:bcoords}
\xi^j = x_{(0)}^j + \delta x^j
\end{equation}
where $x_{(0)}^j$ is the unperterbed location of particle $B$ and $\delta x^j$
is the change is the position of $B$ caused by the gravitational wave.
Substituting equation (\ref{eq:bcoords}) into the equation of geodessic
deviation, we obtain
\begin{equation}
\label{eq:particledev}
\frac{\partial^2 \delta x^j}{\partial t^2} = - \tensor{R}{^j_{0k0}} x_{(0)}^j =
-R_{j0k0} x_{(0)}^j
\end{equation}
where we have used $\eta_{\alpha\beta}$ to lower the spatial index $j$ of the
Riemann tensor. For a weak gravitational wave, all the components of
$R_{\alpha\beta\gamma\delta}$ are completely determined by $R_{j0k0}$.
Furthermore, it can be shown that the $3\times3$ symmetric matrix $R_{j0k0}$,
which we would expect to have $6$ independent componets, has only $2$
independent components due to the Einstein equations and the Biancci identity.

Let us define the (transverse traceless) gravitational wave field,
$h_{jk}^\mathrm{TT}$, by
\begin{equation}
-\frac{1}{2} \frac{\partial^2 h_{jk}^\mathrm{TT}}{\partial t^2}
\end{equation}
and so equation (\ref{eq:particledev}) becomes
\begin{equation}
\delta x^j = \frac{1}{2} h_{jk}^\mathrm{TT} x_{(0)}^j.
\end{equation}
If we orient our coordinates so the gravitational waves propagate in the
$z$-direction, so $h_{jk}^\mathrm{TT}(t-z)$, then the only non-zero components
of $h_{jk}^\mathrm{TT}$ are $h_{xx}^\mathrm{TT}$, $h_{yy}^\mathrm{TT}$,
$h_{xy}^\mathrm{TT}$ and $h_{yx}^\mathrm{TT}$. Since $h_{jk}^\mathrm{TT}$ is
transverse and traceless, these components satisfy
\begin{align}
h_{xx}^\mathrm{TT} &= - h_{yy}^\mathrm{TT} \\
h_{xy}^\mathrm{TT} &= h_{yx}^\mathrm{TT}
\end{align}
and we define the two independent components of the gravitational wave to be
\begin{align}
h_{+} &= h_{xx}^\mathrm{TT} = - h_{yy}^\mathrm{TT}, \\
h_{\times} &= h_{xy}^\mathrm{TT} = h_{yx}^\mathrm{TT}
\end{align}
which we call the ``plus'' and ``cross'' polarizations of the gravitational
wave respectively.

Now consider two more particles $C$ and $D$ lying on the $y$-axis of the LLF
separated by 
\begin{equation}
\zeta^j = y^j_{(0)} + \delta y^j
\end{equation}
as shown in figure \ref{f:rings}. A linearly polarized gravitational wave is
one that only contains the $+$ or the $\times$ polarization. The influence of
a linearly $+$-polarized gravitational wave propagating in the $z$-direction
is
\begin{align}
\delta x &= \frac{1}{2} h_{xx}^\mathrm{TT} x_{(0)}\\
\delta y &= -\frac{1}{2} h_{yy}^\mathrm{TT} y_{(0)}
\end{align}
and for the and for a linearly $\times$-polarized gravitational wave
propagating in the $z$-direction is
\begin{align}
\delta x &= \frac{1}{2} h_{xy}^\mathrm{TT} y_{(0)}\\
\delta y &= \frac{1}{2} h_{yx}^\mathrm{TT} x_{(0)}.
\end{align}
Figure \ref{f:rings} shows the effect of $h_{+}$ and $h_{\times}$ on a ring
of particles that lie in the $xy$ plane. We can see for the plus polarization
that the that the effect of a gravitational wave is to stretch the particles
in the $x$ direction, while squeezing them in the $y$ direction for the first
half of a cycle and then squeeze in the $x$ direction and stretch in the $y$
direction for latter half of the cycle.  There is therefore a relative change
in length between the two particles $AB$ and $CD$ over the course of a cycle.
For a circularly polarized gravitational wave (which contains both plus and
cross polarizations) propagating in the $z$ direction, $h(t-z) =
h_{+}(t-z) + h_{\times}(t-z)$, the overall effect is
\begin{align}
\delta x &= \frac{1}{2}\left(h_{+}(t-z) x_{(0)} + h_{\times}(t-z) y_{(0)}\right),
\delta y &= \frac{1}{2}\left(-h_{+}(t-z) y_{(0)} + h_{\times}(t-z) x_{(0)}\right).
\end{align}
It is this relative change in the lengths of two perpendicular pairs of
particles that we attempt to detect with LIGO. 

\section{Gravitational Waves from Binary Inspiral}

\subsection{The post$^2$-Newtonian Waveform}

\subsection{The Stationary Phase Approximation}
\label{ss:stationaryphase}

The inspiral waveforms that we need in the matched filter are $\tilde{h}_c(f)$
and $\tilde{h}_s(f)$, rather than the waveforms $h_c(t)$ and $h_t(t)$ computed
in chapter \ref{ch:inspiral}. There are two ways of obtaining $\tilde{h_c}(f)$.
The first is to compute $h_c(t)$ then Fourier transform the signal. This is
compulationally expensive, however, as it requires an additional FFT for each
template that we wish to filter. The second method is to use the stationary phase
approximation\cite{Mathews:1992} to express the chirp directly in the frequency
domain\cite{WillWiseman:1996,Cutler:1994}. Given a function
\begin{equation}
B(t) = A(t) \cos \phi(t)
\end{equation}
where
\begin{equation}
\frac{d\ln A}{dt} \ll \frac{d\phi}{dt}
\end{equation}
and
\begin{equation}
\frac{d^2\ln A}{dt^2} \ll \left(\frac{d\phi}{dt}\right)^2
\end{equation}
then the stationary phase approximation to the Fourier transform of $B(t)$ is
given by
\begin{equation}
\tilde{B}(f) = \frac{1}{2} A(t) \left(\frac{df}{dt}\right)^{-\frac{1}{2}}
\exp\left[ -i \left(2\pi f t' - \phi(f) - \frac{\pi}{4} \right)\right]
\end{equation}
where $t'$ is the time at which
\begin{equation}
\frac{d\phi(t)}{dt} = 2\pi f
\end{equation}
and
\begin{equation}
\phi(f) = \phi\left[t(f)\right].
\end{equation}

For the restricted post$^2$-Newtonian chirps discussed in chapter
\ref{ch:inspiral}, the orbital frequency at any instant is given by
\begin{equation}
\Omega = \frac{M^{\frac{1}{2}}}{r^{\frac{3}{2}}}.
\end{equation}
The inspiral rate for circular orbits is given by
\begin{equation}
\frac{dr}{dt} = - \frac{r}{E} \frac{dE}{dt} = 
- \frac{64}{5} \frac{\mu M^2}{r^3}.
\end{equation}
Thus
\begin{align}
r^3 \, dr &= - \frac{64}{5} \mu M^2 \, dt \\
\frac{r^4}{4} &= \frac{64}{5} \mu M^2 (t - t_c)
\end{align}
and so
\begin{equation}
r(t) = \left(\frac{256}{5} \mu M^2 \right)^{\frac{1}{4}}
       \left(t_c - t\right)^{\frac{1}{4}}.
\end{equation}
Since the emitted radiation is quadrupolar
\begin{equation}
f = \frac{\Omega}{\pi}.
\end{equation}
The gravitational wave strain is
\begin{equation}
h(t) = \frac{Q}{D} \left(\frac{384}{5}\right)^\frac{1}{2} \pi^\frac{2}{3} \mu M
r^{-1}(t) \cos \left( \int 2\pi f(t) \, dt \right).
\end{equation}
Now
\begin{align}
\frac{df}{dt} 
     &= \frac{d}{dt} \left(\frac{\Omega}{\pi}\right) 
      = \frac{d}{dt} \left(\frac{M^\frac{1}{2}}{\pi} r^{-\frac{3}{2}}\right) \\
     &= \frac{M^\frac{1}{2}}{\pi} \frac{dr}{dt} \left(-\frac{3}{2} r^{-frac{5}{2}}\right) \\
     &= \frac{M^\frac{1}{2}}{\pi} \left(-\frac{64}{5} \frac{\mu M^2}{r^3}\right)
        \left(-\frac{3}{2} r^{-frac{5}{2}}\right) \\
     &= \frac{96}{5} \frac{M^\frac{5}{2} \mu}{\pi} r^{-\frac{11}{2}}. 
     \label{eq:sp:dfdt}
\end{align}
Now
\begin{equation}
f = \frac{M^\frac{1}{2}}{\pi} r^{-\frac{3}{2}}.
\end{equation}
Solving this for $r$, we obtain
\begin{equation}
r = \frac{M^\frac{1}{3}}{\pi^\frac{2}{3} f^\frac{2}{3}}
\end{equation}
which we can substitute into equation (\ref{eq:sp:dfdt}) and obtain
\begin{align}
\frac{df}{dt} &= \frac{96}{5} \pi^\frac{8}{3} \mu M^\frac{2}{3} f^\frac{11}{3} \\
&= \frac{96}{5} \pi^\frac{8}{3} \mathcal{M}^\frac{5}{3} f^\frac{11}{3}
\end{align}
where we have defined the \emph{chirp mass} by
\begin{equation}
\mathcal{M} = \mu^\frac{3}{5} M^\frac{2}{5}.
\end{equation}
Now the phase is given to post$^2$-Newtonian order by\cite{Blanchet:1996pi}
\begin{equation}
\phi(t) = \int^t 2\pi f(t') \, dt'
\end{equation}
we won't worry about this yet.

Therefore using the stationary phase approximation, we obtain
\begin{align}
\tilde{h}(f) &= \frac{1}{2} \frac{Q}{D} \left(\frac{384}{5}\right)^\frac{1}{2} \pi^{2}{3} \mu M
                r^{-1}(t) \left(\frac{df}{dt}\right)^{-\frac{1}{2}} \exp\left[-i \Psi\right] \\
             &= \frac{1}{2} \frac{Q}{D} \left(\frac{384}{5}\right)^\frac{1}{2} \pi^{2}{3} \mu M
                \frac{\pi^\frac{2}{3} f^\frac{2}{3}}{M^\frac{1}{3}}
                \left(\frac{df}{dt}\right)^{-\frac{1}{2}} \exp\left[-i \Psi\right] \\
             &= \frac{1}{2} \frac{Q}{D} \left(\frac{384}{5}\right)^\frac{1}{2} \pi^{4}{3} \mu
                M^\frac{2}{3} r^\frac{2}{3}
                \left(\frac{df}{dt}\right)^{-\frac{1}{2}} \exp\left[-i \Psi\right].
\end{align}
Now
\begin{equation}
\left(\frac{df}{dt}\right)^{-\frac{1}{2}} =
\left(\frac{5}{96}\right)^\frac{1}{2} \pi^{-\frac{4}{3}} 
\mu^{-\frac{1}{2}} M^{-\frac{1}{3}} f^{-\frac{11}{6}}
\end{equation}
and so 
\begin{align}
\tilde{h}(f) &= \frac{1}{2} \frac{Q}{D} \left(\frac{384}{5} \frac{5}{96}\right)^\frac{1}{2}
                \mu^\frac{1}{2} M^\frac{1}{3} f^{-\frac{7}{6}} \exp\left[-i \psi\right] \\
             &= \frac{Q}{D} \mathcal{M}^\frac{5}{6} f^{-\frac{7}{6}} \exp\left[-i \psi\right].
             \label{eq:sp:template}
\end{align}
Equation (\ref{eq:sp:template}) is the stationary phase binary inspiral
template that we will use in the matched filter.

Discussion of validity of SP waveform.

\begin{equation}
t = \frac{5m}{256\eta} 
  x^{-8}\left(1 + \alpha x^2 + \beta x^3 + \gamma x^4 \right)
 \label{eq:chirplength}
\end{equation}
where
\begin{eqnarray}
x & = & \left(\pi m T_\odot f_{\mathrm{low}}\right)^{\frac{1}{3}}, \\
\alpha & = & \frac{743}{252} + \frac{11}{3}\eta, \\
\beta & = & -\frac{32\pi}{3}, \\
\gamma & = & \frac{3058673}{508032}+\frac{5429}{504}\eta+\frac{617}{71}\eta^2,
\end{eqnarray}
$m$ is the mass of the more massive object and $f_{\mathrm{low}}$ is the low
frequency cutoff of the inspiral template.


\subsection{The Innermost Stable Circular Orbit of Schwartzchild}

\section{Interferometric Gravitational Wave Detectors}

\subsection{Calibration of the Data}
\label{ss:calibration}

\section{Previous Searches for Compact Binary Inspiral}

40m, TAMA, S1. Plot of rate vs time.


The Laser Interferometric Gravitational Wave Observatory (LIGO)\cite{barish}
has completed three ``science runs'' during which all three interferometers
were collecting data simultaneously under stable operation. Analysis of the
data for gravitational waves from coalescing compact binaries has been
completed for the first two runs\cite{bns1,bns2,macho} and is in progress for
the third run. Two of the target populations in these searches are binary
neutron stars\cite{300yrs} and binary black hole
MACHOs\cite{sammacho,nakamura}. For these low mass systems, which have
component masses below $3 M_\odot$, the waveforms of the gravitational
radiation emitted are well known\cite{bdiww,biww}.  Matched filtering is a
common and effective technique for extracting known signals from
noise\cite{wz}. We have implemented a matched filter to extract inspiral
signals from interferometer noise in the package \emph{findchirp} which can be
found in the LIGO Algorithm Library (LAL)\cite{lal}.


