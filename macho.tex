% $Id$

\section{Dark Matter In The Galactic Halo}

Let us consider a simple rotational model for the disk of a spiral galaxy.
Consider a star with mass $m_s$ orbiting at radius $r$ in the outer part
of the galaxy's disk. Newtonian dynanics tells us that if the mass inside
radius $r$ is $m_g$ then
\begin{equation}
\frac{Gm_g m_s}{r^2} = \frac{m_s v_s^2}{r}
\end{equation}
where $v_s$ is the velocity of the star and $G$ is the gravitational constant. 
If we assume that as we increse $r$, the change in the $m_g$ is negligable, a
reasonable assumption if

If 
we consider $m_g$ constant, so
we consider the contribution of the additional mass 


we would expect that at the edges of spiral galaxies 

We can see immediately that we would expect that the velocity of stars at the
edge of the galactic disk would fall off as 
\begin{equation}
v_s \propto \frac{1}{\sqrt{r}}.
\end{equation}


\section{Gravitational Microlensing}

\section{Gravitational Waves Binary Black Hole MACHOs}

Motivation for search: dark matter problem in the galaxy, microlensing
results, Nakamura proposal that MACHOs may be (B)BHs. Predicted rate is
$5\times10^{-2}\times2^{\pm 1}$ events/yr/galaxy is higher than BNS rate.
Waveforms well modelled, same pipeline as S2 BNS. Brief description of sciruns
with reference to instument paper and S2 BNS paper.

\section{Binary Black Hole MACHO Population Models}

Review Ioka et al paper on BBH formation in the early universe. Density of
local dark matter. Review population models used and make some plots of MACHO
distribution.

