%%%%%%%%%%%%%%%%%%%%%%%%%%%%%%%
%%%%%%% CONVENTIONS
%%%%%%%%%%%%%%%%%%%%%%%%%%%%%%%

We adopt the Einstein summation convention where repeated indices are
summed over.  
%
The signature of spacetime is taken to be $(-1,+1,+1,+1)$.

\noindent Except where otherwise noted we work in geometric units where
$G=c=1$.  We will often measure masses in multiples of the mass of the
sun $1 \msun \approx 1.99 \times 10^{30}$ kg 

\vspace{0.5cm}

\noindent We define the Fourier transform of a function of time $g(t)$ to be
$\tilde{g}(f)$, where
%
\begin{equation*}
\tilde{g}(f)=\int_{-\infty}^\infty g(t)\, e^{- 2 \pi i f t}\, dt
\end{equation*}
%
and the inverse Fourier transform is
%
\begin{equation*}
g(t)=\int_{-\infty}^\infty \tilde{g}(f)\, e^{2 \pi i f t}\, dt
\end{equation*}
%
This convention differs from that used in some gravitational-wave
literature, but is the adopted convention in the LIGO Scientific
Collaboration.

% \vspace{0.5cm}
% 
% \noindent The time-stamps of interferometer data are measured in
% Global Positioning System (GPS) seconds: seconds since 00:00.00 UTC
% January 6, 1980 as measured by an atomic clock.

