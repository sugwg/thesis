\listfiles
\documentclass[aps,
prd,
amsmath,
amssymb,
twocolumn,
%preprint,
floatfix,
groupedaddress]{revtex4-1}
%\documentclass[aps,prl,preprint,superscriptaddress]{revtex4-1}
%\documentclass[aps,prl,reprint,groupedaddress]{revtex4-1}
\usepackage[pdftex]{graphicx}% Include figure files
\usepackage{natbib}
\usepackage{lipsum}
%\usepackage{float}
%\usepackage{stfloats}
%\usepackage{fixltx2e}
\usepackage{placeins}
\usepackage{dcolumn}% Align table columns on decimal point
%\usepackage{multicol}
\usepackage{multirow}
%\usepackage{amsmath}
\usepackage{amsthm, amsfonts, amssymb}
%\usepackage{supertabular}
\usepackage{array}
\usepackage{times}
\usepackage{latexsym}
\usepackage{hyperref}
\hypersetup{backref,  
%pdfpagemode=FullScreen,  
colorlinks=true,
linkcolor=red,
filecolor=red,
citecolor=blue}
\usepackage{epsfig}
\usepackage{subfigure}
% \usepackage{multicol}
%\usepackage{acronym}
%\usepackage[caption=false]{caption}
\usepackage{bm}% bold math
\usepackage[usenames,dvipsnames]{color}
\usepackage[normalem]{ulem}
\newcommand{\Sum}{\displaystyle\sum\limits}
\newcommand{\Int}{\displaystyle\int\limits}
\newcommand{\ii}{{\rm i}}
\newcommand{\D}{\mathrm{d}}
\newcommand{\eff}{\mathrm{eff}}
\newcommand{\phys}{\mathrm{phys}}
\newcommand{\real}{\mathrm{real}}
\newcommand{\peak}{\mathrm{peak}}
\newcommand{\EOB}{\mathrm{EOB}}
\newcommand{\NR}{\mathrm{NR}}
\newcommand{\RD}{\mathrm{RD}}
\newcommand{\M}{\mathrm{M}}
\newcommand{\EFF}{\mathrm{EFF}}
\newcommand{\X}{\mathrm{X}}
\newcommand{\Y}{\mathrm{Y}}
\newcommand{\Z}{\mathrm{Z}}
\newcommand{\horizon}{\mathrm{Horizon}}
\newcommand{\opt}{\mathrm{opt}}
\newcommand{\bnk}{\mathrm{bank}}
\newcommand{\iso}{\mathrm{iso}}
\newcommand{\refr}{\mathrm{ref}}
\newcommand{\start}{\mathrm{start}}
\newcommand{\mn}{\mathrm{min}}
\newcommand{\mx}{\mathrm{max}}
\newcommand{\tr}{\mathrm{tr}}
\newcommand{\mm}{\mathrm{mm}}
\newcommand{\Hyb}{\mathrm{Hyb}}
\newcommand{\leftn}{\left|\left|}
\newcommand{\rightn}{\right|\right|}
\newcommand{\lefti}{\left\langle}
\newcommand{\righti}{\right\rangle}
\newcommand{\Mis}{\mathcal{M}}
\newcommand{\Olap}{\mathcal{O}}
\newcommand{\FF}{\mathrm{FF}}
\newcommand{\MM}{\mathrm{MM}}
\newcommand{\N}{\mathrm{N}}
\newcommand{\cyc}{\mathrm{cyc}}
\newcommand{\E}{\mathcal{E}}

\newcommand{\red}{\textcolor{red}}
\newcommand{\harald}[1]{\textcolor{RawSienna}{#1}}

\makeatletter
\newcommand{\etal}{\textit{et~al}\@ifnextchar{\relax}{.\relax}{\ifx\@let@token.\else\ifx\@let@token~.\else.\@\xspace\fi\fi}}
\makeatother

\def\l({\left(}
\def\r){\right)}


%%%%%%%%%%%%%%%%%%%%%%%%%%%%%%%%%%%%%%%%%%%%%%%%%%%%%%%%%%%%%%%%
% Affiliations
%%%%%%%%%%%%%%%%%%%%%%%%%%%%%%%%%%%%%%%%%%%%%%%%%%%%%%%%%%%%%%%%
\newcommand{\CITA}{\affiliation{Canadian Institute for Theoretical
    Astrophysics, 60 St.~George Street, University of Toronto,
    Toronto, ON M5S 3H8, Canada}} %
\newcommand{\CIFAR}{\affiliation{Canadian Institute for Advanced Research, 180 Dundas St.~West, Toronto, ON M5G 1Z8, Canada}} %
\newcommand{\DAA}{\affiliation{Department of Astronomy and Astrophysics, 50 St.\ George Street, University of Toronto, Toronto, ON M5S 3H4, Canada}}






%opening
\begin{document}
\author{}
\affiliation{}
\date{\today}
\title{Binary black hole template banks with Numerical Relativity waveforms.}
\begin{abstract}
Coalescing black hole binaries are expected to be important sources for
detection by the Advanced Laser Interferometer Gravitational-wave
Observatory (LIGO) and Advanced Virgo. With a 10-fold increase in
sensitivity from their initial incarnations, these second generation detectors will 
be able to detect such systems at distances up to a few Gpc. Current searches for
binary black holes (BBHs) use banks of waveforms, modelled using the weak-field
slow-motion post-Newtonian (PN) approximation, as filters with which to
matched-filter the detector data. \red{[HP: This sounds like {\em all} current searches use {\em exclusively} PN templates.  Please clarify wording and remove this red]} These approximate waveforms do not
model the merger phase of the binary's evolution, where strong-field
effects dominate. For massive binaries, the merger will occur in the
sensitive frequency band of LIGO. Recent advances in Numerical
Relativiy (NR) have led to high-accuracy simulations of the
late-inspiral and mergers of BBHs, with fully-numerical solutions of Einstein's field
equations. 
\red{[HP: Note that the rest of abstract is rewritten]}
This paper explores the direct use of NR waveforms for the
  construction of detection banks.  Utilizing only six already
  completed NR simulations for non-spinning BBH systems, we construct
  a template bank that covers mass-ratios $q\le 10$ and chirp-masses
  ${\cal M}_c\gtrsim 28M_\odot$.  Hybridizing these six NR waveforms
  with PN inspiral waveforms, we construct a template bank that covers
  a larger region of the non-spinning parameter space, namely $q\le
  10$ and component masses $m_1,m_2\gtrsim 12M_\odot$.  The region of
  parameter space coverable is limited by the available mass-ratios in
  NR simulations and the truncation error of the post-Newtonian
  waveforms used in the hybrids.  We estimate that 26 NR simulations
  of approximately 50 orbits in length will be needed to construct a
  NR-PN hybrid template covering the complete non-spinning parameter
  space for mass ratios $1 \leq q \leq 10$.  These template banks are
  constructed such that the combined SNR loss from the discrete
  template spacing and the modeling error due to PN truncation error
  is below 3 per cent.
 % Current literature already has simulations available for
 %    mass-ratios $(m1/m2) = \{1,2,3,4,6,8\}$, that span $15-33$
 %    inspiral orbits before merger.  We propose to use gravitational
 %    waves (GWs) extracted from these NR simulations as a detection
 %    bank for LIGO, which is possible down to a total mass between
 %    about 60$M_\odot$ and 100$M_\odot$. To cover this parameter space
 %    at lower masses, one can use hybrid gravitational waveforms, where
 %    NR waveforms have been stitched to PN waveforms, which lowers the
 %    minimum mass to a component mass of about $m_2 = 12M_\odot$. At
 %    lower masses than this, the error between consecutive mass ratios
 %    becomes large, and more NR simulations are necessary. In addition,
 %    the errors within the hybrid waveforms themselves become large due
 %    to unknown higher-order terms in the PN waveforms. We find that we
 %    are able to completely cover the non-spinning parameter space for
 %    mass ratios $1 \leq q \leq 10$ with 26 NR simulations of
 %    approximately 50 orbits in length.
\end{abstract}
\maketitle
\section{Introduction}
In 1915, Albert Einstein published his theory of general relativity, a 
geometric theory of gravitation that sought to expand upon Newtonian 
mechanics and provide a complete description of gravity and its 
relationship with space and time. Einstein theorized that space 
and time were deeply related and existed together as a manifold 
called spacetime. Matter with energy and momentum 
existing in this manifold would create 
curvature in spacetime. Gravitational forces were the result of 
matter following geodesic curves in spacetime. This concept can 
be summarized in the Einstein field equation, which is presented 
as,
\begin{equation}
G_{\mu\nu} = 8\pi T_{\mu\nu}
\label{eq:EFE}
\end{equation}
where $G_{\mu\nu}$ is the Einstein tensor, which describes the 
curvature of spacetime, $T_{\mu\nu}$ is the 
stres-energy tensor, which describes the energy and momentum in 
spacetime, and  $G=c=1$. The Einstein tensor is defined as,
\begin{equation}
G_{\mu\nu} = R_{\mu\nu} - \frac{1}{2}Rg_{\mu\nu}
\end{equation}
where $R_{\mu\nu}$ is the Ricci curvature tensor and $g_{\mu\nu}$ is 
the metric tensor for the manifold.

An interesting result that arises from this formalism is the 
existence of gravitational waves, which are perturbations in 
spacetime caused by certain types of time-varying mass distributions. 
To describe gravitational waves, we consider 
a Minkowski metric with a small perturbation. The Minkowski metric 
is a flat spacetime metric defined as
\begin{equation}
\eta_{\mu\nu} = 
  \begin{pmatrix}
   -1 & 0 & 0 & 0 \\
    0 & 1 & 0 & 0 \\
    0 & 0 & 1 & 0 \\
    0 & 0 & 0 & 1
  \end{pmatrix}
\end{equation}
where $\mu = 0$ corresponds to the time coordinate and $\mu = {1,2,3}$ 
correspond to the spatial coordinates. In examples, we will use the coordinate 
convention $(x^0,x^1,x^2,x^3) = (ct,x,y,z)$. 
The full spacetime metric, $g_{\mu\nu}$, is then constructed as a 
linear perturbation on the Minkowski metric,
\begin{equation}
g_{\mu\nu} = \eta_{\mu\nu} + h_{\mu\nu}
\end{equation}
where $h_{\mu\nu}$ is the metric perturbation and $|h_{\mu\nu}| \ll 1$.
From here, we follow the convention of Saulson \cite{Saulson:1994} to arrive at the general 
form of a gravitational wave.
At this point it is very useful to move into the transverse traceless 
gauge where coordinates on the manifold are defined by the geodesic 
motion of freely-falling test masses. In this gauge, the weak field 
vacuum solution of the Einstein field equation becomes a wave equation: 
\begin{equation}
\square h_{\mu\nu} = 0.
\end{equation}
The solutions to this differential equation will be plane waves of 
the form
\begin{equation}
h_{\mu\nu} = C_{\mu\nu}e^{i(2\pi ft - \vec{k}\cdot\vec{x})}
\end{equation}
where $C_{\mu\nu}$ is the wave amplitude, $f$ is the frequency, 
and $\vec{k}$ is the wave vector which indicates the direction of 
propagation \cite{Carroll}.

For example, consider the case of a gravitational 
wave propogating along the $\hat{z}$-axis.
When the conditions of the transverse traceless gauge are applied, 
the resulting form of $h_{\mu\nu}$ is 
\begin{equation}
h_{\mu\nu} = 
  \begin{pmatrix}
    0 & 0 & 0 & 0 \\
    0 & h_+ & h_x & 0 \\
    0 & h_x & -h_+ & 0 \\
    0 & 0 & 0 & 0
  \end{pmatrix}
\end{equation}
where the diagonal and off-diagonal terms represent two polarizations 
of the resulting gravitational wave, called "h-plus" and "h-cross" 
respectively.
We can see the effects of this perturbation by observing the  
spacetime interval on the manifold. The spacetime interval is defined as 
\begin{equation}
ds^2 = dx^\mu g_{\mu\nu}dx^\nu.
\end{equation}
Substituting in our perturbed metric for $g_{\mu\nu}$, we find that 
the spacetime interval can be broken up into a standard Minkowski line 
element and a perturbation due to $h_{\mu\nu}$.
\begin{equation}
ds^2 = dx^\mu (\eta_{\mu\nu} + h_{\mu\nu})dx^\nu \\
\end{equation}
\begin{equation}
ds^2 = dx^\mu \eta_{\mu\nu} dx^\nu + dx^\mu h_{\mu\nu}dx^\nu
\label{eq:spacetime}
\end{equation}

As an example, we present the case of a plus-polarized gravitational wave 
propagating in the $\hat{z}$ direction and observe the effect of the perturbation 
on the spacetime interval. The perturbation will have the form 
\begin{equation}
h_{\mu\nu} = 
  \begin{pmatrix}
    0 & 0 & 0 & 0 \\
    0 & h_+ & 0 & 0 \\
    0 & 0 & -h_+ & 0 \\
    0 & 0 & 0 & 0
  \end{pmatrix}
\end{equation}
Using the coordinate convention of $(ct,x,y,z)$, the unperturbed
spacetime interval is given as: 
\begin{equation}
ds^2 = -c^2 dt^2 + dx^2 + dy^2 + dz^2.
\end{equation}
Since the perturbation is spatially transverse to the direction of 
propagation, the ct- and z-coordinates will not be modulated by the 
gravitational wave. The x- and y-coordinates will be modulated  
according to equation \ref{eq:spacetime}. The resulting spacetime 
interval is
\begin{equation}
ds^2 = -c^2 dt^2 + (1 + h_+)dx^2 + (1 - h_+)dy^2 + dz^2.
\end{equation}
This shows that a gravitational wave propogating along the $\hat{z}$-axis 
will differentially stretch and squeeze spacetime in the transverse 
axes. The exact form of $h_+$ will depend on the source of the 
gravitational waves. A visualization of this stretching and squeezing 
is shown in Figure \ref{fig:polarizations}\cite{Polarization}. The cross polarization  
stretches and squeezes at a 45 degree angle relative to the plus 
polarization.

\begin{figure}[ht!]
\includegraphics[width=\textwidth]{figures/introduction/polarisations2}
\caption[Plus and cross polarizations]{Plus and cross polarizations %
         of a gravitational wave.}
\label{fig:polarizations}
\end{figure}

The Advanced LIGO interferometers are designed to be sensitive 
to this differential stretching and squeezing by constructing orthogonal 
optical cavities. A gravitational wave passing through an aLIGO interferometer 
will differentially 
modulate the lengths of the optical cavities, creating an interference 
pattern at the output of the instrument that can be searched for 
gravitational wave signals. The layout and gravitational wave readout scheme 
of the interferometers is discussed below.

\section{The Advanced LIGO Interferometers}\label{sec:aligo}

The Advanced LIGO (aLIGO) interferometers are a pair of dual-recycled Michelson interferometers 
that employ 4km long Fabry-Perot cavities in their arms to increase the interaction time with a 
gravitational wave signal. 
Figure \ref{fig:aligo} shows a simplified layout of an aLIGO interferometer. 

\begin{figure}[ht!]
\includegraphics[width=\textwidth]{figures/introduction/ALIGO_layout}
\caption[Layout of Advanced LIGO]{Layout of Advanced LIGO}
\label{fig:aligo}
\end{figure}

At the input to an aLIGO interferometer is a solid-state Nd:YAG laser that provides laser light 
at a wavelength of 1064 nm. Not included in Figure \ref{fig:aligo} are frequency and 
intensity stabilization control loops designed to provide as stable a laser source as 
possible for the experiment. This stabilized laser is called the pre-stabilized laser 
(PSL). The laser light is passed through a series of 
electro-optic modulators (EOM) where radio-frequency (RF) sidebands are generated 
and imparted onto the light. These RF sidebands are used to control auxiliary optical 
degrees of freedom in the interferometer. The beam is then passed through the 
input mode cleaner (IMC), which rejects higher order spatial modes of the beam 
and transmits a circular TEM00 mode to be used in the instrument.

Once the beam has been stabilized in frequency and intensity and the higher order 
optical modes have been stripped away, it is transmitted through the power 
recycling mirror and enters the vertex of the interferometer. In the vertex, 
the beam is split 50/50 by the beamsplitter. Half of the light is directed toward  
the input test mass (ITM) of the X-arm and half of the light is directed  
toward the ITM of the Y-arm. As mentioned previously, the aLIGO arms are not 
single bounce cavities; they are comprised of Fabry-Perot cavities that allow the 
light to circulate in the arm cavities multiple times. The light is stored in 
the arm cavities for $\sim$1ms, trapped between the highly reflective surfaces 
of the ITM and the end test mass (ETM), before it is transmitted back through 
the ITM and into the vertex.

When a gravitational wave passes through an aLIGO inteferometer, the distance
between the ITM and ETM of each arm is modulated, causing the light to have a
longer or shorter travel time as it traverses the arm. Since gravitational
waves expand space in one direction while the orthogonal direction contracts,     
the X- and Y-arms will experience differential changes in length. When light
from the arms is recombined at the beamsplitter, there will be a difference
in phase between the two beams as they have traveled different paths. The 
resulting light from this recombination of phase shifted beams is called the 
antisymmetric part of the output. The part of the beam that is recombined 
in phase is called the symmetric part of the output.

The beams returning from each arm are recombined at the beamsplitter. The 
symmetric part of the beam 
will be sent back toward the power recycling mirror. The power recycling mirror 
forms a resonant cavity with the ITMs, allowing for light at the symmetric 
port of the beamsplitter to be added coherently to incoming light from the PSL and 
increasing the effective power in the vertex. This increase in effective power 
is known as the power recycling gain. 

The antisymmetric part of the beam is sent toward the signal recycling mirror. 
The signal recycling cavity is used to tune the frequency response of the 
interferometer by adjusting the effective finesse of the coupled cavity 
formed by the signal recycling cavity and the arm cavities. 
If the light returning from the arms has accumulated some differential amount of 
phase as it traveled 
along the arms, perhaps from a gravitational wave modulating the length of each 
arm differentially, it will be transmitted through the signal recycling cavity 
and into the output mode cleaner (OMC). The OMC behaves similarly to the IMC, 
stripping away higher order optical modes and isolating the TEM00 mode of the 
beam. The transmitted, mode cleaned signal is then read out using a homodyne 
detection scheme on a DC photodiode. 

\subsection{DC Readout}

When a gravitational wave modulates the length of an arm cavity, the light 
traveling in that arm experiences a phase modulation. This phase modulation 
can be visualized by picturing the beam in frequency space. In figure 
\ref{fig:omc-freq}, the carrier beam frequency is designated as $f_0$. 
The phase modulation due to 
a gravitational wave signal introduces a frequency sideband at the 
gravitational wave frequency, which is in the 30-2000 Hz range. 
The 
RF sidebands used for auxiliary optical cavity control are offset from the 
carrier frequency by 9, 24, and 45 MHz. 
The RF sidebands, which in a 
homodyne detection scheme would only contribute shot noise to the output signal, 
are rejected by the OMC. The gravitational wave sidebands, however, are at a 
low enough frequency offset that they are within the cavity pole of the OMC 
and are allowed to transmit through the cavity.

Since the OMC DC photodiode measures power, it measures the square of the 
incident optical field and witnesses beat frequencies between different 
components of the light. If the RF sidebands have been filtered out by 
the OMC, the only remaining beat note will be that of the carrier beam ($f_0$) 
beating against the gravitational wave sideband ($f_0 + f_{GW}$). This beat note will 
appear as the difference in frequency between the two optical fields, 
leaving behind a signal in the 30-2000 Hz range ($f_{GW}$) and providing a 
natural demodulation inherent to the measurement process. 
The process of recovering the gravitational wave sideband using the 
carrier field as a reference is known as homodyne detection. 

\begin{figure}[ht!]
\includegraphics[width=\textwidth]{figures/introduction/omc-freq}
\caption[Sidebands and OMC cavity pole]{Frequency domain visualization of beam %
         at OMC. Grey dotted lines indicate the cavity pole. The gravitational %
         wave sidebands are within the cavity pole and are transmitted through %
         the OMC. The RF sidebands are in the MHz range and are rejected by the %
         OMC.}
\end{figure}\label{fig:omc-freq}

\section{Sources of Gravitational Waves}
Include that box with modeled, unmodeled, transient, and continuous.

CBCs are the bread and butter, expect BNS, NSBH, and BBH sources
Continuous waves from pulsars
Bursts from supernovae
Stochastic background


\section{Searching for Compact Binary Coalescences}

Steal this from O1 CBC DQ paper


\section{The Advanced Detector Network}

The Advanced Laser Interferometer Gravitational-Wave Observatory (aLIGO) is 
part of a worldwide effort to detect gravitational waves from astrophysical 
sources. The two aLIGO interferometers, one in Hanford, WA and one in 
Livingston, LA, are part of a growing network of ground-based interferometric 
gravitational wave detectors. Each aLIGO interferometer has 4km long arms 
arranged in an L-shaped configuration. A gravitational wave passing through 
an aLIGO interferometer will cause the arms to expand and contract, 
creating an interferometric signal at the output of the instrument. 
Section \ref{sec:aligo} contains a more detailed description of the aLIGO 
interferometers. 

There are a number of other interferometric gravitational wave detectors 
being built and commissioned for future use in collaboration with aLIGO.
The Advanced VIRGO detector is being built and commissioned in Cascina, Italy. 
When it is fully commissioned, VIRGO will be joining LIGO in observing runs. 
The VIRGO interferometer has 3km arms, which should provide enough 
sensitivity to allow for triangulation of astrophysical sources.

The GEO600 detector, located in Hanover, Germany is an interferometer built in 
collaboration between Germany and the United Kingdom. 
GEO600 is an extremely valuable test bed for interferometric technologies,
including quantum optics and homodyne detection. However, with 600m arms, GEO600 
is unlikely to be sensitive enough to witness expected astrophysical sources.

The KAGRA detector, located underground in the Kamioka mine in Japan, 
is in its commissioning phase. KAGRA has 3 km long arms and, 
unlike other gravitational wave interferometers, employs cryogenics to 
reduce thermal noise in its optics. When complete, KAGRA should be 
sensitive enough to contribute to the worldwide detector network.

Include that cool picture of the advanced detector network.


\section{Methodology}\label{s1:methodology}

\subsection{post-Newtonian waveforms}\label{s2:PNwaveforms}


Post-Newtonian (PN) theory is a perturbative approach to describing the
motion of a compact object binary, during the slow-motion and weak-field 
regime, i.e. the inspiral phase. The conserved energy of a binary in orbit,
$E$, has been calculated to 3PN order in literature~\citep{Jaranowski:1997ky,
Jaranowski:1999ye,Jaranowski:1999qd,Damour:2001bu,Blanchet:2003gy,
Damour:2000ni,Blanchet:2002mb}.
Using the adiabatic approximation, we treat the course of inspiral as a series
of radially shrinking circular orbits. This is valid during the inspiral when
the angular velocity of the binary evolves more slowly than the orbital 
time-scale. The radial separation shrinks as the binary loses energy to 
gravitational radiation that propagates outwards from the system. 
% For a binary with individual component masses $m_1$ and $m_2$ and total mass
%$M=m_1+m_2$, the conserved energy accurate to 3PN
% can be written as \citep{Jaranowski:1997ky,Jaranowski:1999ye,Jaranowski:1999qd,Damour:2001bu,Blanchet:2003gy,Damour:2000ni,Blanchet:2002mb}
% \begin{equation}
% \begin{split}\label{eq:E3PN}
% E_3(v)=&-\dfrac{1}{2}\eta v^2 \left[1- \l(\dfrac{3}{4}+\dfrac{1}{12}\eta\r)v^2 - \l(\dfrac{27}{8}-\dfrac{19}{8}\eta\right.\right.\\
% +&\left.\left.\dfrac{1}{24}\eta^2 \r)v^4 - \l(\dfrac{675}{64}-\l(\dfrac{34445}{576}-\dfrac{205}{96}\pi^2\r)\eta\right.\right.\\
% +&\left.\left.\dfrac{155}{96}\eta^2 +\dfrac{35}{5184}\eta^3\r) v^6\right],
% \end{split}
% \end{equation}
% where $\eta=m_1m_2/(m_1+m_2)^2$, $v=(\pi Mf)^{1/3}$ is the characteristic velocity of the binary, and $f$ denotes the frequency of the emitted gravitational wave throughout.
The energy flux from a binary $F$ is known in PN theory to 3.5PN 
order~\cite{FluxandE3-5PN,Blanchet:2004ek,Blanchet:2005tk,Blanchet:2004bb}.
% \begin{equation}
% \begin{split}\label{eq:Ft3.5PN}
% F_{3.5}(v)=&\dfrac{32}{5}\eta^2 v^{10}\left[1 - \l(\dfrac{1247}{336}+\dfrac{35}{12}\eta\r)v^2+4\pi v^3\right.\\
% -&\left.\l(\dfrac{44711}{9072}-\dfrac{9271}{504}\eta -\dfrac{65}{18}\eta^2 \r)v^4\right.\\
% -&\left.\l(\dfrac{8191}{672}+\dfrac{583}{24}\eta\r)\pi v^5+ \l(\dfrac{6643739519}{69854400}\right.\right.\\
% +&\left.\left.\dfrac{16}{3}\pi^2 -\dfrac{1712}{105}\gamma +\l(\dfrac{41}{48}\pi^2 -\dfrac{134543}{7776}\r)\eta \right.\right.\\
% -&\left.\left.\dfrac{94403}{3024}\eta^2 -\dfrac{775}{324}\eta^3 -\dfrac{856}{105}\textrm{log}(16v^2)\r)v^6\right.\\ 
% -&\left.\l(\dfrac{16285}{504}-\dfrac{214745}{1728}\eta -\dfrac{193385}{3024}\eta^2 \r)\pi v^7\right],
% \end{split}
% \end{equation}
% where $\gamma$ is Euler's constant. In the limit $\dot{\omega}/\omega \ll 1$, 
% we can approximate the energy of the system to be the energy averaged over a 
% period. 
Combining the energy balance equation, $\D E/\D t = -F$, with Kepler's law 
gives a description of the radial and orbital phase evolution of the binary. 
We start the waveform where the GW frequency enters the sensitive frequency 
band of advanced LIGO, i.e. at $15$Hz. 
% \begin{subequations}\label{eq:PNOrbitalEvolution01}
% \begin{align}
% \dfrac{\D\phi}{\D t} - \dfrac{v^3}{M} &= 0,\label{eq:PNOrbitalEvolution01_01}\\
% \dfrac{\D v}{\D t} + \dfrac{F(v)}{M\l(\D E/\D v\r)} &= 0.\label{eq:PNOrbitalEvolution01_02}
% \end{align}
% \end{subequations}
Depending on the way the expressions for orbital energy and flux
%in Eq.~\ref{eq:E3PN} \& \ref{eq:Ft3.5PN} 
are combined to obtain the coordinate evolution for the binary,
%to solve Eq.~\ref{eq:PNOrbitalEvolution01}, 
we get different Taylor\{T1,T2,T3,T4\} time-domain approximants. Using the 
stationary phase approximation~\cite{MatthewsWalker}, frequency-domain 
equivalents of these approximants, i.e. TaylorF$n$, can be constructed. 
Past GW searches have extensively used the TaylorF2 approximant, as it has a
closed form and mitigates the computational cost of generating and numerically
fourier-transforming time-domain template~\cite{Colaboration:2011nz,Abadie:2010yb,
Abbott:2009qj,Abbott:2009tt,Messaritaki:2005wv}.
We refer the reader to Ref.~\cite{PNtheoryLivingReviewBlanchet,JolienGWPhysAst}
for an overview.
From the coordinate evolution, we obtain the emitted gravitational waveform;
approximating it by the quadrupolar multipole $h_{2,\pm 2}$ which is the 
dominant mode of the waveform.
% in previous waveform-related studies~\cite{CompTemplates2001,CompTemplates2009,
% Miller:2005qu,NRPNComparisonBoyleetal,NRPNComparisonBaker2007,Boyle:2011dy,
% MacDonald:2011ne,Brown:2012nn}.



\subsection{Effective-One-Body waveforms}\label{s2:EOBwaveforms}


Full numerical simulations are available for a limited number of binary 
mass combinations. We use a recently proposed EOB 
model~\cite{BuonannoEOBv2Main}, which we refer to as EOBNRv2, as a substitute
to model the signal from binaries with arbitrary component masses
in this paper. This model was calibrated to most of the numerical simulations
that we use to construct templates banks, which span the range of masses 
we consider here well. So we expect this approximation to hold. We describe the
model briefly here.

The EOB formalism maps the dynamics of a two-body system onto an effective-mass
moving in a deformed Schwarzschild-like background~\citep{EOBOriginalBuonannoDamour}.
The formalism has evolved to use Pad\'{e}-resummations of perturbative 
expansions calculated from PN theory, and allows for the introduction of higher
(unknown) order PN terms that are subsequently calibrated against NR 
simulations of BBHs 
\cite{EOBdevel01,EOBdevel02,EOBNRdevel03,DamourFluxhlm01,EOBNRdevel01}. The EOB 
model proposed recently in Ref.~\citep{BuonannoEOBv2Main} has been calibrated 
to SpEC NR waveforms for binaries of mass-ratios $q=\{1,2,3,4,6\}$, where 
$q\,\equiv \, m_1/m_2$, and is the one that we use in this paper (we will refer
to this model as EOBNRv2).

The dynamics of the binary enters in the metric coefficient of the deformed
Schwarzschild-like background, the EOB Hamiltonian \cite{EOBOriginalBuonannoDamour}, 
and the radiation-reaction force. 
% In polar coordinates $(r,\Phi)$, the EOB 
% metric is written as
% \begin{equation}\label{eq:dsEOB}
% \D s_{\eff}^2 = -A(r)\D t^2 + \dfrac{A(r)}{D(r)}\D r^2 + r^2\left(\D\Theta^2 + \sin^2\Theta \D\Phi^2\right).
% \end{equation}
These
%metric coefficients $A(r)$ and $D(r)$ 
are known to 3PN order \cite{EOBOriginalBuonannoDamour,PadeAD} from PN theory.
The 4PN \& 5PN terms were introduced in the potential $A(r)$, which was 
Pad\'{e} resummed and calibrated to NR simulations
\citep{EOBNRdevel01,EOBNRdevel02,EOBNRdevel03,EOBNRdevel04,BuonannoEOBv2Main}.
We use the resummed potential calibrated in Ref.~\cite{BuonannoEOBv2Main} 
(see Eq.~(5-9)). The geodesic dynamics of the reduced mass 
$\mu\,=\,m_1 m_2 / M$ in the deformed background 
%of Eq.~\eqref{eq:dsEOB} 
is described by
the Hamiltonian $H^{\eff}$ given by Eq.~(3) in \cite{BuonannoEOBv2Main}.
% , which can be written as \cite{EOBOriginalBuonannoDamour,PadeAD}
% \begin{equation}
% \begin{split}
% H^{\eff} =\, & \mu\hat{H}^{\eff} \\
%          =\, & \mu\sqrt{A(r) \left( 1 +  \dfrac{A(r)}{D(r)}p_r^2 + 2(4 - 3\eta)\eta \dfrac{p_r^4}{r^2} + \dfrac{p^2_{\Phi}}{r^2} \right)},
% \end{split}
% \end{equation}
% where $(p_r,p_{\Phi})$ are momenta conjugate to the $(r,\Phi)$ coordinates. 
The Hamiltonian describing the conservative dynamics of the binary
(labeled the \textit{real} Hamiltonian $H^{\real}$) is related to 
$\hat{H}^{\eff}$ as in Eq.~(4) of \cite{BuonannoEOBv2Main}.
% \begin{equation}
% H^{\real} = \mu\hat{H}^{\real} = M \sqrt{1 + 2\eta (\hat{H}^{\eff} - 1)}.
% \end{equation}
The inspiral-merger dynamics can be obtained by numerically solving the 
Hamiltonian equations of motion for $H^{\real}$, see e.g. Eq.(10)
of~\cite{BuonannoEOBv2Main}. 

The angular momentum carried away from the binary
by the outwards propagating GWs results in a radiation-reaction force
%$\hat{F}_{\Phi}$ 
that causes the orbits to shrink.
% ,
% \begin{eqnarray}
%\dfrac{\D r}{\D\hat{t}} &\equiv & \dfrac{\partial \hat{H}^{\real}}{\partial p_r} = \dfrac{A(r)}{\sqrt{D(r)}}\dfrac{\partial \hat{H}^{\real}}{\partial p_{r*}} (r, p_{r*}, p_{\Phi}) ,\\
%\dfrac{\D\Phi}{\D\hat{t}} &\equiv & \hat{\Omega} = \dfrac{\partial \hat{H}^{\real}}{\partial p_{\Phi}} (r, p_{r*}, p_{\Phi}) , \\ 
%\dfrac{\D p_{r_*}}{\D\hat{t}} &=& -\dfrac{A(r)}{\sqrt{D(r)}} \dfrac{\partial \hat{H}^{\real}}{\partial r} (r, p_{r*}, p_{\Phi}) ,\\
% \dfrac{\D p_{\Phi}}{\D\hat{t}} &=& \hat{F}_{\Phi},%(r, p_{r*}, p_{\Phi}) ;
% \end{eqnarray}
% where, $\hat{t}\equiv t/M$ is time in dimensionless units; and
% \begin{equation}
%\hat{F}_{\Phi} = -\dfrac{1}{\eta \hat{\Omega}} \dfrac{\D E}{\D t} = -\dfrac{1}{\eta v^3} \dfrac{\D E}{\D t},
% \hat{F}_{\Phi} = -\dfrac{1}{\eta v^3} \dfrac{\D E}{\D t},
% \end{equation}
% where, $v=\hat{\Omega}^{1/3}=(\pi Mf)^{1/3}$ as before, and $f$ is the 
% instantaneous frequency of the emitted GWs. 
This is due to the flux of energy from the binary, which
%flux $\D E/\D t$ 
is obtained by summing over the contribution from each term in the multipolar
decomposition of the inspiral-merger EOB waveform.
% , i.e.
% \begin{equation}\label{eq:definedEdt}
% \frac{\D E}{\D t} = \frac{\hat{\Omega}^2}{8\pi} \Sum_{l}\Sum_{m} \left|\frac{\mathcal{R}}{M} h_{lm}\right|^2,
% \end{equation}
% where $\mathcal{R}$ is the physical distance to the binary, and 
%$h_{lm}$, which are 
% the EOB waveform multipoles, 
%defined implicitly as
%  \begin{equation}\label{eq:hlmdef}
%  h_+ - \ii h_{\times} = \dfrac{M}{\mathcal{R}} \Sum^{\infty}_{l=2} \Sum^{m=l}_{m = -l} Y^{lm}_{-2}\, h_{lm},
%  \end{equation}
%  where $Y^{lm}_{-2}$ are spin $-2$ weighted spherical harmonics, 
%  and $h_{+,\times}$ are the two orthogonal polarizations of the incoming GW. 
% These multipoles 
% %$h_{lm}$ 
% are functions of the orbital phase of the binary $\propto e^{-\ii m\Phi}$.
Complete resummed expressions for these multipoles~\cite{DamourFluxhlm01} can be 
read off from Eq.(13)-(20) of Ref.~\cite{BuonannoEOBv2Main}. In this paper, 
as for PN waveforms, we model the inspiral-merger part 
%$l=2\ldots 8$ to compute the energy flux, and approximate the 
%summation in Eq.~\ref{eq:hlmdef} 
by summing over the dominant $h_{2,\pm 2}$ multipoles.

The EOB merger-ringdown part is modeled as a sum of $N$ quasi-normal-modes 
(QNMs),
% \begin{equation}
% h_{lm}^{\RD}(t) = \Sum^{N-1}_{n=0}A_{lmn}e^{-\ii\sigma_{lmn}(t-t_{lm}^{\mathrm{match}})},
% \end{equation}
where $N=8$ for EOBNRv2~\citep{EOBNRdevel01,EOBNRdevel02,EOBNRdevel04,BHRDQNMs}.
The ringdown frequencies depend on the mass and spin of the BH that is formed 
from the coalescence of the binary. The inspiral-merger and ringdown parts are
attached by matching them at the time at which the amplitude of the 
inspiral-merger waveform peaks.
%, i.e. at $t_{lm}^{\mathrm{match}}=t^{lm}_{\peak}$
~\citep{EOBNRdevel01,BuonannoEOBv2Main}. The matching procedure followed
% involves equating the complex amplitude and
% phase (and its derivative) of $h_{lm}(t)$ and $h_{lm}^{\RD}(t)$ over a small 
% interval of time ending at $t_{lm}^{\mathrm{match}}$, from which we obtain the
% complex amplitudes $A_{lmn}$. 
is explained in detail Sec.~II~C of Ref.~\citep{BuonannoEOBv2Main}.
%gives further details of this procedure. 
By combining them, we obtain the complete waveform for a BBH system.
% We combine the inspiral-merger waveform $h_{lm}(t)$ and the ringdown 
% waveform $h^{\RD}(t)$ to obtain the complete inspiral-merger-ringdown EOB 
% waveform $h^{\textrm{IMR}}(t)$, i.e.
% \begin{equation}
% h^{\textrm{IMR}}_{lm}(t) = h_{lm}(t)\Theta(t^{\mathrm{match}}_{lm}-t) + h^{\RD}(t)\Theta(t-t^{\mathrm{match}}_{lm}),
% \end{equation}
% where $\Theta(x)=1$ for $x\geq 0$, and 0 otherwise. These multipoles are then
% combined to give the two orthogonal polarizations of the gravitational 
% waveform, $h_+$ and $h_{\times}$, as in Eq.~\ref{eq:hlmdef}.



%%%%%%%%%%%%%%%%%%%%%%%%%%%%%%%%%%%%%%%%%%%%%%%%%%%%%%%%%%%%%%%%
\subsection{Numerical Relativity Waveforms}\label{s2:NRwaveforms}

The numerical relativity waveforms used in this paper were produced
with the SpEC code~\cite{spec}, a multi-domain pseudospectral code to solve
Einstein’s equations. SpEC uses Generalized Harmonic coordinates,
spectral methods, and a flexible domain decomposition, all of which
contribute to it being one of the most accurate and efficient codes
for computing the gravitational waves from binary black hole
systems. High accuracy numerical simulations of the late-inspiral,
merger and ringdown  of coalescing binary black-holes have been
recently performed for mass-ratios $q\equiv m_1/m_2\in\{1,2,3,4,6,8\}$
~\cite{Buchman:2012dw,Scheel:2008rj,NRPNComparisonBoyleetal,Mroue:2012kv}.

The equal-mass,
non-spinning waveform covers 33 inspiral orbits and was first discussed
in~\cite{MacDonald:2012mp,Mroue:2012kv}. 
This waveform was obtained with numerical techniques similar to those 
of~\cite{Buchman:2012dw}. The unequal-mass waveforms of mass ratios 
$2, 3, 4,$ and $6$ were presented in detail in Ref.~\cite{Buchman:2012dw}.
The simulation with mass ratio $6$ covers about 20 orbits and the
simulations with mass ratios 2, 3, and 4 are somewhat shorter and
cover about 15 orbits. The unequal mass waveform with mass ratio 8 was
presented as part of the large waveform catalog
in~\cite{Mroue:2013xna,Mroue:2012kv}. It is approximately 25 orbits in
length. 
% As it becomes possible to simulate BBH evolution for longer times, in our 
% template bank construction we assume that simulations that span $\gtrsim 20$
% orbits, including the inspiral, merger and ringdown cycles, are (will be) 
% available by the time Advanced LIGO begins observation runs. 
We summarize the NR simulations used in this study in 
Table~\ref{table:etalist4}.

\begin{table}
\begin{tabular}{| c | c | c |}
\hline
$\eta$ & q & Length (in orbits)\\ \hline
0.25 & 1 & 33 \\
0.2222 & 2 & 15 \\
0.1875 & 3 & 18 \\
0.1600 & 4 & 15 \\
0.1224 & 6 & 20 \\
0.0988 & 8 & 25 \\
%0.0884 & 9.2 & ?? \\
\hline
\end{tabular}
\caption{SpEC BBH simulations used in this study.  Given are symmetric mass-ratio $\eta$, mass-ratio $q=m_1/m_2$, and the length in orbits of the simulation.}
\label{table:etalist4}
\end{table}



%%%%%%%%%%%%%%%%%%%%%%%%%%%%%%%%%%%%%%%%%%%%%%%%%%%%%%%%%%%%%%%%
\subsection{Hybridization procedures}\label{s2:NRpNhybridwaveforms}
The hybridization procedure used for this investigation is described in Sec.~3.3 of Ref.~\cite{MacDonald:2011ne}: The PN waveform, $h_\text{PN}(t)$, is time and phase shifted to match the NR waveform, $h_\text{NR}(t)$, and they are smoothly joined together in a GW frequency interval centered at $\omega_m$ with width $\delta\omega$: 
\begin{equation}\label{eq:omega_match}
\omega_m-\frac{\delta\omega}{2} \le \omega \le \omega_m+\frac{\delta\omega}{2}.
\end{equation}
This translates into a matching interval $t_{\rm min}<t<t_{\rm max}$ because the GW frequency continuously increases during the inspiral of the binary. As argued in Ref.~\cite{MacDonald:2011ne}, we
choose $\delta\omega = 0.1\omega_m$ because it offers a good compromise of suppressing residual oscillations in the matching time, while still allowing $h_\text{PN}(t)$ to be matched as closely as possible to the beginning of $h_\text{NR}(t)$.

The PN waveform depends on a (formal) coalescence time, $t_c$, and phase, $\Phi_c$. These two parameters are determined by minimizing the GW phase difference in the matching interval $[t_\text{min}, t_\text{max}]$ as follows:
\begin{equation}\label{eq:tcphic_bymin}
t_c', \Phi_c' = \mathrm{ \mathop{arg min}_{t_c, \Phi_c}}\int^{t_{\rm max}}_{t_{\rm min}} \big(
  \phi_{\rm PN}(t;t_c,\Phi_c) - \phi_{\rm NR} (t) \big)^2 \rm{d}t,
\end{equation}
where $t_c'$ and $\Phi_c'$ are the time and phase parameters for the best matching between $h_\text{PN}(t)$ and $h_\text{NR}(t)$, and $\phi(t)$ is the phase of the (2,2) mode of the gravitational
radiation. Since we consider only the (2,2) mode, this procedure is identical to time and phase shifting the PN waveform until it has best agreement with NR as measured by the integral in Eq.~(\ref{eq:tcphic_bymin}). The hybrid waveform is then constructed in the form
\begin{equation}
h_\text{H}(t) \equiv \mathcal{F}(t) h_\text{PN}(t;t'_c,\Phi'_c) + \big[1- \mathcal{F}(t)\big]  h_\text{NR} (t), 
\end{equation}
where $\mathcal{F}(t)$ is a blending function defined as
\begin{eqnarray}
\mathcal{F}(t) \equiv 
\left\{
\begin{array}{ll}
  1, &  t < t_{\rm min} \\ 
 \cos^2\frac{\pi(t - t_{\rm min})}{2(t_{\rm max} - t_{\rm min})},\quad\quad &  t_{\rm min}
  \leq t < t_{\rm max} \\ 
  0. & t\geq t_{\rm max}  .
\end{array}
\right.\label{eq:BlendingFunction}
\end{eqnarray}
In this work, we construct all hybrids using the same procedure, Eqs.~(\ref{eq:omega_match})--(\ref{eq:BlendingFunction}), and vary only the PN approximant and the matching frequency $\omega_m$.


\subsection{Quantifying waveform accuracy \& bank
  effectualness}\label{s2:quantifyingerrors} 
%%%%%%%%%%%%%%%%%%%%%%%%%%%%%%%%%%%%%%%%%%%%%%%%%%%%%%%%%%%%%%%%%%%%%%%%%%%%%%%

To assess the recovery of SNR from template banks with NR waveforms or NR-PN 
hybrids as templates, we use the measures proposed in
Ref.~\cite{FittingFactorApostolatos,Sathyaprakash:1991mt,Balasubramanian:1995bm}. 
The gravitational waveform emitted during and driving a BBH coalescence is
denoted as $h(t)$, or simply $h$. The inner product between two 
waveforms $h_1$ and $h_2$ is
\begin{equation}\label{eq:overlap}
(h_1|h_2) \equiv 
4\int^{f_\mathrm{Ny}}_{f_\mathrm{min}}\dfrac{\tilde{h}_1(f)\tilde{h}_2^*(f)}{S_n(f)}\D f,
\end{equation}
where $S_n(f)$ is the one-sided power spectral density (PSD) of the detector
noise, which is assumed to be stationary and Gaussian with zero mean; 
$f_\mathrm{min}$ is the lower frequency cutoff for filtering; $f_\mathrm{Ny}$
is the Nyqyuist frequency corresponding to the waveform sampling rate; and 
$\tilde{h}(f)$ denotes the Fourier transform of $h(t)$.
% The noise PSD $S_n(f)$ is defined by
% \begin{equation}
%  \langle\tilde{n}(f)\tilde{n}^*(f')\rangle = \dfrac{1}{2}S_n(f)\delta(f-f'),
% \end{equation}
% where $\tilde{n}(f)$ is the Fourier transform of the detector noise $n(t)$, and
% $\langle\dots \rangle$ denotes an average over an ensemble of noise 
% realizations. 
In this paper, we take $S_n(|f|)$ to be the \textit{zero-detuning high power} 
noise curve for aLIGO, for both bank placement and overlap
calculations~\cite{aLIGONoiseCurve}; and set the lower frequency cutoff 
$f_\mathrm{min} =15$~Hz. The peak GW frequency for the lowest binary masses
that we consider, i.e. for $m_1+m_2\simeq 12M_\odot$, is $\sim 2.1$~kHz during
the ringdown phase. We sample the waveforms at $8192$~Hz, preserving the 
information content up to the Nyquist frequency $f_\mathrm{Ny}=4096$~Hz.
% The normalized overlap between the two waveforms,
% \begin{equation}
% (\hat{h}_1|\hat{h}_2) = \dfrac{(h_1|h_2)}{\sqrt{(h_1|h_1)(h_2|h_2)}},
% \end{equation}
% is also sensitive to a relative constant phase and time offset between $h_1$ 
% and $h_2$, $\phi_c$ and $t_c$, apart from the intrinsic mass parameters. These
A waveform, h, is normalized (made to be a unit vector) by 
$\hat{h} = h/\sqrt{h | h}$. In addition to being senstive to their 
intrinsic mass parameters, the inner product of two normalized waveforms is 
sensitive to phase and time shift differences between the two, $\phi_{c}$ and
$t_{c}$.  These two parameters ($\phi_c$ and $t_c$) can be analytically
maximized over to obtain the maximized overlap $\Olap$,
\begin{equation}\label{eq:maxnormolap}
\Olap(h_1,h_2) = 
\underset{\phi_c,t_c}{\mathrm{max}}\,\l(\hat{h}_1|\hat{h}_2(\phi_c,t_c)\r),
\end{equation}
which gives a measure of how ``close'' the two waveforms are in the waveform
manifold, disregarding differences in overall amplitude. The \textit{mismatch}
$\mathcal{M}$ between the two waveforms is then
\begin{equation}\label{eq:mismatch}
\mathcal{M}(h_1,h_2) = 1 - \Olap(h_1,h_2).
\end{equation}

Matched-filtering detection searches employ a discrete bank of modeled
waveforms as filters. The optimal signal-to-noise ratio (SNR) is obtained when
the detector strain $s(t)\equiv h^{\tr}(t) + n(t)$ is filtered with the 
\textit{true} waveform $h^{\tr}$ itself, i.e.
\begin{equation}
 \rho_{\mathrm{opt}} = \underset{\phi_c,t_c}{\mathrm{max}}\,\l(h^{\tr}|\hat{h}^{\tr}(\phi_c,t_c)\r) = \leftn h^{\tr}\rightn,
\end{equation}
where $\leftn h^{\tr}\rightn\equiv\sqrt{\left(h^{\tr},h^{\tr}\right)}$ is the
noise weighted norm of the waveform. With a discrete bank of filter templates, 
the SNR we recover
\begin{equation}\label{eq:realoptimalSNR}
 \rho\simeq \Olap(h^{\tr},h_b)\leftn h^{\tr}\rightn = \Olap(h^{\tr},h_b)\,\rho_{\mathrm{opt}},
\end{equation}
where $h_b$ is the filter template in the $b$ank (subscript $b$) that has the
highest overlap with the signal $h^{\tr}$.
The furthest distance to which GW signals can be detected is proportional to 
the matched-filter SNR that the search algorithm finds the signal with. 
Note that $0\leq\Olap(h^{\tr},h_b)\leq 1$, so the recovered SNR
$\rho\leq \rho_{\mathrm{opt}}$ (c.f. Eq.~(\ref{eq:realoptimalSNR})). 
For a BBH population uniformly distributed in spacial volume, the 
detection rate would decrease as $\Olap(h^{\tr},h_b)^3$. Searches that aim at
restricting the loss in the detection rate strictly below 
$10\%\,(\mathrm{or}\, 15\%)$, would require a bank of template waveforms that
have $\Olap$ above $0.965\,(\mathrm{or}\, 0.947)$ with \textit{any} incoming
signal~\citep{WaveformAccuracy2008,WaveformAccuracy2010}.


Any template bank has two sources for loss in SNR: 
(i) the discreteness of the bank grid in the physical parameter space of the 
BBHs, and, (ii) the disagreement between the actual GW signal $h^{\tr}$ and the 
modeled template waveforms used as filters. We de-coupled these to estimate
the SNR loss. Signal waveforms are denoted as $h^\tr_x$ in what follows, 
where the superscript $\tr$ indicates a $tr$ue signal, and the subscript
$x$ indicates the mass parameters of the corresponding binary. Template
waveforms are denoted as $h^\M_b$, where $\M$ denotes the waveform $m$odel, and
$b$ indicates that it is a member of the discrete \textit{b}ank.
% To put a bound on the 
% loss in SNR due to the first, we compute the \textit{minimal match} $\MM$ of 
% the bank,
% \begin{equation}
% \MM_{\mathcal{B}} = \underset{a}{\textrm{min}}\,\underset{g\, \in\, 
% \mathrm{bank}}{\textrm{max}}\,\Olap(h^{\M}_a,h^{\M}_g),
% \end{equation}
% \red{[Harald:  Define ``a'' in this equation and for use below]}
% which gives the highest fractional loss in SNR over the entire region of 
% physical parameters $\mathcal{B}$ that the bank covers. As both the signal 
% and the template 
% are modeled with the same approximant, $\MM$ measures the loss due to the 
% discreteness of the bank grid alone. The combined fractional SNR loss at any
% point $a$ in the parameter space can be measured by computing the 
% \textit{fitting factor} $\FF$ of the bank~\cite{FittingFactorApostolatos},
% \begin{equation}\label{eq:defFF}
% \FF_{\M}(a) = \underset{g\, \in\, 
% \mathrm{bank}}{\textrm{max}}\,\Olap(h^{\tr}_a,h^{\M}_g).
% \end{equation}
% which measures the same in a small neighborhood of $a$. A detection
% search that aims at strictly less than $10\%\,(15\%)$ loss in the observable  volume of the universe (module the effect of the cosmological redshift),
% requires a bank of template waveforms that has $\FF$ above $0.965\,
% (0.947)$~\citep{WaveformAccuracy2008,WaveformAccuracy2010,CompTemplates2009}
% over the entire parameter space.
\begin{figure}
 \centering
\includegraphics[width=\columnwidth]{Eff1v2.png}%\quad
\caption{We show the \textit{true} (upper) and the \textit{hybrid} (lower) 
waveform manifolds here, with the signal residing in the former, and a discrete
bank of templates placed along lines of constant mass-ratio in the latter. 
Both manifolds are embedded in the same space of all possible waveforms.
The true signal waveform is denoted as $h^{\tr}_x$, while the templates in the
bank are labelled $h^{\M}_b$. The hybrid waveform that matches the signal $H^{\tr}_x$
best is shown as $h^{\M}_\perp$. Also shown is the ``distance'' between
the signal and the hybrid template that has the highest overlap with it.
This figure is qualitatively similar to Fig.~3 of
Ref.~\cite{WaveformAccuracy2008}.}
\label{fig:EFFdiag1}
\end{figure}
% \begin{figure}
%  \centering
% \includegraphics[width=0.8\columnwidth]{Eff2.png}%\quad
% \caption{We show the \textit{true} and the \textit{hybrid} waveform manifolds 
% relative to it. We assume that the hybrid waveform manifolds envelope 
% \textit{true} one, and so the shortest distance between a point on the true 
% manifold and any hybrid manifold would be lesser than the same between the two 
% mutually farthest hybrid manifolds; as for Eq.~(\ref{eq:hybridMMTn}).}
% \label{fig:EFFdiag2}
% \end{figure}
Fig.~\ref{fig:EFFdiag1} shows the signal $h^\tr_x$ in its manifold, and the
bank of templates $h^\M_b$ residing in the model waveform manifold, both being
embedded in the same space of all possible waveforms. The 
point $h^\M_\perp$ is the waveform which has the smallest mismatch
in the entire (continuous) model manifold with $h^\tr_x$, i.e.
$h^\M_\perp :\mathcal{M}(h^\tr_x,h^\M_\perp) = \underset{y}{\mn}\,\,\mathcal{M}(h^\tr_x,h^\M_y)$.
The fraction of the optimal SNR recovered at different points $x$ in the
binary mass space can be quantified by measuring the fitting factor $\FF$ of
the bank~\cite{FittingFactorApostolatos},
\begin{equation}\label{eq:ffmismatch}
 \FF(x) = 1 - \underset{b}{\mn}\,\,\mathcal{M}(h^\tr_x, h^\M_b).
\end{equation}
For two waveforms $h_1$ and $h_2$ close to each other in the
waveform manifold: $\leftn h_1\rightn \simeq\leftn h_2\rightn$, and mutually
aligned in phase and time such that the overlap between them is maximized, 
\begin{equation}
%  \begin{align}
  \leftn h_1 - h_2\rightn^2 \simeq 2\l( h_1 |h_1\r)\left(1 - \dfrac{\l( h_1 |h_2\r)}{\sqrt{\l( h_1 |h_1\r)}\sqrt{\l( h_1 |h_1\r)}}\right).
%  &= 2\leftn h_1\rightn^2 \Mis\left(h_1,h_2\right)
% \end{align}
\end{equation}
The mismatch can, hence, be written as 
(c.f. Eq.~(\ref{eq:mismatch}))
\begin{equation}
 \Mis\left(h_1,h_2\right) = \dfrac{1}{2\leftn h_1\rightn^2}\leftn h_1 - 
h_2\rightn^2.
\end{equation}
We note that this equation is an upper bound for Eq.~(25) of
Ref.~\cite{Cannon:2012gq}. 
From this relation, and treating the space embedding the true and model 
waveform manifolds as Euclidean at the scale of template separation, we
can separate out the effects of bank coarseness and template inaccuracies as
\begin{subequations}
\begin{align}
 \FF(x) &= 1 - \underset{b}{\mn}\dfrac{1}{2\leftn h^\tr_x\rightn^2}\leftn h^\tr_x - h^\M_b\rightn^2 ,\\
 &= 1 - \Gamma_\Hyb(x) - \Gamma_\bnk(x)\label{eq:FFGammas};
 %&= 1 - \underset{b}{\mn}\dfrac{1}{2\leftn h^\tr_x\rightn^2}\leftn h^\tr_x - h^\M_\perp\rightn^2 - \underset{b}{\mn}\dfrac{1}{2\leftn h^\tr_x\rightn^2}\leftn h^\M_\perp - h^\M_b\rightn^2
 \end{align}
\end{subequations}
where 
\begin{equation}
\Gamma_\Hyb(x) \equiv \dfrac{1}{2\leftn h^\tr_x\rightn^2}\leftn h^\tr_x - h^\M_\perp\rightn^2 = \mathcal{M}(h^\tr_x,h^\M_\perp) 
\end{equation}
is the SNR loss from model waveform errors out of the manifold of true signals;
and 
\begin{equation}\label{eq:GammaBank}
\Gamma_\bnk(x) \equiv \underset{b}{\mn}\dfrac{1}{2\leftn h^\tr_x\rightn^2}\leftn h^\M_\perp - h^\M_b\rightn^2 = \underset{b}{\mn}\,\,\mathcal{M}(h^\M_\perp,h^\M_b) 
\end{equation}
is the loss in SNR from the distant spacing of templates in the bank.
The decomposition in Eq.~(\ref{eq:FFGammas}) allows for the measurement of the 
two effects separately. 
% When we use the NR simulations as templates, $\Gamma_\Hyb \simeq 0$, but 
% hybridizing them with PN inspirals will introduce waveform errors. 
NR-PN hybrids have the inspiral portion of the waveform, from PN theory, 
joined to the available late-inspiral and merger portion from NR (as described
in Sec.~\ref{s2:NRpNhybridwaveforms}). Towards the late inspiral, the PN
waveforms accumulate phase errors, contaminating the
hybrids~\cite{MacDonald:2011ne,MacDonald:2012mp}. For each hybrid, we constrain
this effect using mismatches between hybrids constructed from the same NR 
simulation and different PN models, i.e.
\begin{equation}
 \Gamma_\Hyb(x) \leq \mathcal{M}(h^\tr_x,h^\Hyb_x) \lesssim \underset{(i,j)}{\mx}\,\,\mathcal{M}(h^{\M_i}_x,h^{\M_j}_x),
\end{equation}
where  $\M_i = $ TaylorT[1,2,3,4]+NR.
However, this is only possible for a few values of mass-ratio for which NR
simulations are available. We assume $\Gamma_\Hyb$ to be a slowly and smoothly 
varying quantity over the component-mass space at the scale of template grid
separation. At any arbitrary point $x$ in the mass space we approximate 
$\Gamma_\Hyb$ with its value for the ``closest'' template, i.e.
\begin{equation}\label{eq:GammaHybfinal}
 \Gamma_\Hyb(x) \leq \underset{(i,j)}{\mx}\,\,\mathcal{M}(h^{\M_i}_x,h^{\M_j}_x) \simeq \underset{(i,j)}{\mx}\,\,\mathcal{M}(h^{\M_i}_b,h^{\M_j}_b),
\end{equation}
where $h^\M_b$ is the hybrid template in the bank with the highest overlap with 
the signal at $x$. 
% Since NR waveforms have a limited number of 
% orbits, to obtain $\Gamma_\Hyb$ for hybrids with lower matching frequencies, 
% their NR portion is replaced with EOBNRv2 waveforms. As only the PN
% portion is changing in these comparisons, and the EOBNRv2 model was calibrated 
% against most of the NR simulations that we use here~\cite{BuonannoEOBv2Main}, 
% using EOBNRv2-PN hybrids gives the same measurements as using NR-PN hybrids for
% hybrid mismatches.

The other contribution to SNR loss comes from the discrete placement of 
templates in the mass space. In Fig.~\ref{fig:EFFdiag1}, this is shown in the
manifold of the template model. As NR waveforms (or hybrids) are available
for a few values of mass-ratio, we measure this in the manifold of EOBNRv2
waveforms. The EOBNRv2 model reproduces most of the NR simulations that
were consider here well~\cite{BuonannoEOBv2Main}, allowing for this 
approximation to hold. For the same reason, we expect $h^\EOB_x$ to be close to 
$h^\EOB_\perp$, with an injective mapping between the two. This allows us to 
approximate (c.f. Eq.~(\ref{eq:GammaBank}))
\begin{eqnarray}\label{eq:GammaBankEOB}
\label{eq:Gammabnkfinal}
 \Gamma_\bnk(x) &\simeq & \underset{b}{\mn}\,\,\mathcal{M}(h^\EOB_x,h^\EOB_b).
\end{eqnarray}

In Sec.~\ref{s1:NRonlybank}, we construct template banks that use purely-NR
templates, which have negligible waveform errors. The SNR recovery from such 
banks is characterized with
\begin{equation}\label{eq:NRFFGammas}
 \FF(x) = 1 - \Gamma_\bnk(x),
\end{equation}
where the SNR loss from bank coarseness is obtained using 
Eq.~(\ref{eq:Gammabnkfinal}). In 
Sec.~\ref{s1:NRpNhybridbank},~\ref{s1:futureNRpNhybridbank}, we construct 
template banks aimed at using NR-PN hybrid templates. Their SNR recovery is
characterized using Eq.~(\ref{eq:FFGammas}), where the additional contribution
from the hybrid waveform errors are obtained using Eq.~(\ref{eq:GammaHybfinal}).


% For the NR-PN hybrid template bank, as we do not know the \textit{true} 
% waveforms at arbitrary points in the signal parameter space (the component
% masses of the binary), we estimate the $\FF$ as follows. Let us
% parametrize the mass parameter space by $\eta$ and $M$, noting the bijective
% mapping
% $\eta =q/(1+q)^2$. 
% For any point in the mass space $(\eta,M)$, we re-write the $\FF$ as
% \begin{subequations}
% \begin{align}
%  \FF(\eta,M) &\equiv 1 - 
% \underset{\eta',M'}{\mn}\Mis\left(h^{\tr}(\eta,M),h^{\M}_g(\eta',M')\right) \\
%  %
%  \simeq & 1 - \dfrac{1}{2\leftn h\rightn^2}\,\, \underset{\eta',M'}{\mn} \leftn 
% h^{\tr}(\eta,M)-h^{\M}_g(\eta',M')\rightn^2 \label{eq:FFasnormsq} \\
%  \equiv & 1 - \dfrac{1}{2\leftn h\rightn^2}\,\E(\eta,M)
% \end{align}
% \end{subequations}
% where $\leftn h\rightn\equiv \leftn h^{\tr}(\eta,M)\rightn \simeq \leftn 
% h^{\M}_g(\eta',M')\rightn$, and $\E(\eta,M)\equiv \underset{\eta',M'}{\mn} \leftn 
% h^{\tr}(\eta,M)-h^{\M}_g(\eta',M')\rightn^2$. 
% %What we want to constrain, at each point $(\eta ,M)$ in the region of 
% %component mass space that our NR-PN-hybrid bank aims to cover is
% % \begin{multline}
% %  \underset{\eta_i,M_A}{\mn} \leftn h^{\tr}(\eta,M) - 
% % h^{\M}_g(\eta_i,M_A)\rightn^2 \notag\\
% %  = 2\rho\,(1-\FF_{\Hyb}(\eta,M)) \equiv \E(\eta,M),
% % \end{multline}
% %where $\rho$ in the expected matched-filter signal-to-noise-ratio (SNR) for the 
% %signal with source parameters $(\eta, M)$.
% As in Ref.~\cite{WaveformAccuracy2008,WaveformAccuracy2010}, we visualize the
% waveforms in their manifold in Fig.~\ref{fig:EFFdiag1}, where the top manifold
% is that of the \textit{true} waveforms, and the bottom is the model waveform 
% manifold; while the parallel lines are contours of constant mass-ratio with 
% total mass $M$ increasing from left to right (clockwise). The NR and the NR-PN
% hybrid bank grids are restricted to occupy these contours as NR simulations are 
% only available for a discrete set of mass-ratios, but can be rescaled to 
% different values of total mass. The quantity $\E$ is depicted in the figure as
% the ``distance'' between the \textit{true} signal and the \textit{closest}
% template in the bank of hybrid waveforms (squared). We drop a \red{[Word missing here?]} perpendicular 
% from the signal at $(\eta, M)$ in the \textit{true} manifold to the hybrid 
% waveform manifold at $(\eta_{\perp}'', M_{\perp}'')$, i.e.
% \begin{multline}
% (\eta_{\perp}'', M_{\perp}''):\,\leftn h^{\tr}(\eta,M) - h^{\M}(\eta_{\perp}'', M_{\perp}'')\rightn \notag\\
%  = \underset{\eta'',M''}{\mn} \leftn h^{\tr}(\eta,M) - h^{\M}(\eta'', 
% M'')\rightn.
% \end{multline}
% As waveforms are smooth functions of the masses, their manifolds are
% expected to be smooth and the mapping $T:(\eta,M)\rightarrow (\eta_{\perp}'', 
% M_{\perp}'')$ to be unique. Assuming both true and approximant waveform 
% manifolds are in a locally Euclidean space (over ``distances''
% at the scale of template separation) we can write $\E(\eta,M)$ as
% \begin{subequations}
% \begin{align}
%  \E(\eta,M) &= \underset{\eta'',M''}{\mn} \leftn h^{\tr}(\eta,M) - h^{\M}(\eta'', M'')\rightn^2 \notag\\
%  & + \underset{\eta',M'}{\mn} \leftn h^{\M}(\eta_{\perp}'', M_{\perp}'') - 
% h^{\M}_g(\eta', M')\rightn^2 \label{eq:MMaddQuad1}\\
%  &\equiv \Delta_1 + \Delta_2(\eta_{\perp}'', M_{\perp}''),
% \end{align}
% \end{subequations}
% where $\Delta_1 =\Delta_1(\eta,M) =\Delta_1\left(T^{-1}(\eta_{\perp}'', M_{\perp}'')\right)$ 
% and $\Delta_2(\eta_{\perp}'', M_{\perp}'')$ are 
% abbreviations for the first and second terms in Eq.~\ref{eq:MMaddQuad1}, 
% respectively. 
% 
% \red{[Harald:  The description here seems overly complicated.  Please reconsider whether all symbols introduced are really needed.  Also reconsider ordering to make the flow of ideas as smooth and uninterrupted as possible.  Specifically: (i) Is a symbol needed for the mapping $T$, or is it sufficient to define $\eta_\perp$ and $M_\perp$?  (ii) Do the symbols $\Delta_1$ and $\Delta_2$ need to have parameters attached to it inside parentheses?  If so, they are defined as functions, so define what the functions are.  I see $\Delta_1(...)$ used below {\em once} with different arguments.  If this is the only use, then it's easier to just spell out that one use, rather than to define a function (iii) if $\Delta_1$ needs to remain a function, be consistent with how you write it.  Either always with arguments or never.]}
% 
% The first of these
% \begin{equation}
%   \Delta_1(\eta,M)\leq \underset{M''}{\mn} \leftn h^{\tr}(\eta,M) - h^{\M}(\eta, M'')\rightn^2,
% \end{equation}
% where the RHS of this inequality is the quantity that we put a bound on to
% estimate the waveform errors for the NR-PN hybrids~\cite{MacDonald:2012mp}. 
% The hybridization of NR 
% simulations with different PN waveforms gives us different hybrids, all of which 
% reside in their own manifolds, as shown in Fig.~\ref{fig:EFFdiag2}. If we assume 
% that the \textit{true} manifold is enveloped between the different hybrid 
% manifolds, we can conservatively estimate
% \begin{equation}\label{eq:hybridMMTn}
%  \begin{split}
%   \Delta_1 & \leq \underset{M''}{\mn} \leftn h^{\tr}(\eta,M) - h^{\M}(\eta, M'')\rightn^2 \\
%   &\lesssim \underset{(i,j)}{\max}\, \underset{M''}{\mn} \leftn h^{\M_i}(\eta,M) - h^{\M_j}(\eta, M'')\rightn^2,
%  \end{split}
% \end{equation}
% \red{[Harald:  I don't think we ever did the minimization over $M''$ when computing PN-hybrid errors.]}
% \red{[Harald: The text presently discusses twice how PN-hybrid errors are determined, here and in Sec IIF.  It would be good to incorporate IIF into IIE.  However, spend an entire paragraph on how they are determined, they are too complicated to do in passing in a sentence or two.]}
% 
% 
% where $(\M_i,\M_j)$ are pairs of the same NR waveform hybridized with 
% different PN approximants. As we have NR simulations for restricted values of
% mass-ratio, estimation of $\Delta_1$ at arbitrary points in the component-mass
% space is difficult. Assuming $\Delta_1$ to be a slowly varying quantity over the 
% component-mass space at the scale of template grid separation, we approximate 
% $\Delta_1 (\eta_{\perp}'', M_{\perp}'')\simeq\Delta_1(\eta_{g_c},M_{g_c})$,
% where $(\eta_{g_c},M_{g_c})$ is the closest template grid point to 
% $(\eta_{\perp}'', M_{\perp}'')$, obtained by maximizing
% $\Olap(h^{\M}(\eta_{\perp}'', M_{\perp}''),h^{\M}(\eta_{g_c},M_{g_c}))$.
% Then, the fitting factor at a point $(\eta,M)$ in the true manifold, and at
% $(\eta_{\perp}'', M_{\perp}'')$ in the model waveform manifold becomes
% %\begin{equation}
% \begin{subequations}\label{eq:FFConstraint}
%  \begin{align}
% \FF(\eta,M) &\lesssim 1 - \underset{(i,j)}{\mx}\,\underset{M''}{\mn}\,\,  \mathcal{M}\left(h^{\M_i}(\eta_{g_c},M_{g_c}), h^{\M_j}(\eta_{g_c}, M'')\right) \label{eq:effdef1}\notag\\
%  &\,\, - \underset{\eta_i,M_A}{\mn}\mathcal{M}\left(h^{\M}(\eta_{\perp}'', M_{\perp}''),h^{\M}_g(\eta_i, M_A)\right)  \\
%  &\equiv 1 - \Gamma_{\Hyb}(\eta_{\perp}'', M_{\perp}'') - \Gamma_{\bnk}(\eta_{\perp}'', M_{\perp}'') \\
%  &= 1 - \Gamma_{\Hyb}\left(T(\eta, M)\right) - \Gamma_{\bnk}\left(T(\eta, M)\right) ,
%  \end{align}
% \end{subequations}
% %\end{equation}
% where $\Gamma_{\Hyb}$ and $\Gamma_{\bnk}$ are just the last two terms on the 
% RHS of Eq.~\ref{eq:effdef1}. To construct an effectual bank, therefore, we
% restrict $\Gamma_{\Hyb} + \Gamma_{\bnk}$ below $0.035\,(0.053)$ over the region
% the NR-only or the NR-PN hybrid bank is to cover, as discussed above.
% 
% 
% \red{[Harald: I didn't manage to follow the calculation in Sec IIE
%   within the time I spent on it.  Perhaps it would be clearer to
%   reorganize the argument to arrive quickly at
% \begin{equation}
% FF(\eta, M) \lesssim 1- \Gamma_{\rm Hyb} - \Gamma_{\rm bank},
% \end{equation}
% as a sum of terms orthogonal to the model manifold, and tangential to
% the model manifold.  After this equation has been obtained, consider
% each of the $\Gamma_{\rm bank/Hyb}$ separately with its own sequence
% of inequalities.]}
% 
% 
% 
% 
% % From Eq.~\ref{eq:FFasnormsq}
% % \begin{equation}
% %  \FF(\eta,M) = 1 - \dfrac{1}{2\leftn h\rightn^2}\,\, \underset{\eta',M'}{\mn}
% %\leftn h^{\tr}(\eta,M)-h^{\M}_g(\eta',M')\rightn^2,
% % \end{equation}
% % we need to find an upper bound on
% % $\leftn h^{\tr}(\eta,M)-h^{\M}_g(\eta',M')\rightn^2$, which we can constrain
% % over the component-mass region. Let 
% % \begin{equation}
% %  \epsilon_{\Hyb}(\eta,M) \equiv \leftn h^{\tr}(\eta,M)-h^{M}(\eta,M)\rightn^2,
% % \end{equation}
% % be the error norm for the Hybrid waveform ($\M = \Hyb$ here), and
% % \begin{equation}
% %  \epsilon_{\Hyb,\eta,\M}(\eta,M) \equiv \underset{\eta'',M''}{\mn}\leftn 
% % h^{\tr}(\eta,M)-h^{M}(\eta'',M'')\rightn^2
% % % \end{equation}
% % % is the waveform error norm margenalized over the subscripted mass parameters. 
% % % Also define $(\eta'_m,M'_m)$ as the parameter values on the model manifold
% % % \begin{multline}
% % %  (\eta'_m,M'_m): \leftn h^{\tr}(\eta,M) - h^{\M}(\eta'_m,M'_m)\rightn^2 = \\
% % %  \underset{\eta'',M''}{\mn}\leftn h^{\tr}(\eta,M) - 
% % h^{\M}(\eta'',M'')\rightn^2,
% % % \end{multline}
% % % at which the line drawn from the manifold of true waveforms starting at 
% % ($\eta,M$)
% % % is orthogonal to the $\M$ manifold (which is the manifold of Hybrid waveforms 
% % in
% % % our case). We can then write
% % % \begin{subequations}
% % %  \begin{align}
% % %   &\underset{\eta',M'}{\mn}\leftn h^{\tr}(\eta,M)-h^{\M}_g(\eta',M')\rightn^2 
% % \simeq \notag\\
% % %   &\quad\quad \leftn h^{\tr}(\eta,M) - h^{\M}(\eta'_m,M'_m)\rightn^2 +\notag\\
% % %   &\quad\quad\quad\quad\underset{\eta',M'}{\mn} \leftn h^{\M}(\eta'_m,M'_m) - 
% % h^{\M}_g(\eta',M')\rightn^2 \\
% % %   %
% % %   &= \epsilon_{\Hyb,\eta,\M}(\eta,M) + \epsilon_{\mm}(\eta'_m,M'_m) \\
% % %   %
% % %   &\leq \epsilon_{\Hyb,\M}(\eta,M) + 
% % \epsilon_{\mm}(\eta'_m,M'_m)\label{eq:mmtogridmm}
% % %  \end{align}
% % % \end{subequations}
% % % where $\epsilon_{\Hyb,\M}(\eta,M)$ is the hybrid waveform error minimized 
% % over 
% % only the total mass at the point ($\eta,M$) and $\epsilon_{\mm}(\eta'_m,M'_m)$ 
% % is the loss due to 
% % % discretization of the bank grid at the point ($\eta'_m,M'_m$). Now, if we 
% % term 
% % the
% % % maximum loss due to a discrete grid $\epsilon_g$ (and call it \textit{grid} 
% % maximal
% % % mismatch), i.e.
% % % \begin{align}
% % %  \epsilon_g &\equiv 
% % \underset{\eta'_m,M'_m}{\mx}\epsilon_{\mm}(\eta'_m,M'_m)\notag\\
% % %  &\equiv \underset{\eta'_m,M'_m}{\mx}\underset{\eta',M'}{\mn} \leftn 
% % h^{\M}(\eta'_m,M'_m) - h^{\M}_g(\eta',M')\rightn^2,
% % % \end{align}
% % % we have, from Eq.~\ref{eq:mmtogridmm},
% % % \begin{equation}
% % %  \underset{\eta',M'}{\mn}\leftn h^{\tr}(\eta,M)-h^{\M}_g(\eta',M')\rightn^2 
% % \leq  \epsilon_{\Hyb,\M}(\eta,M) + \epsilon_g.
% % % \end{equation}
% % % \textit{If we assume that $\epsilon_{\Hyb,\M}(\eta,M)$ is a slowly varying 
% % function of the 
% % % mass-parameters compared to the grid density}, 
% % % i.e. $\epsilon_{\Hyb,\M}(\eta,M)\simeq \epsilon_{\Hyb,\M}(\eta',M')$,
% % % where $(\eta',M')$ are the parameters values of the closest point on the grid 
% % to
% % % ($\eta,M$), we get
% % % \begin{subequations}
% % % \begin{align}\label{eq:Finmms}
% % %  \underset{\eta',M'}{\mn}\leftn h^{\tr}(\eta,M)-h^{\M}_g(\eta',M')\rightn^2 
% % \leq  \epsilon_{\Hyb,\M}(\eta',M') + \epsilon_g \\ \label{eq:FinFFs}
% % %  \Rightarrow \FF(\eta,M) \geq 1 - \dfrac{1}{2\leftn 
% % h\rightn^2}\,\,\left(\epsilon_{\Hyb,\M}(\eta',M') + \epsilon_g\right)
% % % \end{align}
% % % \end{subequations}
% % % If we construct a bank, with 
% % % \begin{equation}
% % % \boxed{ \dfrac{1}{2\leftn h\rightn^2}\,\,\left(\epsilon_{\Hyb,\M}(\eta',M') + 
% % \epsilon_g\right) \leq 0.035;\,\,\forall\,(\eta',M')\in\mathrm{bank}},
% % % \end{equation}
% % % it should restrict the detection rate loss to below $\sim 10\%$. In other 
% % words,
% % % the tolerances we have to construct the NR-PN hybrid bank is
% % % \begin{equation}
% % %  \boxed{\MM_{\Hyb,\M} + \MM_g \leq 0.035}
% % % \end{equation}
% % % where $\MM_{\Hyb,\M}\equiv \underset{\eta',M'}{\mx}\left(\dfrac{1}{2\leftn 
% % h\rightn^2}\epsilon_{\Hyb,\M}(\eta',M')\right)$ is the hybrid waveform 
% % mismatch, 
% % % minimized over total-mass for each point in the bank, and subsequently 
% % maximized
% % % over the entire bank; and $\MM_g\equiv\dfrac{1}{2\leftn 
% % h\rightn^2}\,\epsilon_g$
% % % is the discretization mismatch measured using the same model as signal and 
% % template.
% 
% %\red{\em Describe how we derive an error-bound on a PN-NR hybrid waveform.}
% 
% The largest source of error in hybrid gravitational waveforms is
% caused by the higher-order unknown terms in the PN component. These
% cause different PN approximants to diverge as the binary black holes approach
% merger. In order to ascertain what this error is, we compare many
% types of PN waveforms and find the maximum mismatch between hybrids
% which use different PN approximants, in this case, Taylor T1, T2, T3,
% and T4. We assume this to be close to the actual PN error, as discussed in 
% Sec.~\ref{s2:quantifyingerrors}.
% 
% More specifically, we take four hybrid waveforms $h_\text{Tn}$, where $n = [1,2,3,4]$, and find their mismatches as defined in Eq.~\ref{eq:mismatch}:
% \begin{equation}
% \mathcal{M}_\text{max}(\eta,M) = \underset{(i,j)}{\mx}\,\underset{M''}{\mn}\,\,  \mathcal{M}\left(h^{\mathrm{T}i}(\eta,M), h^{\mathrm{T}j}(\eta,M'')\right) 
% \end{equation}
% $\mathcal{M}_\text{max}$ is what we call the hybridization error.
% 
% Because NR waveforms have a limited number of orbits, to obtain results for hybrids with lower matching frequencies, we replace the NR part of the hybrids with EOBNRv2 waveforms as described in Sec.~\ref{s2:EOBwaveforms}. Since only the PN waveforms are changing in these hybrid comparisons, using EOB hybrids gives the same results as using NR hybrids.


%%%%%%%%%%%%%%%%%%%%%%%%%%%%%%%%%%%%%%%%%%%%%%%%%%%%%%%%%%%%%%%%

\subsection{post-Newtonian uncertainties in hybrid waveforms}\label{s2:pNuncertainties}

%\red{\em Describe how we derive an error-bound on a PN-NR hybrid waveform.}

The largest source of error in hybrid gravitational waveforms is
caused by the higher-order unknown terms in the PN component. These
cause different PN approximants to diverge as the binary black holes approach
merger. In order to ascertain what this error is, we compare many
types of PN waveforms and find the maximum mismatch between hybrids
which use different PN approximants, in this case, Taylor T1, T2, T3,
and T4. We assume this to be close to the actual PN error, as discussed in 
Sec.~\ref{s2:quantifyingerrors}.

More specifically, we take four hybrid waveforms $h_\text{Tn}$, where $n = [1,2,3,4]$, and find their mismatches as defined in Eq.~\ref{eq:mismatch}:
\begin{equation}
\mathcal{M}_\text{max}(\eta,M) = \underset{(i,j)}{\mx}\,\underset{M''}{\mn}\,\,  \mathcal{M}\left(h^{\mathrm{T}i}(\eta,M), h^{\mathrm{T}j}(\eta,M'')\right) 
\end{equation}
$\mathcal{M}_\text{max}$ is what we call the hybridization error.

Because NR waveforms have a limited number of orbits, to obtain results for hybrids with lower matching frequencies, we replace the NR part of the hybrids with EOBNRv2 waveforms as described in Sec.~\ref{s2:EOBwaveforms}. Since only the PN waveforms are changing in these hybrid comparisons, using EOB hybrids gives the same results as using NR hybrids.


%%%%%%%%%%%%%%%%%%%%%%%%%%%%%%%%%%%%%%%%%%%%%%%%%%%%%%%%%%%%%%%%

\section{Results}\label{s1:results}

\subsection{NR-only template bank}\label{s2:NRonlybank}
\input{nrbankresults.tex}

\subsection{NR-pN hybrid template bank}\label{s2:NRpNhybridbank}

%\textcolor{red}{Why do we need a bank of hybrids?}\newline
The template bank contructed in Sec.~\ref{s1:NRonlybank} is effectual for 
GW detection searches focussed at relatively massive binaries with 
$\mathcal{M}_c \gtrsim 27M_\odot$. As the NR waveforms are restricted to a
small number of orbits, it is useful to consider NR-PN hybrids to bring the
lower mass limit down on the template bank. PN waveforms can be generated
for an arbitrarily large number of inspiral orbits, reasonably accurately and
relatively cheaply. Thus, a hybrid waveform comprised of a long PN early-inspiral 
and an NR late-inspiral, merger, and ringdown could also be arbitrarily long. 
There are, however, uncertainties in the PN waveforms, due to
the unknown higher-order terms. During the late-inspiral and merger phase,
these terms become more important and the PN description becomes
less accurate. In addition, when more of the late-inspiral is
in the detector's sensitivity frequency range, hybrid waveform mismatches 
due to the PN errors become increasingly large, and reduce the
recovered SNR. Thus, when
hybridizing PN and NR waveforms, there must be enough NR orbits that the PN
error is sufficiently low for the considered detector noise-curve. In 
this section, we construct an NR + PN hybrid template bank, for currently
available NR waveforms, and determine the lowest value of binary masses to 
which it covers.


%\subsubsection{Including hybrid PN error}\label{s2:HybridPNerror}
\begin{figure}
\includegraphics[width=0.9\columnwidth, trim=20 17 75 75]{maxmismatchVSmass_allq.pdf}
\caption{\label{fig:Current-NR-PN-Errors}Bounds on mismatches of PN-NR
  hybrid waveforms, for the currently existing NR simulations. The PN
  error is for hybrids matched at $M\omega_m=0.025$ for $q=1$,
  $M\omega_m=0.038$ for $q=2$, and $M\omega_m=0.042$ for
  $q=3,4,6,8$. The black circles indicate the lower bound of the
  template bank in Table~\ref{table:etalist4}. The black square show the
  lower bound with a hybrid error of 1.5\%. The inset shows these
  lower bounds as a function of mass ratio.} 
\end{figure}

%\textcolor{red}{Details of hybrids and how we estimate their errors?}\newline
The hybrids we use are constructed by joining the PN and NR portions, as 
described in Sec.~\ref{s2:NRpNhybridwaveforms}. The number of orbits before 
merger at which they are joined depends on the length of the available NR 
waveforms. We estimate the PN waveform errors using hybridization
mismatches $\Gamma_\Hyb$, as discussed in Sec.~\ref{s1:quantifyingerrors}. 
Fig.~\ref{fig:Current-NR-PN-Errors} shows the same for all the hybrids, as a 
function of total mass. In terms of orbital frequency, these are
matched at $M\omega_m=0.025$ for $q=1$, $M\omega_m=0.038$ for $q=2$,
and $M\omega_m=0.042$ for $q=3,4,6,8$. In terms of number of orbits
before merger, this is 26.9 orbits for $q=1$, 13.6 orbits for $q=2$,
12.6 orbits for $q=3$, 14.3 orbits for $q=4$, 17.8 orbits for $q=6$,
and 21.4 orbits for $q=8$. The dotted line indicates a mismatch of
$1.5\%$, a comparatively tight bound that leaves flexibility to accommodate
errors due to template bank discreteness. The black circles show the hybrid
mismatches at the lower mass bound of the NR-only template bank in
Table~\ref{table:etalist4}, which are negligible. The inset shows this minimum 
mass as a function of mass ratio, as well as the minimum attainable mass if we
accept a hybrid error of $1.5\%$. At lower masses, the mismatches increase
sharply with more of the PN part moving into the Advanced LIGO sensitivity band.
This is due to the nature of the frequency dependence of the detector 
sensitivity. The detectors will be most sensitive in a comparatively
narrow frequency band. As the hybridization frequency sweeps through this band, 
the hybrid errors rise sharply. They fall again at the lowest masses, for which
mostly the PN portion stays within the sensitive band.


\begin{figure*}
\begin{center}
\includegraphics[width=\columnwidth]{bank001_lowM_01_stochastic_mtot200_logMq_NOhybMM-tiny.png}
\includegraphics[width=\columnwidth]{bank001_lowM_01_stochastic_mtot200_logMq_hybMM-tiny.png}
\caption{\label{fig:Current-hybrids-stochastic-FF}These figures show fitting
  factors $\FF$ obtained when using a discrete mass-ratio template bank for
  $q=1,2,3,4,6,8$. For each mass-ratio, the templates are extended down 
  to a total mass where the NR-PN hybridization mismatch becomes
  $3\%$. The bank is placed using the stochastic algorithm, similar to 
  Ref.~\cite{Harry:2009ea,Ajith:2012mn,Manca:2009xw}. 
  The black dots show the location
  of the templates. The fitting factor on the left plot does 
  {\em not} take into account the hybridization error, and therefore shows the
  effect of the sparse placement of the templates alone. 
  The right plot accounts for the hybridization error
  and gives the actual fraction of the optimal SNR that would be recovered
  with this bank of NR-PN hybrid templates. The region bounded by the magenta 
  (solid) line in both plots indicates the lower end of the coverage of the 
  bank of un-hybridized NR waveforms. Lastly, the shaded grey dots show the 
  points where the fitting factor was below $96.5\%$.}
\end{center}
\end{figure*}
\begin{figure*}
\begin{center}
\includegraphics[width=\columnwidth]{bank_26022013_02_mtot200_logMq_NOhybMM-tiny.png}
\includegraphics[width=\columnwidth]{bank_26022013_02_mtot200_logMq_hybMM-tiny.png}
\caption{\label{fig:Current-hybrids-FF}These figures are similar to 
  Fig.~\ref{fig:Current-hybrids-stochastic-FF}. The figures show fitting
  factors $\FF$ obtained when using a discrete mass-ratio template bank for
  $q=1,2,3,4,6,8$. For each mass-ratio, the templates are extended down 
  to a total mass where the NR-PN hybridization mismatch becomes
  $3\%$. Templates are placed independently for each mass-ratio, and span the 
  full range of total masses. For each mass-ratio, neighboring templates are 
  required to have an overlap of $97\%$. The union of the six single-$q$ 
  one-dimensional banks is taken as the final bank. The black dots show the 
  location of the templates. The fitting factor on the left plot does 
  {\em not} take into account the hybridization error, and therefore shows the
  effect of the sparse placement of the templates alone. The right plot accounts
  for the hybridization error
  and gives the actual fraction of the optimal SNR that would be recovered
  with this bank of NR-PN hybrid templates. The region bounded by the magenta 
  (solid) line in both plots indicates the lower end of the coverage of the 
  bank of un-hybridized NR waveforms. Lastly, the shaded grey dots show the 
  points where the fitting factor was below $96.5\%$.}
\end{center}
\end{figure*}

\begin{figure*}
\begin{center}
\includegraphics[width=\columnwidth]{bank_26022013_02_hybrids_mtot200_logMq_NOhybMM-tiny.png}
\includegraphics[width=\columnwidth]{bank_26022013_02_hybrids_mtot200_logMq_hybMM-tiny.png}
\caption{\label{fig:Current-real-hybrids-FF}This figure is similar to 
  Fig.~\ref{fig:Current-hybrids-FF}. The figures show fitting
  factors $\FF$ obtained when using a discrete mass-ratio template bank for
  $q=1,2,3,4,6,8$. Templates are placed independently for each mass-ratio, and 
  span the range of total masses, down to the region where the hybrid errors
  become $3\%$. For each mass-ratio, neighboring templates are 
  required to have an overlap of $97\%$. The union of the six single-$q$ 
  one-dimensional banks is taken as the final bank. The black dots show the 
  location of the templates. The GW signals are modeled using the EOBNRv2
  approximant~\cite{BuonannoEOBv2Main}, while TaylorT4+NR hybrids are used as
  templates. The fitting factor on the left plot shows the combined effect of 
  the sparse placement of the templates, and the (relatively small) 
  disagreement between the hybrid and EOBNRv2 waveforms. The right plot
  explicitly accounts for the hybridization error and gives the (conservative)
  actual fraction of the optimal SNR that would be recovered
  with this bank of NR-PN hybrid templates. The region bounded by the magenta 
  (solid) line in both plots indicates the lower end of the coverage of the 
  bank of un-hybridized NR waveforms. Lastly, the shaded grey dots show the 
  points where the fitting factor was below $96.5\%$.  }
\end{center}
\end{figure*}

%\textcolor{red}{How do we construct the stochastic bank, and how well does it perform?}\newline
We now consider template banks viable for hybrids constructed from currently 
available NR waveforms at mass ratios $q=1,2,3,4,6,8$. The lower mass limit,
in this case, is extended down to masses where the hybridization error 
exceeds $3\%$. We demonstrate two independent methods of laying
down the bank grid. First, we use the stochastic placement method that proceeds
as described in Sec.~\ref{s1:NRonlybank}. The templates are sampled over the
total mass - mass-rato $(M,q)$ coordinates, sampling $q$ from the restricted
set. The total mass $M$ is sampled from the continuous interval between the 
lower mass limit, which is different for each $q$, and the upper limit of
$200M_\odot$. To assess the SNR loss from the sparse placement
of the templates, we simulate a population of $100,000$ BBH signal waveforms,
with masses sampled with $3M_\odot\leq m_{1,2}\leq 200M_\odot$ and 
$M\leq 200M_\odot$, and filter them through the bank. This portion of the SNR
loss needs to be measured with both signals and templates in the same waveform
manifold. We use the EOBNRv2 approximant~\cite{BuonannoEOBv2Main} to model both, 
as it has been calibrated to most of the NR waveforms we consider here, and it
allows us to model waveforms for arbitrary systems. The left panel of
Fig.~\ref{fig:Current-hybrids-stochastic-FF} shows the
fraction of the optimal SNR that the bank recovers, accounting for its 
discreteness alone. We observe that, with just six mass-ratios, the bank 
can be extended to much lower masses before it is limited by the restricted
sampling of mass-ratios for the templates. For binaries with both black-holes 
more massive than $\sim 12M_\odot$, the spacing between mass-ratios was found 
to be sufficiently dense. The total SNR loss, after subtracting out 
the hybrid mismatches from Fig.~\ref{fig:Current-NR-PN-Errors}, are shown in 
the right panel of Fig.~\ref{fig:Current-hybrids-stochastic-FF}. 
At the lowest masses, the coverage shrinks between the lines of constant $q$ 
over which the templates are placed, due to the hybrid errors increasing 
sharply. We conclude that this bank is viable for hybrid templates for GW 
searches for BBHs with $m_{1,2}\geq 12M_\odot$, $1\leq q\leq 10$, and 
$M\leq 200M_\odot$. Over this region the bank will recover more than $96.5\%$
of the optimal SNR. This is a significant increase over the coverage allowed 
for with the purely-NR bank, the region of coverage of which is shown in the 
right panel of Fig.~\ref{fig:Current-hybrids-FF}, bounded at lowest masses by 
the magenta (solid) curve.


%\textcolor{red}{What is the second method, why do we use it, and how well does it compare?}\newline
Second, we demonstrate a non-stochastic algorithm of bank placement, with 
comparable results. We first construct six independent bank grids, each
restricted to one of the mass-ratios $q=1,2,3,4,6,8$, and spanning the full 
range of total masses. The template with the lowest total mass is chosen 
by requiring the hybrid mismatch to be $3\%$ at that point.
The spacing between neighboring templates is
given by requiring that the overlap between them be $97\%$. 
We take the union of these banks as the final 
two-dimensional bank. As before, we measure the SNR loss due to discreteness of
the bank and the waveform errors in the templates separately. To estimate the
former, we simulate a population of $100,000$ BBH systems, and filter
them through the bank. The signals and the templates are both modeled
with the EOBNRv2 model. The left panel of Fig.~\ref{fig:Current-hybrids-FF}
reveals the fraction of SNR recovered over the mass space, accounting for the 
sparsity of the bank alone, i.e. $1-\Gamma_\mathrm{bank}$. At lower masses, 
we again start to see gaps between the lines of constant mass ratio which
become significant at $m_{1,2} \leq 12M_\odot$. 
The right panel of Fig.~\ref{fig:Current-hybrids-FF} shows the final fraction 
of the optimal SNR recovered, i.e. the $\FF$ as defined in Eq.~(\ref{eq:FFGammas}). 
As before, these are computed by subtracting out the hybrid mismatches 
$\Gamma_\Hyb$ in addition to the discrete mismatches, as described in
Sec.~\ref{s1:quantifyingerrors}. 

The efficacy of both methods of template bank construction
can be compared from Fig.~\ref{fig:Current-hybrids-stochastic-FF} and 
Fig.~\ref{fig:Current-hybrids-FF}. We observe that the final banks from either
of the algorithms have very similar SNR recovery, and are both effectual over
the range of masses we consider here. Both were also found to give a very 
similar number of templates. The uniform-in-overlap method yields a grid 
with $2,325$ templates. The stochastic bank, on the other hand, was placed with a
requirement of $98\%$ minimal mismatch, and had $2,457$ templates. 
This however includes templates with $m_{1,2} < 12M_\odot$. Restricted
to provide coverage over the region with $m_{1,2}\geq 12M_\odot$, $1\leq q\leq 10$,
and $M\leq 200M_\odot$, the two methods yield banks with $627$ and $667$
templates respectively. The size of these banks is comparable to one 
constructed using the second-order post-Newtonian TaylorF2 hexagonal 
template placement method~\cite{SathyaBankPlacementTauN,BabaketalBankPlacement,
SathyaMetric2PN,Cokelaer:2007kx}, 
which yields a grid of $522$ and $736$ templates, for a minimal
match of $97\%$ and $98\%$, respectively.

%\textcolor{red}{How do we show the robustness of the banks using NR hybrids?}\newline
Finally, we test the robustness of these results using TaylorT4+NR hybrids 
as templates. As before, we simulate a population of $100,000$ BBH signal 
waveforms. As we do not have hybrids for arbitary binary
masses, we model the signals as EOBNRv2 waveforms. This population is filtered
against a bank of hybrid templates. The SNR recovered is shown in the left 
panel of Fig.~\ref{fig:Current-real-hybrids-FF}. Comparing with the left panels 
of Fig.~\ref{fig:Current-hybrids-stochastic-FF},~\ref{fig:Current-hybrids-FF}, 
we find that the EOBNRv2 manifold is a reasonable approximation for the hybrid 
manifold; and that, at lower masses, there is a small systematic bias in the 
hybrids towards EOBNRv2 signals with slightly higher mass-ratios.
The right panel of Fig.~\ref{fig:Current-real-hybrids-FF} 
shows the fraction of optimal SNR recovered after subtracting out the hybrid
mismatches from the left panel. The similarity of the $\FF$ distribution between 
the right panels of Fig.~\ref{fig:Current-real-hybrids-FF} and 
Fig.~\ref{fig:Current-hybrids-stochastic-FF},~\ref{fig:Current-hybrids-FF}
is remarkable. This gives us confidence that the EOBNRv2 model is a good 
approximation for testing NR/hybrid template banks, as we do in this paper; 
and that a template bank of NR+PN hybrids is indeed effectual for binaries 
with $m_{1,2}\geq 12M_\odot$, $M\leq 200M_\odot$ and $q\leq 10$.




\subsection{Complete PN-NR hybrid bank for non-spinning BBH}\label{s2:futureNRpNhybridbank}
\input{futurehybridbankresults.tex}

%\FloatBarrier
\section{Conclusions}\label{s1:conclusions}
%%%%%%%%%%%%%%%%%%%%%%%%%%%%%%%%%%%%%%%%%%%%%%%%%%%%%%%%%%%%%%%%%%%%%%%%%%%%%%%
%%% Describe the wondrous conclusions of the thesis.
%%%%%%%%%%%%%%%%%%%%%%%%%%%%%%%%%%%%%%%%%%%%%%%%%%%%%%%%%%%%%%%%%%%%%%%%%%%%%%%


The first observation runs of Advanced LIGO and Advanced Virgo detectors 
are scheduled for $2015$. By $2018$, these detectors will reach 
their design sensitivity. These second-generation terrestrial detectors
will be able to see up to $10$ times further out in the universe 
than their earlier counterparts. For a compact binary population
uniformly distributed in co-moving volume, this translates to 
a thousandfold increase in the expected detection rate.
% 
Gravitational wave searches make use of theoretical knowledge of
binary dynamics and employ modeled waveforms as filter templates.
With the increase in sensitivity, the resolution of the detectors 
for small errors in modeled waveforms also increases. In this dissertation,
we primarily focus on selecting and developing optimal waveform filters
for Advanced LIGO searches. We also validate gravitational-wave 
search algorithms using accurate numerically simulated signals injected 
into emulated detector noise.

Past binary black hole searches have used post-Newtonian (pN) and 
Effective-One-Body (EOB) waveforms as filters. While the pN waveforms are 
computationally inexpensive, they are restricted to the inspiral
regime of binary coalescence. EOB waveforms include the complete
coalescence process through inspiral, merger and ringdown, and also
the sub-dominant waveform harmonics. However, they are also 
computationally more expensive. For low mass binary black holes 
($m_1,m_2\leq 25M_\odot$),
we explore the region of the parameter space over which pN waveform
templates are sufficiently accurate, in the sense of being able to 
recover more than $97\%$ of the optimal signal-to-noise ratio, 
and where in the parameter space would searches need EOB
waveform templates.
% 
% For binaries with masses $m_1,m_2\leq 25M_\odot$, we compare the 
% inspiral-only post-Newtonian waveforms with the recently proposed
% Effective-One-Body (EOB) model~\cite{BuonannoEOBv2Main}. 
% As this EOB model is calibrated
% against high-accuracy numerical simulations of non-spinning binary 
% black holes, it is demonstrably accurate for {\it comparable}
% mass-ratio binaries. However, it is computationally more expensive
% than the post-Newtonian approximants. 
% We investigate the region of the parameter
% space of non-spinnning binaries where the accuracy of post-Newtonian
% approximants is sufficient and we can win with computational cost, as
% well as the region where EOB waveforms would be required. 
Here we approximate the waveforms with their dominant multipoles. Next,
we study the impact of ignoring sub-dominant waveform multipoles in 
searches. We find that including sub-dominant harmonics could increase
the reach of aLIGO and Virgo for binaries which have their orbital
angular momentum highly inclined to the line of sight connecting them
to the detector.

Numerical Relativity (NR) has seen recent breakthroughs and rapid progress
in simulating the merger of orbiting black holes. These are the most 
accurate solutions to Einstein's field equations available. Still, 
due to their computational cost, numerical relativity simulations 
span only the last stages of the binary inspiral, alongwith the merger
and ringdown. It is possible to join these short but accurate strong-field
simulations with post-Newtonian waveforms that cover the slow-motion
regime, to construct pN-NR {\it hybrids}. We demonstrate that, within 
the limits of current NR technology, it is possible and viable to use 
hybrid waveforms in gravitational wave searches. In addition, we show
that hybrid waveforms can cover the entire region of the binary black
hole parameter space where pN waveforms are insufficient for Advanced 
LIGO searches.


Apart from having applications as search templates, and in enhancing the
accuracy of waveform models, NR simulations
can be used to validate gravitational-wave search algorithms.
We do precisely this within the purview of the NINJA-2 project. 
Several numerical relativity groups contributed
post-Newtonian-hybridized simulations to the project. These were subsequently
injected in emulated advanced detector noise. We demonstrate the ability of
existing search algorithms to successfully {\it detect} these simulations
embedded within realistic noise. This is different from the NINJA-1 project
on a few counts, one of them being the nature of the emulated noise. In the 
NINJA-2 project, initial LIGO data with its non-Gaussian transient noise was
recolored to the expected sensitivity of the Advanced LIGO-Virgo detectors, as
opposed to colored Gaussian noise that was used in NINJA-1. 
Therefore this project provided a more robust test of our search methods, and 
provided a benchmark against which future search developments could be compared.


While the above concerns primarily comparable mass-ratio binaries, we 
also develop a waveform model for intermediate mass-ratio ones with 
$m_1/m_2 \in [10, 100]$. 
Intermediate mass-ratio systems, containing intermediate mass  
and stellar mass black holes will also be relatively more massive than stellar
mass binaries.
This would shift the frequency of the emitted gravitational radiation to 
lower values, and their late-inspiral and merger would occur in the most
sensitive frequency band of the Advanced detectors. This makes the modeling 
of the later portion of their waveforms crucial to their detection. 
%
First-order conservative self-force corrections have been derived for a
test-particle moving in the background of a supermassive Schwarzschild 
black hole. Using the form of these calculations, we formulate a 
prescription to model the early and late inspiral
of such binaries. Then, using the implicit rotation source picture
(due to Baker et al~\cite{Baker:2008}), we develop a model for the plunge and merger,
where the black holes are close and the orbits are no longer quasi-circular.
We then complete the description by stitching the quasi-normal modes emitted
by the black hole formed at merger. 
Therefore, we complete a model that captures the entire coalescence process
for intermediate mass-ratio binaries of non-spinning black holes.

To summarize, for {\it comparable} mass ratio binaries, we show that a combination
of post-Newtonian and post-Newtonian--Numerical-Relativity hybrid waveforms
would be sufficient for gravitational wave searches. This is true for the 
entire stellar-mass non-spinning binary black hole parameter space.
We also successfully validate gravitational wave search algorithms 
that have been used in the most recent LIGO-Virgo searches, using accurate 
numerical simulations injected in emulated detector noise. 
For {\it intermediate} mass ratios, we develop an accurate waveform 
model that captures the binary dynamics from the weak-field slow-motion
regime to the strong-field regime up to the merger of both compact objects. 
Therefore the work presented in this dissertation is an effort towards
arriving at optimal search filters for non-spinning binary black holes 
which are prospective sources detectable by the second-generation terrestrial
gravitational wave detectors; as well as towards validating existing search 
algorithms using an improved testing methodology.














\acknowledgments
We are grateful to the SXS collaboration. 

Part of the computations for the paper
were done on the

This work was supported by 
NSERC of Canada, the Canada Chairs Program, 
and the Canadian Institute for Advanced Research.

Simulations used in this work
were computed with the \texttt{SpEC} code~\cite{spec}.  Computations
were performed on the  SUGAR cluster, which is supported by \red{NSF grant PHY-XXXXX}; the Zwicky cluster at Caltech, which is supported by
the Sherman Fairchild Foundation and by NSF award PHY-0960291; on the
NSF XSEDE network under grant TG-PHY990007N; and on the GPC
supercomputer at the SciNet HPC Consortium~\cite{scinet}. SciNet is
funded by: the Canada Foundation for Innovation under the auspices of
Compute Canada; the Government of Ontario; Ontario Research
Fund--Research Excellence; and the University of Toronto.



\FloatBarrier
\bibliographystyle{apsrev4-1}
\bibliography{paper}

\end{document}
