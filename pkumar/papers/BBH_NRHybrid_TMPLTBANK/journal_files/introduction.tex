

Upgrades to the LIGO and Virgo observatories are
underway~\cite{Harry:2010zz,aVIRGO}, with first observation runs planned for
$2015$~\cite{Aasi:2013wya}. The construction of the Japanese detector KAGRA 
has also begun~\cite{Somiya:2011np}. The advanced detectors will be
sensitive to gravitational waves at frequencies down to 
$\sim 10$Hz, with an order of magnitude increase in sensitivity across the
band. This is a significant improvement over the lower cutoff of $40$Hz
for initial LIGO. Estimates for the expected rate of detection have
been placed between $0.4 - 1000$ stellar-mass binary black hole (BBH)
mergers a year~\cite{LSCCBCRates2010}. 
The uncertainty in these estimates comes from the uncertainties in the various
factors that govern the physical processes in the BBH formation 
channels~\cite{1973NInfo..27...86T,1973NInfo..27...70T}. 
In sub-solar metallicity environments, stars (in binaries) are expected to 
lose relatively less mass to stellar winds and form more massive remnants 
\cite{Webbink:1984ti,Kowalska:2012bb,Fryer:2011cx}. 
Population synthesis studies estimate that sub-solar metallicity environments
within the horizon of advanced detectors could increase the detection rates 
to be as high
as a few thousand per year~\cite{Dominik:2012kk,Belczynski:2012cx}. 
On the other hand, high recoil momenta during core-collapse and 
merger during the common-envelope phase of the binary star evolution
could also decrease the detection 
rates drastically~\cite{Fryer:2011cx,Dominik:2012kk}. 

% The coalescence of a BBH can be divided into three phases, (i) the 
% early-inspiral, where the component objects' motion is only mildly relativistic;
% (ii) late-inspiral and merger, where the separation between the objects becomes 
% of the order of the inner-most stable circular orbits of the component 
% black-holes and their motion becomes highly relativistic; and (iii) ringdown, 
% which involves the quasi-normal ringing of the black-hole formed from the 
% merger of the binary. The slow-motion weak-field post-Newtonian (PN) 
% approximation scheme to Einstein's field 
% equations~\cite{1938AnMat..39...65E,Einstein:1940mt,einstein1949motion}  
% accurately describes the early inspiral. In PN theory, the binary
% orbital energy and energy flux are derived as Taylor expansions in the 
% characteristic velocity of the binary $v$~\cite{Cutler:1994ys,Blanchet:2004ek,
% Blanchet:2004bb,Jaranowski:1999qd, Jaranowski:1999ye, Damour:2001bu,Droz:1999qx,
% Kidder:2007rt, Blanchet3PN}. 
% As the motion of the objects becomes more relativistic, the PN equations 
% become increasingly inaccurate, especially during the late-inspiral and 
% merger phases~\cite{CompTemplates2001,CompTemplates2009}.

Past GW searches  have focused on GW bursts~\cite{Abadie:2010mt,
Abadie:2010wx,Abadie:2012rq}; coalescing compact
binaries~\cite{Colaboration:2011nz,Abadie:2010yb,Abbott:2009qj,
Abbott:2009tt,Messaritaki:2005wv,Abadie:2011kd,Aasi:2012rja},
and ringdowns of perturbed black holes~\cite{Abbott:2009km}, amongst
others~\cite{Abbott:2003yq,Abbott:2005pu,Sintes:2005fp,Abadie:2011md,
Palomba:2012wn}. For coalescing BBHs, detection searches involve 
matched-filtering~\cite{Wainstein:1962,Allen:2005fk} of the instrument
data using large banks of theoretically modeled waveform templates
as filters~\cite{Sathyaprakash:1991mt,SathyaMetric2PN,OwenTemplateSpacing,
BabaketalBankPlacement,SathyaBankPlacementTauN,Cokelaer:2007kx}.
The matched-filter is the optimal linear filter to maximize the
signal-to-noise ratio (SNR), in the presence of stochastic 
noise~\cite{1057571}. It requires an accurate modeling of the gravitational 
waveform emitted by the source binary. Early LIGO-Virgo searches 
employed template banks of Post-Newtonian (PN) inspiral 
waveforms~\cite{Colaboration:2011nz,Abadie:2010yb,Abbott:2009qj,
Abbott:2009tt,Messaritaki:2005wv}, while more recent
searches targeting high mass BBHs used complete inspiral-merger-ringdown
(IMR) waveform templates~\cite{Abadie:2011kd,Aasi:2012rja}. 

%In early LIGO-Virgo searches,
%and more recent ones targeted at low mass BBHs, the waveforms used as filters 
%were modeled within the Post-Newtonian theory~\cite{Colaboration:2011nz,
%Abadie:2010yb,Abbott:2009qj,Abbott:2009tt,Messaritaki:2005wv}. 
%For more massive binaries that merge at frequencies that LIGO was
%most sensitive to, complete inspiral-merger-ringdown (IMR) waveforms
%were used, during the LIGO-Virgo observation period
%from 2005-07 and 2009-10~\cite{Abadie:2011kd,Aasi:2012rja}. 

Recent developments in Numerical Relativity (NR) have provided complete 
simulations of BBH dynamics in the strong-field regime, i.e. during the
late-inspiral and merger phases~\cite{Pretorius2005,Baker:2005vv,Campanelli:2005dd,
Pretorius2006,Lindblom:2005qh}. These simulations have contributed 
unprecedented physical insights to the understanding of BBH mergers
%, like black hole recoil due to 
%asymmetric emission of gravitational radiation~\cite{Herrmann:2006ks,
%Baker:2006vn,Herrmann:2007ac,Koppitz:2007ev,Campanelli:2007ew,Gonzalez:2007hi,
%Campanelli:2007cga,Baker:2007gi,Herrmann:2007ex,Brugmann:2007zj,
%Schnittman:2007ij,Pollney:2007ss,Healy:2008js,Gonzalez:2008bi},
%and prediction of the physical parameters of the Kerr black hole 
%remnant~\cite{Campanelli:2006fg,Campanelli:2006fy,Rezzolla:2007xa,
%Boyle:2007sz,Rezzolla:2007rd,Marronetti:2007wz,Rezzolla:2008sd,Lousto:2009mf,
%Hemberger:2013hsa}.
(see, e.g., ~\cite{Pretorius:2007nq,Hannam:2009rd,Hinder:2010vn,Pfeiffer:2012pc} 
for recent overviews of the field).
Due to their high computational cost, fully numerical simulations currently 
span a few tens of inspiral orbits before merger. For mass-ratios
$q=m_{1}/m_{2}=1,2,3,4,6,8$, the multi-domain Spectral Einstein code 
(SpEC)~\cite{spec} has been used to simulate 15--33 inspiral merger 
orbits~\cite{Buchman:2012dw,Mroue:2012kv,Mroue:2013inPrep}.
These simulations have been used to calibrate waveform models, for example,
within the effective-one-body (EOB) formalism~\cite{EOBOriginalBuonannoDamour,
EOBNRdevel01,BuonannoEOBv2Main,Taracchini:2012ig}. 
Alternately, inspiral waveforms from PN theory can be
joined to numerical BBH inspiral and merger waveforms, to construct longer 
\textit{hybrid} waveforms~\cite{Boyle:2011dy,MacDonald:2011ne,MacDonald:2012mp,
Ohme:2011zm,Hannam:2010ky}. NR-PN hybrids have been used to calibrate 
phenomenological waveform models~\cite{Ajith:2007qp,Santamaria:2010yb},
%,Huerta:2012zy
and within the NINJA project~\cite{Aylott:2009tn,Ajith:2012tt}
to study the efficacy of various GW search algorithms towards realistic (NR)
signals~\cite{Santamaria:2009tm,Aylott:2009ya}.

Constructing template banks for gravitational wave searches has
been a long sought goal for NR. Traditionally, intermediary waveform
models are calibrated against numerical simulations and then
used in template banks for LIGO searches~\cite{Abadie:2011kd,
Aasi:2012rja}. In this paper we explore an alternative to this
traditional approach, proposing the use of NR waveforms themselves
and hybrids constructed out of them as search templates.
For a proof of principle, we focus on the non-spinning BBH space, 
with the aim of extending to spinning binaries in future work. We
investigate exactly where in the mass space can the existing NR 
waveforms/hybrids be used as templates, finding that only six 
simulations are sufficient to cover binaries with 
$m_{1,2}\gtrsim 12M_\odot$ upto mass-ratio $10$. This method can
also be used as a guide for the placement of parameters for future
NR simulations. Recent work has shown that existing PN waveforms 
are sufficient for aLIGO searches for 
$M=m_1+m_2\lesssim 12M_\odot$~\cite{CompTemplates2009,Brown:2012nn}.
To extend the NR/hybrid bank coverage down to 
$M\simeq 12M_\odot$, we demonstrate that a total of $26$ 
simulations would be sufficient. The template banks are 
constructed with the requirement that the net SNR recovered for 
any BBH signal should remain above $96.5\%$ of its optimal value.
Enforcing this tells that that these $26$ simulations would be 
required to be $\sim 50$ orbits long. This goal is achievable,
given the recent progress in simulation 
technology~\cite{MacDonald:2012mp,Mroue:2013xna,BelaLongSimulation}. 
Our template banks are viable for GW searches with aLIGO, and the 
framework for using hybrids within the LIGO-Virgo software 
framework has been demonstrated in the NINJA-2
collaboration~\cite{NINJA2:2013inPrep}. In this paper, we also 
derive waveform modeling error bounds which are independent of
analytical models. These can be extended straightforwardly to 
assess the accuracy of such models.

% This work can also be extended to quantify the errors of analytic
% and phenomenological waveform models. 

% IMR waveform models, such as the EOB and phenomenological models, 
% build on resummed or phenomenological extensions of results from PN 
% theory. The free parameters in the models are fitted to extend them
% to arbitrary values of physical paraemters. This is done by requiring
% agreement with a sample of NR waveforms that cover a few tens
% of inspiral cycles before merger. Their intrinsic modeling errors
% for the late-inspiral phase, however, are difficult to quantify 
% precisely, especially for mass (and spin) values to which the models 
% are interpolated or extrapolated to. On the other hand, we have 
% hybrid waveforms that are available for a restricted set of 
% binary masses and spins. Their intrinsic errors arise out of the
% yet-unknown higher order PN terms, which become significant as the 
% characteristic velocity increases. We can put a (conservative) 
% quantitative bound on these~\cite{MacDonald:2011ne,MacDonald:2012mp}.
% This allows us to consider the effect of template modeling errors
% on the SNR recovery of matched-filtering searches, which could be
% significant for stellar mass BBHs.
% 
% Therefore we propose the use of purely-NR and NR-PN hybrid waveforms
% as templates in GW searches with aLIGO. In this paper, we demonstrate
% the feasibility of doing so for non-spinning BBHs, with the aim of 
% extending the method to spinning binaries in future work. We find 
% that the currently available hybrids are sufficient to cover a 
% significant portion of the binary masses, including BBHs with 
% component masses above $12M_\odot$. 
% It has been a long standing aim of Numerical Relativity to provide
% simulations that can be directly applied to GW searches. We show that
% aiming at a complete hybrid template bank is useful in choosing the
% physical parameters for future NR simulations. Towards this aim, we 
% give the set of mass-ratios for which NR simulations would be required,
% and sufficient, for complete non-spinning hybrid banks for aLIGO.
% The detectory sensitivity is modeled using the final design zero-detuning
% high-power noise curve~\cite{aLIGONoiseCurve,aLIGOsensitivity}, with
% filtering starting at $15$~Hz.


First, we construct a bank for purely-NR templates, restricting to
currently available simulations~\cite{MacDonald:2012mp,Mroue:2012kv,
Buchman:2012dw,Mroue:2013xna,Mroue:2012kv}. We use a stochastic algorithm 
similar to Ref.~\cite{Harry:2009ea,Ajith:2012mn,Manca:2009xw}, 
and place a template bank grid
constrained to $q=m_1/m_2=\{1,2,3,4,6,8\}$. The bank placement 
algorithm uses the EOB model from Ref.~\cite{BuonannoEOBv2Main} (EOBNRv2).
As this model was calibrated against NR for most of these mass-ratios, 
we expect the manifold of EOBNRv2 to be a reasonable approximation 
for the NR manifold. In Sec.~\ref{s1:NRpNhybridbank}, we demonstrate that
this approximation holds well for NR-PN hybrids as well.
To demonstrate the efficacy of the 
bank, we measure its fitting-factors (FFs)~\cite{FittingFactorApostolatos} over
the BBH mass space. We simulate a population of $100,000$ BBH waveforms with
masses sampled uniformly over 
$3M_\odot\leq m_{1,2}\leq 200M_\odot$ and $M=m_1+m_2\leq 200M_\odot$, and filter
them through the template bank to characterize its SNR recovery. For a
bank of NR templates, any SNR loss accrued will be due to the coarseness
of the bank grid. We measure this requiring both signals and templates
to be in the same manifold, using the EOBNRv2 model for both. We find 
that for systems with chirp mass 
$\mathcal{M}_c \equiv (m_1 + m_2)^{-1/5} (m_1 m_2)^{3/5}$ above 
$\sim 27M_{\odot}$ and $1\leq q\leq 10$, this bank has FFs $\geq 97\%$ and
is sufficiently accurate to be used in GW searches.
We also show that the coverage of the purely-NR bank can be extended to
include $10\leq q\leq 11$, if we instead constrain it to templates with
mass-ratios $q=\{1,2,3,4,6,9.2\}$.

Second, we demonstrate that currently available PN-NR hybrid waveforms can be 
used as templates to search for BBHs with much lower masses. The hybrids
used correspond to mass-ratios $q=\{1,2,3,4,6,8\}$. We use two distinct methods
of bank placement to construct a bank with these mass-ratios, and compare the
two. The first method is the stochastic algorithm we use for purely-NR 
templates. The second is a deterministic algorithm, that constructs the 
two-dimensional bank (in $M$ and $q$) through a union of six one-dimensional
banks, placed separately for each allowed value of mass-ratio. Templates are
placed over the total mass dimension by requiring that all pairs of neighboring
templates have the same noise weighted overlap. As before, we measure the SNR 
loss from both banks, due to the discrete placement of the templates, by 
simulating a population of $100,000$ BBH signals, to find the SNR recovered.
We measure the intrinsic hybrid errors using the method of
Ref.~\cite{MacDonald:2011ne,MacDonald:2012mp}, and subsequently account for 
them in the SNR recovery fraction. We find that the NR-PN
hybrid bank is effectual for detecting BBHs with $m_{1,2}\geq 12M_{\odot}$, 
with FFs $\geq 96.5\%$. The number of templates required was found
to be close to that of a bank constructed using the second-order TaylorF2
hexagonal bank placement algorithm~\cite{Sathyaprakash:1991mt,SathyaMetric2PN,
OwenTemplateSpacing,BabaketalBankPlacement,
SathyaBankPlacementTauN,Cokelaer:2007kx}. We note that by pre-generating the
template for the least massive binary for each of the mass-ratios that 
contribute to the bank, we can re-scale it on-the-fly to different total 
masses in the frequency domain~\cite{Sathyaprakash:2000qx}. 
Used in detection searches, such a bank would be computationally inexpensive 
to generate relative to a bank of time-domain modeled waveforms.

Finally, we determine the minimal set of NR simulations that we would need to 
extend the bank down to $M\simeq 12M_\odot$. We find that a bank that
samples from the set of $26$ mass-ratios listed in Table~\ref{table:fullqlist}
would be sufficiently dense, even at the lowest masses, for binaries with 
mass-ratios $1\leq q\leq 10$. We show that this bank recovers more than
$98\%$ of the optimal SNR, not accounting for hybrid errors. 
To restrict the loss in event detection 
rate below $10\%$, we restrict the total SNR loss below $3.5\%$. 
This implies the hybrid error mismatches stay below $1.5\%$, which 
constrains the length of the NR part for each hybrid.
We find that NR simulations spanning about $50$ orbits of late-inspiral, merger
and ringdown would suffice to reduce the PN truncation error to the desired 
level. With such a bank of NR-PN hybrids and purely-PN templates for lower
masses, we can construct GW searches for stellar-mass BBHs with mass-ratios 
$q\leq 10$.

The paper is organized as follows, in Sec.~\ref{s2:NRwaveforms}, we discuss
the NR waveforms used in this study, in Sec.~\ref{s2:PNwaveforms} we
describe the PN models used to construct the NR-PN hybrids,  and in Sec.~\ref{s2:NRpNhybridwaveforms} we describe the
construction of hybrid waveforms. In Sec.~\ref{s2:EOBwaveforms} we 
describe the EOB model that we use to place and test the template
banks. In Sec.~\ref{s1:quantifyingerrors} we describe the accuracy
measures used in quantifying the loss in signal-to-noise ratio in a
matched-filtering search when using a discrete bank of templates and
in the construction of hybrid waveforms. In Sec.~\ref{s1:NRonlybank}
we describe the construction and efficacy of the NR-only banks, while in
Sec.~\ref{s1:NRpNhybridbank} we discuss the same for the NR-PN hybrid 
template banks constructed with currently available NR waveforms. In
Sec.~\ref{s1:futureNRpNhybridbank}, we investigate the parameter and length
requirements for future NR simulations in order to cover the entire 
non-spinning parameter space with $12 M_\odot\leq M\leq 200M_\odot$, 
$m_{1,2} \geq 3M_\odot$, and $1 \leq q \leq 10$. Finally, in 
Sec.~\ref{s1:conclusions} we summarize the results. 
