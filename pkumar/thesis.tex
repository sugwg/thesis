%-----------------------------------------------------------------------
%
% File Name: thesis.tex
%
% Author: Kumar, P.
%
% Revision: $Id$
%
%-----------------------------------------------------------------------
%%%%%%%%%%%%%%%%%%%%%%%%%%%%%%%%%%%%%%%%%%%%%%%%%%%%%%%%%%%%%%%%%%%%%%%%%%%%%%%
%%%%%%%%%%%%%%%%%%%%%%%%%%%%%%%%%%%%%%%%%%%%%%%%%%%%%%%%%%%%%%%%%%%%%%%%%%%%%%%

% document class and packages
\documentclass[12pt,notitlepage]{report}
\usepackage{setspace}
\usepackage{graphicx}\usepackage{latexsym}
% \usepackage{iopams}

\usepackage{syrthesis}
% \usepackage{color}
\usepackage{amsmath,mathtools}
\usepackage{amssymb}
\usepackage{amsfonts}
\usepackage{rotating}
\usepackage{tensor}
\usepackage{lscape}
\usepackage{bm}
\usepackage{units} %  do things like \units[1.234]{sec}i
% \usepackage{chapterfolder} % Organize the graphics etc of each chapter in a sub-directory
\usepackage{amsopn, amstext, wasysym}
\usepackage{colonequals}

\usepackage{subfigure}
\usepackage{rotating}
\usepackage{braket}
\usepackage{multirow}
\usepackage{xspace}
\usepackage[usenames,dvipsnames]{color}

\usepackage[bookmarksnumbered, bookmarksopen, breaklinks, colorlinks, linkcolor=blue, citecolor=magenta]{hyperref}
\usepackage{bibunits}

\usepackage{url}
\usepackage{alltt}

\usepackage{ulem}
\normalem

%%%%%%%%%%%%%%%%%%%%%%%%%%%
%%% NINJA-2 packages
%%%%%%%%%%%%%%%%%%%%%%%%%%%%%
\usepackage{dcolumn}
\usepackage{units}
%%%%%%%%%%%%%%%%%%%%%%%%%%%
\makeatletter
\let\protect\relax
{\catcode`\|=\active
  \xdef\InnerProduct{\protect\expandafter\noexpand\csname InnerProduct
\endcsname}
  \expandafter\gdef\csname InnerProduct \endcsname#1{%
    \begingroup
    \ifx\SavedDoubleVert\relax
    \let\SavedDoubleVert\|\let\|\IpDoubleVert
    \fi
    \mathcode`\|32768\let|\IPVert
    \left({#1}\right)
    \endgroup
  }
}
\def\IPVert{\@ifnextchar|{\|\@gobble}% turn || into \|
     {\egroup\,\mid@vertical\,\bgroup}}
\def\IPDoubleVert{\egroup\,\mid@dblvertical\,\bgroup}
\let\SavedDoubleVert\relax
\def\midvert{\egroup\mid\bgroup}
\def\SetVert{\@ifnextchar|{\|\@gobble}% turn || into \|
    {\egroup\;\mid@vertical\;\bgroup}}
\def\SetDoubleVert{\egroup\;\mid@dblvertical\;\bgroup}
\def\mid@vertical{\mskip1mu\vrule\mskip1mu}
\def\mid@dblvertical{\mskip1mu\vrule\mskip2.5mu\vrule\mskip1mu}
\makeatother
\usepackage{braket}
\newcommand{\Overlap}{\Braket}

\pdfoutput=1
\DeclareGraphicsExtensions{.pdf,.png,.eps}

\hbadness=10000

%%%%%%%%%%%%%%%%%%%%%%%%%%%%%%%%%%%%%%%%%%%%%%%%%%%%%%%%%%%%%%%%%%%%%%%%%%%%%%%
% new command definitions
%%%%%%%%%%%%%%%%%%%%%%%%%%%%%%%%%%%%%%%%%%%%%%%%%%%%%%%%%%%%%%%%%%%%%%%%%%%%%%%
\newcommand{\half}{\frac{1}{2}}
\newcommand{\ospsd}{\ensuremath{S_n\left(\left|f_{k}\right|\right)}}

\newcommand{\msun}{M_\odot}
\newcommand{\chisq}{\chi^2}
\newcommand{\newsnr}{\rho_{\textrm{new}}}
\newcommand{\erf}{\mathrm{erf}}

%%%%%%%%%%%%%%%%%%%%%%%%%%%%%%%%%%%%%%%%%%%%%%%%%%%%%%%%%%%%%%%%%%%%%%%%%%%%%%%
% more new commands
%%%%%%%%%%%%%%%%%%%%%%%%%%%%%%%%%%%%%%%%%%%%%%%%%%%%%%%%%%%%%%%%%%%%%%%%%%%%%%%
\newcommand{\Sum}{\displaystyle\sum\limits}
\newcommand{\Int}{\displaystyle\int\limits}
\newcommand{\ii}{{\rm i}}
\newcommand{\D}{\mathrm{d}}
\newcommand{\eff}{\mathrm{eff}}
\newcommand{\phys}{\mathrm{phys}}
\newcommand{\real}{\mathrm{real}}
\newcommand{\peak}{\mathrm{peak}}
% \newcommand{\EOB}{\mathrm{EOB}}
\newcommand{\NR}{\mathrm{NR}}
\newcommand{\RD}{\mathrm{RD}}
\newcommand{\Olap}{\mathcal{O}}
\newcommand{\FF}{\mathcal{FF}}
\newcommand{\MM}{\mathrm{MM}}
\newcommand{\EFF}{\mathrm{EFF}}
\newcommand{\X}{\mathrm{X}}
\newcommand{\Y}{\mathrm{Y}}
\newcommand{\Z}{\mathrm{Z}}
\newcommand{\horizon}{\mathrm{Horizon}}
\newcommand{\opt}{\mathrm{opt}}
\newcommand{\iso}{\mathrm{iso}}
\newcommand{\refr}{\mathrm{ref}}
\newcommand{\start}{\mathrm{start}}

\def\l({\left(}
\def\r){\right)}
\newcommand{\lt}{\left}
\newcommand{\rt}{\right}

%%%%%%%%%%%%%%%%%%%%%%%%%%%%%%%%%%%%%%%%%%%%%%%%%%%%%%%%%%%%%%%%%%%%%%%%%%%%%%%
% more new commands
%%%%%%%%%%%%%%%%%%%%%%%%%%%%%%%%%%%%%%%%%%%%%%%%%%%%%%%%%%%%%%%%%%%%%%%%%%%%%%%
\newcommand{\EOB}{\mathrm{EOBNRv2}}
\newcommand{\M}{\mathit{M}}
\newcommand{\bnk}{\mathrm{bank}}
\newcommand{\mn}{\mathrm{min}}
\newcommand{\mx}{\mathrm{max}}
\newcommand{\tr}{\mathrm{tr}}
\newcommand{\mm}{\mathrm{mm}}
\newcommand{\Hyb}{\mathrm{Hyb}}
\newcommand{\leftn}{\left|\left|}
\newcommand{\rightn}{\right|\right|}
\newcommand{\lefti}{\left\langle}
\newcommand{\righti}{\right\rangle}
\newcommand{\Mis}{\mathcal{M}}
\newcommand{\N}{\mathrm{N}}
\newcommand{\cyc}{\mathrm{cyc}}
\newcommand{\E}{\mathcal{E}}

\newcommand{\red}{\textcolor{red}}
\newcommand{\blue}{\textcolor{blue}}
\newcommand{\etal}{\textit{et~al}\@ifnextchar{\relax}{.\relax}{\ifx\@let@token.\else\ifx\@let@token~.\else.\@\xspace\fi\fi}}


%% Macros for comments
\newcommand\fake[1]{\textcolor{red}{#1}}
\newcommand\checkme[1]{\textcolor{blue}{\textbf{#1}}}
\newcommand{\Note}[1]{\textcolor{red}{\textbf{[#1]}}}
% \newcommand\checked[1]{#1}
\newcommand\marginnote[2]{%
  \mbox{}\marginpar{\raggedleft\hspace{0pt}\scriptsize
    \textcolor{blue}{#1: #2}}}


\newcommand\weakheader[1]{
\vspace*{5mm}
\noindent {\it #1}
\vspace*{5mm}
}


\usepackage{hyperref} 

\begin{document}

\title{
Topics in Gravitational-Wave Astronomy
}
\author{\bf Prayush Kumar}
\majorprof{Duncan Brown}
\submitdate{August 2014}
\degree{Doctor of Philosophy}
\program{Physics}
\copyrightyear{2014}
\majordept{Physics}
\havededicationtrue
\dedication{to\\ my parents}
\haveminorfalse
\copyrighttrue
\doctoratetrue
\figurespagetrue
\tablespagetrue


\Abstract{
\color{blue}{Gravitational waves are a consequence of the general theory of
relativity.  Direct detection of such waves will provide a wealth of
information about physics, astronomy, and cosmology.  A worldwide
effort is currently underway to make the first  direct detection of
gravitational waves.  The global network of detectors includes the
Laser Interferometer Gravitational-wave Observatory (LIGO), which
recently completed its sixth science run.

A particularly promising source of gravitational waves is a binary
system consisting of two neutron stars and/or black holes.  As the
objects orbit each other they emit gravitational radiation, lose
energy, and spiral inwards.  This produces a characteristic ``chirp''
signal for which we can search in the LIGO data.  Currently this is
done using matched-filter techniques, which correlate the detector
data against analytic models of the emitted gravitational waves.
Several choices must be made in constructing a search for signals from
such binary coalescences. 

Any discrepancy between the signals and the models used will reduce
the effectiveness of the matched filter.  However, the analytic models
are based on approximations which are not valid through the entire
evolution of the binary.  In recent years numerical relativity has had
impressive success in simulating the final phases of the coalescence
of binary black holes.  While numerical relativity is too
computationally expensive to use directly in the search, this progress
has made it possible to perform realistic tests of the LIGO searches.
The results of such tests can be used to improve the efficiency of
searches.

Conversely, noise in the LIGO and Virgo detectors can reduce the
efficiency.  This must be addressed by characterizing the quality of
the data from the detectors, and removing from the analysis times that
will be detrimental to the search.

In this thesis we utilize recent results from numerical relativity to
study both the degree to which analytic models match realistic
waveforms and the ability of LIGO searches to make detections.  We
also apply the matched-filter search to the problem of removing times
of excess noise from the search.}
}
\beforepreface
\prefacesection{Preface}
The work presented in this thesis stems from my participation in the
LIGO Scientific Collaboration (LSC) and the NINJA-2 Collaboration.  This
work does not reflect the scientific opinion of the LSC and it was not
reviewed by the collaboration.

\vspace*{0.5cm}

\noindent Chapter \ref{ch:EOBF2_Effectualness} is based on material from

\vspace*{0.25cm}

\noindent D. A. Brown, P. Kumar and A. H. Nitz, ``Template banks to 
search for low-mass binary black holes with Advanced Gravitational-wave 
detectors,''
{\it Phys. Rev. D} {\bf 87}, 082004 (2013)

\vspace*{0.5cm}

\noindent Chapter \ref{ch:NRHyb_bank} is based on material from

\vspace*{0.25cm}

\noindent P. Kumar, I. MacDonald, D. A. Brown, H. P. Pfeiffer, K. Cannon,
M. Boyle, L. E. Kidder, A. H. Mroue, B. Szilagyi and A. Zeninoglu, 
``Template banks for binary black hole searches with numerical relativity
waveforms,''
{\it Phys. Rev. D} {\bf 89}, 042002 (2014)

\vspace*{0.5cm}

\noindent Chapter \ref{ch:ninja2} is based on material from

\vspace*{0.25cm}

\noindent The LIGO Scientific Collaboration, the Virgo Collaboration, 
the NINJA-2 Collaboration, ``The NINJA-2 project: 
Detecting and characterizing gravitational waveforms modelled using 
numerical binary black hole simulations,''
{\it arXiv:1401.0939} (2014) {\rm Submitted to Classical and Quantum Gravity}

\vspace*{0.5cm}

\noindent Chapter \ref{ch:insSFIMRI} is based on material from

\vspace*{0.25cm}

\noindent E. A. Huerta, P. Kumar, D. A. Brown, 
``Accurate modeling of intermediate-mass-ratio inspirals:
Exploring the form of the self-force in the intermediate-mass-ratio regime,''
{\it Phys. Rev. D} {\bf 86}, 024024 (2012)

\vspace*{0.5cm}

\noindent Chapter \ref{ch:imrSFIMRI} is based on material from

\vspace*{0.25cm}

\noindent E. A. Huerta, P. Kumar, J. R. Gair, S. T. McWilliams, 
``Self-forced evolutions of an implicit rotating source:
a natural framework to model comparable and intermediate mass-ratio 
systems from inspiral through ringdown,''
{\it arXiv:1403.0561} (2014) {\rm Submitted to Phys. Rev. D}

\vspace*{0.5cm}


\prefacesection{Acknowledgments}

I have been fortunate to work with many wonderful people through my
membership in the LIGO and NINJA collaborations.  Sadly space makes it
impossible to thank them all individually, and I hope the vast
majority I must omit will forgive me.

I would first and foremost like to thank my advisor, Duncan Brown..


\prefacesection{Conventions}

We adopt the Einstein summation convention where repeated indices are
summed over.  Parenthesis are shorthand for the fully symmetric sum
%
\begin{equation*}
A_{(\alpha\beta)}
= \frac{1}{2} \left(A_{\alpha\beta} 
+ A_{\beta\alpha} \right)
\end{equation*}
%
and square brackets are fully antisymmetric
%
\begin{equation*}
A_{[\alpha\beta]}
= \frac{1}{2} \left(A_{\alpha\beta} 
- A_{\beta\alpha} \right)
\end{equation*}
%
We take the signature of spacetime to be $(-1,+1,+1,+1)$.

\noindent Except where otherwise noted we work in geometric units where
$G=c=1$.  We will often measure masses in multiples of the mass of the
sun $1 \msun \approx 1.99 \times 10^{30}$ kg 

\vspace{0.5cm}

\noindent We define the Fourier transform of a function of time $g(t)$ to be
$\tilde{g}(f)$, where
%
\begin{equation*}
\tilde{g}(f)=\int_{-\infty}^\infty g(t)\, e^{- 2 \pi i f t}\, dt
\end{equation*}
%
and the inverse Fourier transform is
%
\begin{equation*}
g(t)=\int_{-\infty}^\infty \tilde{g}(f)\, e^{2 \pi i f t}\, dt
\end{equation*}
%
This convention differs from that used in some gravitational-wave
literature, but is the adopted convention in the LIGO Scientific
Collaboration.

\vspace{0.5cm}

\noindent The time-stamps of interferometer data are measured in
Global Positioning System (GPS) seconds: seconds since 00:00.00 UTC
January 6, 1980 as measured by an atomic clock.


\afterpreface

\Chapter{Introduction}
\label{ch:introduction}
In 1915, Albert Einstein published his theory of general relativity, a 
geometric theory of gravitation that sought to expand upon Newtonian 
mechanics and provide a complete description of gravity and its 
relationship with space and time. Einstein theorized that space 
and time were deeply related and existed together as a manifold 
called spacetime. Matter with energy and momentum 
existing in this manifold would create 
curvature in spacetime. Gravitational forces were the result of 
matter following geodesic curves in spacetime. This concept can 
be summarized in the Einstein field equation, which is presented 
as,
\begin{equation}
G_{\mu\nu} = 8\pi T_{\mu\nu}
\label{eq:EFE}
\end{equation}
where $G_{\mu\nu}$ is the Einstein tensor, which describes the 
curvature of spacetime, $T_{\mu\nu}$ is the 
stres-energy tensor, which describes the energy and momentum in 
spacetime, and  $G=c=1$. The Einstein tensor is defined as,
\begin{equation}
G_{\mu\nu} = R_{\mu\nu} - \frac{1}{2}Rg_{\mu\nu}
\end{equation}
where $R_{\mu\nu}$ is the Ricci curvature tensor and $g_{\mu\nu}$ is 
the metric tensor for the manifold.

An interesting result that arises from this formalism is the 
existence of gravitational waves, which are perturbations in 
spacetime caused by certain types of time-varying mass distributions. 
To describe gravitational waves, we consider 
a Minkowski metric with a small perturbation. The Minkowski metric 
is a flat spacetime metric defined as
\begin{equation}
\eta_{\mu\nu} = 
  \begin{pmatrix}
   -1 & 0 & 0 & 0 \\
    0 & 1 & 0 & 0 \\
    0 & 0 & 1 & 0 \\
    0 & 0 & 0 & 1
  \end{pmatrix}
\end{equation}
where $\mu = 0$ corresponds to the time coordinate and $\mu = {1,2,3}$ 
correspond to the spatial coordinates. In examples, we will use the coordinate 
convention $(x^0,x^1,x^2,x^3) = (ct,x,y,z)$. 
The full spacetime metric, $g_{\mu\nu}$, is then constructed as a 
linear perturbation on the Minkowski metric,
\begin{equation}
g_{\mu\nu} = \eta_{\mu\nu} + h_{\mu\nu}
\end{equation}
where $h_{\mu\nu}$ is the metric perturbation and $|h_{\mu\nu}| \ll 1$.
From here, we follow the convention of Saulson \cite{Saulson:1994} to arrive at the general 
form of a gravitational wave.
At this point it is very useful to move into the transverse traceless 
gauge where coordinates on the manifold are defined by the geodesic 
motion of freely-falling test masses. In this gauge, the weak field 
vacuum solution of the Einstein field equation becomes a wave equation: 
\begin{equation}
\square h_{\mu\nu} = 0.
\end{equation}
The solutions to this differential equation will be plane waves of 
the form
\begin{equation}
h_{\mu\nu} = C_{\mu\nu}e^{i(2\pi ft - \vec{k}\cdot\vec{x})}
\end{equation}
where $C_{\mu\nu}$ is the wave amplitude, $f$ is the frequency, 
and $\vec{k}$ is the wave vector which indicates the direction of 
propagation \cite{Carroll}.

For example, consider the case of a gravitational 
wave propogating along the $\hat{z}$-axis.
When the conditions of the transverse traceless gauge are applied, 
the resulting form of $h_{\mu\nu}$ is 
\begin{equation}
h_{\mu\nu} = 
  \begin{pmatrix}
    0 & 0 & 0 & 0 \\
    0 & h_+ & h_x & 0 \\
    0 & h_x & -h_+ & 0 \\
    0 & 0 & 0 & 0
  \end{pmatrix}
\end{equation}
where the diagonal and off-diagonal terms represent two polarizations 
of the resulting gravitational wave, called "h-plus" and "h-cross" 
respectively.
We can see the effects of this perturbation by observing the  
spacetime interval on the manifold. The spacetime interval is defined as 
\begin{equation}
ds^2 = dx^\mu g_{\mu\nu}dx^\nu.
\end{equation}
Substituting in our perturbed metric for $g_{\mu\nu}$, we find that 
the spacetime interval can be broken up into a standard Minkowski line 
element and a perturbation due to $h_{\mu\nu}$.
\begin{equation}
ds^2 = dx^\mu (\eta_{\mu\nu} + h_{\mu\nu})dx^\nu \\
\end{equation}
\begin{equation}
ds^2 = dx^\mu \eta_{\mu\nu} dx^\nu + dx^\mu h_{\mu\nu}dx^\nu
\label{eq:spacetime}
\end{equation}

As an example, we present the case of a plus-polarized gravitational wave 
propagating in the $\hat{z}$ direction and observe the effect of the perturbation 
on the spacetime interval. The perturbation will have the form 
\begin{equation}
h_{\mu\nu} = 
  \begin{pmatrix}
    0 & 0 & 0 & 0 \\
    0 & h_+ & 0 & 0 \\
    0 & 0 & -h_+ & 0 \\
    0 & 0 & 0 & 0
  \end{pmatrix}
\end{equation}
Using the coordinate convention of $(ct,x,y,z)$, the unperturbed
spacetime interval is given as: 
\begin{equation}
ds^2 = -c^2 dt^2 + dx^2 + dy^2 + dz^2.
\end{equation}
Since the perturbation is spatially transverse to the direction of 
propagation, the ct- and z-coordinates will not be modulated by the 
gravitational wave. The x- and y-coordinates will be modulated  
according to equation \ref{eq:spacetime}. The resulting spacetime 
interval is
\begin{equation}
ds^2 = -c^2 dt^2 + (1 + h_+)dx^2 + (1 - h_+)dy^2 + dz^2.
\end{equation}
This shows that a gravitational wave propogating along the $\hat{z}$-axis 
will differentially stretch and squeeze spacetime in the transverse 
axes. The exact form of $h_+$ will depend on the source of the 
gravitational waves. A visualization of this stretching and squeezing 
is shown in Figure \ref{fig:polarizations}\cite{Polarization}. The cross polarization  
stretches and squeezes at a 45 degree angle relative to the plus 
polarization.

\begin{figure}[ht!]
\includegraphics[width=\textwidth]{figures/introduction/polarisations2}
\caption[Plus and cross polarizations]{Plus and cross polarizations %
         of a gravitational wave.}
\label{fig:polarizations}
\end{figure}

The Advanced LIGO interferometers are designed to be sensitive 
to this differential stretching and squeezing by constructing orthogonal 
optical cavities. A gravitational wave passing through an aLIGO interferometer 
will differentially 
modulate the lengths of the optical cavities, creating an interference 
pattern at the output of the instrument that can be searched for 
gravitational wave signals. The layout and gravitational wave readout scheme 
of the interferometers is discussed below.

\section{The Advanced LIGO Interferometers}\label{sec:aligo}

The Advanced LIGO (aLIGO) interferometers are a pair of dual-recycled Michelson interferometers 
that employ 4km long Fabry-Perot cavities in their arms to increase the interaction time with a 
gravitational wave signal. 
Figure \ref{fig:aligo} shows a simplified layout of an aLIGO interferometer. 

\begin{figure}[ht!]
\includegraphics[width=\textwidth]{figures/introduction/ALIGO_layout}
\caption[Layout of Advanced LIGO]{Layout of Advanced LIGO}
\label{fig:aligo}
\end{figure}

At the input to an aLIGO interferometer is a solid-state Nd:YAG laser that provides laser light 
at a wavelength of 1064 nm. Not included in Figure \ref{fig:aligo} are frequency and 
intensity stabilization control loops designed to provide as stable a laser source as 
possible for the experiment. This stabilized laser is called the pre-stabilized laser 
(PSL). The laser light is passed through a series of 
electro-optic modulators (EOM) where radio-frequency (RF) sidebands are generated 
and imparted onto the light. These RF sidebands are used to control auxiliary optical 
degrees of freedom in the interferometer. The beam is then passed through the 
input mode cleaner (IMC), which rejects higher order spatial modes of the beam 
and transmits a circular TEM00 mode to be used in the instrument.

Once the beam has been stabilized in frequency and intensity and the higher order 
optical modes have been stripped away, it is transmitted through the power 
recycling mirror and enters the vertex of the interferometer. In the vertex, 
the beam is split 50/50 by the beamsplitter. Half of the light is directed toward  
the input test mass (ITM) of the X-arm and half of the light is directed  
toward the ITM of the Y-arm. As mentioned previously, the aLIGO arms are not 
single bounce cavities; they are comprised of Fabry-Perot cavities that allow the 
light to circulate in the arm cavities multiple times. The light is stored in 
the arm cavities for $\sim$1ms, trapped between the highly reflective surfaces 
of the ITM and the end test mass (ETM), before it is transmitted back through 
the ITM and into the vertex.

When a gravitational wave passes through an aLIGO inteferometer, the distance
between the ITM and ETM of each arm is modulated, causing the light to have a
longer or shorter travel time as it traverses the arm. Since gravitational
waves expand space in one direction while the orthogonal direction contracts,     
the X- and Y-arms will experience differential changes in length. When light
from the arms is recombined at the beamsplitter, there will be a difference
in phase between the two beams as they have traveled different paths. The 
resulting light from this recombination of phase shifted beams is called the 
antisymmetric part of the output. The part of the beam that is recombined 
in phase is called the symmetric part of the output.

The beams returning from each arm are recombined at the beamsplitter. The 
symmetric part of the beam 
will be sent back toward the power recycling mirror. The power recycling mirror 
forms a resonant cavity with the ITMs, allowing for light at the symmetric 
port of the beamsplitter to be added coherently to incoming light from the PSL and 
increasing the effective power in the vertex. This increase in effective power 
is known as the power recycling gain. 

The antisymmetric part of the beam is sent toward the signal recycling mirror. 
The signal recycling cavity is used to tune the frequency response of the 
interferometer by adjusting the effective finesse of the coupled cavity 
formed by the signal recycling cavity and the arm cavities. 
If the light returning from the arms has accumulated some differential amount of 
phase as it traveled 
along the arms, perhaps from a gravitational wave modulating the length of each 
arm differentially, it will be transmitted through the signal recycling cavity 
and into the output mode cleaner (OMC). The OMC behaves similarly to the IMC, 
stripping away higher order optical modes and isolating the TEM00 mode of the 
beam. The transmitted, mode cleaned signal is then read out using a homodyne 
detection scheme on a DC photodiode. 

\subsection{DC Readout}

When a gravitational wave modulates the length of an arm cavity, the light 
traveling in that arm experiences a phase modulation. This phase modulation 
can be visualized by picturing the beam in frequency space. In figure 
\ref{fig:omc-freq}, the carrier beam frequency is designated as $f_0$. 
The phase modulation due to 
a gravitational wave signal introduces a frequency sideband at the 
gravitational wave frequency, which is in the 30-2000 Hz range. 
The 
RF sidebands used for auxiliary optical cavity control are offset from the 
carrier frequency by 9, 24, and 45 MHz. 
The RF sidebands, which in a 
homodyne detection scheme would only contribute shot noise to the output signal, 
are rejected by the OMC. The gravitational wave sidebands, however, are at a 
low enough frequency offset that they are within the cavity pole of the OMC 
and are allowed to transmit through the cavity.

Since the OMC DC photodiode measures power, it measures the square of the 
incident optical field and witnesses beat frequencies between different 
components of the light. If the RF sidebands have been filtered out by 
the OMC, the only remaining beat note will be that of the carrier beam ($f_0$) 
beating against the gravitational wave sideband ($f_0 + f_{GW}$). This beat note will 
appear as the difference in frequency between the two optical fields, 
leaving behind a signal in the 30-2000 Hz range ($f_{GW}$) and providing a 
natural demodulation inherent to the measurement process. 
The process of recovering the gravitational wave sideband using the 
carrier field as a reference is known as homodyne detection. 

\begin{figure}[ht!]
\includegraphics[width=\textwidth]{figures/introduction/omc-freq}
\caption[Sidebands and OMC cavity pole]{Frequency domain visualization of beam %
         at OMC. Grey dotted lines indicate the cavity pole. The gravitational %
         wave sidebands are within the cavity pole and are transmitted through %
         the OMC. The RF sidebands are in the MHz range and are rejected by the %
         OMC.}
\end{figure}\label{fig:omc-freq}

\section{Sources of Gravitational Waves}
Include that box with modeled, unmodeled, transient, and continuous.

CBCs are the bread and butter, expect BNS, NSBH, and BBH sources
Continuous waves from pulsars
Bursts from supernovae
Stochastic background


\section{Searching for Compact Binary Coalescences}

Steal this from O1 CBC DQ paper


\section{The Advanced Detector Network}

The Advanced Laser Interferometer Gravitational-Wave Observatory (aLIGO) is 
part of a worldwide effort to detect gravitational waves from astrophysical 
sources. The two aLIGO interferometers, one in Hanford, WA and one in 
Livingston, LA, are part of a growing network of ground-based interferometric 
gravitational wave detectors. Each aLIGO interferometer has 4km long arms 
arranged in an L-shaped configuration. A gravitational wave passing through 
an aLIGO interferometer will cause the arms to expand and contract, 
creating an interferometric signal at the output of the instrument. 
Section \ref{sec:aligo} contains a more detailed description of the aLIGO 
interferometers. 

There are a number of other interferometric gravitational wave detectors 
being built and commissioned for future use in collaboration with aLIGO.
The Advanced VIRGO detector is being built and commissioned in Cascina, Italy. 
When it is fully commissioned, VIRGO will be joining LIGO in observing runs. 
The VIRGO interferometer has 3km arms, which should provide enough 
sensitivity to allow for triangulation of astrophysical sources.

The GEO600 detector, located in Hanover, Germany is an interferometer built in 
collaboration between Germany and the United Kingdom. 
GEO600 is an extremely valuable test bed for interferometric technologies,
including quantum optics and homodyne detection. However, with 600m arms, GEO600 
is unlikely to be sensitive enough to witness expected astrophysical sources.

The KAGRA detector, located underground in the Kamioka mine in Japan, 
is in its commissioning phase. KAGRA has 3 km long arms and, 
unlike other gravitational wave interferometers, employs cryogenics to 
reduce thermal noise in its optics. When complete, KAGRA should be 
sensitive enough to contribute to the worldwide detector network.

Include that cool picture of the advanced detector network.


\Chapter{Gravitational waves and Compact binaries in General Relativity}
\label{ch:theory}
In this chapter we briefly survey the theoretical issues behind
gravitational-wave astronomy.  We start in
Sec.~\ref{sec:general_relativity} with a review of general
relativity, beginning with the relevant mathematics.  In
Sec.~\ref{sec:gravitational_radiation} we show how general
relativity predicts the existence of gravitational radiation and
discuss some of the properties of this radiation.  Then in
Sec.~\ref{sec:effects_of_waves} we show how gravitational radiation
affects freely-falling particles.  This will motivate the design of
the LIGO experiment to search for gravitational waves, an overview of
which is presented in the next chapter.  We then move to the
generation of gravitational waves and discuss two approaches to
modeling the waves produced by the inspiral and eventual merger of
pairs of compact objects.  These are analytic models, discussed in
Sec.~\ref{sec:PNWaveforms} and numerical models, discussed in
Sec.~\ref{sec:NRWaveforms}.


\section{General Relativity}
\label{sec:general_relativity}

We start with an overview of differential geometry and build to
Einstein's equations.  This is of necessity very brief, readers are
referred to the textbooks by Misner, Thorne and Wheeler~\cite{MTW} and
Carroll~\cite{carrollTextbook} for more complete treatments.

\subsection{Elements of Differential Geometry}

An $n$-dimensional ($C^\infty$) manifold $\mathcal{M}$ is a set of
points plus an \emph{atlas}, a set of \emph{charts} $\{\phi_i\}$ which
are invertible maps from open subsets of $\mathcal{M}$ to open subsets
of $\mathcal{R}^n$ such that


\begin{itemize}
\item For all points $p \in \mathcal{M}$ there exists an $\phi_i$ 
such that $p$ is in the domain of $\phi_i$.
\item The composition $\phi_i \circ \phi_j^{-1}$ on the 
intersections of the domains of $\phi_i$ and $\phi_j$ is a
($C^\infty$) function from $\mathcal{R}^n \to \mathcal{R}^n$.
\end{itemize}
%
Two natural structures on a manifold are curves, maps from
$\mathcal{R}\to\mathcal{M}$, and scalar functions,  maps from
$\mathcal{M}\to\mathcal{R}$.  Compositing a function $f$ with a curve
$\gamma(\lambda)$ gives a map from $\mathcal{R} \rightarrow
\mathcal{R}$ which may be differentiated in the usual way at a point
$p$.

\iffalse
\begin{equation*}
\frac{d}{d \lambda} fi \big|_p = 
  \frac{d}{d\lambda} (f \circ \gamma) \big|_p
\end{equation*}


Expanding this in terms of a chart whose domain includes $p$ and then
applying the chain rule
 
\begin{align*}
\frac{d}{d \lambda} fi \big|_p &= 
 \frac{d}{d\lambda} ( (f \circ \phi^{-1}) \circ (\phi \circ
\lambda) ) \\
&= \frac{d(\phi^{-1} \circ \gamma)^\mu}{d\lambda} 
\frac{\partial (f \circ \phi^{-1}) }{\partial x^\mu} \big|_p \\
&= \frac{dx^\mu}{d\lambda} \partial_\mu f \big|_p
\end{align*}

where $x^\mu$ are the coordinates on $\mathcal{R}^n$.  Finally, since
the function $f$ is arbitrary we can define

\begin{equation}
\label{eq:tangent_vector}
\frac{d}{d\lambda} = \frac{dx^\mu}{d\lambda} \partial_\mu
\end{equation}
\fi

Geometrically, taking the derivative gives the tangent vector to the
curve at a point $p$.  It is possible to associate the set of such
vectors with the set of directional derivatives, taking the partial
derivatives along the coordinates as the basis.  Henceforth this basis
will be denoted both $\partial_\mu$ and $\vec{e}_\mu$.  Note that the
tangent to a curve is defined at the point $p$.  Each point in the
manifold possesses its own space of tangent vectors.  These spaces are
distinct, which will be important in what follows.

We next define \emph{one-forms} as linear maps from vectors to
$\mathcal{R}$.  The set of one-forms at a point can be shown to form a
vector space, a natural basis for which can be obtained by requiring
%
\begin{equation*}
\vec{e}_\mu \tilde{\omega}^\nu = \delta_\mu^\nu
\end{equation*}
%
The components of an arbitrary form $\omega$ in this basis may be
found by applying the form to the basis vectors.

\iffalse
\begin{align*}
\omega(\vec{e}_\nu)
&= \tilde{\omega}^\mu \omega_\mu (\vec{e}_\nu) \\
&= \omega_\mu \tilde{\omega}^\mu (\vec{e}_\nu) \\
&= \omega_\mu \delta^\nu_\mu \\
&= \omega_\nu\\
\end{align*}
\fi

We can then build up arbitrary ${m \choose n}$ tensors as linear maps
from tensor products of $m$ vectors and $n$ one-forms to
$\mathcal{R}$.  The components of a tensor $T$ in a choice of
coordinates may found by applying it to combinations of the basis
vectors and basis 1-forms.  Finally, a ${m \choose n}$ tensor field is
a map that associates to each point $p$ in $\mathcal{M}$ an element in
the space of  ${m \choose n}$ tensors at $p$.

\subsection{The Metric Tensor}

A particularly important tensor in general relativity is the
\emph{metric}, a ${2 \choose 0}$ tensor that is symmetric ($g_{\mu\nu}
= g_{\nu\mu})$ and non-degenerate (the determinant of $g$ taken as a
matrix $|g_{\mu\nu}| \neq 0$.  The latter feature makes it possible to
define the inverse metric $g^{\mu\nu}$ as
%
\begin{equation*} g^{\mu_\rho} g_{\rho_\nu} = \delta^\mu_\nu
\end{equation*}
%
Given a vector $x^\mu$ the object $g_{\mu\nu} x^\mu$ maps another
vector to a real number, and is therefore a one-form.  The metric
therefore maps between the space of one-forms and the space of vectors
at each point.  Most importantly, the metric defines a notion of
distance on the manifold.  Infinitesimally
%
\begin{equation} ds^2 = g_{\mu\nu} dx^\mu dx^\nu \end{equation}
%
In special relativity, in Cartesian coordinates, the metric has
components $(-1,1,1,1)$ along the diagonal, all other components are
zero.   The metric will be denoted $\eta_{\mu\nu}$ and called the
\emph{flat space metric}.

\subsection{Covariant Derivatives}

Since vectors are only defined at a point we need additional structure
to define derivatives of vector fields, as there is no natural way to
compare vectors that live in different spaces.  We seek an operator
$\nabla$ with the following properties
%
\begin{itemize}
\item Maps ${m \choose n}$ tensors to ${m \choose {n+1}}$ tensors.
This is so we may consider the directional derivative of a tensor $T$
along a vector $x$ as $x^\mu \nabla_\mu T$.
\item Reduces to partial differentiation when applies to a scalar
field.
\item Linear.
\item Satisfies the Leibniz rule, $\nabla(a b) = a\nabla b + b \nabla a$.
\end{itemize}
%
Such an operator applied to a vector field gives
%
\begin{equation*}
\nabla_\mu (x^\nu \vec{e}_\nu)
= (\partial_\mu x^\nu) \vec{e}_\nu + x^\nu (\nabla_\mu
\vec{e}_\nu)
\end{equation*}
%
In flat space in Cartesian coordinates the basis vectors do not
change and so the last term is zero.  But in a curved space, or even
flat space in non-Cartesian coordinates, they may.  However, the new
vector must be expressible as a linear combination of the original
basis vectors.  The components are called \emph{connection
coefficients} and are denoted as $\Gamma^\rho_{\nu\mu}$ so
%
\begin{align}
\label{eq:covariant_derivative}
\nabla_\mu (x^\nu \vec{e}_\nu) &= 
(\partial_\mu x^\nu) \vec{e}_\nu + 
x^\nu \Gamma^\rho_{\mu\nu} \vec{e}_\rho \\
&= (\partial_\mu x^\nu + x^\rho \Gamma^\nu_{\mu\rho}) \vec{e}_\nu \\
\nabla_\mu x^\nu &= \partial_\mu x^\nu + x^\rho \Gamma^\nu_{\mu\rho}
\end{align}
%
In general relativity the connection is usually assumed to be
\emph{torsion-free}, that is
%
\begin{equation}
\label{eq:torsion}
 \Gamma^\nu_{\mu\rho} =  \Gamma^\nu_{\rho\mu}
\end{equation}
%
which thus far has been borne out by experiment.  However, it is
possible to construct theories where this condition does not hold.

We will henceforth occasionally use commas and semicolons to denote
partial and covariant differentiation, respectively:
%
\begin{align*}
{x^\mu}_{,\nu} &\equiv \partial_\nu x^{\mu} \\
{x^\mu}_{;\nu} &\equiv \nabla_\nu x^{\mu} \\
\end{align*}

By considering the covariant derivative of a scalar constructed from a
one-form acting on a vector, $\nabla (x^\nu \omega_\nu)$, it can be shown
that
%
\begin{equation*}
\nabla_\mu \omega_\nu = \partial_\mu \omega_\nu - 
\Gamma^\rho_{\mu\nu} \omega_\rho
\end{equation*}
%
The covariant derivative of a ${m \choose n}$ tensor generalizes this
and has a partial derivative term, $m$ positive therms in $\Gamma$ and
$n$ negative terms in $\Gamma$.

\subsection{Parallel Transport}
\label{ssec:parallel}

Covariant differentiation provides a way to ``move a vector without
changing it.''  We can \emph{parallel transport} a vector $v^\mu$
infinitesimally along a curve whose tangent vector is $u^\nu$ by
requiring
%
\begin{equation*}
u^\nu \nabla_\nu v^\mu = 0
\end{equation*}
%
As an example of such transport, consider an arrow on the equator of
the Earth pointing towards the north pole.  This arrow can be carried
halfway around the equator without rotating it, so it ends up on the
other side of the globe, still pointing north.  If the vector is then
parallel transported northward to the pole and then continued until it
returns to its starting point it will return pointing south.  Although
the vector was never rotated locally it has returned rotated.  This is
an indication that the underlying space is curved.

Of particular interest is the case where a vector is
parallel-transported along itself
%
\begin{equation*}
0 = v^\mu \nabla_\mu v^\nu 
= v^\mu (\partial_\mu v^\nu + \Gamma^\nu_{\mu\rho} v^\rho)
\end{equation*}
%
Now consider a curve $x(\lambda)$ such that $v$ is the tangent to this
curve, $v^\mu = d x^\mu/d\lambda$, then
%
\begin{align}
\label{eq:geodesic}
\frac{d x^\mu}{d\lambda}
  \frac{\partial }{\partial x^\mu}
  \frac{d x^\nu}{d\lambda}  
+ \Gamma^\nu_{\mu\rho} 
\frac{d x^\mu}{d\lambda}
\frac{d x^\rho}{d\lambda} &= 0 \nonumber \\
\frac{d^2 x^\nu}{d\lambda^2}
+ \Gamma^\nu_{\mu\rho} 
\frac{d x^\mu}{d\lambda}
\frac{d x^\rho}{d\lambda} &= 0 \nonumber \\
\end{align}
%
This is the \emph{geodesic equation}, solutions to which are
\emph{geodesics}.  The same equation can be derived by extremizing the
path length, $\sqrt{g_{\mu\nu} dx^\mu dx^\nu}$.  In general relativity
test masses acting under the influence of gravity and no other forces
follow geodesics.  

\subsection{The Christoffel Symbols}

If we now require that scalars do not change under parallel transport
we have, for arbitrary vectors fields $u^\alpha, v^\beta$ and $x^\mu$
%
\begin{align*}
0 &= x^\mu \nabla_\mu (g_{\alpha\beta} u^\alpha v^\beta) \\
&= x^\mu (\nabla_\mu g_{\alpha\beta}) u^\alpha v^\beta
+ g^{\alpha\beta} (x^\mu \nabla_\mu u^\alpha) v^\beta
+ g^{\alpha\beta} u^\alpha (x^\mu \nabla_\mu v^\beta)
\end{align*}
%
We can now specialize such that $u^\alpha, v^\beta$ are constant
and so the last two terms vanish, which implies the  \emph{metric
compatibility} condition:
%
\begin{equation}
\label{eq:metric_compatibility}
\nabla_\mu g_{\alpha\beta} = 0
\end{equation}

Equations~\ref{eq:metric_compatibility} and~\ref{eq:torsion} together
fix the connection coefficients in terms of the metric:
%
\begin{equation}
\Gamma^{\rho}i_{\mu\nu}
= \frac{1}{2} g^{\rho\sigma}\left[
\partial_\nu g_{\mu\sigma}
+ \partial_\mu g_{\nu\sigma}
- \partial_\sigma g_{\mu\nu}
\right]
\end{equation}
% 
Combining this with the previous section we see that the motion of
particles are completely specified once we know the metric and their
initial positions and velocities.

\subsection{The Riemann Tensor}

We now generalize the example given in Sec.~\ref{ssec:parallel}, and
ask how a vector $A^\mu$ changes as it is parallel-transported around
an infinitesimal parallelogram with sides defined by the vectors
$B^\mu$ and $C^\nu$.  Recalling that vectors and directional
derivatives are the same thing, it can be shown that this is
equivalent to asking how covariant derivatives fail to commute.  The
result must be linear in $A^\mu$ and so we may write
%
\begin{equation}
\label{eq:riemann_def}
\left[\nabla_\mu \nabla_\nu - \nabla_\nu \nabla_\mu\right] A^\rho
= R^\rho_{\sigma\mu\nu} A^\sigma
\end{equation}
%
which defines the \emph{Riemann tensor} $R^\rho_{\sigma\mu\nu}$.  A
number of properties follow from this definition (which are either
obvious or may be proven by substituting the definition of the
covariant derivative, eqn.~\ref{eq:covariant_derivative}).  First,
the symmetry properties
%
\begin{equation}
\label{eq:symmetries}
R_{\rho\sigma\mu\nu}
= -R_{\sigma\rho\mu\nu}
= -R_{\rho\sigma\nu\mu}
= R_{\mu\nu\rho\sigma}
\end{equation}
%
which in turn imply
%
\begin{align}
R^\rho_{[\sigma\mu\nu]} = 0
\end{align}
%
Second, the Bianchi identity,
%
\begin{equation}
\label{eq:bianchi}
R_{\rho\sigma\mu\nu;\alpha}
+R_{\rho\sigma\nu\alpha;\mu}
+R_{\rho\sigma\alpha\mu;\nu} = 0
\end{equation}

We may now generalize Eqn.~\ref{eq:riemann_def} and ask how an
arbitrary tensor changes after being parallel-transported 
around a loop.  It can be shown that
%
\begin{equation}
\label{eq:higher_order_riemann}
\left[\nabla_\mu \nabla_\nu - \nabla_\nu \nabla_\mu\right] 
B^{\rho_1 \rho_2 \ldots \rho_n}
= - R^{\rho_1}_{\sigma \mu\nu} B^{\sigma \rho_2 \ldots \rho_n}
- R^{\rho_2}_{\sigma \mu\nu} B^{\rho_1 \sigma \ldots \rho_n}
- \ldots -
- R^{\rho_n}_{\sigma \mu\nu} B^{\rho_1 \rho_2 \ldots \sigma }
\end{equation}
%
which may be proved by expanding
%
\begin{equation*}
\left[\nabla_\mu \nabla_\nu - \nabla_\nu \nabla_\mu\right] 
(\vec{e}_\rho \otimes \vec{e}_\sigma)
\end{equation*}
%
and then proceeding by induction.  

The symmetries of the Riemann tensor imply that there is, up to sign,
only one non-trivial contraction
%
\begin{equation}
R_{\mu\nu} = R^\sigma_{\mu\sigma\nu}
\end{equation}
%
which defines the \emph{Ricci tensor}.  This may be contracted again
%
\begin{equation}
R = R^\mu_\mu
\end{equation}
%
to obtain the \emph{Ricci scalar}.

Now, contracting the Bianchi identity twice gives
%
\begin{equation*}
g^{\sigma\nu}
\left(\nabla_\alpha R_{\sigma\nu}
+ \nabla^\rho R_{\rho\sigma\nu\alpha}
+ \nabla_\nu R^\mu_{\sigma\alpha\mu}\right) = 0
\end{equation*}
%
Using the symmetries of the Riemann tensor (eqn.~\ref{eq:symmetries})
this can be written
%
\begin{equation*}
\nabla_\alpha R
- \nabla^\rho R_{\rho\alpha}
- \nabla^\sigma R_{\sigma\alpha} = 0
\end{equation*}
%
Relabeling the dummy indices and using metric compatibility gives
%
\begin{equation*}
\nabla^\rho \left(g_{\rho\alpha} R - 2 R_{\rho\alpha} \right) = 0
\end{equation*}
%
This motivates the definition of the \emph{Einstein Tensor} as
%
\begin{equation}
\label{eq:einstein_tensor}
G_{\mu\nu} = R_{\mu\nu} - \frac{1}{2} g_{\mu\nu} R
\end{equation}
%
The previous result implies this is divergentless
%
\begin{equation*}
\nabla^\nu G_{\mu\nu} = 0
\end{equation*}
%
Note also that $G$ is symmetric, $G_{\mu\nu} = G_{\nu\mu}$.

We now relate this to physics by noting that the matter and energy
content of a region is described by the stress-energy tensor
$T_{\mu\nu}$ where each component is ``the flow of $\mu$ momentum in the
$\nu$ direction.''  For example, the $0,0$ component is energy density
and the $0,i$ components are the $i^\mathrm{th}$ components of
momentum.

Conservation of energy requires that the difference in momentum
($\p^i$) across each face of a cube be balanced by a change of energy
($\rho$), within the cube,
%
\begin{equation*}
\partial_t \rho = \partial_i p^i
\end{equation*}
%
In terms of the stress-energy tensor this becomes
%
\begin{equation*}
0 = - \nabla^0 T_{00} \nabla^i T_{0i} = 0
= \nabla^\nu T_{0 \nu}
\end{equation*}
%
However the time direction is not uniquely specified as a change of
coordinates will mix space and time components, so this must
generalize to 
%
\begin{equation*}
\nabla^\nu T_{\mu\nu} = 0
\end{equation*}
%
That is, $T$ is also divergentless, like $G$, and like $G$ it is also
symmetric.  It is therefore reasonable to suggest the ansatz
%
\begin{equation*}
G_{\mu\nu} \propto T_{\mu\nu}
\end{equation*}
%
Requiring agreement with Newton's law of gravity in the appropriate
low-energy limit ($T_{00} \gg$ all other components) fixes the
constant of proportionality and gives us \emph{Einstein's field
equation}
%
\begin{equation}
\label{eq:einsteins_equation}
G_{\mu\nu} = 8\pi T_{\mu\nu}
\end{equation}
%
Note that $G_{\mu\nu}$ is entirely determined by the metric.
Equation~\ref{eq:einsteins_equation} may therefore be thought of as a
set of coupled, non-linear differential equations for $g_{\mu\nu}$.

\section{Gravitational Radiation}
\label{sec:gravitational_radiation}

We now move to the prediction of gravitational waves.  We begin with
Einstein's equation in empty space,
%
\begin{equation*}
G_{\mu\nu} = R_{\mu\nu} - \frac{1}{2} g_{\mu\nu} R = 0
\end{equation*}
%
By taking the trace and substituting into Eqn.~\ref{eq:einsteins_equation}
it can be shown that this implies that in empty space $R_{\mu\nu} =
0$.  Similary, using the Bianchi identity and symmetries of the Riemann tensor gives,
in empty space,
%
\begin{equation}
\label{eq:divergence_in_empty_space}
R_{\beta\delta;\gamma}  -R_{\beta\gamma;\delta} = 0
\end{equation}

We next consider the application of the wave operator to the Riemann
tensor.  From the Bianchi identity (eqn.~\ref{eq:bianchi}) this becomes
%
\begin{equation*}
\label{eq:wave_expanded}
g^{\mu\nu} R_{\alpha\beta\gamma\delta;\mu\nu}
= - g^{\mu\nu}
\left[R_{\alpha\beta\delta\mu;\gamma\nu}
+ R_{\alpha\beta\mu\gamma;\delta\nu} \right]
\end{equation*}
%
Consider the first term on the right-hand side:
%
\begin{align*}
g^{\mu\nu} R_{\alpha\beta\delta\mu;\gamma\nu}
&= g^{\mu\nu} R_{\alpha\beta\delta\mu;\nu\gamma}
+ g^{\mu\nu} R_{\alpha\beta\delta\mu;\gamma\nu}
- g^{\mu\nu} R_{\alpha\beta\delta\mu;\nu\gamma} \\
&= g^{\mu\nu} R_{\alpha\beta\delta\mu;\nu\gamma}
+ g^{\mu\nu} 
\left[\nabla_\nu,\nabla_\gamma\right] R_{\alpha\beta\delta\mu}
\end{align*}
%
The first term vanishes by
Eqn.~\ref{eq:divergence_in_empty_space}.  The second term involves
products of the Riemman tensor by~\ref{eq:higher_order_riemann}.  The
second term on the right in Eqn.~\ref{eq:wave_expanded} has the
same form.

We now specialize to the case where the Riemann tensor is small, so
that terms involving multiple factors can be neglected.  This is
equivalent to considering the Riemann tensor as a field on a flat
background.  This gives
%
\begin{equation}
\label{eq:riemann_wave}
g^{\mu\nu}
R_{\alpha\beta\gamma\delta;\mu\nu}
=
\Box R_{\alpha\beta\gamma\delta}
= 0
\end{equation}
%
That is, each component of the Riemann tensor independently 
satisfies the vacuum wave equation.  We can immediately write the
solution:
%
\begin{equation}
R^\alpha_{\beta\gamma\delta} = 
\textrm{Re}\, A^\alpha_{\beta\gamma\delta} \exp(i k_\mu x^\mu)
\end{equation}
%
where $A^\alpha_{\beta\gamma\delta}$ is a set of amplitudes and
$k^\mu$ is the wave vector.  In a chosen coordinate system it has
components $(\omega, k_x, k_y, k_z)$ where $\omega$ is the angular
frequency and the spacial $k$ components are wavelengths in each
direction.  It can be shown that 
%
\begin{align*}
\nabla_{\vec k} \vec{k} &= 0 \\
k_\mu k^\mu &= 0 \\
\end{align*}
%
which together imply that gravitational waves travel along geodesics 
at the speed of light.


\section{Effect of Gravitational Waves}
\label{sec:effects_of_waves}

We now derive the effect of gravitational waves on matter.  Consider
two particles moving along world-lines $A^\mu$ and $B^\mu$.  Choose
coordinates so that $A$ remains fixed at the origin, $A^\mu =
(1,0,0,0)$.  We may further specialize our coordinates such that at
the origin $g_{\mu\nu} = \eta_{\mu\nu}$.  It can be shown that we may
also require that the first derivatives of the metric vanish at this
point.  We may not, however, make the second derivatives vanish in
general.  This corresponds to the fact that the Riemann tensor is
defined in terms of second derivatives.  We call the coordinate system
thus constructed a \emph{Local Lorentz Frame}.

We now define the separation between the two particles as 
%
\begin{equation*}
\xi^\mu = B^\mu - A^\mu
\end{equation*}
%
We fix $\xi$ to be perpendicular to $A$, so that $\xi^0 = 0$.  If space
is curved it can readily be seen that $\xi$ will not remain constant.
For example, if the particles are initially at rest some distance from
the surface of the Earth they will both move towards the center of the
Earth and $\xi$ will decrease.  It can be shown that $\xi$ obeys the
equation of \emph{geodesic deviation},
%
\begin{equation}
\label{eq:geodesic_deviation}
\frac{d^2}{dt^2} \xi^\rho = -R^\rho_{\mu\nu\sigma} A^\mu \xi^\nu A^\sigma
=-R^\rho_{0 \nu 0} \xi^\nu 
\end{equation}
%
Using the condition that $\xi$ has no time component reduces this to 
%
\begin{equation}
\frac{d^2}{dt^2} \xi^i = -R^\rho_{\mu\nu\sigma} A^\mu \xi^\nu A^\sigma
=-R^i_{0 j 0} \xi^j
\end{equation}
%
That is, the change in separation between two
infinitesimally-separated test masses at rest with respect to each
other in an arbitrary gravitational field is entirely specified by
$R^i_{0 j 0}$.  From the symmetries of the Reimann tensor this is
symmetric in $i$ and $j$, and hence appears to have 6 independent
components.  However, it can be shown that these can entirely be
specified by two values, which without loss of generality we take to
be $R^x_{0 x 0}$.  $R^x_{0 y 0}$.  We now recall that in empty space
$R_{\mu\nu} = 0$, which implies that $R^y_{0 y 0} = - R^x_{0 x 0}$.
We summarize this by saying that R is \emph{traceless}.  

We now specialize to the case of gravitational waves, so that the
Riemann tensor satisfies Eqn.~\ref{eq:riemann_wave} and we choose
coordinates such that the wave is traveling in the $z$ direction.
Using the fact that the speed of light is 1 in dimensionless units the
solution can then be written
%
\begin{equation*}
R_{i 0 j 0} = A_{ij}(t - z)
\end{equation*}
%
where we have lowered the first index to simplify notation.

Now, using the fact that $\partial_x R_{i0j0} = \partial_y R_{i0j0}
= 0$ and integrating the Bianchi identity we can show that
$R_{x 0 y 0} = R_{y 0 x 0}$ and that all other components vanish.
It can also be shown that in addition to being traceless $R$ is
\emph{transverse}, $k^j R_{i0j0} = 0$.  We denote these two facts by
adding the superscript $TT$, and define the gravitational-wave field as
%
\begin{equation}
\label{eq:wave_field}
-\frac{1}{2} \frac{\partial^2 h_{ij}^{TT}}{\partial t^2}
\equiv R^{TT}_{i0j0}
\end{equation}

We now decompose the separation vector $\xi$ into the initial
separation and a time-dependant perturbation, $\xi = \xi_0 + \delta
\xi$.  In terms of this Eqn.~\ref{eq:geodesic_deviation} becomes
%
\begin{equation}
\label{eq:geodesic_deviation_delta}
\frac{d^2}{dt^2} \delta \xi^i = -R^{0 i 0 j} \xi_0^j
\end{equation}
%
where we drop the initial portion from the left-hand side because it
is constant, and we drop the perturbation from the right hand side
because it is much smaller than the initial portion.  Comparing
eqn.~\ref{eq:wave_field} and eqn.~\ref{eq:geodesic_deviation_delta}
we obtain the equation for the effect of a gravitational wave on
free-falling test masses:
%
\begin{equation}
\label{eq:wave_effect}
\delta \xi^i = \frac{1}{2} h^{TT}_{ij} \xi^j
\end{equation}
%
We note in passing that this is the same result obtained in other
treatments by expanding the metric in terms of the flat-space metric
plus a perturbation, $g_{\mu\nu} = \eta_{\mu\nu} + h_{\mu\nu}$,
substituting into the Einstein equation and expanding to first order
in $h$, and then choosing a gauge in which $h$ is transverse and
traceless.

Now, define
%
\begin{align*}
h_+ &\equiv h_{xx} = - h_{yy} \\
h_\times &\equiv h_{xy} = h_{yx}
\end{align*}
%
which we refer to as the \emph{plus} ($+$) and \emph{cross} ($\times$)
polarizations, respectively.  Consider the case where $h_\times = 0$.
If particle $B$ is initially on the $x$ axis then 
%
\begin{align*}
\delta \xi^x &= \frac{1}{2} h^{TT}_{xx} \xi^x \\
\delta \xi^y &= \frac{1}{2} h^{TT}_{xy} \xi^y \\
&= 0
\end{align*}
%
The particle remains on the $x$ axis.  For an oscillatory wave the
distance between the two particles likewise oscillates.  We can
describe this as an induced \emph{strain}, $\Delta L/L$ where $L$ is
the initial separation.  If $B$ is initially on the $y$ axis
%
\begin{align*}
\delta \xi^x &= \frac{1}{2} h^{TT}_{xx} \xi^x \\
&= 0 \\
\delta \xi^y &= \frac{1}{2} h^{TT}_{yy} \xi^y \\
&= - \frac{1}{2} h^{TT}_{xx} \xi^y
\end{align*}
%
The particle remains on the $y$ axis and oscillates out of phase with
a corresponding particle on the $x$ axis.  The net effect is that,
after a quarter cycle, a set of masses initially in a circle are moved
into an ellipse flattened along one axis and stretched along the other
such that the area remains constant.  After another quarter cycle they
return to a circle, in the next quarter cycle they are in an ellipse
with the axes flipped, and so on. It is similarly straightforward to
show that for a wave cross-polarized wave the eigendirections are on
the lines $x=\pm y$.  The effects are the same as for the $+$
polarization, rotated 45 degrees.

Now consider a thought experiment, originally due to Feynman, where we
place a bead on a stick in the path of a gravitational wave.  The wave
will cause the bead to slide back and forth, heating the stick trough
friction and imparting energy to the system.  This implies that
gravitational waves carry energy.  This argument can be made
precise~\cite{RevModPhys.29.509}.

\section{Modeling Gravitational Waves}

Having demonstrated the predicted existence of gravitational waves and
their effects on matter we now turn to the question of their
generation.  From Eqn.~\ref{eq:wave_field} we can see that, since
the components of the Riemann tensor obey the wave equation, the
components of $h^{TT}$ do as well.  We now consider the solution of
the wave equation when the source, the stress-energy tensor, is not
zero.  By analogy with electromagnetism we can immediately write down
the solution in terms of retarded fields
%
\begin{equation}
\label{eq:h_from_t}
h^{TT}_{ij} = 4 \left[ \int \frac{T_{ij}(x', t-r)}{r}\, d^3 x'
\right]^{TT}
\end{equation}
%
where we integrate over the source distribution $x'$.  When the energy
and momentum densities are small, so that the curvature of spacetime
is likewise small, the coordinates may be taken to have their
conventional flat-space meaning.  We may also replace the covariant
derivative by regular partial differentiation.

Starting from the conservation of energy and momentum written in terms
of the stress-energy tensor
%
\begin{align*}
{T^{00}}_{,0} &= - {T^{0j}}_{,j} \\
{T^{i0}}_{,0} &= - {T^{ij}}_{,j} \\
\end{align*}
%
Differentiating with respect to time and going through some algebra
yields expressions for the first and second moments of the stress
energy tensor, ${T^{lm}}_{,ml} x^j x^k$ and ${T^{jm}}_{,m} x^k$.  These,
plus Stoke's theorem, can be used to simplify
Eqn.~\ref{eq:h_from_t} to give
%
\begin{equation*}
h^{TT}_{ij} = \frac{2}{r} \left[
\int {T^{00}}_{,00}\, x^i x^j d^3 x' \right]^{TT}
\end{equation*}
%
We recognize $T^{00}$ as the mass/energy density, and therefore this
can be written as the \emph{quadrapole formula}
%
\begin{equation}
\label{eq:quadrupole_formula}
h^{TT}_{jk} = \frac{2}{r} \ddot{\mathcal{I}}(t-r)
\end{equation}
%
where $\mathcal{I}$ is the quadrupole moment of the source
%
\begin{equation*}
\mathcal{I}_{ij} = \int \rho(\mathbf{x})x_i x_j\,d^3 x
\end{equation*}

Any system of mass with a time-dependant quadrupole moment will give
rise to gravitational radiation.  However, restoring physical units to
Eqn.~\ref{eq:quadrupole_formula} scales the right-hand side by
$G/c^4$.  We therefore need very large masses and/or rapid changes in
order to generate waves we have any chance of detecting.  There are
many such sources of astrophysical and cosmological interest:
%
\begin{itemize}
\item supernovae
\item rotating neutron stars with axial asymmetry
\item processes in the early universe, which would have produced
\emph{relic} gravitational waves in principle still detectable today
\item topological defects
\item compact bodies, such as neutron stars and black holes, in 
orbit around each other.
\end{itemize}
%
We will focus on the last of these for the remainder of the thesis.

It is straightforward to start from the mass distribution of two point
masses in orbit around their center:
%
\begin{align*}
\rho(\mathbf{x}) &= m_1\delta(x - r_1\cos(\Omega t)) \delta(y-r_1
\sin(\Omega t)) \delta(z) \\
&\quad + m_2\delta(x + r_2\cos(\Omega t)) \delta(y + r_2 \sin(\Omega t))
\delta(z)
\end{align*}
%
and calculate $h^{TT}$. By choosing coordinates centered on a
terrestrial gravitational-wave detector and a basis we can write
the gravitational-wave strains as~\cite{DBrownThesis}
%
\begin{align}
\label{eq:h_plus_cross}
h_+(t)   &= - \frac{2G}{c^4 r} \mu (\pi G M f)^{\frac{2}{3}}
(1+\cos^2(\iota)) \cos(2\pi f t - 2\phi_0) \\ \nonumber
h_\times(t)  &= - \frac{4G}{c^4 r} \mu (\pi G M f)^{\frac{2}{3}}
\cos(\iota) \sin(2\pi f t - 2\phi_0) \nonumber
\end{align}
%
where $M = m_1+m2$, $\mu = m_1 m_2 / M$, $f$ is the
\emph{gravitational-wave frequency} which is twice the orbital
frequency $f = 2\Omega/2\pi$, $\phi_0$ is the orbital phase at a
specified time, and $\iota$ is the \emph{inclination} of the binary
with respect to the detector, the angle between the normal to the
plane of the binary and the line joining the detector to the binary.

We noted above that gravitational waves carry energy.  This energy
must be balanced by a loss of energy by the system.  This energy can
not be localized to any one point of the wave, since any point can be
placed in a Local Lorentz Frame where there is no wave.  However, by
averaging over a cycle it can be shown~\cite{carrollTextbook} that
%
\begin{equation}
t_{\mu\nu} = \frac{1}{32 \pi} \left\langle h^{TT}_{\rho \sigma,\mu}
(h^{TT})^{\rho\sigma}_{,\nu} \right\rangle
\end{equation}
%
where $t_{\mu\nu}$ is the \emph{stress-energy pseudo-tensor}.  The
stress energy tensor itself is zero in a region of spacetime
containing only gravitational radiation.  However, $t$ may be used to
describe the energy content of such radiation, and we find the flux of
energy in the radial direction out of a sphere enclosing a gravitating
system is
%
\begin{equation}
\label{eq:energy_loss}
\frac{dE}{dt} = T_{0r} = \frac{1}{8 \pi r^2} \left\langle \ddot{\mathcal{I}}^{ij}
\ddot{\mathcal{I}}_{ij} \right\rangle
\end{equation}

When the separation between the masses, $a$ is large and the masses are
moving slowly the gravitational energy of the system is approximately
given by the Newtonian expression
%
\begin{equation*}
E = - \frac{1}{2} \frac{G \mu M}{a}
\end{equation*}
%
Therefore, orbiting bodies will emit gravitational radiation and
\emph{inspiral} until they inevitably collide.

\section{Post-{N}ewtonian Approximations}
\label{sec:PNWaveforms}

Substituting Eqn.~\ref{eq:h_plus_cross} into
Eqn.~\ref{eq:energy_loss} shows that the power emitted goes as
$a^{-5}$.  Most of the power from an inspiral, and therefore our best
chance of detecting such systems, occurs when the masses have become
close and are about to merge.  The approximations we made in the
previous section are no longer valid in this regime.  In order to
obtain analytic models of gravitational waves from inspirals to a
precision that will be useful in searches we must therefore consider
higher-order corrections beyond Newtonian mechanics of the binary.
This leads to \emph{post-Newtonian} (pN) waveforms.

The key to doing this is the requirement that the energy flux carried
away by gravitational radiation must be balanced by loss of energy of the
system
%
\begin{equation}
\label{eq:equivilent_exchange}
\frac{dE}{dt} = - \mathcal{F}
\end{equation}
%
It can be shown~\cite{MTW} that the energy of a body of mass $\mu$ moving
on a geodesic in the Schwarzschild metric with mass $M$ is
%
\begin{equation}
\label{eq:hamiltonian}
E = \mu \frac{1-2M}{\sqrt{1-3M/r}} 
\end{equation}
%
We now introduce the parameter $v = (\pi M f)$, which is the velocity
in Newtonian mechanics.  We now consider the \emph{adiabatic
approximation}, where we treat the system as moving through a series
of circular orbits.  On a circular geodesic $v= \sqrt{M/r}$ and the
energy becomes
%
\begin{equation}
E = \mu \frac{1-2v^2}{\sqrt{1-3v^2}} 
\end{equation}

We have written down the flux to first pN order above in
Eqn.~\ref{eq:energy_loss}, in terms of the parameter $v$ it is
%
\begin{equation}
\mathcal{F} = \frac{32}{5} \left( \frac{\mu}{M} \right) v^{10}
\end{equation}
%
Going to higher order requires techniques that are beyond the scope of
this thesis.

Both the energy and flux may now be expanded in powers of $v$.  We
could then obtain $t$ as a function of $v$ by rewriting
Eqn.~\ref{eq:equivilent_exchange} and integrating.  However, we
will instead obtain an expression relating time and the orbital phase
$\phi = \int 2\pi f\, dt$ by writing
%
\begin{equation*}
\mathcal{F} = - \frac{dE}{dv} \frac{dv}{d\phi} \frac{d\phi}{dt}
\end{equation*}
%
which leads to the expression
%
\begin{equation}
\label{eq:expansion_for_phi}
\phi = \phi_0 + \frac{2}{M} \int_v^{v_\mathrm{ref}} \frac{v^3
dE/dv}{\mathcal{F}}\, dv
\end{equation}
%
where $v_\mathrm{ref}$ is the velocity at a given reference velocity,
or equivalently reference frequency.  If $\mathcal{F}$ is given as a
Taylor series in $v$ then $\mathcal{F}^{-1}$ may be expanded to the
same order.  $dE/dv$ may be similarly expanded, and the integrand
becomes the product of polynomials with rational coefficients.

Motivated by Eqn.~\ref{eq:h_plus_cross} we model the waveform as a
time-dependant amplitude and evolving phase:
%
\begin{equation}
\label{eq:general_waveform}
h(t) = A(t) \cos(\phi(t)) = \frac{1}{2} A(t) \left(e^{i\phi(t)} +
e^{-i\phi(t)} \right)
\end{equation}
%
For reasons that will become clear in the next chapter it is often more
convenient to work in the frequency domain, so we take the Fourier
transform of Eqn.~\ref{eq:general_waveform}
%
\begin{align}
\tilde{h}(f) &= \frac{1}{2} \int A(t) \exp(2\pi i f t + i
\phi(t))\, dt \\ \nonumber
&\qquad + \frac{1}{2} \int A(t) \exp(2\pi i f t - i
\phi(t))\, dt \\ \nonumber
\end{align}

We next apply the \emph{stationary-phase approximation}.  Oscillatory
integrands cancel except where the phase is stationary, at extrema of
the exponent.  We therefore expand the exponent around the time $t_0$
of the extremum and discard the linear term
%
\begin{equation}
2 \pi i f t + i \phi(t) \approx (2 \pi i f t_0 + i \phi(t_0))
+ \frac{1}{2} i \ddot{\phi}(t_0) (t-t_0)^2
\end{equation}

Substituting back, the first term leads to a constant factor and the
second leads to a Gaussian integral.  Going through the calculation
yields the approximate waveform
%
\begin{equation}
\label{eq:spa_waveform}
\tilde{f}(f) = \frac{2 G \msun}{c^2 r}
\left(\frac{5 \mu}{96 \msun} \right)^{\frac{1}{2}}
\left( \frac{M}{\pi^2 \msun} \right)^{\frac{1}{3}}
\left( \frac{G \msun}{c^3}\right)^{-\frac{1}{6}}
f^{-\frac{7}{6}} \Theta(f - f_c) e^{i \Psi(f;M,mu)}
\end{equation}
%
where $\msun$ is the mass of the sun and is introduced to so that $M$
is measured in solar masses, a useful unit in astrophysical work.
$\Psi$ is the result of the expansion~\ref{eq:expansion_for_phi}
written as a function of $f$.  The coefficients coming from the
expansion of the flux will depend on the mass $M$ and ration $\mu$ or
equivalently mass and \emph{symmetric mass ratio} $\eta = \mu/M$.  In
general waveforms may also depend on other parameters such as the
spins of the objects.  We introduce the step function to terminate the
waveform at a \emph{cutoff frequency} $f_c$.  This reflects the fact
that we do not trust this approximation all the way through the
evolution of the binary to merger.  The appropriate value for the
cutoff frequency is one of the questions we will address in
chapter~\ref{ch:comparison}.

There are many different ways to perform post-Newtonian expansions, in
both the time and frequency domains.  These lead to different
waveforms which have been assigned standard names, a summary of the
variations we will encounter throughout this thesis is given in
Appendix A.  We note here in particular that the stationary-phase
approach discussed above leads to waveforms denoted \emph{TaylorF2}.
The degree to which these different waveforms agree as been studied in
Ref.~\cite{BuonannoIyerOchsnerYiSathya2009}. 


\subsection{Effective One-Body}
\label{ssec:EOB}

A relatively recent development in post-Newtonian theory is the
\emph{effective one-body} (EOB)
formulation~\cite{BuonannoDamour:1999}.  The basic idea is familiar
from non-relativistic classical mechanics, where it is common to write
the Hamiltonian for the Kepler problem in terms of a particle with
reduced mass $\mu = m_1m_2 /(m_1 + m_2)$ moving in a central potential
due to a fixed body of mass $M = m_1 + m_2$.  

In general relativity we have noted that the energy, or equivalently
the Hamiltonian, takes the form of Eqn.~\ref{eq:hamiltonian} for
circular geodesics in a spherically symmetric spacetime.  The EOB
approach proceeds by seeking a map from the general two-body problem
to an equivalent spherically symmetric metric
%
\begin{equation}
ds^2 = -A(R) c^2 dT^2 + \frac{D(R)}{A(R)} dR^2 + R^2(d\theta^2 + sin^2 \theta
d\varphi^2)
\end{equation}
%
where the functions are written as expansions in $(GM/c^2R)$.  The
Hamiltonian of geodesic motion in this metric captures the
conservative portion of the motion (as can be seen by the fact that
the metric has no $T$ dependence).  Radiation-reaction terms can then
be added to this Hamiltonian to capture energy carried away by
gravitational waves.  This process results in coupled differential
equations which can be evolved numerically and from which the
gravitational waveforms can be obtained.

A key feature of the EOB approach is the use of \emph{Pad\'{e}
resummation}.  Given a function $f(x)$ with Taylor series $\sum a_i
x^i$ one ``resums'' the series into the ratio of polynomials $\sum b_i
x^i/\sum c^i x_i$ by expanding this ratio in a Taylor series and
matching the coefficients to the original series, then solving for the
$b_i$ and $c_i$.  The resulting function can converge to $f$ faster
than the Taylor series.

The expansion of the unknown functions in the metric to useful order
leads to unknown coefficients that can not be determined from the
problem alone.  These must be found by fitting the results to a
waveform obtained through other means.  This is one application of
waveforms from Numerical Relativity, discussed in the next section.
The resulting waveforms are called
\emph{EOBNR}~\cite{Buonanno:2009qa}.


\section{Numerical Relativity}
\label{sec:NRWaveforms}

Analytic methods themselves can not provide complete gravitational
waveforms valid through the merger of the two objects and the
evolution of the resulting single object to a steady state.  Even
where there is hope of being able to capture the full physics, as in
EOB or phenomenological models that have been developed, the
results must be tuned against complete waveforms which must be
calculated using other means.

There is a long history of solving partial differential equations
numerically on computers, and while Einstein's equations are in a
particularly difficult class of such equations, we can hope to obtain
complete solutions numerically.  This proved to be a remarkably
difficult problem, and progress through the 1990s was slow due to both
conceptual difficulties and limited computational power.  However,
following Pretorius' successful simulation of two merging black
holes~\cite{Pretorius:2005gq} and breakthroughs by researchers at
Goddard~\cite{Campanelli:2005dd} and the University of Texas at
Brownville and Florida Atlantic University~\cite{Campanelli:2005dd} in
2005 the field has expanded rapidly.  Readers are referred to the
textbook by Alcubierre~\cite{alcubierreTextbook} and review articles
by Husa~\cite{Husa:2007zz} and Hindler~\cite{Hinder:2010vn} for more
details than can be provided here.

The basis of most approaches in numeric relativity (hence ``NR'') is the
``3+1'' decomposition of Einstein's equation.  There is a similar
situation in electromagnetism.  Maxwell's equations may be expressed
in a geometric, four-dimensional form
%
\begin{align*}
F^{\mu\nu}_{,\nu} &= j^\mu \\
\mathcal{F}^{\mu\nu}_{,\nu} &= 0
\end{align*}
%
where $F$ is the electromagnetic field tensor and $\mathcal{F}$ is its dual.
However, this is not the most usual form in which to do calculations.
Instead we choose a time direction and rewrite in terms of separate
spatial and temporal differential operators,
%
\begin{align}
\label{eq:constraints1}
\nabla \cdot \mathbf{E} &= 4\pi \rho \\
\label{eq:constraints2}
\nabla \cdot \mathbf{B} &= 0 \\
\label{eq:evolution1}
\partial_t \mathbf{E}   &= \nabla \times \mathbf{B} - 4\pi j^\mu \\
\label{eq:evolution2}
\partial_t \mathbf{B}   &= - \nabla \times \mathbf{E} \\ \nonumber
\end{align}
%
Note that equations~\ref{eq:constraints1} and \ref{eq:constraints2}
are \emph{constraint equations} that impose conditions on valid
solutions at any given time, and in particular for initial conditions.
Equations~\ref{eq:evolution1} and \ref{eq:evolution2} are
\emph{evolution equations} which determine how to evolve the system.

Wile much work in numerical relativity includes matter, for the
purpose of this thesis  we restrict our attention to \emph{vacuum
spacetimes}, that is, situations where $T_{\mu\nu} = 0$ everywhere.
It is a remarkable fact that, although black holes are characterized
by a mass parameter, a spacetime containing only black holes and
gravitational waves is a vacuum.  We need therefore only be concerned
with general relativity, without considering the equations governing
any other fields.

In general relativity the split into space+time  is accomplished by
writing the metric as 
%
\begin{equation*}
ds^2 = -(\alpha^{2}-\gamma_{ij}\beta^{i}\beta^{j})dt^{2}
   + 2 \gamma_{ij}\beta^{j}dt\,dx^{i}
   + \gamma_{ij}dx^{i}\,dx^{j}, 
\end{equation*}
%
where $\gamma_{ij}$ is a metric on the slices of constant time $t$,
and the scalar function $\alpha$ and  vector field $\beta^i$ are
commonly used to encode the freedom of coordinate choice.  This
results in a division of the Einstein Eqn.~\ref{eq:einsteins_equation}
into constraint and evolution equations.  However, this split is not
uniquely determined and care must be taken to ensure that the
resulting systems are well-behaved.  In particular, there may be
issues ensuring that the constraints remain satisfied as the system
evolves.  While this is guaranteed mathematically in the continuum
limit it may not be true once the system is discretized.

Once a scheme for performing the 3+1 split has been decided upon it is
then necessary to find initial data that captures the situation of
interest and satisfies the constrain equations.  There are many ways
to do this.  Most approaches assume the spatial metric on the initial
slice is \emph{conformally flat}, meaning each point has a
neighborhood which can be mapped to flat space.  We have seen that at
any one point the metric can be made to be the flat-space metric with
vanishing first derivatives, but this does not extend into a
neighborhood in general.  In particular, NR systems do not include the
history of gravitational waves that would have been produced by the
system prior to the start of the simulation.  This deviation from the
true physics manifests as a burst of spurious ``junk radiation'' at
the start of the system.

The creation of initial conditions is complicated for black holes by
the presence of singularities, which can not be captured in a
simulation.  Most codes adopt the ``moving puncture'' approach.  The
idea is to consider black holes as ``wormholes'' with an internal
asymptotically flat region, and then compactify this region.  A
conceptually simpler approach is to simply excise the points near the
singularity from the computational grid, but care must be taken to
ensure that the excised region is causally disconnected from the
computational domain in order to avoid unphysical results.

There are a number of techniques for performing the evolution.  A
common approach is to evaluate the metric on a grid and replace
derivatives with differences.  An alternative is to use \emph{spectral
methods}, which involve expanding the solution in terms of a set of
basis functions, and then evolving the coefficients.  In this scheme
differentiation can be performed analytically.  In either scheme there
is a tradeoff between accuracy and performance.  A finer grid will
give more accurate results, but at the cost of higher computational
cost.  As simulations typically take weeks to months even on large
computer clusters, improving the performance is desirable.  Most codes
therefore use some form of \emph{adaptive mesh refinement}, employing
a finer grid only in regions close to the holes where the metric is
changing rapidly.

In order to extract gravitational-wave information most approaches use
the \emph{Newman-Penrose scalar}, which in vacuum may be written
%
\begin{equation*}
\Psi_4 = R_{\alpha \beta \gamma \delta} n^\alpha \bar{m}^\beta n^\gamma
\bar{m}^\delta
\end{equation*}
%
where $m$ and $n$ are vectors constructed from the basis vectors,
which in spherical coordinates are
%
\begin{align*}
n &= \frac{1}{\sqrt{2}} \left(\hat{t} - \hat{r}\right) \\
m &= \frac{1}{\sqrt{2}} \left(\hat{\theta} +i  \hat{\phi}\right)
\end{align*}
%
and $\bar{m}$ is the complex conjugate of $m$. The gravitational-wave
strain is related to $\Psi_4$ by two time derivatives,
%
\begin{equation*}
\Psi_4 = \ddot{h}_+ - i \ddot{h}_\times
\end{equation*}


\subsection{Hybrid Waveforms}
\label{sec:HybridWaveforms}

As noted, NR simulations require a great deal of time and resources;
typically a single simulation starting 10 orbits before the merger
will require a few weeks of runtime on approximately 50-100
processors~\cite{Scheel:2008rj}.  These requirements scale with the
length of the waveform extracted.  Long waveforms are therefore
prohibitively expensive; the longest currently available span about 30
cycles before merger.  Systems of astrophysical interest less massive
than about $36 \msun$ will spend many more cycles in the frequency
range to which Initial LIGO is most sensitive.  We therefore desire
much longer waveforms.  Post-Newtonian waveforms can not be extended
upwards into the late inspiral and merger, and NR waveforms can not be
extended downwards to the early inspiral.  However, we can consider
``stitching'' a pN waveform to an NR in order to create a \emph{hybrid
waveforms}.

This turns out to be possible, although as we will see in
chapter~\ref{ch:ninja2} there are subtleties in doing so that have not
yet been fully resolved.  One approach described 
in Ref.~\cite{Boyle2008a} is to work with $\Psi_4$ and match the numerical
waveform to the pN waveform by adjusting the time and phase offsets of
the pN waveform to minimize the quantity
%
\begin{equation}
  \label{eq:MatchingChiSquared}
  \Xi(\Delta t, \Delta \phi) = \int_{t_{1}}^{t_{2}}\, \left[
    \phi_{\mathrm{NR}}(t) - \phi_{\mathrm{pN}}(t - \Delta t) - \Delta \phi
  \right]^{2}\, d t \ .
\end{equation}
%
This technique has been shown~\cite{Boyle2007} to give good results
when the pN waveform is taken to be the time-domain \textit{TaylorT4}
waveform (see Appendix A) with terms up to 3.5-pN in phase and 3.0-pN
in amplitude.


\iffalse
% Taken from Boyle et. al, left here as a reminder of
% material to cover.

\section{NR}
\label{sec:NR}

Optimal searches for gravitational waves use matched filtering, which
requires accurate knowledge of the waveform~\cite{thorne.k:1987}.
Previous searches in LIGO data have used post-Newtonian and
phenomenological templates to search for the coalescence of black-hole
binaries~\cite{Abbott:2005pf,Abbott:2007xi,Abbott:2008}. Over the last
several years numerical relativity has made remarkable breakthroughs
in simulating the late inspiral, merger and ringdown of black-hole
binaries. The computational cost of these simulations is high,
however, making it impractical to use them directly as template
waveforms for use in a matched-filter search. It has been shown that
there is good agreement between the waveforms generated by numerical
relativity with analytic post-Newtonian waveforms to within just a few
orbits of merger~\cite{Buonanno-Cook-Pretorius:2007, Baker2006d,
  Pan2007, Buonanno2007, Hannam2007, Boyle2007, Gopakumar:2007vh,
  Hannam2007c, Boyle2008a, Mroue2008, Hinder2008b}.


\subsection{Hybrid Waveform}
\label{sec:HybridWaveform} %
Numerical simulations cannot simulate a very large portion of the
inspiral of a black-hole binary system.  Indeed, the longest such
simulation currently in the literature is the one used here---which
extends over just 32 gravitational-wave cycles before merger.
Fortunately, this is the only stage in which simulations are needed.
It has been shown previously~\cite{Boyle2007} that the
\textit{TaylorT4} waveform with 3.5-pN phase and 3.0-pN amplitude
matches the early part of this simulation to very high accuracy.  We
generate a \textit{TaylorT4} waveform of over 8000 gravitational-wave
cycles ($t \sim 1.2\times 10^{6}M$, starting at $M f=0.004$), and
transition between the two to create a hybrid.  This long waveform is
sufficient to ensure that---even for the lowest-mass systems we will
consider---the waveform begins well before it enters the frequency
band of interest to LIGO.

We begin with $\Psi_4$ data, which will later be integrated to obtain
$h$.  Following Ref.~\cite{Boyle2008a}, we match the numerical
waveform to the pN waveform by adjusting the time and phase offsets of
the pN waveform to minimize the quantity
\begin{equation}
  \label{eq:MatchingChiSquared}
  \Xi(\Delta t, \Delta \phi) = \int_{t_{1}}^{t_{2}}\, \left[
    \phi_{\NR}(t) - \phi_{\pN}(t - \Delta t) - \Delta \phi
  \right]^{2}\, d t \ .
\end{equation}
Here, we choose $t_{1}=900\,M$ and $t_{2}=1730\,M$, which is closer to
the beginning of the waveform than in the previous paper.  This
particular interval is chosen to begin and end at troughs of the small
oscillations due to the residual eccentricity $e\sim 5\times 10^{-5}$
in our numerical waveform.  Taking a range from trough to trough or
peak to peak---rather than node to node, for example---of the
eccentricity effects minimizes their influence on the matching.  The
eccentricity oscillations can be seen more easily after low-pass
filtering the waveform, though we find filtering to be unnecessary for
this paper.  The junk radiation apparent in the waveform as shown here
has no effect on the resulting match---as we have verified by
filtering, and redoing the match.  Because the final waveform will
incorporate no numerical data before $t_{1}$ and very little
immediately thereafter (as explained below), the junk radiation will
have no effect on any of our results---as we have also explicitly
verified.  In particular, by integrating $\Psi_{4}$ to obtain $h$, we
will effectively smooth the junk radiation.

%%%%%%%%%%%%%%%%%%%%%%%%%%%%%%%%%%%%%%%%%%%%%%%%%%%%%%%%%%%%%%%%%%%%%%
\begin{figure}
  \begin{center}
    \includegraphics[width=0.55\linewidth]{figures/comparison/PlotDifferences}
  \end{center}
  \caption{Amplitude and phase differences between the numerical and
    post-Newtonian waveforms, $\Psi_4$, that are blended to create the
    hybrid waveform.  The vertical lines at $900M$ and $1730M$ denote
    the region over which matching and hybridization occur.  Note that
    the agreement is well within the numerical accuracy of the
    simulation, represented by the horizontal bands, throughout the
    matching region.  Also note that the phase difference is fairly
    flat for a significant period of time after the matching range,
    which indicates that the match is not sensitive to the particular
    range chosen for matching.}
  \label{fig:MatchingPhaseComparison}
\end{figure}%
%%%%%%%%%%%%%%%%%%%%%%%%%%%%%%%%%%%%%%%%%%%%%%%%%%%%%%%%%%%%%%%%%%%%%%
In Fig.~\ref{fig:MatchingPhaseComparison} we compare the phase of the
numerical and pN waveforms.  The quantities plotted are
\begin{eqnarray}
  \delta \phi & \equiv & \phi_{\pN} - \phi_{\NR}\ , \\
  \frac{\delta A}{A} & \equiv & \frac{A_{\pN} - A_{\NR}}{A_{\NR}}\ , 
\end{eqnarray}
shown over the interval on which both data sets exist.  The vertical
bars denote the matching region.  Note that the phase difference is
well within the accuracy of the simulation (about 0.01 radians,
represented by the horizontal band) over a range extending later than
the matching region.  Also, the difference between the two is fairly
flat, which implies that the match is not very sensitive to the region
chosen for matching.  Because of this, we expect that the phase
coherence between the early pN data and the late NR data will be
physically accurate to high precision.

The hybrid waveform is then constructed by blending the two matched
waveforms together according to
\begin{eqnarray}
  \label{eq:HybridWaveform}
  A_{\hyb}(t) &= & \tau(t)\, A_{\NR} + \left[ 1 - \tau(t) \right]\,
  A_{\pN}(t)\ , \\
  \phi_{\hyb}(t) &= & \tau(t)\, \phi_{\NR} + \left[ 1 - \tau(t)
  \right]\, \phi_{\pN}(t)\ .
\end{eqnarray}
The blending function $\tau$ is defined by

\begin{equation}
  \label{eq:BlendingFunction}
  \tau(t) = \left\{\begin{array}{ll}
      0 & \mathrm{if}\quad t<t_{1}  \\
      \frac{t-t_{1}}{t_{2}-t_{1}} & \mathrm{if}\quad t_{1} \leq t < t_{2} \\
      1 & \mathrm{if}\quad t_{2} \leq t
    \end{array} \right.
\end{equation}

The values of $t_{1}$ and $t_{2}$ are the same as those used for the
matching.  The amplitude discrepancy between the pN waveform and the
NR waveform over this interval is within numerical
uncertainty---roughly $0.4\%$.  As with the matching technique
(Eq.~(\ref{eq:MatchingChiSquared})), this method is similar to that of
Ref.~\cite{Ajith-Babak-Chen-etal:2007b}, but distinct, in that we
blend the phase and amplitude, rather than the real and imaginary
parts.  This leads to a smoothly blended waveform, shown in
Fig~\ref{fig:WaveformSnapshot}.
%%%%%%%%%%%%%%%%%%%%%%%%%%%%%%%%%%%%%%%%%%%%%%%%%%%%%%%%%%%%%%%%%%%%%%
\begin{figure}
  \begin{center}
    \includegraphics[width=0.55\linewidth]{figures/comparison/Waveform}
  \end{center}
  \caption{The last $t=5000\MSun$ of the hybrid waveform used in this
    analysis: the $h_{+}$ waveform seen by an observer on the positive
    $z$ axis.  The vertical lines denote the matching and
    hybridization region.  The $0$ on the time axis corresponds to the
    beginning of data from the numerical simulation.}
  \label{fig:WaveformSnapshot}
\end{figure}%
%%%%%%%%%%%%%%%%%%%%%%%%%%%%%%%%%%%%%%%%%%%%%%%%%%%%%%%%%%%%%%%%%%%%%%

Up to this point, the waveform has been $\Psi_{4}$ data.  With the
long waveform in hand, we numerically integrate twice to obtain $h$,
and set the four integration constants so that the final waveform has
zero average and first moment~\cite{Pfeiffer-Brown-etal:2007}.
Because of the very long duration of the waveforms, this gives a
reasonable result, which is only incorrect at very low
frequencies---lower than any frequency of interest to us.  We have
also checked that our results do not change when we effectively
integrate in the frequency domain by taking
\begin{equation}
  \label{eq:PsiFourIntegration}
  \tilde{h} = -\frac{\tilde{\Psi}_{4}}{4\,\pi\, f^{2}}\ ,
\end{equation}
which is the frequency-domain analog of the equation $\Psi_{4} =
\ddot{h}$.
\fi


\section{Conclusions}

In this chapter we reviewed the basic properties of gravitational
waves and methods used to model such waves from the inspiral and
merger of systems of compact binaries.  In the next chapter we discuss
the principles behind the LIGO detectors, which are looking for
gravitational-wave signals.  Then in chapter~\ref{ch:search} we
discuss how pN waveforms are used to detect signals in the LIGO data.


% Todo:
% Discuss the Weyl tensor, if needed for NR


\Chapter{Search template banks for low-mass binary black holes in the 
Advanced gravitational-wave detector era}
\label{ch:EOBF2_Effectualness}
\input{eobpn_1-5pnmetric_effectualness}

\Chapter{Binary black hole search template banks with Numerical 
Relativity waveforms}
\label{ch:NRHyb_bank}
\input{nrhyb_tmpltbank}

\Chapter{NINJA-2: Detecting gravitational waveforms modelled
using numerical binary black hole simulations}
\label{ch:ninja2}
The NINJA-1 project was a huge success in bringing the numerical
relativity and gravitational-wave astronomy communities together.  The
project also resulted in several intriguing qualitative results.
However, it only began the process of testing detection and parameter
estimation pipelines against realistic signals.  The follow-up
project, NINJA-2, is ongoing as of the time of writing.  NINJA-2 aims
to remove some of the shortcomings of NINJA-1 and allow quantitative
studies of the behaviors of pipelines in varying regions of signal
parameter space.  Specifically, NINJA-2 addresses issues with both the
waveform submissions and the noise used to construct the data sets.

This chapter describes the contributed waveforms, the studies that
have been done to verify them, and the construction of the first round
of data sets.  The next chapter will present preliminary results from
running the CBC pipelines on the longest and most carefully
constructed of these data sets.

\section{Contributed waveforms}

NINJA-1 had an open policy towards waveforms submission in order to
encourage wide participation.  This meant there were no requirements
on either waveform quality or length.  The lack of quality
requirements allowed for the possibility of unphysical features in the
waveforms.  There were also no requirements to perform the kind of
convergence testing reported in section~\ref{sec:PNNRHybridWaveform},
although such validation is typically done by numerical
relativistists.  The loose requirements limited the conclusions that
could be drawn, for example it makes it difficult to say whether an
injection was missed due to the parameters of the signal or an
unintended feature of the waveform.

The lack of length requirement limited the available mass range to $M
< 36 \msun$ for reasons that can be seen in
figure~\ref{fig:StildesAndInitialPSD}.   Had the waveform in that plot
not had a post-Newtonian component, the NR component to the right of
the triangles would have had to be placed below 40 Hz in order to
prevent turning on in-band.  This mass range limited the tests that
could be done, for example it entirely excludes the standard CBC
low-mass pipeline.

To address these issues NINJA-2 specifies the following minimal
requirements~\cite{ninja2-wiki}.  The raw numerical simulation should
include at least five orbits of usable data before merger (i.e., not
counting bursts of junk radiation or other significant noise).  Given
the computation cost of extending the NR waveforms, we instead require
stitching to a post-Newtonian inspiral approximant, which should be
performed at a GW frequency of $M\omega \leq 0.075$, where $M\omega$ is
the frequency of the $(l = 2, m = \pm 2)$ harmonic. The full waveform
should be long enough to be entirely within the sensitivity bands of
LIGO and Virgo down to $10 \msun$ with a lower cutoff frequency of 10
Hz, which corresponds to a starting GW frequency of $M\omega = 0.003$.
The numerical-waveform (before any hybridization) amplitude should be
accurate to within 5\%, and the phase (as a function of GW frequency)
should have an accumulated uncertainty over the entire inspiral,
merger and ringdown, of no more than 0.5 radian. The PN approximants
used for hybridization should ideally use the highest PN orders
available, both in phase and amplitude.  

These minimal accuracy requirements are motivated by the results of
the Samurai project~\cite{Hannam:2009hh}, and studies performed in
preparation for the NR-AR collaboration project~\cite{ninja-wiki}.
The question of how many NR cycles are needed in order to produce a
robust waveform is an area of current research~\cite{MacDonald:2011}. 

The NINJA-2 project encourages although does not require the addition
of higher-order modes.  We chose to restrict attention to non-spinning
waveforms and waveforms with spins aligned or anti-aligned with the
orbital angular momentum.  There are sufficient open questions
regarding these restricted cases to make this analysis interesting,
without adding the additional complications on both the NR and data
analysis sides associated with precession. 

A total of 60 waveforms from 8 groups were contributed, these are
summarized in tables ~\ref{tab:ninja2_bam}, \ref{tab:ninja2_fau},
\ref{tab:ninja2_gatech}, \ref{tab:ninja2_lean},
\ref{tab:ninja2_llama}, \ref{tab:ninja2_rit}, \ref{tab:ninja2_spec},
\ref{tab:ninja2_uiuc} and a map of the parameter values is shown in
figure~\ref{f:ninja2_param_map}.

\begin{table}
\begin{center}
\begin{tabular}{|l|r|r|r|l|c|}
\hline
Run & $q$ & Spin1${}_z$ & Spin2${}_z$ & pN Approx. & Refs \\
\hline
BAM\_D10spp85\_80.T4.hyb.n2 & 1 & 0.85 & 0.85 & TaylorT4 & \cite{Hannam:2007wf,Brugmann:2008zz} \\
BAM\_D10spp85\_80.T1.hyb.n2 & 1 & 0.85 & 0.85 & TaylorT1 & \cite{Hannam:2007wf,Brugmann:2008zz} \\
BAM\_D125smm50Nep\_80.T1.hyb.n2 & 1 & -0.50 & -0.50 & TaylorT1 & \cite{Hannam:2007wf,Brugmann:2008zz} \\
BAM\_D125smm50Nep\_80.T4.hyb.n2 & 1 & -0.50 & -0.50 & TaylorT4 & \cite{Hannam:2007wf,Brugmann:2008zz} \\
BAM\_D13smm75Nep\_96.T4.hyb.n2 & 1 & -0.75 & -0.75 & TaylorT4 & \cite{Hannam:2007wf,Brugmann:2008zz} \\
BAM\_D13smm75Nep\_96.T1.hyb.n2 & 1 & -0.75 & -0.75 & TaylorT1 & \cite{Hannam:2007wf,Brugmann:2008zz} \\
BAM\_D13smm85Nep\_88.T4.hyb.n2 & 1 & -0.85 & -0.85 & TaylorT4 & \cite{Hannam:2007wf,Brugmann:2008zz} \\
BAM\_D13smm85Nep\_88.T1.hyb.n2 & 1 & -0.85 & -0.85 & TaylorT1 & \cite{Hannam:2007wf,Brugmann:2008zz} \\
BAM\_D11spp50\_96.T4.hyb.n2 & 1 & 0.50 & 0.50 & TaylorT4 & \cite{Hannam:2007wf,Brugmann:2008zz} \\
BAM\_D11spp50\_96.T1.hyb.n2 & 1 & 0.50 & 0.50 & TaylorT1 & \cite{Hannam:2007wf,Brugmann:2008zz} \\
BAM\_D10spp75\_80.T1.hyb.n2 & 1 & 0.75 & 0.75 & TaylorT1 & \cite{Hannam:2007wf,Brugmann:2008zz} \\
BAM\_D10spp75\_80.T4.hyb.n2 & 1 & 0.75 & 0.75 & TaylorT4 & \cite{Hannam:2007wf,Brugmann:2008zz} \\
BAM\_D12smm25Nep\_80.T4.hyb.n2 & 1 & -0.25 & -0.25 & TaylorT4 & \cite{Hannam:2007wf,Brugmann:2008zz} \\
BAM\_D12smm25Nep\_80.T1.hyb.n2 & 1 & -0.25 & -0.25 & TaylorT1 & \cite{Hannam:2007wf,Brugmann:2008zz} \\
BAM\_EP\_um4\_D10-n96.T4.hyb.n2 & 4 & 0.00 & 0.00 & TaylorT4 & \cite{Hannam:2007wf,Brugmann:2008zz} \\
BAM\_EP\_um4\_D10-n96.T1.hyb.n2 & 4 & 0.00 & 0.00 & TaylorT1 & \cite{Hannam:2007wf,Brugmann:2008zz} \\
BAM\_um3\_88.T4.hyb.n2 & 3 & 0.00 & 0.00 & TaylorT4 & \cite{Hannam:2007wf,Brugmann:2008zz} \\
BAM\_um3\_88.T1.hyb.n2 & 3 & 0.00 & 0.00 & TaylorT1 & \cite{Hannam:2007wf,Brugmann:2008zz} \\
BAM\_um2\_88.T1.hyb.n2 & 2 & 0.00 & 0.00 & TaylorT1 & \cite{Hannam:2007wf,Brugmann:2008zz} \\
BAM\_um2\_88.T4.hyb.n2 & 2 & 0.00 & 0.00 & TaylorT4 & \cite{Hannam:2007wf,Brugmann:2008zz} \\
BAM\_R6\_PN\_80.T1.hyb.n2 & 1 & 0.00 & 0.00 & TaylorT1 & \cite{Hannam:2007wf,Brugmann:2008zz} \\
BAM\_R6\_PN\_80.T4.hyb.n2 & 1 & 0.00 & 0.00 & TaylorT4 & \cite{Hannam:2007wf,Brugmann:2008zz} \\
BAM\_D12spp25\_96.T4.hyb.n2 & 1 & 0.25 & 0.25 & TaylorT4 & \cite{Hannam:2007wf,Brugmann:2008zz} \\
BAM\_D12spp25\_96.T1.hyb.n2 & 1 & 0.25 & 0.25 & TaylorT1 & \cite{Hannam:2007wf,Brugmann:2008zz} \\
BAM\_q2a0a025\_T\_96\_344.T1.hyb.n2.bbh & 2 & 0.25 & 0.00 & {} & \cite{,Brugmann:2008zz} \\
BAM\_q2a0a025\_T\_96\_344.T4.hyb.n2.bbh & 2 & 0.25 & 0.00 & {} & \cite{,Brugmann:2008zz} \\
\hline
\end{tabular}
\end{center}
\caption[BAM submissions to NINJA-2]{
\label{tab:ninja2_bam}
BAM submissions to NINJA-2}
\end{table}

\begin{table}
\begin{center}
\begin{tabular}{|l|r|r|r|l|c|}
\hline
Run & $q$ & Spin1${}_z$ & Spin2${}_z$ & pN Approx. & Refs \\
\hline
BAM\_hybrid\_om0.025etmq3S0.4- & 3 & 0.40 & 0.60 & TaylorT4 & \cite{none,???} \\
0\_0\_S0.6\_0\_0\_72 &  &  &  &  &  \\
\hline
\end{tabular}
\end{center}
\caption[FAU submissions to NINJA-2]{
\label{tab:ninja2_fau}
FAU submissions to NINJA-2}
\end{table}

\begin{table}
\begin{center}
\begin{tabular}{|l|r|r|r|l|c|}
\hline
Run & $q$ & Spin1${}_z$ & Spin2${}_z$ & pN Approx. & Refs \\
\hline
MayaKranc\_D12\_a0.00\_m129\_nj & 1 & 0.00 & 0.00 & TaylorT4 & \cite{,} \\
MayaKranc\_D10\_a0.90\_m129\_nj & 1 & 0.90 & 0.90 & TaylorT4 & \cite{,} \\
MayaKranc\_D10\_a0.20\_m77\_nj & 1 & 0.20 & 0.20 & TaylorT4 & \cite{,} \\
MayaKranc\_D10\_a0.60\_m77\_nj & 1 & 0.60 & 0.60 & TaylorT4 & \cite{,} \\
MayaKranc\_D12\_a0.60\_m103\_nj & 1 & 0.60 & 0.60 & TaylorT4 & \cite{,} \\
MayaKranc\_Sp02py0935th90\_gr & 1 & 0.80 & 0.00 & TaylorT4 & \cite{,} \\
MayaKranc\_D12\_a0.80\_m103\_nj & 1 & 0.80 & 0.80 & TaylorT4 & \cite{,} \\
MayaKranc\_D12\_a0.00\_q2\_m90\_nj & 2 & 0.00 & 0.00 & TaylorT4 & \cite{,} \\
MayaKranc\_D11\_a0.20\_q2\_m90\_nj & 2 & 0.02 & 0.09 & TaylorT4 & \cite{,} \\
MayaKranc\_D10\_a0.40\_m90\_nj & 1 & 0.40 & 0.40 & TaylorT4 & \cite{,} \\
MayaKranc\_D10\_a0.80\_m90\_nj & 1 & 0.80 & 0.80 & TaylorT4 & \cite{,} \\
MayaKranc\_D12\_a0.40\_m103\_nj & 1 & 0.40 & 0.40 & TaylorT4 & \cite{,} \\
MayaKranc\_D12\_a0.20\_m103\_nj & 1 & 0.20 & 0.20 & TaylorT4 & \cite{,} \\
\hline
\end{tabular}
\end{center}
\caption[GATech submissions to NINJA-2]{
\label{tab:ninja2_gatech}
GATech submissions to NINJA-2}
\end{table}

\begin{table}
\begin{center}
\begin{tabular}{|l|r|r|r|l|c|}
\hline
Run & $q$ & Spin1${}_z$ & Spin2${}_z$ & pN Approx. & Refs \\
\hline
dq4 & 4 & 0.00 & 0.00 & TaylorT1 & \cite{,Sperhake:2006cy} \\
\hline
\end{tabular}
\end{center}
\caption[LEAN submissions to NINJA-2]{
\label{tab:ninja2_lean}
LEAN submissions to NINJA-2}
\end{table}

\begin{table}
\begin{center}
\begin{tabular}{|l|r|r|r|l|c|}
\hline
Run & $q$ & Spin1${}_z$ & Spin2${}_z$ & pN Approx. & Refs \\
\hline
Llama\_d550-h64-Hybrid & 1 & 0.00 & 0.00 & 3.5pNTaylorF2 & \cite{Reisswig:2009rx,Reisswig:2009rx} \\
Llama\_d4d4-q1--D10-h64-r250.T4.hybrid & 1 & -0.40 & -0.40 & TaylorT4 & \cite{Pollney:2010hs,Pollney:2009yz,} \\
Llama\_d4d4-q1--D10-h64-r250.T1.hybrid & 1 & -0.40 & -0.40 & TaylorT1 & \cite{Pollney:2010hs,Pollney:2009yz,} \\
Llama\_u4u4-q1--D8-h64-r250.T1.hybrid & 1 & 0.40 & 0.40 & TaylorT1 & \cite{Pollney:2010hs,Pollney:2009yz,} \\
Llama\_u4u4-q1--D8-h64-r250.T4.hybrid & 1 & 0.40 & 0.40 & TaylorT4 & \cite{Pollney:2010hs,Pollney:2009yz,} \\
Llama\_d5q2-h016-Hybrid & 2 & 0.00 & 0.00 & 3.5pNTaylorF2 & \cite{,Reisswig:2009rx} \\
Llama\_u2u2-q1--D8-h64-r250.T1.hybrid & 1 & 0.20 & 0.20 & TaylorT1 & \cite{Pollney:2010hs,Pollney:2009yz,} \\
Llama\_u2u2-q1--D8-h64-r250.T4.hybrid & 1 & 0.20 & 0.20 & TaylorT4 & \cite{Pollney:2010hs,Pollney:2009yz,} \\
Llama\_d2d2-q1--D10-h64-r250.T1.hybrid & 1 & -0.20 & -0.20 & TaylorT1 & \cite{Pollney:2010hs,Pollney:2009yz,} \\
Llama\_d2d2-q1--D10-h64-r250.T4.hybrid & 1 & -0.20 & -0.20 & TaylorT4 & \cite{Pollney:2010hs,Pollney:2009yz,} \\
\hline
\end{tabular}
\end{center}
\caption[Llama submissions to NINJA-2]{
\label{tab:ninja2_llama}
Llama submissions to NINJA-2}
\end{table}

\begin{table}
\begin{center}
\begin{tabular}{|l|r|r|r|l|c|}
\hline
Run & $q$ & Spin1${}_z$ & Spin2${}_z$ & pN Approx. & Refs \\
\hline
LazEV\_D8.4\_10to1\_nj\_hybrid & 10 & 0.00 & 0.00 & TaylorT4 & \cite{Campanelli:2005dd} \\
\hline
\end{tabular}
\end{center}
\caption[RIT submissions to NINJA-2]{
\label{tab:ninja2_rit}
RIT submissions to NINJA-2}
\end{table}

\begin{table}
\begin{center}
\begin{tabular}{|l|r|r|r|l|c|}
\hline
Run & $q$ & Spin1${}_z$ & Spin2${}_z$ & pN Approx. & Refs \\
\hline
SpEC\_q6s0 & 6 & 0.00 & 0.00 & TaylorT1 & \cite{SpECWebsite} \\
SpEC\_q4s0 & 4 & 0.00 & 0.00 & TaylorT2 & \cite{SpECWebsite} \\
SpEC\_EqualMassAntiAlignedSpins & 1 & -0.44 & -0.44 & NA & \cite{chu-2009,SpECWebsite} \\
SpEC\_q1s-0.95 & 1 & -0.95 & -0.95 & TaylorT1 & \cite{SpECWebsite} \\
SpEC\_q2s0 & 2 & 0.00 & 0.00 & TaylorT2 & \cite{SpECWebsite} \\
SpEC\_EqualMassAlignedSpins & 1 & 0.44 & 0.44 & NA & \cite{chu-2009,SpECWebsite} \\
SpEC\_q3s0 & 3 & 0.00 & 0.00 & TaylorT2 & \cite{SpECWebsite} \\
SpEC\_EqualMassNonspinning & 1 & 0.00 & 0.00 & TaylorT4 & \cite{Scheel:2008rj,SpECWebsite} \\
\hline
\end{tabular}
\end{center}
\caption[SpEC submissions to NINJA-2]{
\label{tab:ninja2_spec}
SpEC submissions to NINJA-2}
\end{table}

\begin{table}
\begin{center}
\begin{tabular}{|l|r|r|r|l|c|}
\hline
Run & $q$ & Spin1${}_z$ & Spin2${}_z$ & pN Approx. & Refs \\
\hline
UIUC\_spin\_-0.25\_om0.0528\_22-HYBRID & 1 & -0.25 & -0.25 & NA & \cite{none} \\
UIUC\_spin\_0.85\_om0.0536\_22-HYBRID & 1 & 0.85 & 0.85 & NA & \cite{none} \\
\hline
\end{tabular}
\end{center}
\caption[UIUC submissions to NINJA-2]{
\label{tab:ninja2_uiuc}
UIUC submissions to NINJA-2}
\end{table}


\begin{figure}
  \includegraphics[width=\linewidth]{figures/ninja2/ninja2_cat.png}
  \caption[Parameters of the NINJA-2 submissions]{
  \label{f:ninja2_param_map}
Parameters of the NINJA-2 hybrid waveform submissions showing the
symmetric mass ratio $\eta=m_1 m_2 /(m_1+m_2)^2$ and dimensionless
spin parameter $\chi=(S_1/m_1 + S_2/m_2)/(m_1+m_2)$ after scaling the
waveforms to a total mass of 10 $\msun$.  The numbers indicate how
many distinct waveforms with the specified parameters were submitted.}
\end{figure}%

\section{Verifying the hybrid waveforms}

Each NR group verified that their waveforms met the minimum NINJA-2
requirements before submission.  Once submitted, a series of checks
were performed in order to validate the waveforms against each other.

In the first stage the post-Newtonian expressions and codes were
compared against each other and the literature.  This required several
iterations, but resulted in a set of codes in various languages that
produce waveforms that all agree in both phase and amplitude. 

\subsection{Time-domain and frequency-domain checks}

In the second stage the complete hybrid waveforms were examined.
We first plotted the last 40 cycles of each waveform -- enough to
include the full NR portion, the hybridization region, and some of the
pN portion -- and looked for any anomalies such as those present in
some of the NINJA-1 waveforms in figure~\ref{fig:NR-Reh22}.  A few such
features were indeed visible, spotting them in this way allowed them
to be corrected.  One example is shown in figure (\Note{show the dq4
waveform before and after, note this was due to a problem integrating
psi4, and discuss this stage a little in the NR section of chapter
3}).

The amplitude of the Fourier transform of the complete waveforms were
also plotted.  This analysis also revealed unphysical features, 
primarily due to hybridization.  An example is shown in
figure~\ref{f:ninja2_freq_hybrids}, which shows a visible ``kink'' in
the waveform at the hybridization frequency, which vanishes after the
waveform was reconstructed.

\begin{figure}
  \includegraphics[width=0.5\linewidth]{figures/ninja2/bam_d125smm50nep_80_t1_hyb_n2_amp.png}
  \includegraphics[width=0.5\linewidth]{figures/ninja2/bam_d125smm50nep_80_t1_hyb_n2_amp_v2.png}
  \caption[Frequency-domain hybrid NINJA-2 waveforms]{
  \label{f:ninja2_freq_hybrids}
Fourier amplitude of the (2,2) mode of a sample NINJA-2 hybrid
waveform from the BAM/AEI group.  The waveform has been scaled to 10
$\msun$ and placed 1 Mpc from the detector to give it physical units.
\Note{I need to dig up the old version of the waveform and remake it
with the new code} The waveform on the left is the version
initially submitted, note there is a visible ``kink'' in the waveform
at the hybridization frequency.  The waveform on the right has been
re-hybridized and there is no longer a visible kink.  This feature did
not show up in the time domain view of the waveform.}
\end{figure}%


\subsection{Overlap comparisons}

\Note{Explain why we have to sample the overlaps at high rate, refer to
figure~\ref{f:overlap_sample_frequency}}

\begin{figure}
  \includegraphics[width=0.5\linewidth]{figures/ninja2/resolutions}
  \includegraphics[width=0.5\linewidth]{figures/ninja2/overlap_time_series}
  \caption[Sensitivity of the overlaps to sample rate]{
  \label{f:overlap_sample_frequency}
\Note{TODO: explain that the maximum is sharply peaked, undersampling
can miss the true maximum, as the waveforms beat against each other we
get cycles}}
\end{figure}%

In this check the waveforms were compared against each other using
standard data-analysis techniques, in particular the overlap defined
in equation~\ref{eq:OverlapDefinition}  using the initial LIGO noise
curve.  The waveforms were grouped into sets with identical
parameters.  For each set one waveform was chosen as the reference and
the overlap with all the other waveforms calculated over a range of
masses, optimizing over the unknown coalescence time and phase as
usual.  This process was then repeated, taking each of the other
waveforms as the reference in turn.

The initial set of contributions showed unexpectedly large mismatches
at masses $\approx 20 \msun$, at the point where the hybridization
frequency for several waveforms passes through the most sensitive
portion of the LIGO band.  This prompted a number of the NR groups to
revise their hybridization procedures, after which the overlaps were
more in line with the expected values.  A sample of these plots before
and after rehybridization is shown in
figure~\ref{f:ninja2_overlap_test}.  It is worth noting that even
after rehybridizing there are still mismatches.  In particular, the
SpEC and MayaKranc submissions use the same hybridization method, the
same pN approximant, and are simulating the same physical system.  The
overlaps approach 1 at higher masses, where the NR portion dominates.
This is an important validation, as the two groups use completely
different codes based on different principles.  The overlap is also
close to 1 at the lowest masses, where pN dominates.  This is expected
following the pN cross-checks done previously.  That the overlap
diminishes slightly at intermediate masses must be due to the
different hybridization frequencies \Note{get these values}, and
indicates the sensitivity of the waveform to the hybridization
details.


\begin{figure}
  \includegraphics[width=0.5\linewidth]{figures/ninja2/q_1_z_0_figure03}
  \includegraphics[width=0.5\linewidth]{figures/ninja2/figure2_1_0_16}
  \caption[Overlaps between NINJA-2 submissions maximized over time
and phase]{
  \label{f:ninja2_overlap_test}
Overlaps between the equal-mass, non-spinning NINJA-2 contributions,
maximized over time and phase.   For the original submissions (left)
overlaps are as low as 0.70 between waveforms using different pN
approximants and 0.94 for waveforms using the same approximant.  After
rehybridization (right) the waveforms achieve much higher overlaps,
with minima above 0.94 for different approximants and above 0.98 for
identical approximants.  The residual differences  between waveforms
using TaylorT4 are due to hybridization details.  The Llama waveform
was accidentally omitted from the original runs.}
\end{figure}%

%%%%%%%%%%%%%%%%
\clearpage

\begin{figure}
  \includegraphics[width=0.5\linewidth]{figures/ninja2/pn_figure03.png} 
  \includegraphics[width=0.5\linewidth]{figures/ninja2/pn_figure06.png} \\
  \includegraphics[width=0.5\linewidth]{figures/ninja2/pn_figure09.png} 
  \includegraphics[width=0.5\linewidth]{figures/ninja2/pn_figure12.png} \\
  \caption[Overlap of time-domain pN waveforms, $q=1$ $S_{z1} = S_{z2} = 0$]{
  \label{f:figure_pn}
Overlap of time-domain pN waveforms, $q=1$ $S_{z1} = S_{z2} = 0$}
\end{figure}%

\begin{figure}
  \includegraphics[width=\linewidth]{figures/ninja2/figure_1_-0d5_01.png}
  \caption[Overlap plots for $q=1$ $S_{z1} = S_{z2} = -0.5$]{
  \label{f:figure_1_-0d5}
Overlap plots for $q=1$ $S_{z1} = S_{z2} = -0.5$}
\end{figure}%


\begin{figure}
  \includegraphics[width=\linewidth]{figures/ninja2/figure_1_-0d4_01.png}
  \caption[Overlap plots for $q=1$ $S_{z1} = S_{z2} = -0.4$]{
  \label{f:figure_1_-0d4}
Overlap plots for $q=1$ $S_{z1} = S_{z2} = -0.4$}
\end{figure}%


\begin{figure}
  \includegraphics[width=0.5\linewidth]{figures/ninja2/figure_3_0_02.png} 
  \includegraphics[width=0.5\linewidth]{figures/ninja2/figure_3_0_04.png} \\
  \includegraphics[width=0.5\linewidth]{figures/ninja2/figure_3_0_06.png} 
  \caption[Overlap plots for $q=3$ $S_{z1} = S_{z2} = 0$]{
  \label{f:figure_3_0}
Overlap plots for $q=3$ $S_{z1} = S_{z2} = 0$}
\end{figure}%


\begin{figure}
  \includegraphics[width=0.5\linewidth]{figures/ninja2/figure_1_0d2_03.png} 
  \includegraphics[width=0.5\linewidth]{figures/ninja2/figure_1_0d2_06.png} \\
  \includegraphics[width=0.5\linewidth]{figures/ninja2/figure_1_0d2_09.png} 
  \includegraphics[width=0.5\linewidth]{figures/ninja2/figure_1_0d2_12.png} \\
  \caption[Overlap plots for $q=1$ $S_{z1} = S_{z2} = 0.2$]{
  \label{f:figure_1_0d2}
Overlap plots for $q=1$ $S_{z1} = S_{z2} = 0.2$}
\end{figure}%


\begin{figure}
  \includegraphics[width=0.5\linewidth]{figures/ninja2/figure_1_-0d25_02.png} 
  \includegraphics[width=0.5\linewidth]{figures/ninja2/figure_1_-0d25_04.png} \\
  \includegraphics[width=0.5\linewidth]{figures/ninja2/figure_1_-0d25_06.png} 
  \caption[Overlap plots for $q=1$ $S_{z1} = S_{z2} = -0.25$]{
  \label{f:figure_1_-0d25}
Overlap plots for $q=1$ $S_{z1} = S_{z2} = -0.25$}
\end{figure}%


\begin{figure}
  \includegraphics[width=\linewidth]{figures/ninja2/figure_1_-0d2_01.png}
  \caption[Overlap plots for $q=1$ $S_{z1} = S_{z2} = -0.2$]{
  \label{f:figure_1_-0d2}
Overlap plots for $q=1$ $S_{z1} = S_{z2} = -0.2$}
\end{figure}%


\begin{figure}
  \includegraphics[width=0.5\linewidth]{figures/ninja2/figure_2_0_04.png} 
  \includegraphics[width=0.5\linewidth]{figures/ninja2/figure_2_0_08.png} \\
  \includegraphics[width=0.5\linewidth]{figures/ninja2/figure_2_0_12.png} 
  \includegraphics[width=0.5\linewidth]{figures/ninja2/figure_2_0_16.png} \\
  \includegraphics[width=0.5\linewidth]{figures/ninja2/figure_2_0_20.png} 
  \caption[Overlap plots for $q=2$ $S_{z1} = S_{z2} = 0$]{
  \label{f:figure_2_0}
Overlap plots for $q=2$ $S_{z1} = S_{z2} = 0$}
\end{figure}%


\begin{figure}
  \includegraphics[width=\linewidth]{figures/ninja2/figure_1_0d8_01.png}
  \caption[Overlap plots for $q=1$ $S_{z1} = S_{z2} = 0.8$]{
  \label{f:figure_1_0d8}
Overlap plots for $q=1$ $S_{z1} = S_{z2} = 0.8$}
\end{figure}%


\begin{figure}
  \includegraphics[width=0.5\linewidth]{figures/ninja2/figure_1_0_04.png} 
  \includegraphics[width=0.5\linewidth]{figures/ninja2/figure_1_0_08.png} \\
  \includegraphics[width=0.5\linewidth]{figures/ninja2/figure_1_0_12.png} 
  \includegraphics[width=0.5\linewidth]{figures/ninja2/figure_1_0_16.png} \\
  \includegraphics[width=0.5\linewidth]{figures/ninja2/figure_1_0_20.png} 
  \caption[Overlap plots for $q=1$ $S_{z1} = S_{z2} = 0$]{
  \label{f:figure_1_0}
Overlap plots for $q=1$ $S_{z1} = S_{z2} = 0$}
\end{figure}%


\begin{figure}
  \includegraphics[width=\linewidth]{figures/ninja2/figure_1_0d5_01.png}
  \caption[Overlap plots for $q=1$ $S_{z1} = S_{z2} = 0.5$]{
  \label{f:figure_1_0d5}
Overlap plots for $q=1$ $S_{z1} = S_{z2} = 0.5$}
\end{figure}%


\begin{figure}
  \includegraphics[width=\linewidth]{figures/ninja2/figure_1_0d25_01.png}
  \caption[Overlap plots for $q=1$ $S_{z1} = S_{z2} = 0.25$]{
  \label{f:figure_1_0d25}
Overlap plots for $q=1$ $S_{z1} = S_{z2} = 0.25$}
\end{figure}%


\begin{figure}
  \includegraphics[width=0.5\linewidth]{figures/ninja2/figure_1_0d4_03.png} 
  \includegraphics[width=0.5\linewidth]{figures/ninja2/figure_1_0d4_06.png} \\
  \includegraphics[width=0.5\linewidth]{figures/ninja2/figure_1_0d4_09.png} 
  \includegraphics[width=0.5\linewidth]{figures/ninja2/figure_1_0d4_12.png} \\
  \caption[Overlap plots for $q=1$ $S_{z1} = S_{z2} = 0.4$]{
  \label{f:figure_1_0d4}
Overlap plots for $q=1$ $S_{z1} = S_{z2} = 0.4$}
\end{figure}%


\begin{figure}
  \includegraphics[width=\linewidth]{figures/ninja2/figure_1_-0d75_01.png}
  \caption[Overlap plots for $q=1$ $S_{z1} = S_{z2} = -0.75$]{
  \label{f:figure_1_-0d75}
Overlap plots for $q=1$ $S_{z1} = S_{z2} = -0.75$}
\end{figure}%


\begin{figure}
  \includegraphics[width=\linewidth]{figures/ninja2/figure_1_-0d85_01.png}
  \caption[Overlap plots for $q=1$ $S_{z1} = S_{z2} = -0.85$]{
  \label{f:figure_1_-0d85}
Overlap plots for $q=1$ $S_{z1} = S_{z2} = -0.85$}
\end{figure}%


\begin{figure}
  \includegraphics[width=0.5\linewidth]{figures/ninja2/figure_4_0_03.png} 
  \includegraphics[width=0.5\linewidth]{figures/ninja2/figure_4_0_06.png} \\
  \includegraphics[width=0.5\linewidth]{figures/ninja2/figure_4_0_09.png} 
  \includegraphics[width=0.5\linewidth]{figures/ninja2/figure_4_0_12.png} \\
  \caption[Overlap plots for $q=4$ $S_{z1} = S_{z2} = 0$]{
  \label{f:figure_4_0}
Overlap plots for $q=4$ $S_{z1} = S_{z2} = 0$}
\end{figure}%


\begin{figure}
  \includegraphics[width=\linewidth]{figures/ninja2/figure_1_0d75_01.png}
  \caption[Overlap plots for $q=1$ $S_{z1} = S_{z2} = 0.75$]{
  \label{f:figure_1_0d75}
Overlap plots for $q=1$ $S_{z1} = S_{z2} = 0.75$}
\end{figure}%


\begin{figure}
  \includegraphics[width=0.5\linewidth]{figures/ninja2/figure_1_0d85_02.png} 
  \includegraphics[width=0.5\linewidth]{figures/ninja2/figure_1_0d85_04.png} \\
  \includegraphics[width=0.5\linewidth]{figures/ninja2/figure_1_0d85_06.png} 
  \caption[Overlap plots for $q=1$ $S_{z1} = S_{z2} = 0.85$]{
  \label{f:figure_1_0d85}
Overlap plots for $q=1$ $S_{z1} = S_{z2} = 0.85$}
\end{figure}%

\clearpage
%%%%%%%%%%%%%%%%

We also calculated the overlaps between waveforms with identical
parameters, optimizing over mass as well as time and phase.  This gives
a sense of the range over which parameter estimation pipelines could 
be biased by unphysical features of the waveform.  Example plots
using the equal-mass, non-spinning MayaKranc waveform as the signal and
BAM plus two different approximants as the template are shown in 
figure~\ref{f:ninja2_max_over_mass_bam}.


\begin{figure}
  \includegraphics[width=0.5\linewidth]{figures/ninja2/maya_bamt4_max_over_m}
  \includegraphics[width=0.5\linewidth]{figures/ninja2/maya_bamt1_max_over_m}
  \caption[Overlaps between NINJA-2 submissions maximized over mass]{
  \label{f:ninja2_max_over_mass_bam}
Overlaps between the equal-mass, non-spinning MayaKranc waveform taken
as the signal, and the equal-mass, non-spinning BAM waveform
hybridized with TaylorT4 (left) and TaylorT1 (right) taken as
templates.  Maximization is done over the mass of the template, as well
as over time and phase.  Note the lower overall overlaps and mass bias
at the low-mass end of the figure on the right, where the two different
pN waveforms dominate the overlap.}
\end{figure}%

At the high-mass end the overlap is dominated by NR data, and as in
figure~\ref{f:ninja2_overlap_test} the overlaps are high without
needing to move off the signal mass.  At the low-mass end the same
result would be expected in a pure pN/pN comparison.  However, there
is enough of the hybridization in-band to reduce the overlaps.  However,
changing the mass introduces a phase difference that accumulates over
all the cycles in-band, and so higher overlaps can not be achieved.
The result is optimal mass values close to the correct mass value, but
with a low overlap.

In the middle region these factors compete.  At higher masses the
overlap is reduced less by changing the mass and so the recovered
value can stray further from the injected value.  However as the
hybridization passes out of band this adjustment is no longer needed.


\noindent \Note{TODO: Fill in more details, discuss the need to sample
overlap time series at 16384, include the rest of the overlap plots after
remaking them for latest submissions}




% TODO:
% make tables of submissions
% plot of paramater space in eta, chi scaled to 10 M
% plot of injection set (use chi instead of sum of magnitudes)
% text
% results
% explain move to 16384 (plot showing aliasing)



\Chapter{Modeling of intermediate mass-ratio inspirals: 
Exploring the form of the self-force in the intermediate mass-ratio regime}
\label{ch:insSFIMRI}
%%%%%%%%%%%%%%%%%%%%%%%%%%%%%%%%%%%%%%%%%%%%%%%%%%%%%%%%%%%%%%%%%%%%%%%%%%%%%%%
%%% Describe the inspiral model for IMRIs.
%%%%%%%%%%%%%%%%%%%%%%%%%%%%%%%%%%%%%%%%%%%%%%%%%%%%%%%%%%%%%%%%%%%%%%%%%%%%%%%

\newcommand {\MBH}{{\cal{M}_{\bullet}}}
\newcommand {\Msun}{\ensuremath{M_{\odot}}}
\newcommand {\Mstar}{\ensuremath{M_{\ast}}}
\newcommand {\Nstar}{\ensuremath{{\cal{N}_{\star}}}}
\newcommand {\Rsun}{\ensuremath{R_{\odot}}}
\newcommand {\Mpthree}{\ensuremath{M_{\odot}\,\mathrm{pc}^{-3}}}
\newcommand {\RS}{\ensuremath{R_{\mathrm{S}}}}
\newcommand {\kms}{\ensuremath{\mathrm{km\,s}^{-1}}}
\newcommand {\peryr}{\ensuremath{\mathrm{yr}^{-1}}}
\newcommand {\nb}{{\sc Nbody }}
% \newcommand{\Note}[2]{{\bf [Note from #1] #2}}
\newcommand{\eeq}{\end{equation}}
\newcommand{\bea}{\begin{eqnarray}}
\newcommand{\rmd}{{\rm d}}
\def\ltsima{$\; \buildrel < \over \sim \;$}
\def\simlt{\lower.5ex\hbox{\ltsima}}
\def\gtsima{$\; \buildrel > \over \sim \;$}
\def\simgt{\lower.5ex\hbox{\gtsima}}

\newcommand{\hGR}{\mathbf{h}_\mathrm{GR}}
\newcommand{\hAP}{\mathbf{h}_\mathrm{AP}}
\newcommand{\tbf}{\theta}
\newcommand{\ttr}{\hat \theta}
\newcommand{\amp}{2\frac{m}{d}}
\newcommand{\rtthr}{\sqrt{3}}
\newcommand{\sint}{\sin\left(2\pi \frac{t}{T}\right)}
\newcommand{\cost}{\sin\left(2\pi \frac{t}{T}\right)}
\newcommand{\csthsky}{\cos \theta_S}
\newcommand{\sinthsky}{\sin \theta_S}
\newcommand{\sphis}{\sin\phi_S}
\newcommand{\cphis}{\cos \phi_S}
\newcommand{\csthk}{\cos \theta_K}
\newcommand{\sinthk}{\sin \theta_K}
\newcommand{\sphik}{\sin \phi_K}
\newcommand{\cphik}{\cos \phi_K}
\newcommand{\csth}{x_{1}(t)}
\newcommand{\ctphi}{x_{2}(t)}
\newcommand{\stphi}{x_{3}(t)}
\newcommand{\ctpsi}{x_{4}(t)}
\newcommand{\stpsi}{x_{5}(t)}
\newcommand{\tphi}{x_{6}(t)}
\newcommand{\tdphi}{x_{7}(t)}


\newcommand{\PsiFourPhase}{\phi}
\newcommand{\hDotPhase}{\varphi}
\newcommand{\OrbitalPhase}{\Phi}
\newcommand{\PsiFourFreq}{\omega}
\newcommand{\hDotFreq}{\varpi}
\newcommand{\OrbitalFreq}{\Omega}
\newcommand{\pPhi}{\ensuremath{p_{\OrbitalPhase}}}
\newcommand{\vPhi}{\ensuremath{v_{\OrbitalPhase}}}
\newcommand{\vOmega}{\ensuremath{v_{\OrbitalFreq}}}
\newcommand{\prstar}{\ensuremath{p_{r_{\ast}}}}
\newcommand{\Hhatreal}{\ensuremath{\hat{H}^{\text{real}}}}
\newcommand{\abs}[1]{\left\lvert #1 \right\rvert}
% \newcommand{\define}{\equiv}
\newcommand{\NQC}{\ensuremath{N}}
% \newcommand{\MM}{\ensuremath{\text{MM}}}

\newcommand{\Real}{\mbox{Re}}
\newcommand{\Imag}{\mbox{Im}}
\newcommand{\Pade}{Pad\'{e}\xspace}
\newcommand{\pade}{\Pade}
% \newcommand{\etal}{et al.}



%
% \def\etal{{\it et al.}} 
\def\ie{{\it i.e.}}  \def\eg{{\it e.g.}}
\def\lap{\hbox{${_{\displaystyle<}\atop^{\displaystyle\sim}}$}}
\def\gap{\hbox{${_{\displaystyle>}\atop^{\displaystyle\sim}}$}}
\def\lesssim{\mathrel{\hbox{\rlap{\hbox{\lower4pt\hbox{$\sim$}}}\hbox{$<$}}}}
\def\gtrsim{\mathrel{\hbox{\rlap{\hbox{\lower4pt\hbox{$\sim$}}}\hbox{$>$}}}}
\def\alt{\mathrel{\hbox{\rlap{\hbox{\lower4pt\hbox{$\sim$}}}\hbox{$<$}}}}
\def\agt{\mathrel{\hbox{\rlap{\hbox{\lower4pt\hbox{$\sim$}}}\hbox{$>$}}}}
\def\PRD{{\it Phys. Rev.} D~}
\def\PRL{{\it Phys.Rev.} Lett~}
\def\apjl{{\it Astrophys. J.} Lett~}
\def\Msun{M_\odot}
\def\PR{{\it Phys. Rev.}}
\def\CQG{{\it Class. Quantum Grav.}}
\def\aaps{{\it A\&AS~ }}
\def\pasj{{\it PASJ }}
\def\mnras{{\it MNRAS}} 
\def\gta{\ifmmode {\mathbin{\lower 3pt\hbox   %> or of order
    {$\,\rlap{\raise 5pt\hbox{$\char'076$}}\mathchar"7218\,$}}}
    \else {${\mathbin{\lower 3pt\hbox
    {$\rlap{\raise 5pt\hbox{$\char'076$}}\mathchar"7218\,$}}}
    $}\fi}
\def\lta{\ifmmode {\,\mathbin{\lower 3pt\hbox   %< or of order
    {$\,\rlap{\raise 5pt\hbox{$\char'074$}}\mathchar"7218\,$}}}
    \else {${\mathbin{\lower 3pt\hbox
    {$\rlap{\raise 5pt\hbox{$\char'074$}}\mathchar"7218\,$}}}
    $}\fi}

    


\section{Introduction}    

Testing Einstein's theory of general relativity in the strong-field regime by directly detecting the gravitational waves (GWs) emitted by black hole (BH) binaries of extreme mass-ratio has revived interest in a fundamental problem in general relativity: that of the gravitational self-force acting on a mass particle that moves in the background of a more massive BH. The gravitational self-force arises as  a result of the back reaction between the  small compact object and its own gravitational field, which, at linear order in mass-ratio, corresponds to a linear perturbation of the central BH geometry. Theoretical work in this field has been making steady progress since the seminal contributions of Dirac on the electromagnetic self-force in flat spacetime \cite{dirac}, and the extension of this analysis to curved spacetime by DeWitt an Brehme \cite{dewitt}.   The generalization of these early studies to the case of the gravitational self-force was accomplished independently by Quinn and Wald \cite{qwald} and 
Mino, Sasaki and Tanaka \cite{mino}. More recent developments have introduced mathematical rigor in the theoretical derivation of the gravitational self-force, and have relaxed previous assumptions with regard to the internal structure of the small compact object, i.e, it can now be a small Kerr BH or a small compact object made up of ordinary matter \cite{grallaI,grallaII}. 


Likewise, the actual computation of the gravitational self-force has evolved from  simplified scalar-field models~\cite{scasf}, to more sophisticated solutions that involve electromagnetic and gravitational problems in the context  of Schwarzschild circular orbits. At present, the self-force program has succeeded in developing numerical codes to compute the gravitational self-force along generic orbits around a Schwarzschild BH, and actually implementing these computations to develop an accurate waveform model that describes the inspiral evolution of non-spinning stellar mass BHs onto supermassive non-spinning BHs \cite{wargar}. The development of numerical algorithms to compute the self-force on a scalar charge moving along an eccentric-equatorial orbit of a Kerr BH has also been accomplished \cite{war}. The extension of this algorithm to compute the gravitational self-force for Kerr inspirals is under development  \cite{war,warleor}. 

The fact that the self-force program has focused on the computation of the self-force for spinless particles that inspiral into more massive BHs has a physical rationale. It is not just that such a problem would be more difficult to solve. It has also been shown that in the context of extreme-mass ratio inspirals (EMRIs), with typical mass-ratios 1:\(10^{6}\), the inclusion of small-body spin corrections in search templates will not allow us to measure the small body spin parameter with good accuracy  \cite{smallbody}. Hence, from a data analysis perspective, the inclusion of small body spin effects is not necessary.  Additionally, before attempting to compute the self-force for spinning particles, one may need to address a more pressing modeling issue for spinless particles: it has been shown that including conservative self-force corrections in the orbital phase of EMRIs may not be necessary for source detection, but they may still be necessary for accurate parameter reconstruction. Additionally, second 
order radiative corrections may contribute to the phase evolution at the same level as first-order conservative corrections \cite{cons,conspro}.  This then suggests that a waveform template that aims to provide an accurate description of the inspiral evolution of EMRIs may have to include both first-order conservative corrections, and second-order radiative corrections, as pointed out in \cite{wargar}. 


In sharp contrast, modeling BH binaries with intermediate mass ratio, i.e., 1:10-1:1000, (IMRIs) presents new challenges that can be neglected in the EMRI arena.  In the absence of fully general relativistic gravitational self-force corrections in this mass-ratio regime, some studies have assessed the importance of including post-Newtonian self-force corrections in search templates for spinning BHs of intermediate-mass that inspiral into supermassive Kerr BHs. This work has shown that the implementation of first-order post-Newtonian self-force corrections for spin-spin and spin-orbit couplings is essential to ensure the reliability of parameter estimation results \cite{higherspin}.  These studies suggest that the computation and implementation of gravitational self-force corrections for spinning binaries is a pressing, important problem both from a theoretical and data analysis perspective. 

 
In this article we shed light on the importance of including gravitational self-force corrections in search templates for non-spinning stellar mass BHs that inspiral into Schwarzschild BHs of intermediate-mass. As, at present, we have no access to accurate self-force calculations in this mass-ratio regime, it is important to shed light on the regime in which current self-force corrections do not render an accurate dynamical evolution of GW sources, and explore the range of applicability of the information we have at hand to develop accurate waveform templates in that mass-ratio regime. This study is particularly important in view of the ongoing upgrade of the LIGO detector~\cite{aLIGO}. Once advanced LIGO (aLIGO) begins observations, it will be possible to target the inspirals of neutron stars (NSs) and stellar-mass BHs into intermediate-mass BHs with masses  \(\sim 50M_{\odot} - 350M_{\odot}\)~\cite{brown} ---events which may take place in core-collapsed globular clusters~\cite{evidence,firstpaper}.  To 
perform this study we will make use of the recently developed effective-one-body (EOB) model that has been calibrated using numerical relativity (NR)  simulations for non-spinning BH binaries of mass-ratio    \(q=m_1/m_2\) \(=1,2,3,4\) and 6 \cite{BuonannoEOBv2Main}. It is worth pointing out that the actual calibration of this EOBNRv2 (EOBNR version two) model reproduces with great accuracy the features of true inspirals, and hence we will use this model as a benchmark to explore the form of the self-force in the intermediate-mass ratio regime. By construction, the EOBNRv2 model reproduces results in the test-mass particle limit, and also encodes self-force corrections that have been derived for small mass-ratios. In our analysis we will explicitly show these important modeling ingredients when we compare the predictions made by the EOBNRv2 and those obtained through black hole perturbation theory (BHPT).  

Finally, we will make use of the EOBNRv2 model and pertubative results to present a new prescription for the orbital frequency shift at the innermost stable circular orbit (ISCO), originally derived in \cite{inner} in the context of EMRIs. Our prescription reproduces exactly the self-force prediction for extreme-mass ratios and provides an accurate prediction in the intermediate and comparable-mass ratio regimes.   

This paper is organized as follows. In Section~\ref{s1} we present a succinct description of the EOBNRv2 model. In Section~\ref{s2} we derive an IMRI waveform model that accurately captures the features of true inspirals, as compared with the EOBNRv2 model. The derivation of this model will enable us to explore what the form of the self-force should be in the intermediate-mass-ratio regime so as to accurately reproduce the orbital dynamics obtained through NR simulations. In Section~\ref{s3} we derive a new prescription for the gravitational self-force correction to the orbital frequency at the ISCO that encodes results from the extreme, intermediate and comparable-mass ratio regimes. Finally, we summarize our results in Section~\ref{s4}.


\section{Effective One Body model}
\label{s1}

The Effective One Body (EOB) model was recently calibrated for non-spinning BH binaries of mass ratios   \(q=m_1/m_2\) \(=1,2,3,4\) and 6 by comparison to NR simulations \cite{BuonannoEOBv2Main}. In this Section we briefly describe this model.

\subsection{EOB dynamics}

The EOB is a scheme that maps the dynamics of the two body problem in general relativity to that of one object moving in the background of an effective metric. In the non-spinning limit this metric takes the form 

\begin{equation}
  ds_\mathrm{eff}^2 = -A(r)\,dt^2 + \frac{D(r)}{A(r)}\,dr^2 +
  r^2\,\Big(d\Theta^2+\sin^2\Theta\,d\OrbitalPhase^2\Big) \,,
  \label{eq:EOBmetric}
\end{equation}

\noindent where  $(r,\OrbitalPhase)$ stand for the dimensionless radial and polar coordinates, respectively. The conjugate momenta of these quantities is given by $(p_r,p_\OrbitalPhase)$. Since \(p_r\) diverges near the horizon, it is convenient to replace it  by the momentum conjugate to the EOB \textit{tortoise} radial coordinate $r_*$, i.e., 

%
\begin{equation}
  \frac{dr_*}{dr}=\frac{\sqrt{D(r)}}{A(r)}\,.
\end{equation}
%

\noindent Using this coordinate transformation, the effective EOB Hamiltonian can be written as  \cite{BuonannoEOBv2Main}

\begin{equation}
  \label{eq:genexp}
  H^\mathrm{eff}(r,p_{r_*},p_\OrbitalPhase) \equiv \mu\,\widehat{H}^\mathrm{eff}(r,p_{r_*},p_\Phi)  \\
  = \mu\,\sqrt{p^2_{r_*}+A (r) \left[ 1 +
      \frac{p_\OrbitalPhase^2}{r^2} +
      2(4-3\eta)\,\eta\,\frac{p_{r_*}^4}{r^2} \right]} \,,
\end{equation}

\noindent where \(\mu = m_1 m_2/(m_1+m_2)\), \(M=m_1+m_2\) and \(\eta=\mu/M\) stand for the reduced and total mass of the system, and the symmetric-mass ratio, respectively.  Additionally, the real EOB Hamiltonian is given by \cite{BuonannoEOBv2Main}

\begin{equation}
  \label{himpr}
  H^\mathrm{real}(r,p_{r_*},p_\OrbitalPhase) \equiv \mu\hat{H}^\mathrm{real}(r,p_{r_*},p_\Phi)  \\
  = M\,\sqrt{1 + 2\eta\,\left ( \frac{H^\mathrm{eff} - \mu}{\mu}\right )}
  -M\,.
\end{equation}

\noindent Furthermore, to ensure the existence and \(\eta\)-continuity of a last stable orbit (ISCO) as well as the existence and \(\eta\)-continuity of an \(\eta\)-deformed analog of the light-ring (the last stable orbit of a massless particle), these metric coefficients must be Pad\'e resummed. At present these coefficients are available at (pseudo) 5PN and 3PN order for  \(A(r)\) and \(D(r)\), respectively,

\begin{equation}
  D(r)=\frac{r^{3}} {(52\,\eta - 6\,\eta^{2}) + 6\, \eta\, r +
    r^{3}}\,,
\end{equation}

\noindent and 

\begin{equation}
  A(r) = \frac{\mathrm{Num}(A)}{\mathrm{Den}(A)}\,,
\end{equation}
%
with
%
\begin{eqnarray} \mathrm{Num}(A) &=& r^4\,\left[-64 +
    12\,a_4+4\,a_5+a_6+64 \eta-4 \eta ^2 \right]
  \nonumber\\
  &+& r^5\,\left[32-4\,a_4-a_5-24 \eta \right]\,,
\end{eqnarray}
%
and
%
\begin{eqnarray}
  && \mathrm{Den}(A) = 4\,a_4^2+4\,a_4\,a_5+a_5^2-a_4\,a_6+16\,a_6+ (32\,a_4 \nonumber \\
  &&\qquad + 16\,a_5-8\,a_6)\,\eta + 4\,a_4\,\eta^2+32\,\eta^3 + r\,\left[4\,a_4^2+a_4\,a_5 \right. 
  \nonumber\\
  &&\qquad \left. +16\,a_5+8\,a_6+(32\,a_4 -2\,a_6)\,\eta + 32\,\eta^2+8\,\eta^3\right] 
  \nonumber\\
  &&\qquad + r^2\,\left[16\,a_4+8\,a_5+4\,a_6+(8\,a_4+2\,a_5)\,\eta +32\,\eta^2\right] \nonumber \\
  &&\qquad + r^3\,\left[8\,a_4+4\,a_5+2\,a_6+32\,\eta-8\,\eta^2\right] 
  \nonumber\\
  &&\qquad + r^4\,\left[4\,a_4+2\,a_5+a_6+16\,\eta-4\,\eta^2\right] \nonumber \\
  &&\qquad + r^5\,\left[32-4\,a_4-a_5-24\,\eta\right]\,,
\end{eqnarray}
%
\noindent where $a_4=[94/3-(41/32)\,\pi^2]\,\eta$, and \(a_5\), \(a_6\) are adjustable parameters which were determined by minimizing the inspiral phase difference between the NR and EOB (2,2) modes in \cite{BuonannoEOBv2Main}. 

The EOB equations of motion that describe the orbital dynamics of the BH binary are given by \cite{bur}

\begin{subequations} \label{eq-eob}
  \begin{align}
    \frac{dr}{d \widehat{t}} &=
    \frac{A(r)}{\sqrt{D(r)}}\frac{\partial
      \widehat{H}^\mathrm{real}} {\partial
      p_{r_*}}(r,p_{r_*},p_\OrbitalPhase)\,,
    \label{eq:eobhamone} \\
    \frac{d \OrbitalPhase}{d \widehat{t}} &= \frac{\partial
      \widehat{H}^\mathrm{real}} {\partial
      p_\OrbitalPhase}(r,p_{r_*},p_\OrbitalPhase)\,,
    \label{eq:eobhamtwo}\\
    \frac{d p_{r_*}}{d \widehat{t}} &=
    -\frac{A(r)}{\sqrt{D(r)}}\,\frac{\partial \widehat{H}^\mathrm{
        real}} {\partial r}(r,p_{r_*},p_\OrbitalPhase) +{}^\mathrm{
      nK}\widehat{\cal F}_\OrbitalPhase \,
    \frac{p_{r_*}}{p_\OrbitalPhase}\,, \label{eq:eobhamthree}\\
    \frac{d p_\OrbitalPhase}{d \widehat{t}} &= {}^\mathrm{
      nK}\widehat{\cal F}_\OrbitalPhase\,,
    \label{eq:eobhamfour}
  \end{align}
\end{subequations}

\noindent where $\hat{t}\equiv t/M$, $\widehat{\OrbitalFreq}\equiv d \OrbitalPhase/d \widehat{t} \equiv M\Omega$ and the radiation-reaction force \({}^\mathrm{nK}\widehat{\cal F}_\OrbitalPhase \) is given by \cite{BuonannoEOBv2Main}
 
\begin{equation}\label{RadReacForce} {}^\mathrm{nK}\widehat{\cal
    F}_\OrbitalPhase = -\frac{1}{\eta\,v_\Omega^3}\,\frac{dE}{dt}\,,
\end{equation}

\noindent with $v_\Omega \equiv \widehat{\OrbitalFreq}^{1/3}$. An important improvement in the EOB formalism over models that used Pad\'e resummation of Taylor approximants to the energy flux is the implementation of a resummed energy flux of the form

 \begin{equation}\label{resflux}
  \frac{dE}{dt}=\frac{v_\Omega^6}{8\pi}\,
  \sum_{\ell=2}^{8}\,\sum_{m=1}^{\ell}\, m^2\, \abs{
    \frac{\mathcal{R}}{M}\, h_{\ell m}}^2~,
\end{equation}

\noindent where \( \mathcal{R}\) is the distance to the source and $h_{lm}$'s represent the multipoles of the waveform, defined through the following relation 

\begin{equation}
\label{mulwav}
h_{+} - i h_{\times} = \frac{M}{\mathcal{R}} \sum^{\infty}_{l=2} \sum^{m=l}_{m = -l} Y^{lm}_{-2}\, h_{lm},
\end{equation}

\noindent where $Y^{lm}_{-2}$ represent the spin weighted -2 spherical harmonics and  $h_+$ and $h_{\times}$ stand for the two gravitational wave polarizations. Since the numerical relativity simulations used to calibrate the EOB model were found to satisfy the condition $h_{\ell  m}=(-1)^{\ell}\,h_{\ell\,-m}^*$ with great accuracy, where $^*$ denotes complex conjugate, one can also assume that the analytical modes that enter the sum in Eq.~\ref{resflux} also satisfy this property. Furthermore, because $\abs{h_{\ell, m}}=\abs{h_{\ell,-m}}$, the sum in  Eq.~\ref{resflux} extends only over positive \(m\) modes. 
  
  
\clearpage 

\subsection{Modelling of the inspiral and plunge evolution}  

In the EOB scheme, the inspiral and plunge phases are described by the product of several factors, namely, 

\begin{equation}\label{hip}
  h^\mathrm{insp-plunge}_{\ell m} = h^\mathrm{F}_{\ell m}\,N_{\ell m}\,,
\end{equation}

 \noindent where the function \(N_{\ell m}\) is introduced to ensure that the EOB model reproduces: a) the shape of the NR amplitudes \(|h_{\ell m}|\) near their maxima; and b) the timelag between the maxima of the \(|h_{\ell m}|\) and the maxima of \(|h_{22}|\), obtained from NR data. On the other hand, the factorized resummed modes \(h^\mathrm{F}_{\ell m}\) are given by
 
 \begin{equation}\label{hlm}
  h^\mathrm{F}_{\ell m}=h_{\ell m}^{(N,\epsilon)}\,\hat{S}_\mathrm{eff}^{(\epsilon)}\, T_{\ell m}\, e^{i\delta_{\ell m}}\,\left(\rho_{\ell m}\right)^\ell\,,
\end{equation}

\noindent where \(\epsilon\) stands for the parity of \(h_{\ell m}\), i.e., \(\epsilon\,=\,1\) if \( \ell+m\) is even, and \(\epsilon\,=\,0\) for odd \( \ell+m\). 


The factor  \(h_{\ell m}^{(N,\epsilon)}\) stands for the Newtonian contribution, defined in Eqs.(15)-(18) of \cite{BuonannoEOBv2Main}. The remaining terms \(\hat{h}_{\ell m}^{(\epsilon)}= \hat{S}_\mathrm{eff}^{(\epsilon)}\, T_{\ell m}\, e^{i\delta_{\ell m}}\left(\rho_{\ell m}\right)^\ell\) represent a resummed version of all PN corrections, which have the structure \(\hat{h}_{\ell m}^{(\epsilon)}= 1+ {\mathcal{O}}(x)\), where \(x\) is the gauge-invariant object \(x=\hat{\Omega}^{2/3}\). 


Regarding the structure of the \(\hat{S}_\mathrm{eff}^{(\epsilon)}\) factor, we note that in the even parity case, which corresponds to mass moments, the leading order source of GW radiation is given by the energy density. Therefore, the source factor can be defined as  \(\hat{S}_\mathrm{eff}^{(\epsilon=0)} =  \hat{H}^\mathrm{eff}(r, p_{r_*}, p_\Phi)\)  \cite{resu}. On the other hand,  in the odd-parity case, which is associated to current modes, the angular momentum \(\hat{L}_\mathrm{eff}\) turns out to be a factor in the Regee-Wheeler-Zerilli odd-parity multipoles in the limit of small mass-ratio \(\eta\) \cite{resu}. Hence, one can define \(\hat{S}_\mathrm{eff}^{(\epsilon=1)} =  \hat{L}_\mathrm{eff}= p_\Phi\, v_\Omega\)  \cite{BuonannoEOBv2Main}.

Furthermore, considering a Schwarzschild background of mass \(M_{\rm ADM} = H^\mathrm{real}\), the tail term \(T_{\ell m}\) is a resummed version of an infinite number of logarithmic terms that enter the transfer function between the nearÐ-zone and far-Ðzone waveforms. Since this complex object only resums the leading logarithms of tail effects, one needs to introduce an additional dephasing factor \(\delta_{\ell m}\), which is related to subleading logarithms.  The final building block is given by \(\left(\rho_{\ell m}\right)^\ell\), which was introduced to enhance the agreement of the EOB model with NR in the strong-field regime.  The explicit expressions for these various quantities can be found in Eqs. (19)-(21) and Appendix B of \cite{BuonannoEOBv2Main}.
 
 
 Having described the building blocks of the EOB model, we will now describe how to go about in the actual construction of the EOB model using NR simulations.  The first step in the calibration of the EOBNR model consists of aligning the waveforms at low frequency, following the procedure outlined in \cite{BuonannoEOBv2Main}. This approach is used to minimize the phase difference between the NR and EOB \((\ell,m)\) modes using the prescription 
 
  
 \begin{equation}
 \varUpsilon (\Delta t, \Delta \phi)= \int^{t_2}_{t1} \left(\phi^{\rm EOB}(t+\Delta t) + \Delta \phi - \phi^{\rm NR}(t)\right)^2 \, dt,
 \label{alignment}
 \end{equation}
 
 \noindent where \(\Delta t/\Delta \phi\) are time/phase shifts, respectively, over which the minimization is performed. The time window \((t_1,t_2\)) is chosen so as to maximize the length of the NR waveform, but making sure that junk radiation does not contaminate the numerical data.  
 
The  numerical \(h_{22}\) is usually called the leading multipolar waveform because, compared to all the other multipoles, it provides the leading contribution to the amplitude of the full waveform \(h(t)\). Additionally, once the NR and EOB \((\ell,m)=(2,2)\) modes are aligned using the prescription given by Eq.~\eqref{alignment}, the peak of the numerical  \(h_{22}\) takes place at the same time the orbital frequency  \(\hat{\Omega}\) reaches its peak. Put in different words, the time at which the numerical \(h_{22}\) reaches its maximum and the EOB light-ring time (innermost circular orbit for a massless particle) are coincident. 

To calibrate the EOB dynamics, determined by Eqs.~\eqref{eq:eobhamone}-\eqref{eq:eobhamfour}, one minimizes the phase difference between the leading NR and EOB \((\ell,m)=(2,2)\) modes during the inspiral phase. This phase minimization procedure enables us to constrain the value of the doublet (\(a_5,\,a_6\)). As pointed out in \cite{BuonannoEOBv2Main}, the calibration of the adjustable parameters   (\(a_5,\,a_6\)) is not unique. For instance, \cite{BuonannoEOBv2Main} and \cite{rev} provide different values for the doublet \((a_5,a_6)\) which reproduce with great accuracy NR simulations for equal and comparable-mass non-spinning BH binaries. Even if the calibration of these parameters is degenerate, one can instill some physics in the way   (\(a_5,\,a_6\)) are determined.  In \cite{BuonannoEOBv2Main}, these parameters are modeled  as smooth functions of \(\eta\), such that they reproduce the self-force prediction of the orbital frequency shift at the ISCO in the test-mass particle limit \(\eta \rightarrow 0\)
, i.e., for  \(a_6(\eta\rightarrow 0)/\
eta\) and  \(a_5(\eta\rightarrow 0)/\eta\) the EOB model reproduces the result \cite{inner}

\begin{equation}
M \Omega_{\rm ISCO} = \frac{1}{6\sqrt{6}}\left(1+1.2512\eta + {\cal{O}}(\eta^2)\right).
\label{ishi}
\end{equation}

It is worth pointing out that in contrast with the model developed in~\cite{tara}, the development of the EOBNRv2 model did not require the computation of higher-order PN corrections in the adjustable parameters \(\rho_{22}\) and \(\delta_{22}\). The actual expression for these two parameters had already been determined analytically in previous studies \cite{rev}. 

Having determined the EOB dynamics, the \(N_{22}\) coefficients are computed and implemented in the energy flux resummed prescription given by Eq.~\eqref{resflux}, one can determine the rest of the EOB adjustable parameters, namely, higher-order PN corrections in \(\rho_{\ell m}\)/\(\delta_{\ell m}\) for \((\ell,m) \neq (2,2)\), by minimizing the amplitude/phase difference between the remaining numerical and EOB multipolar waveforms used in the calibration. These higher-order corrections for \((\ell,m) \neq (2,2)\) are included in the inspiral waveform prescription, Eq.~\eqref{hip}, but not in the energy flux, Eq.~\eqref{resflux}. 
  
 
\subsection{Merger and ring-down calibration of the EOB model}

The merger of two non spinning BHs generates a distorted Kerr BH whose gravitational radiation can be modeled using a superposition of quasi normal modes (QNMs), which are described by the indices (\(\ell,\, m,\,n\)), where \((\ell,\,m\)) denote the mode and \(n\) specifies the tone.  Each of these modes has a complex frequency \(\sigma_{\ell m n}\) given by

\begin{equation}
\sigma_{\ell mn} = \omega_{\ell m n} -i/\tau_{\ell m n},
\label{qnms}
\end{equation}

\noindent where the real/imaginary part \( \omega_{\ell m n}/\tau_{\ell m n}^{-1}\) corresponds to the frequency/inverse damping time of each QNM. These two observables are uniquely determined by the mass and spin of the Kerr BH formed after merger \cite{rdeq}. The prescription used to compute these quantities is given by  \cite{BuonannoEOBv2Main}

\begin{subequations}
\label{finalMS}
\begin{eqnarray}
    \!\!\!\!\!\!\frac{M_f}{M} &=& 1+\left (\sqrt{\frac{8}{9}}-1\right )\eta-0.4333 \eta^2-0.4392 \eta^3, \\
    \!\!\!\!\!\!\!\!\frac{a_f}{M_f} &=& \sqrt{12}\eta-3.871 \eta^2+4.028 \eta^3.
\end{eqnarray}
\end{subequations}

It is worth pointing out that a mode \((\ell,m\)) always consists of a superposition of two different frequencies/damping times. These `twin modes' are given by \( \omega'_{\ell m n} = -  \omega_{\ell -m n}\) and \(\tau'_{\ell m n} = \tau_{\ell -m n}\).  However, when considering two initially non-spinning BHs, the mirror solutions are degenerate in the modulus of the frequency and damping time, and hence one has that \( \omega_{\ell m n}>0\) and \( \tau_{\ell m n} >0\). 

Following \cite{BuonannoEOBv2Main}, the merger-ringdown waveform may be written as follows 

\begin{equation}
  \label{RD}
  h_{\ell m}^\mathrm{merger-RD}(t) = \sum_{n=0}^{N-1} A_{\ell mn}\,e^{-i\sigma_{\ell mn} (t-t_\mathrm{match}^{\ell m})},
\end{equation}
 
 \noindent where \(N\) is the number of overtones included in the model, i.e., \(N=8\), and \(A_{\ell mn}\) are complex amplitudes which will be determined by smoothly matching the inspiral-plunge waveform (Eq.~\eqref{hip}) with its merger ring-down counterpart (Eq.~\eqref{RD}).  
 
 To determine the complex amplitudes \(A_{\ell mn}\), one defines $t_\mathrm{match}^{\ell m}$ as the time at the amplitude maximum of the $h^{\rm EOB}_{\ell m}$ mode, namely, $t_\mathrm{match}^{\ell m}=t_\mathrm{max}^{\Omega}+\Delta t_\mathrm{max}^{\ell m}$, and demand continuity of the waveform at \(N-2\) points sampled in the time range \([ t_\mathrm{match}^{\ell m} - \Delta t_\mathrm{match}^{\ell m},\, t_\mathrm{match}^{\ell m}]\), and ensure the continuity and differentiability of the waveforms at \(t_\mathrm{match}^{\ell m} - \Delta t_\mathrm{match}^{\ell m}\) and \(t_\mathrm{match}^{\ell m}\)  (see Eqs. (36a)-(36c) in \cite{BuonannoEOBv2Main}).
 
Finally,  the full waveform can be written as

 \begin{equation}
  \label{eobfullwave}
  h_{\ell m} = h_{\ell m}^\mathrm{insp-plunge}\, {\mathcal H}(t_\mathrm{match}^{\ell m} - t) + h_{\ell m}^\mathrm{merger-RD}\,{\mathcal H} (t-t_\mathrm{match}^{\ell m})\,.
\end{equation}
 
 \noindent where \( {\mathcal H}(t)\) is the Heaviside step function. 
 
 This is the model we shall use in the following Section to develop an IMRI waveform model to explore the form of the self-force in the intermediate-mass ratio regime. We will consider three different systems with mass ratios 17:100, 10:100 and 1:100. We have chosen these systems because: a) the EOBNRv2 model was calibrated by comparison to NR simulations for events with mass-ratio 1:1-1:6, and hence systems with component masses 17:100 should be accurately modeled using EOBNRv2; b) the actual construction of the EOB model is such that it reproduces the expected dynamics of systems with small mass-ratio, e.g., \(\sim\) 1:100. This modeling statement will not be taken for granted in our subsequent analysis. We will show in the following Section that this is indeed the case by comparing results between EOBNRv2 and those obtained using Teukolsky data; c) we will explore an additional case, 1:10, for which the EOBNRv2 is the best model currently available to shed light on the form that self-force corrections 
should have so as to reproduce the dynamical evolution of these type of GW sources. Studying these type of sources is also important to try to bridge the gap in the parameter space covered by current waveform models.  
 
  
 \section{Self-force corrections in the intermediate-mass-ratio regime}
 \label{s2}
 
 In the previous Section, we described the model we shall now use to build a waveform model for intermediate-mass-ratio inspirals (IMRIs). Using this IMRI model we will explore whether the inclusion of available self-force corrections, which have been computed in the extreme-mass-ratio regime~\cite{baracknewphi,sago,inner}, are able to reproduce the inspiral evolution of intermediate-mass-ratio binary black holes. 
 
 
 \subsection{IMRI self-force model}
 
 In this Section we introduce the key elements we need to build our IMRI model which is simple and flexible enough to explore the form of the self-force in the intermediate-mass-ratio regime, and that, at the same time is able to reproduce with great accuracy the binary's dynamical evolution as predicted by EOBNRv2. In the following we assume that EOBNRv2 dynamics provides a good description of the actual binary black hole dynamical evolution. This is a reasonable assumption because, at present, EOBNRv2 is the best interface to translate NR simulations into a waveform model that uses coordinates which can be related to physical units~\cite{damsh}. This desirable modeling approach is obtained by comparing EOB gravitational waveforms to NR waveforms as seen by an observer at infinity~\cite{NRPNComparisonBoyleetal}.  The fact that this model has been calibrated to NR simulations of mass-ratios 1:1-1:6 also means that such a model provides the adequate arena for the studies we want to carry out.  Having said 
that, we now describe 
the approach we shall follow to build our IMRI model:
 
 \begin{enumerate}
 \item We start with an ansatz for the orbital frequency evolution of our IMRI model which is inspired by recent studies on the inclusion of linear order self-force corrections in the EOB approach~\cite{barus}. We develop this prescription so as to faithfully reproduce the EOBNRv2 orbital frequency evolution (See Figure~\ref{omegafit}).  An accurate modeling of the orbital frequency is necessary to build the gauge-invariant object \(x=\hat\Omega^{2/3}\), which is a crucial element in our analysis. 
 \item We propose an ansatz to model the IMRI gravitational wave angular momentum flux which we calibrate so as to faithfully reproduce its EOBNRv2 counterpart (see Figures~\ref{fluxfitlight} and \ref{fluxfitheavy}). Note that the EOBNRv2 angular momentum flux used for this calibration  is constructed by summing over  35 leading and subleading waveform multipoles \(h_{\ell m}\), with  \(2\leq \ell \leq 8\), \(1\leq m\leq \ell\).  We include all these modes so as to model the radiative part of the self-force in our IMRI model as accurately as possible.
 \item Having derived an accurate prescription for the orbital frequency and the angular momentum flux, we make use of a prescription for the angular momentum that goes beyond the test-mass particle limit and which includes conservative self-force corrections. This expression for \(\hat{L}_z (x)\) encodes conservative self-force corrections in the redshift observable \(z_{\rm SF}\) (see Eq.~\eqref{zedsf}). 
 \item Because the prescriptions for the orbital frequency and the flux of angular momentum of our IMRI model reproduce their EOBNRv2 counterparts with great accuracy (see Figures~\ref{omegafit} and \ref{fluxfitlight}), we can now use these objects to constrain the form of the gauge-invariant expression of the angular momentum \(\hat{L}_z (x)\) in Eq.~\eqref{drdt} below by demanding internal consistency in our model, i.e., by reproducing faithfully the dynamical evolution of the binary black hole as predicted by EOBNRv2. This is equivalent to exploring the form of the red-shift observable \(z_{\rm SF}\) which reproduces the binary's dynamical evolution as predicted by EOBNRv2. This calibration procedure enable us to reproduce the expected inspiral evolution point to point to better than one part in a thousand, as shown in Figures \ref{dv15100} and \ref{randphiaccuracy}.
 \item Having determined the form of \(z_{\rm SF}\) that reproduces the dynamical evolution of the systems considered in this analysis, we compare the gauge-invariant expression of \(\hat{L}_z (x)\)  using our results and self-force corrections obtained in the context of EMRIs. We show that available self-force corrections do not reproduce the late inspiral evolution for systems with mass-ratio 1:6 and 1:10, but that they provide a fair description for systems with mass-ratio 1:100. This result is to be expected because these available self-force corrections were derived in the small mass-ratio limit.
 \end{enumerate}
 
 Having described the approach to be followed, we start off by describing the ansatz we use to model the orbital frequency. To do this,  we follow \cite{rev} and \cite{bernu}, and use the following prescription

\begin{equation}
\Omega=\frac{A}{\hat{H}}\frac{p_{\phi}}{r^2}.
\label{sOM}
\end{equation}

\noindent Furthermore, assuming circular orbits, we can simplify Eq.~\eqref{sOM} using the following relations for the conservative Hamiltonian $\hat{H}$

\begin{equation}
\hat{H}=\sqrt{A(u)\left(1+j_0^2 u^2\right)}, \quad {\rm with} \quad p_\phi=j_0\,, \quad {\rm and} \quad j_0^2=-\frac{A'(u)}{\left(u^2A(u)\right)'}\,,
\label{assum}
\end{equation}
 
\noindent where \(u=1/r\) and \('\) denotes \(d/du\). Using a prescription similar to that introduced in \cite{barus}, we shall use the following ansatz for the potential \(A(u)\)

\begin{equation}
A_{\rm ansatz}(u)= 1-2u+\eta\left( \sqrt{1-3u}\,k_{\rm fit} -u \left(1+\frac{1-4u}{\sqrt{1-3u}}\right)\right), \quad {\rm with} \quad k_{\rm fit} = 2u\frac{1+a_1u+a_2 u^2}{1+a_3 u +a_4 u^2 + a_5 u^3}\,.
\label{ansatz}
\end{equation}

\noindent Under these assumptions, the prescription for the orbital frequency takes the simple form

\begin{equation}
\Omega(u)_{\rm ansatz}= u^{3/2}\sqrt{-\frac{A'_{\rm ansatz}(u)}{2}}\,.
\label{ompres}
\end{equation}

\noindent The values of the \(a_i\) parameters in the function \(k_{\rm fit}\) are given in Table~\ref{coef}. It is worth pointing out that this prescription captures accurately the orbital phase evolution of the sources described above all the way down to ISCO and slightly beyond with much less computational complexity than EOBNRv2.  The actual comparison between this scheme and the EONBRv2 model is shown in Figure~\ref{omegafit}. Notice that our IMRI model does pretty well from large \(r\) to the fast-motion strong-field regime in all three cases shown. Notice that we are modeling the potential  \(A_{\rm ansatz}(u)\) using the coordinate \(u\) and not \(M/x\) as done in \cite{barus}. The rationale for doing this is to use a simple, yet accurate, prescription for the orbital frequency that captures faithfully the EOBNRv2 orbital frequency evolution. We could have also used a different prescription for this object, i.e., a PN series~\cite{cons}, a Pad\'e resummed expression, etc. The purpose this object 
serves at this stage in the analysis is to reproduce accurately the orbital evolution predicted by EOBNRv2 with less computational complexity.
 


\begin{table}[ht!]
\centering
\begin{tabular}{c c c c c c}
\hline\hline
  $\eta$& $a_1$ & $a_2$ & $a_3$ & $a_4$ & $ a_5 $  \\  
    $\frac{1700}{13689}$& -7.458 & 15.0179 & -7.208 & 12.152 & 6.103 \\ [1ex]
   $\frac{10}{121}$& -8.037 & 17.059 & -7.826 & 14.589 & 5.393\\ [1ex]
    $\frac{100}{10201}$&-9.535 & 22.984 & -9.456 & 21.862 & 3.085 \\ [1ex]
\hline
\end{tabular}
\caption{\(\Omega_{\rm ansatz}\) fit coefficients}
\label{coef}
\end{table}




\begin{figure*}[ht]
\centerline{
\includegraphics[height=0.38\textwidth,  clip]{figures/insimri/rom17100bold}
\includegraphics[height=0.38\textwidth,  clip]{figures/insimri/rom10100bold}
}
\centerline{
\includegraphics[height=0.38\textwidth,  clip]{figures/insimri/rom1100bold}
\includegraphics[height=2.73in, width=3.75in,  clip]{figures/insimri/omac}
}
\caption{The panels show the orbital frequency, as a function of the radial coordinate \(r\),  obtained using the fit described in the text (solid red line), and the orbital frequency predicted by the EOB model (solid black line). The systems shown in the panels correspond to binaries of component masses \(17 M_{\odot}+100M_{\odot}\) (top-left panel), \(10 M_{\odot}+100M_{\odot}\) (top-right) and  \(1 M_{\odot}+100M_{\odot}\) (bottom-left panel). The bottom-right panel demonstrates the accuracy with which our IMRI model reproduces the orbital frequency predicted by the EOBNRv2 for the systems 17:100 (dashed blue), 10:100 (dashed-dot red) and 1:100 (solid black). Note that the spikes are due to artifacts in the interpolation function used to plot the EOBNRv2 orbital frequency and the numerical fit used to reproduce it. Also notice that the discrepancy between the data and the fit is always smaller than one part in a thousand. }
\label{omegafit}
\end{figure*}

Having derived a prescription for the orbital evolution, we require a prescription to generate the inspiralling trajectory of the stellar mass compact object. The first step to achieve this consists of deriving a consistent model for the flux of angular momentum. The model for the flux of angular momentum that we introduce in the following Section sums up the effect of including all the dominant and subdominant modes currently available in the literature, i.e., \(2\leq \ell \leq 8, \, 1\leq m \leq \ell\), with the advantage that the inclusion of extra modes does not impact the computational cost of generating the IMRI waveform. 


\subsection{Radial evolution prescription}

An important component of the IMRI waveform model is the prescription used for the fluxes of energy and angular momentum. Deriving an accurate expression for these quantities is essential to capture the main features of true inspirals, as shown in \cite{improved} in the context of EMRIs. 


As discussed in Section~\ref{s1}, in the EOBNRv2 model the flux of energy is constructed using the prescription given by Eq.~\eqref{resflux}, which is obtained by summing over the modes  \(( h_{\ell m}\)) with \(2\leq \ell \leq 8\), \(1\leq m\leq \ell\). Using this prescription as input data and the fact that in the EOB formalism the following relation is fulfilled~\cite{barus}

\begin{equation}
\dot E= \Omega\dot L_z,
\label{circr}
\end{equation}

\noindent we derive a prescription for the gravitational-wave angular momentum flux which is valid from early inspiral all the way to the ISCO and slightly beyond.  This prescription, which encapsulates the contribution from 35 dominant and subdominant modes, is inspired by the modeling of accurate EMRI waveform models introduced in~\cite{improved}, i.e., 

\begin{equation}
\left(\dot L_z\right)_{\rm fit} = -\frac{32}{5}\frac{\mu^2}{M}\frac{1}{r^{7/2}}\Bigg[1-\frac{1247}{336}\frac{1}{r}+4\pi\frac{1}{r^{3/2}}- \frac{44711}{9072}\frac{1}{r^2} + \frac{1}{r^{5/2}}\left(c^1_{2.5} +  c^2_{3}\frac{1}{r^{1/2}}+c^3_{3.5}\frac{1}{r}\right)\Bigg],
\label{lzfluxfit}
\end{equation}

\noindent and the coefficients in Eq.~\eqref{lzfluxfit} are given in Table~\ref{Lzdotcoef}.

\begin{table}[ht]
\centering
\begin{tabular}{c c c c}
\hline\hline
  $\eta$& $c^1_{2.5}$ & $c^2_{3}$ & $c^3_{3.5}$   \\  
    $\frac{1700}{13689}$& -121.903 & 551.141 &-694.699  \\ [1ex]
   $\frac{10}{121}$& -110.657 & 360.261 &-156.016 \\ [1ex]
    $\frac{100}{10201}$&-75.255 & 333.449 & -363.505  \\ [1ex]
\hline
\end{tabular}
\caption{\(\dot{L}_z\) fit coefficients}
\label{Lzdotcoef}
\end{table}


Figures~\ref{fluxfitlight} and \ref{fluxfitheavy} present the comparison between the EOBNRv2 flux, which includes all currently known dominant and subdominant modes, and the calibrated angular momentum flux \(\left(\dot L_z\right)\). Notice that this computationally inexpensive scheme does pretty well from large \(r\) all the way to the ISCO and slightly beyond. Hence, this approach will enable us to capture the main features of the inspiral evolution of the systems under consideration in the regime of interest with good accuracy. Modelling the prescription of the flux of angular momentum in our IMRI waveform using all the dominant and subdominant modes in the EOBNRv2 model is equivalent to modeling the radiative part of the self-force with the best information currently available in the literature. 



\begin{figure*}[ht]
\centerline{
\includegraphics[height=0.38\textwidth,  clip]{figures/insimri/lzdotp17100}
\includegraphics[height=0.38\textwidth,  clip]{figures/insimri/lzdotp10100}
}
\centerline{
\includegraphics[height=0.38\textwidth,  clip]{figures/insimri/lzdotp1100}
\includegraphics[height=2.6in, width=3.8in,  clip]{figures/insimri/facc}
}
\caption{We compare the calibrated flux of angular momentum described in Eq.~\eqref{lzfluxfit} against the prediction of the EOBNRv2 model for binaries of component masses \(17M_{\odot} + 100M_{\odot}\) (top left panel),  \(10M_{\odot} + 100M_{\odot}\) (top right panel), and \(1M_{\odot} + 100M_{\odot}\) (bottom-left panel). The bottom-right panel shows the accuracy with which the IMRI model reproduces the EOBNRv2 angular momentum flux, \(\dot{L}=dL_z/dt\), for the systems: 17:100 (dashed blue), 10:100 (dashed-dot red) and 1:100 (solid black). The spikes in the bottom-right panel are due to numerical artifacts of the interpolating function used to plot the EOBNRv2 angular momentum flux and the numerical fit to reproduce it. The fit is such that its discrepancy to the EOBNRv2  is always smaller than one part in a thousand.}
\label{fluxfitlight}
\end{figure*}


In Figure~\ref{fluxfitheavy} we present a comparison between the flux of angular momentum predicted by the EOBNRv2 model and the fit to Teukolsky data proposed in \cite{improved} for extreme-mass-ratio inspirals. Note the remarkable agreement between both formalisms all the way down to the ISCO. This comparison also confirms that the construction of the EOBNRv2 captures the main features of inspirals with small mass-ratios which are modeled using black hole perturbation theory. This comparison also shows that the EOBNRv2 model encodes fairly well the radiative part of the self-force for systems with small mass-ratio. 


\begin{figure*}[ht]
\centerline{
\includegraphics[height=0.38\textwidth,  clip]{figures/insimri/TEOBLzdot}
\includegraphics[height=2.6in, width=3.8in,  clip]{figures/insimri/teuvseobflux}
}
\caption{The left panel shows the angular momentum flux for binaries of mass ratio 1:100 computed using the EOBNRv2 model and the fit to Teukolsky data introduced in \cite{improved}. Both formalisms present a remarkable agreement all the way down to the ISCO. Note that the angular momentum flux fit to Teukolsky data is valid only from early inspiral until the ISCO. The right panel shows the relative difference between the two prescriptions for the angular momentum flux \(\dot{L}=dL_z/dt\).}
\label{fluxfitheavy}
\end{figure*}


Having found a prescription for the angular momentum that captures the dynamics of IMRIs, we can now use it to generate the inspiral trajectory of the stellar mass compact object that inspirals into an IMBH using the relation 

\begin{equation}
\frac{d r}{d t}= \frac{d L_z}{d t}\frac{d r}{d L_z}.
\label{drdt}
\end{equation}

\noindent  The first term on the right hand side of Eq.~\eqref{drdt} can be obtained from Eq.~\eqref{lzfluxfit}. With regard to the derivative of the angular momentum with respect to the radial coordinate, we need to use a prescription for the angular momentum that goes beyond the test-mass particle limit, as conservative self-force corrections play a more significant role in this regime. Hence, we use the relation derived by Barausse et al in \cite{barus} which includes conservative self-force corrections, i.e.,

\begin{equation}
\hat{L}_z (x)= \frac{L_z}{\mu M}= \frac{1}{\sqrt{x(1-3x)}} + \eta\left(-\frac{1}{3 \sqrt{x}} z'_{\rm SF}(x) + \frac{1}{6\sqrt{x}}\frac{4-15x}{\left(1-3x\right)^{3/2}}\right) + {\cal O} (\eta^2),
\label{angmom}
\end{equation}

\noindent where \(x=(M\Omega)^{2/3}\) and \('\) stands for \(d/dx\). Furthermore,

\begin{equation}
z_{\rm SF} (x)= 2x \frac{1+b_1 x+b_2 x^2}{1+b_3x +b_4 x^2 + b_5 x^3},
\label{zedsf}
\end{equation}

\noindent and the various coefficients \(b_i\) were derived in \cite{barus} using available self-force data. In this section we derive the coefficients in Eq.~\eqref{zedsf} that reproduce the actual inspiral evolution all the way down to the ISCO. Note that in Eq.~\eqref{drdt} we have in place a prescription for the angular momentum flux which faithfully reproduces the expected loss of angular momentum even beyond the ISCO,  as compared to the EOBNRv2 model. Hence, the only ingredient that needs to be tuned to reproduce the expected inspiral evolution is contained in Eq.~\eqref{angmom}. We have followed this approach to explore the form that \(\hat{L}_z (x)\) should have. Put in different words, the form that the self-force redshift observable \(z_{\rm SF} (x)\) should have to reproduce an inspiral trajectory consistent with EOBNRv2 all the way down to the ISCO.  The various coefficients of Eq.~\eqref{zedsf} that generate such an inspiral orbit are given in Table~\ref{zedfunccoef}.

\begin{table}[ht]
\centering
\begin{tabular}{c c c c c c}
\hline\hline
  $\eta$& $b_1$ & $b_2$ & $b_3$ & $b_4$ & $ b_5 $  \\  
    $\frac{1700}{13689}$& -3.062 & 0.760 & -3.261 & 0.550 & -6.000 \\ [1ex]
   $\frac{10}{121}$& -3.260 & 0.892 & -3.475 & 0.600 & -6.829\\ [1ex]
    $\frac{100}{10201}$&-3.370 & 1.570 & -3.730 & 0.970 & -7.280 \\ [1ex]
\hline
\end{tabular}
\caption{\(z_{\rm SF} (x)\) fit coefficients}
\label{zedfunccoef}
\end{table}

In Figure~\ref{dv15100} we show that this prescription captures with great accuracy the features of EOBNRv2  inspirals. To recap,  the IMRI model incorporates both first-order conservative corrections --through  the construction of the gauge invariant expression of the angular momentum in  Eq.~\eqref{angmom}--- and first-order radiative-corrections through the construction of the flux of angular momentum in Eq.~\eqref{lzfluxfit}. Notice that the latter object encodes the best information currently available in the literature, since it includes the contribution from 35 dominant and subdominant \((\ell,m)\) modes of the multipolar waveform to the angular momentum flux. Using this IMRI model, we have been able to constrain the form of the angular momentum \(L_z(x)\) that reproduces the inspiral evolution predicted by the EOBNRv2 model.

\begin{figure*}[ht]
\centerline{
\includegraphics[height=0.38\textwidth,  clip]{figures/insimri/tphiim17100k}
\includegraphics[height=0.38\textwidth,  clip]{figures/insimri/rtim17100k}
}
\centerline{
\includegraphics[height=0.38\textwidth,  clip]{figures/insimri/tphiim10100k}
\includegraphics[height=0.38\textwidth,  clip]{figures/insimri/rtim10100k}
}
\centerline{
\includegraphics[height=0.38\textwidth,  clip]{figures/insimri/tphiim1100k}
\includegraphics[height=0.38\textwidth,  clip]{figures/insimri/trim1100k}
}
\caption{The panels show the radial and azimuthal evolution for a \(17M_{\odot} + 100M_{\odot}\) system (top panels), \(10M_{\odot} + 100M_{\odot}\) system (middle panels), and \(1M_{\odot} + 100M_{\odot}\) system (bottom panels), obtained using the IMRI model compared against the EOBNRv2 model.}
\label{dv15100}
\end{figure*}

\begin{figure*}[ht]
\centerline{
\includegraphics[height=2.6in, width=3.8in,  clip]{figures/insimri/radacc}
\includegraphics[height=2.6in, width=3.8in,  clip]{figures/insimri/phaacc}
}
\caption{The panels show the accuracy with which the IMRI model proposed in the paper reproduces the radial (right panel) and azimuthal (left-panel) time evolution predicted by the EOBNRv2 model for the systems  \(17M_{\odot} + 100M_{\odot}\) (dashed blue), \(10M_{\odot} + 100M_{\odot}\) (dashed-dot red), and \(1M_{\odot} + 100M_{\odot}\) (solid black). Notice that our model reproduces the EOBNRv2 orbital and azimuthal evolution point to point with an accuracy better than  one part in a thousand. }
\label{randphiaccuracy}
\end{figure*}



In order to show the importance of increasing our knowledge of the self-force in the intermediate-mass-ratio regime, we present in Figure~\ref{emimcomp}  three different curves which describe the time evolution of three binary systems during late inspiral. These plots show that if we use a waveform model (unfitted) that includes: a) an accurate prescription for the orbital frequency \(\hat\Omega\); b) a prescription for the flux of angular momentum that incorporates the contribution from all dominant and subdominant \((\ell,m)\) modes; c) an invariant expression for the angular momentum using the object \(x=\hat\Omega^{2/3}\) (see  Eq.~\eqref{angmom}), and; d) available self-force corrections to constrain the coefficients \(b_i\)~\cite{barus}, then such a model would generate an inspiral trajectory that deviates from the expected orbital evolution, in particular near the ISCO. 


Figure~\ref{emimcomp} also shows that our model actually predicts the expected orbital evolution all the way down to the ISCO, in agreement with the EOBNRv2 model, and requires a different set of coefficients  \(b_i\) in the redshift observable \(z_{\rm SF} (x)\), as compared with results in the extreme-mass-ratio limit quoted in~\cite{barus}. Notice that this is not only a modeling issue. It is an indication that implementing available self-force data into IMRI waveform models will not render the correct inspiral evolution, at the very  least for the cases we have considered. This is an important result of this paper. In order to substantiate this statement, in the following Section we will compute the value of the gauge invariant angular momentum \(L_z(x)\) at the ISCO within both the self-force formalism and our IMRI model. We will also show that the evolution of \(L_z(x)\) is consistent between both formalisms during early inspiral, but that the form of this object differs as we near the ISCO.  

One may also expect that for binaries with small mass-ratios, available self-force corrections may provide a fairly good description of the inspiral evolution. This is what we actually see in the bottom panel of Figure~\ref{emimcomp}. We will also show in the following Section that the evolution of \(L_z(x)\) for binaries with mass-ratio 1:100 is pretty consistent from early inspiral to the ISCO with EOBNRv2. This may not be surprising, since current self-force data have been obtained in the context of EMRIs and the EOBNRv2 has been calibrated so as to reproduce the dynamical evolution of binary black holes of small mass-ratio~\cite{resu,raci,nic,nic1}. This exercise then suggests that it may be necessary to go beyond first-order conservative corrections to reproduce accurately the expected dynamical evolution of intermediate-mass ratio systems with \(\eta \sim 10^{-2}-10^{-1}\).



\begin{figure*}[ht]
\centerline{
\includegraphics[height=0.38\textwidth,  clip]{figures/insimri/imem17100comp}
\includegraphics[height=0.38\textwidth,  clip]{figures/insimri/imem10100comptwo}
}
\centerline{
\includegraphics[height=0.38\textwidth,  clip]{figures/insimri/imem1100comp}
\includegraphics[height=2.6in, width=3.8in,  clip]{figures/insimri/radaccemri}
}
\caption{The panels show the radial evolution using the EOBNRv2 model, the IMRI model described in the text, and a model (Unfitted) which is the same as the IMRI model described in text except for the fact that the coefficients used in Eq.~\eqref{zedsf} for the function  \(z_{\rm SF} (x)\) were derived using available self-force data from extreme-mass ratio calculations~\cite{barus}. The plots correspond to binaries of component masses \(17M_{\odot} + 100M_{\odot}\) (top-left panel), \(10M_{\odot} + 100M_{\odot}\) (top-right panel) and \(1M_{\odot} + 100M_{\odot}\) (bottom-left panel). The bottom-right panel shows that the orbital evolution predicted by a model that incorporates self-force corrections from extreme-mass-ratio inspirals (EMRIs) deviates from the orbital evolution predicted by the EOBNRv2 model at late inspiral. This discrepancy is more noticeable for systems with mass-ratios 17:100 (dashed blue) and 10:100 (dashed-dot red). Furthermore, for smaller mass-ratios, 1:100 (solid black), the 
discrepancy becomes comparatively smaller, as expected. }
\label{emimcomp}
\end{figure*}



To carry out the analysis described above in the following Section, we will start by using the EOBNRv2 model to compute the orbital frequency ISCO shift for binaries of mass ratio 1:1, 1:2, 1:3, 1:4, 1:5, 1:6, 1:10 and 1:100, along with perturbative results in the context of extreme-mass-ratio inspirals. We will use this expression for the orbital frequency ISCO shift  to evaluate the gauge-invariant object \(x\)  and then compute the value of the angular momentum at the ISCO using Eq.~\eqref{angmom}.  We should also acknowledge the fact that EOB has not yet been calibrated using NR simulations of systems with mass-ratio 1:100. It is expected that EOBNRv2 provides a fair description of the dynamical evolution of these types of sources,  and we show in the following section that both EOBNRv2 and perturbative calculations provide a very consistent modeling of the gauge-invariant angular momentum \(\hat{L}_{z}\) from early inspiral to the ISCO for binaries with mass-ratio 1:100. However, this study should be 
compared to accurate NR simulations of mass-ratio 1:100~\cite{carlos} once these are generated with several gravitational waveform cycles before merger.


\clearpage


\section{ISCO shift: connecting the extreme, intermediate and comparable-mass ratio regimes}
\label{s3}

In the previous Sections we have mentioned that during the inspiral of a stellar mass compact object of mass \(m_2\) into a supermassive BH of mass \(m_1\), the radiative part of the self-force drives the inspiral evolution of the small object, whereas its conservative part has a cumulative effect on the orbital phase evolution \cite{SFB}. These two effects have been considered in the development of EMRI waveform templates  \cite{amos, cutler, gairles, kludge, improved, lisacapt, seoane}. 

We shall now consider a novel effect that was explored by Barack \& Sago~\cite{inner}. They have shown that the self-force also introduces shifts in the innermost stable circular orbit radius and frequency. For a test-mass particle, these two quantities are given by

\begin{equation}
r_{\rm ISCO} = 6m_1,  \qquad m_1\Omega_{\rm ISCO} = \frac{1}{6\sqrt{6}},
\label{tmpt}
\end{equation}

\noindent whereas, for finite \(\eta\), in the Lorenz gauge, these two quantities take the form~\cite{inner} 

\begin{equation}
\Delta r_{\rm ISCO} = -3.269(\pm3\times10^{-3}) m_2,  \qquad \frac{\Delta\Omega_{\rm ISCO}}{\Omega_{\rm ISCO}} = 0.4870(\pm6\times10^{-4})\frac{m_2}{m_1}.
\label{shift}
\end{equation}

Since Barack \& Sago carried out these calculations in Lorenz gauge, it was necessary to translate these results into coordinates that are commonly used for GW observations, i.e., asymptotically flat coordinates.  This exercise has been done for the orbital frequency, which is a gauge invariant object, and hence can be compared to results obtained in alternative formalisms, such as PN theory or the EOB approach.  In \cite{baracknewphi} and \cite{damsh}, the authors derive the `renormalization' factor that translates results of Lorenz-gauge calculations into  physical units. Applying this renormalization technique, one finds that the ISCO frequency is given by  

\begin{equation}
M\Omega_{\rm ISCO} = \frac{1}{6\sqrt{6}}\left(1+1.2512\eta + {\cal{O}}(\eta^2)\right),
\label{bsshift}
\end{equation}

\noindent with \(M=m_1+m_2\). This prediction can be compared with PN--based ISCO calculations at different orders of accuracy \cite{favata}

\begin{eqnarray}
M\Omega^{2{\rm PN}}_{\rm ISCO} &=&\frac{1}{6\sqrt{6}}\left(1+\frac{7}{12}\eta + {\cal{O}}(\eta^2)\right), \\\nonumber
M\Omega^{3{\rm PN}}_{\rm ISCO} &=&\frac{1}{6\sqrt{6}}\left(1+ \left(\frac{565}{288} - \frac{41}{768}\pi^2\right)\eta + {\cal{O}}(\eta^2)\right). \\\nonumber
\label{pnshift}
\end{eqnarray}

\noindent The EOB approach has also been used to describe the orbital frequency shift at ISCO. Damour suggested in \cite{damsh} that a fit for the ISCO orbital frequency shift that incorporates results from the gravitational self-force and NR simulations may be a quadratic polynomial in \(\eta\) of the form \cite{damsh} 

\begin{equation}
M\Omega^{\rm Damour}_{\rm ISCO} = \frac{1}{6\sqrt{6}}\left(1+1.25\eta + 1.87\eta^2\right).
\label{damshift}
\end{equation}


\noindent In this paper we build up on this analysis and update this estimate using results from the self-force program, and making use of the EOBNRv2 model, which has been calibrated to NR simulations~\cite{BuonannoEOBv2Main}. The prescription for the orbital frequency ISCO shift that we propose below reproduces accurately the results predicted by the self-force program for EMRIs, and also reproduces the best data currently available for intermediate and comparable-mass systems.  To compute the ISCO orbital frequency shift within the EOB formalism, we use the equation derived in \cite{damsh}, namely,


\begin{equation}
2A(u)A'(u) + 4u\left(A'(u)\right)^2 - 2uA(u)A''(u)=0,
\label{damisco}
\end{equation}

\noindent where \(u=1/r\) and \('\) stands for \(d/du\). This ISCO condition can be rewritten in terms of the radial coordinate as \cite{favata}

\begin{equation}
rA(r)A''(r) -2r\left(A'(r)\right)^2 +3A(r)A'(r)=0,
\label{iscoeq}
\end{equation}

\noindent where \('=d/dr\). We shall use the metric coefficient \(A(r)\) quoted in \cite{BuonannoEOBv2Main}, which includes the Pad\'e expression for \(A(r)\) at 5PN order.  Having obtained the value for \(r_{\rm  isco}\) using Eq.~\eqref{iscoeq}, we evaluate the angular orbital frequency  \(M\Omega_{\rm ISCO}\) at  this fiducial value using Eq. (10b) of \cite{BuonannoEOBv2Main} with \(p_r =0\). 

Using a variety of events, including extreme, \(\eta\sim 10^{-5}\), and intermediate, \(\eta\sim 10^{-2}-10^{-1}\), mass ratio inspirals, we derive a quadratic polynomial fit  in \(\eta\) for the ISCO orbital frequency shift for these types of events, namely

\begin{equation}
M\Omega^{\rm fit}_{\rm ISCO} = \frac{1}{6\sqrt{6}}\left(1+1.05786\eta + 2.12991\eta ^2\right).
\label{newshift}
\end{equation}

\noindent We found that at second order in \(\eta\), this numerical fit does not reproduce accurately the orbital frequency ISCO shift for small-mass ratios~\cite{inner}. We can fix these problems using a prescription of the form

\begin{equation}
M\Omega^{\rm fit}_{\rm ISCO} = \frac{1}{6\sqrt{6}}\left(1 + 1.2512\eta - 0.0553751\eta^2 + 5.78557\eta^3\right).
\label{comisco}
\end{equation}

\noindent At this level of accuracy we exactly reproduce the prediction for EMRIs for small \(\eta\), as well as the most-up-to-date results for binaries modeled using the EOBNRv2 scheme. We compare the range of applicability of this numerical expression, along the other various approximations mentioned above,  in Figure~\ref{iscoshift}.

 
\begin{figure*}[ht]
\centerline{
\includegraphics[height=0.44\textwidth,  clip]{figures/insimri/mwiscotwo}
}
\caption{ISCO shift using various approximations as described in the text. The `Numerical Data' has been obtained using calculations in the extreme-mass ratio regime~\cite{inner} and the EOBNRv2. The prescription that encapsulates results from the extreme, intermediate and comparable mass-ratio regime is labeled as `Numerical Fit' and is given by Eq.~\eqref{comisco} in the main text.  }
\label{iscoshift}
\end{figure*}


With our prescription for \(M\Omega_{\rm ISCO}\), we can evaluate the value of the gauge-invariant angular momentum flux at the ISCO. To do so, we compute the value of the angular momentum, see Eq.~\eqref{angmom}, using available self-force data (SF Fit), i.e., we use Eq.~\eqref{bsshift} to compute the shift in the orbital frequency at the ISCO, and then use Eq.~\eqref{angmom} in conjunction with the values for the \(b_i\) coefficients of Eq.~\eqref{zedsf} quoted in~\cite{barus}, i.e., self-force corrections derived in the extreme-mass-ratio limit. We also present results for an `incomplete model', in which we use Eq.~\eqref{comisco} to evaluate the value of the orbital frequency at the ISCO, and then use Eq.~\eqref{angmom} with the set of \(b_i\) coefficients quoted in~\cite{barus}. Finally, the IMRI model encodes all the results derived in this paper, namely, we use the prescription for the shift of the orbital frequency at the ISCO in Eq.~\eqref{comisco}, and the prescription for the angular momentum (Eq.
~\eqref{angmom}) using the corrections quoted in Table~\ref{zedfunccoef}.  Table~\ref{lzcomp} shows that, in accord with Figure~\ref{emimcomp}, for binaries with symmetric mass-ratio \(\eta \sim 0.01\), the value of the angular momentum evaluated at ISCO is fairly consistent between the two models. However,  the predicted value of the angular momentum at ISCO becomes more discrepant for binaries with   \(\eta \sim 0.1\). This then suggests that the evolution of intermediate-mass ratio inspirals cannot be fully captured by using self-force calculations from extreme-mass ratio inspirals. This can be better visualized in Figure~\ref{lzfull} where we show the angular momentum during inspiral all the way down to the ISCO using two formalism, namely, the IMRI prescription and a model that includes available self-force corrections, which is labeled as `Self Force'. 


\begin{table}[thb]
\begin{tabular}{|c|c|c|c|}
\hline\multicolumn{1}{|c|}{}&\multicolumn{3}{c|}{\(L_z(x_{\rm ISCO} )\)}\\\cline{2-4}
\multicolumn{1}{|c|}{\(\eta\)}&SF Fit&Incomplete&IMRI \\\cline{1-4}
$\frac{1700}{13689}$&3.3389 &3.3522&3.2918 \\[1ex]\cline{1-4}
$\frac{10}{121}$& 3.3876&3.3878&3.3254\\ [1ex]\cline{1-4}
$\frac{100}{10201}$& 3.4561&3.4561&3.4429\\[1ex]\hline
\end{tabular}
\caption{ The Table shows the value of the angular momentum \(L_z\) as a function of the gauge invariant object \(x=\left(M\Omega \right)^{2/3}\)  evaluated at the ISCO radius (see Eq.~\eqref{angmom}). The SF (self-force) values are computed using the prescription given in Eq.~\eqref{bsshift} to compute the orbital frequency shift, and Eq.~\eqref{angmom} with the coefficients \(b_i\) quoted in \cite{barus}, i.e., evaluated in the context of extreme-mass-ratio inspirals. Incomplete stands for a prescription in which we use the prescription for the orbital frequency given by Eq.~\eqref{comisco}, and the angular momentum prescription given by  Eq.~\eqref{angmom} with the coefficients \(b_i\) quoted in \cite{barus}. The   \(b_i\) values of the IMRI model are obtained using a model that reproduces the dynamical evolution of intermediate-mass-ratio inspirals, as compared with the EOBNRv2 model, and for which we have derived a new prescription for the red-shift observable \(z_{\rm SF} (x)\) which actually 
reproduces the features of true inspirals.}
\label{lzcomp}
\end{table}


\begin{figure*}[ht]
\centerline{
\includegraphics[height=0.38\textwidth,  clip]{figures/insimri/lz17100invbold}
\includegraphics[height=0.38\textwidth,  clip]{figures/insimri/lz10100invbold}
}
\centerline{
\includegraphics[height=0.38\textwidth,  clip]{figures/insimri/lz1100invbold}
\includegraphics[height=2.6in, width=3.8in,  clip]{figures/insimri/angmomacc}
}
\caption{The panels show the invariant angular momentum \(L_{z}(x)\), with \(x=\left(M\Omega\right)^{2/3}\), from early inspiral to the ISCO using two different prescriptions. The `IMRI' prescription reproduces accurately the expected inspiral evolution, as compared with results from the EOBNRv2 model ---which has been calibrated to NR simulations. The `Self-Force' prescription encapsulates self-force corrections derived in the context of EMRIs~\cite{sago,barus}. Note that for the three binary systems studied,  \(17M_{\odot} + 100M_{\odot}\) (top-left panel),  \(10M_{\odot} + 100M_{\odot}\) (top-right panel), and  \(1M_{\odot} + 100M_{\odot}\) (bottom-left panel), the `Self Force' prescription for the angular momentum presents a deviation from its expected value that becomes more noticeable near the ISCO. The bottom-right panel shows the fractional accuracy between the two different prescriptions for the angular momentum for the systems 17:100 (dashed blue), 10:100 (dashed-dot red) and 1:100 (solid black). 
Notice also that, as expected, for binaries with small mass-ratio the `Self Force' and `IMRI' prescriptions are fairly consistent all the way down to the ISCO.  As in the previous plots, the spikes in the bottom-right panel are due to numerical artifacts.}
\label{lzfull}
\end{figure*}

Table~\ref{lzcomp} and Figures~\ref{emimcomp}, \ref{lzfull} show that as the mass-ratio increases, the discrepancy between a model that incorporates available self-force corrections~\cite{sago} and one that has been calibrated to NR simulations becomes more pronounced. For events with mass-ratio 1:10 this difference looks slightly larger than for those of mass-ratio 17:100. Part of the reason for this behavior may be the fact that the model we have used to perform this analysis, EOBNRv2, provides the best prescription currently available for the inspiral evolution of sources with mass-ratios 1:1-1:6. It is expected that the model provides a good description of the inspiral evolution of sources with mass-ratio 1:10, and hence we have extended its realm of applicability to shed light on the form that the self-force is expected to have so as to reproduce the inspiral evolution of these type of sources. Thus, a conservative conclusion we can draw at this stage is that, using the best waveform model currently 
available in the literature, we have shown that  one may not be able to accurately model the dynamics of sources with mass-ratio 1:10 using available self-force calculations.  This trend is also present   in the case of events with mass-ratio 17:100 and the results in this case are more conclusive. We have shown that a waveform model that includes available self-force corrections will not be able to capture faithfully the inspiral evolution of these GW sources. These results suggest that one may need to include higher-order corrections in the self-force to capture the behavior of true inspirals in the intermediate-mass-ratio regime. Finally, our work is also a consistency check on the internal structure of the EOB model, since we have shown that the EOBNRv2 renders a good prescription for the inspiral evolution via the flux of angular momentum and that the angular momentum prescription is consistent with results obtained from perturbative calculations. 
\clearpage

 
 \section{Conclusions}
 \label{s4}
 
 In this paper we have developed a code to reproduce the analysis presented in \cite{BuonannoEOBv2Main}. Using this code we generated the inspiral evolution of three different systems to perform an exploratory study of the importance of including accurate self-force corrections in search templates that aim to detect non-spinning intermediate-mass ratio inspirals. The choice of the systems to perform this study reflects the knowledge we have at present on the dynamical evolution of non-spinning BH binaries (\(17M_{\odot} + 100M_{\odot}\)), what we want to know (\(10 M_{\odot} + 100M_{\odot}\)), and a special system that  provides reassurance that the dynamics of intermediate-mass-ratio inspirals with typical mass-ratios \(\eta\sim 0.01\) can be captured using perturbative theory, as shown in \cite{carlos,ulr}. 
 
 The EOBNRv2 model we have used as a benchmark to carry out this study has the advantage of encoding the best information currently available of non-spinning BH binaries, is capable of reproducing the expected dynamics in the test-mass particle limit, and also includes corrections taken from self-force corrections in the extreme-mass-ratio limit. We have confirmed these statements for systems with  \(\eta\sim 0.01\) by showing  that the EOBNRv2 does predict the expected form of the angular momentum flux, as compared to Teukolsky data, as well as the evolution  of the gauge-invariant angular momentum, as compared to self-force data. 
 
In order to explore the form of the self-force in the intermediate-mass-ratio regime, we developed an IMRI model that reproduces the inspiral evolution predicted by the EOBNRv2 model, and which enable us to explore the form that the self-force should have in this mass-ratio regime so as to reproduce the binary's dynamical evolution as predicted by the best available interface to NR simulations. We have found that for systems with component masses  \(17M_{\odot} + 100M_{\odot}\), available self-force corrections do not accurately reproduce the inspiral evolution. We have shown that there is a clear deviation from the true inspiral trajectory near the ISCO. We have also explored this issue in greater detail by showing that the gauge-invariant angular momentum does deviate from the current self-force prediction near the ISCO. When we extend the realm of applicability of the EOBNRv2 to binaries of mass-ratio 1:10, we observe a similar behaviour.  This exploratory study then suggests that it may be necessary to 
extend conservative corrections beyond the linear order to accurately capture the true inspiral evolution for sources with intermediate-mass-ratio. Furthermore, once NR simulations of systems with mass-ratios 1:10, 1:15 and 1:100 have reached enough resolution near merger, we will be in a good position to calibrate the EOB model so as to reproduce the true dynamical evolution for these type of sources \cite{carlos, carlosI,carlosII}. Such a model will enable us to further probe the parameter space to develop IMRI models, as the one proposed in this paper, that capture the features of true inspirals at a very inexpensive computational cost. 

We have also used our EOBNRv2 code to derive a prescription for the orbital frequency ISCO shift that encapsulates results from the extreme, intermediate and comparable-mass ratio regimes. Our prescription is the first in the literature that reproduces the self-force result in the appropriate limit.   We have made use of this new prescription to estimate the value of the angular momentum at the ISCO using our IMRI prescription and available self-force data. We have found a clear discrepancy for systems with mass-ratios \(\eta \sim 0.1\), but have confirmed that for systems with small mass ratios \(\eta\sim 0.01\) both predictions are fairly consistent. 

This study has also shed light on  a pressing issue that needs to be addressed before aLIGO begins observations, namely, at present we use templates  in searches for GW sources  whose actual dynamics are currently unknown. For instance, searches for BH binaries with total mass  \(25M_{\odot} - 200M_{\odot}\), and individual masses from \(3M_{\odot}\) to \(99M_{\odot}\) have been carried out, but not a single model has been calibrated using high-resolution NR simulations for mass-ratios smaller than 1:6. Hence, it is important that numerical relativists and search template developers interact more closely so as to run NR simulations which cover regions in parameter space where future GW detectors may detect GW sources. This meaningful interaction will be important from a theoretical perspective, as we will be able to further our knowledge of the self-force, and from a data analysis perspective, as we will be in a stronger position to develop accurate search templates for use in data analysis. 

The approach we have outlined in this paper is the initial step to construct inspiral-merger-ringdown IMRI waveforms. Having derived a consistent prescription for the inspiral evolution, we can use a similar approach to that described in~\cite{ori,firstpaper} to include merger and ring-down in a physically consistent way. Developing complete IMRI waveforms may be useful for aLIGO data analysis, as they will provide the accuracy of more complex waveform models at an inexpensive computational cost.  
  
\section{Appendix}

In this Section we show that the approach outlined in the main body of the paper can still be used to model equal-mass (EM) binaries with great accuracy at an inexpensive computational cost. To do so we consider a system with total mass \(20 M_{\odot}\). Following the method described in Section~\ref{s2}, we start by deriving accurate prescriptions for the orbital frequency \(\Omega_{\rm ansatz}\) and angular momentum flux \(\dot{L}_z\). Thereafter we derive the corrections that should be implemented in the gauge-invariant expression for the angular momentum (see Eq.~\eqref{angmom}) to accurately reproduce the inspiral evolution predicted by the EOBNRv2 model. 

\begin{figure*}[ht]
\centerline{
\includegraphics[height=2.6in, width=3.8in,  clip]{figures/insimri/anfreemc}
\includegraphics[height=2.6in, width=3.8in,  clip]{figures/insimri/amfemc}
}
\centerline{
\includegraphics[height=2.6in, width=3.8in,  clip]{figures/insimri/rteobvsemc}
\includegraphics[height=2.6in, width=3.8in,  clip]{figures/insimri/phiteobvsem}
}
\centerline{
\includegraphics[height=2.6in, width=3.8in,  clip]{figures/insimri/lzinveobvcem}
}
\caption{The panels show the accuracy with which our model can reproduce the dynamics of an equal mass (EM) binary system with masses  ( \(10M_{\odot} + 10M_{\odot}\)), as compared with the EOBNRv2 model. The top panels show that our model can reproduce the orbital angular evolution and the flux of angular momentum predicted by the EOBNRv2 model point to point with an accuracy better than one part in a thousand. The middle panels show that our model can reproduce with a similar accuracy the radial and azimuthal evolution of a   \(10M_{\odot} + 10M_{\odot}\) binary system. The bottom panel shows that the expression for the gauge-invariant angular momentum \(L_{z}(x)\) for an EM mass deviates clearly from the EMRI prescription even before nearing the ISCO. }
\label{emcase}
\end{figure*}

The prescriptions for the orbital frequency, angular momentum flux and angular momentum which reproduce the inspiral evolution for the \(10M_{\odot} + 10M_{\odot}\) binary system shown in Figure~\ref{emcase} require the following set of coefficients (compare Tables~\ref{coef}, \ref{Lzdotcoef}, \ref{zedfunccoef})


\begin{eqnarray}
\Omega(u)_{\rm ansatz} &:& a_1= -5.644, \quad a_2 = 11.003, \quad a_3 = -5.390,\quad a_4 = 8.794, \quad a_5 = 0.458\\\nonumber
\dot{L}_z &:& c^{1}_{2.5} = -90.566, \quad c^2_3  =304.941, \quad c^3_{3.5}= -339.500\\\nonumber
z_{\rm SF} &:&  b_1 = -5.803,\quad  b_2= 14.171, \quad b_3 = -7.074, \quad b_4 = 22.810, \quad b_5 =-25.580\\\nonumber
\label{ecmcoe}
\end{eqnarray}

Having shown that the approach outlined in the paper is still effective to model EM binaries, we consider worthwhile developing a model that encapsulates the physics of binaries of comparable and intermediate-mass-ratio. In order to extend our IMRI model into the comparable mass-ratio regime, we require to implement some improvements in the model.  First and foremost, we require expressions for the orbital frequency evolution (see Eq.~\eqref{ompres}) and angular momentum (see Eq.~\eqref{angmom}) which incorporate conservative corrections at second-order in mass ratio \(\eta\). We can draw this conclusion by comparing the bottom-right panel of Figure~\ref{lzfull} with the bottom panel of Figure~\ref{emcase}. The former plot shows that for IMRIs the prescription for the angular momentum  deviates from the self-force prediction near the ISCO. However, the latter plot shows that in the EM case the prescription for the angular momentum deviates considerably from the self-force prediction even during the inspiral 
evolution, far away from the ISCO. Because the prescription we have used for the angular momentum in both cases includes corrections at liner order in mass-ratio, this suggest that to derive a model that unifies both regimes we require  an expression for the angular momentum which includes second-order conservative corrections. We can follow a similar strategy to encapsulate in a single prescription the radiative piece of the self-force for comparable and intermediate mass-ratio systems by deriving second-order radiative corrections in the prescription for the angular momentum flux. 

Having derived expressions for \(\Omega_{\rm ansatz}\), \(\dot{L}_z\) and \(L_{z}(x)\) which include corrections at second-order in mass-ratio, we will be able to tune the coefficients of these objects and derive numerical fits for the coefficients in terms of the mass-ratio \(\eta\) of the system. This approach will provide a unified description for the inspiral evolution of IMRIs and comparable-mass systems. We will pursue these studies in the future.


\Chapter{Self-forced evolutions of an implicit rotating source: 
a natural framework to model comparable and intermediate mass-ratio systems 
from inspiral through ringdown}
\label{ch:imrSFIMRI}
%%%%%%%%%%%%%%%%%%%%%%%%%%%%%%%%%%%%%%%%%%%%%%%%%%%%%%%%%%%%%%%%%%%%%%%%%%%%%%%
%%% Describe the IMR model for IMRIs.
%%%%%%%%%%%%%%%%%%%%%%%%%%%%%%%%%%%%%%%%%%%%%%%%%%%%%%%%%%%%%%%%%%%%%%%%%%%%%%%

%
\def\etal{{\it et al.}}  \def\ie{{\it i.e.}}  \def\eg{{\it e.g.}}
\def\lap{\hbox{${_{\displaystyle<}\atop^{\displaystyle\sim}}$}}
\def\gap{\hbox{${_{\displaystyle>}\atop^{\displaystyle\sim}}$}}
\def\lesssim{\mathrel{\hbox{\rlap{\hbox{\lower4pt\hbox{$\sim$}}}\hbox{$<$}}}}
\def\gtrsim{\mathrel{\hbox{\rlap{\hbox{\lower4pt\hbox{$\sim$}}}\hbox{$>$}}}}
\def\alt{\mathrel{\hbox{\rlap{\hbox{\lower4pt\hbox{$\sim$}}}\hbox{$<$}}}}
\def\agt{\mathrel{\hbox{\rlap{\hbox{\lower4pt\hbox{$\sim$}}}\hbox{$>$}}}}
\def\prd{{\it Phys. Rev.} D~}
\def\PRL{{\it Phys.Rev.} Lett~}
\def\apjl{{\it Astrophys. J.} Lett~}
\def\apj{{\it Astrophys. J.}}
\def\Msun{M_\odot}
\def\PRD{{\it Phys. Rev.} D~}
\def\CQG{{\it Class. Quantum Grav.}}
\def\aaps{{\it A\&AS~ }}
\def\pasj{{\it PASJ }}
\def\mnras{{\it MNRAS}} 
\def\aapr{{\it A\&ARv}}

\def\gta{\ifmmode {\mathbin{\lower 3pt\hbox   %> or of order
    {$\,\rlap{\raise 5pt\hbox{$\char'076$}}\mathchar"7218\,$}}}
    \else {${\mathbin{\lower 3pt\hbox
    {$\rlap{\raise 5pt\hbox{$\char'076$}}\mathchar"7218\,$}}}
    $}\fi}
\def\lta{\ifmmode {\,\mathbin{\lower 3pt\hbox   %< or of order
    {$\,\rlap{\raise 5pt\hbox{$\char'074$}}\mathchar"7218\,$}}}
    \else {${\mathbin{\lower 3pt\hbox
    {$\rlap{\raise 5pt\hbox{$\char'074$}}\mathchar"7218\,$}}}
    $}\fi}

%%%%%%%%%%%%%%%%%%%%%%%%%%%%%%%%%%%%%%%%%%%%%%%%%%%%%%%%%%%%%%%%%%%%%%%%%%%%%%%

% \section{Introduction}   
The black hole (BH) mass function in the local Universe is a strongly bi-modal distribution that identifies two main families: stellar-mass BHs with typical masses \(\sim 10M_{\odot}\) observed in Galactic X-ray binaries~\cite{McClintock:2006} and, more recently, in globular clusters~\cite{Morscher:2013}, and supermassive BHs with masses \(\gtrsim 10^{5} M_{\odot}\) observed to be present in most galactic nuclei~\cite{Merloni:2008, Fukugita:2004}.  However, a population of X-ray sources with luminosities in excess of \(10^{39}\, \rm{ erg}\, \rm{s}^{-1} \) has recently been observed, and {\it{Chandra}} and {\it{XMM-Newton}} spectral observations of these ultra-luminous X-ray sources (ULXs) revealed cool disc signatures that were consistent with the presence of intermediate mass BHs (IMBHs) with masses \(10^{2-4} M_{\odot}\)~\cite{Miller:2004,Miller:2004b,Miller:2006_BOOK}. Subsequent observations have shown that these ULXs have spectral and temporal signatures that are not consistent with the sub-Eddington 
accretion regime that is expected for IMBHs at typical ULX luminosities. Rather, these later studies suggest that many ULXs are powered by super-Eddington accretion onto \(\lesssim 100 M_{\odot}\) BH remnants.  Nevertheless, recent work by Swartz et al.~\cite{Swartz:2011} demonstrates that there is a subpopulation of ULXs that seem to be powered by a separate physical mechanism. These objects have typical luminosities \(L\gtrsim 10^{41}\, \rm{ erg}\, \rm{s}^{-1} \), which cannot be explained by close to maximal radiation from super-Eddington accretion onto massive BHs formed in low metallicity regions~\cite{Zampieri:2009, Belczynski:2010, Ohsuga:2011}. Several hyper-luminous X-ray sources, including M82 X-1, ESO 243-49 HLX-1, Cartwheel N10 and CXO J122518.6+144545, present the best indirect evidence for the existence of IMBHs~\cite{Matsumoto:2001,Farrel:2009,Wolter:2010,Jonker:2010}.  In particular, the colocation of M82 X-1 with a 
massive, young stellar cluster, the features if its power spectrum, and some reported transitions between a hard state and a thermal dominant state, make this object a strong IMBH candidate~\cite{Portegies:2004,Strohmayer:2003,Kaaret:2007,Feng:2010}.  Recent searches of archival {\it{Chandra}} and {\it{XMM-Newton}} data sets have also uncovered two new hyper-luminous X-ray sources with luminosities in excess of \(10^{39}\, \rm{ erg}\, \rm{s}^{-1} \). These sources are the most promising IMBH candidates currently known, although the highest possible super-Eddington accretion rate onto the largest permitted BH remnant cannot yet be ruled out~\cite{Sutton:2012}.  This increasing body of observational evidence~\cite{Trenti:2006,Coleman:2004}, and the fact that the existence of IMBHs provides a compelling explanation for the initial seeding of supermassive BHs present in most galactic nuclei~\cite{Volonteri:2010,Schneider:2002,Yu:2002} has revived the quest for these elusive objects.  

A promising channel for detection of IMBHs is through the emission of gravitational radiation during the coalescence of stellar-mass compact remnants --- neutron stars (NSs) or BHs --- with IMBHs in core-collapsed globular clusters. This expectation is backed up by numerical simulations of globular clusters~\cite{Taniguchi:2000,Miller:2002,Mouri:2002a,Mouri:2002b,Gultekin:2004,Gultekin:2006,Oleary:2006,Oleary:2007}  which suggest that IMBHs could undergo several collisions with stellar-mass compact remnants during the lifetime of the cluster through a variety of mechanisms, including gravitational radiation, Kozai resonances and binary exchange processes.  As discussed in~\cite{man}, the most likely mechanism for the formation of binaries involving a stellar-mass compact remnant and an IMBH is hardening via three body interactions, with an expected detection rate of \(\sim 1-10\, {\rm{yr}}^{-1}\) with ground-based observatories~\cite{man,Abadie:2010}. 


% Since hyper-luminous X-ray sources are rare, and our knowledge about their astrophysical properties is still limited, we may have to use a different means to search for IMBHs in order to improve our knowledge about the channels that lead to the formation of these objects, and to shed light on their astrophysical properties, such as mass and spin distributions~\cite{mandel}. 
% In this chapter we explore the use of observations in the gravitational wave (GW) spectrum to gain insight into the properties of IMBHs. 
% 
The current upgrade of the LIGO and Virgo detectors~\cite{aLIGO, virgo}, will enable the detection of IMBHs with masses \(50 M_{\odot} \lesssim M \lesssim 500 M_{\odot}\), by achieving their target sensitivity at low-frequencies down to 10Hz~\cite{ZDHP:2010}.  Advanced LIGO (aLIGO) and Advanced Virgo are expected to have greatest sensitivity in the 15Hz - 1kHz range, with a peak at \(\sim60\) Hz (see Figure~\ref{ZDHP_promise}). 
% Proposed third generation detectors, such as the Einstein Telescope~\cite{Freise:2009}, aim to extend the frequency range of ground-based detectors down to 1Hz, in order to search for GWs emitted by binaries of \(10^{2-4}M_{\odot}\) BHs~\cite{etgair,Huerta:2011a,Huerta:2011b}. 
% 
% 
\begin{figure*}[ht]
\centerline{
\includegraphics[height=0.6\columnwidth,  clip]{figures/imrimri/nc_normalized.eps}
}
\caption{The panel shows the expected sensitivity for two configurations of the Einstein Telescope (ET), namely, ETD (black), ETB (blue) and LIGO's Zero Detuned High Power (ZDHP) configuration (red). The vertical axis measures \(F_{\rm{normalized}} =  \left(f/f_{\rm{max}}\right)^{-7/6}\sqrt{S_n(f_{\rm{max}})/S_n(f)}\), where \(f_{\rm{max}}\) is the maximum of the corresponding power spectral density, \(S_n(f)\).
}
\label{ZDHP_promise}
\end{figure*}
% 
% 
The frequency of the dominant quadrupolar harmonic in the GWs emitted at the
innermost stable circular orbit (ISCO) for a binary of non-spinning objects is 
\begin{equation}
f_{\rm{ISCO}}= 4.4 {\rm{kHz}} \left(\frac{M_{\odot}}{M}\right).
\label{fIMRIisco}
\end{equation}
For a typical intermediate mass--ratio coalescence (IMRC) with total mass 
\(M\gtrsim 100 M_{\odot}\), advanced detectors will observe the late inspiral,
merger and ringdown; with the latter two phases contributing significantly to the 
overall SNR~\cite{Smith:2013}.
% However, the heaviest IMRCs, with masses $\gtrsim500M_\odot$, will coalesce at or below the low frequency limit of the bandwidth. 
% Hence, merger and ringdown will significantly contribute to the SNR of IMRCs over a considerable portion of the detectable mass-range~\cite{Smith:2013}. 
In order to maximize the science output of GW observations of IMRCs, it is 
therefore necessary to model the inspiral, merger and ringdown consistently. 
% In order to make progress in this direction, we previously developed a waveform model that combined results from Black Hole 
% Perturbation Theory (BHPT) and post-Newtonian (PN) theory to  explore the information that could be obtained from observations of IMRCs with the EinsteinTelescope~\cite{Huerta:2011a,Huerta:2011b}. Although this model provided an important step in exploring the science that could be done with IMRC observations, the model was limited.
% 
% 
% In previous work~\cite{Huerta:2012zy}% and in~\cite{Huerta:2009,Huerta:2010,Huerta:EHE,Huerta:2012}
% we explored using the self-force formalism~\cite{SFB,LRP} to develop a waveform model with a robust description of the dynamical evolution of IMRCs during the inspiral phase. These were found to be effective when used to carry out matched-filter based searches for inspiral-only IMRCs~\cite{Smith:2013}, but searches for IMRCs in the advanced detector era will require waveform models that include not only the inspiral but also the merger and ringdown~\cite{Smith:2013}. The model described in~\cite{Huerta:2011a} included merger and ring down but without the self-force driven inspiral.
While modeling of IMRCs would benefit greatly from NR simulations, simulating
mergers for mass-ratios%
\footnote{Note that the definition of mass-ratio $q$ is different from the
previous chapters, where $q\geq 1$. We choose the definition with $q\leq 1$ here
to ensure that in the extreme mass-ratio regime $q\rightarrow 0$
as $\eta\rightarrow 0$. See Section~\ref{ssec:nomenclature}.}
% 
 \(q=m_1/m_2 \lesssim 1/10\) are prohibitively computationally expensive at
present~\cite{Mroue:2013xna}. On the other hand, recent 
% breakthroughs in the self-force program that have 
results show that the conservative part of self-force can reproduce
results from NR simulations of comparable-mass binary systems~\cite{LeTiec:2012}.  
Furthermore, the recent computation of the self-force inside the ISCO 
equips us to develop models that better reproduce the strong field dynamics of 
BH binaries~\cite{Akcay:2012}.  

\begin{figure*}%[ht]
\centerline{
\includegraphics[height=0.45\columnwidth,  clip]{figures/imrimri/m1m006_phasedifftM_T4EOB_r030tM.eps}
\includegraphics[height=0.45\columnwidth,  clip]{figures/imrimri/m1m008_phasedifftM_T4EOB_r030_tM.eps}
}
\centerline{
\includegraphics[height=0.45\columnwidth,  clip]{figures/imrimri/m1m010_phasedifftM_T4EOB_r030tM.eps}
\includegraphics[height=0.45\columnwidth,  clip]{figures/imrimri/m1m015_phasedifftM_T4EOB_r030_tM.eps}
}
\caption{The phase discrepancy in radians between the PN approximant TaylorT4, and the Effective One Body model, shown as a function of time from \(r=30M\) to the point when the TaylorT4 model reaches the ISCO. The systems have mass-ratio, $q$, total mass, \(M\), and final phase discrepancy, $\Delta\Phi$: $(q, M, \Delta\Phi) = (1/6, 7M_{\odot},21.5\,{\rm rads})$ (top-left), $(1/8, 9M_{\odot},30.2\,{\rm rads})$ (top-right), $(1/10, 11M_{\odot},70.1\,{\rm rads})$ (bottom-left) and $(1/15, 16M_{\odot},83.2\,{\rm rads})$ (bottom-right) respectively.}
\label{pn_approx}
\end{figure*}


In this chapter we combine recent developments in the self-force program, in 
PN theory and in NR to develop a model that describes the inspiral, merger and
ringdown of IMRCs and comparable mass-ratio systems. In Section~\ref{ssec:inspiral} 
we
discuss the modelling of the conservative part of the self-force during inspiral.
Using \(\mathcal{O}(\eta)\)\footnote{\(\eta=m_1 m_2/(m_1+m_2)^2\)} 
self force results for binaries with mass-ratios \(q\gtrsim1/6\) gives a 
system without an ISCO. We discuss the implications of this result for 
the modeling of comparable and intermediate mass--ratio binaries.  
% 
In Section~\ref{ssec:dissipative}, we describe our approach 
to model the radiative part of the self-force for the inspiral evolution. 
% 
In Section~\ref{trans}, we extend the transition scheme of 
Ori and Thorne~\cite{ori} by including finite mass-ratio corrections, and model
the orbital phase evolution using the implicit rotating source (IRS) model. 
We adopt the IRS description for the late-time radiation in order to provide a 
smooth progression from late inspiral to ringdown, as it provides the correct 
orbital frequency evolution in the vicinity of the light-ring. 
% 
Finally, in Section~\ref{RDwav} we construct the ringdown waveform using a 
sum of quasinormal modes. Section~\ref{conclu} presents a summary of our 
findings and future directions of work. 

In addition to IMBH--BH binaries, NS--BH binaries also have mass-ratios 
\(q \lesssim 1/6\). NSBH mergers are promising GW sources for second generation detectors
with an estimated detection rate of \(0.2-300\) mergers a year~\cite{LSCCBCRates2010}. 
Past GW searches for NSBH systems used PN waveforms as templates, which have
been demonstrated to be insufficiently accurate for aLIGO searches~\cite{Nitz:2013mxa}. 
In Figure~\ref{pn_approx}, we show the phase difference between the PN approximant TaylorT4~\cite{TaylorT4Origin} and the EOB model introduced
in~\cite{BuonannoEOBv2Main}. The model we develop here would 
also be applicable to NSBH detection searches.



\section{Modeling}
\label{one}


\subsection{Nomenclature}\label{ssec:nomenclature}
Throughout this chapter we will use units with \(G=c=1\), unless otherwise stated. We consider BH  binaries on circular orbits with component masses \(m_1, m_2\), such that \(m_1 < m_2\). We assume that the binary components are non spinning. We use several combinations of the masses in the following sections, which are summarized in Table~\ref{length}. 

\begin{table}[thb]
\centering
\begin{tabular}{|c| c| }
\hline
\multicolumn{2}{|c|}{Binary masses}  \\\cline{1-2} 
\(m_1\) & mass of inspiralling compact object  \\ [0.7ex] 
\(m_2 \) & mass of central compact object  \\ [0.7ex]  
\(M=m_1+m_2 \) & total mass of binary system \\ [0.7ex]  
\(q=\frac{m_1}{m_2} \) & (with $m_1\leq m_2$) mass--ratio \\ [0.9ex]  
\(\mu=\frac{m_1 m_2}{m_1 + m_2} \) & reduced mass \\ [0.9ex]  
\(\eta= \frac{\mu}{M}\) & symmetric mass--ratio \\ [0.9ex]  
\hline
\end{tabular}
\caption{The table summarizes the nomenclature we will use throughout our analysis.}
\label{length}
\end{table}

Note that the definition of the mass-ratio $q$ is different from the previous chapters, where
$q\geq 1$. We choose the definition with $q\leq 1$ here to ensure that in the 
extreme mass-ratio regime both $q\rightarrow 0$ and $\eta\rightarrow 0$.
% Having defined the variables to be used in the subsequent sections, we shall now describe the construction of the self-forced waveform model. The model consists of four building blocks --- the inspiral, the transition, the plunge and the ringdown phases. The next section describes the inspiral evolution. 

\subsection{Inspiral evolution}
\label{ssec:inspiral}
We model the inspiral phase evolution in the context of the Effective One Body  (EOB) formalism~\cite{EOB:Damour}, i.e., we consider the scenario in which the dynamics of a binary system is mapped onto the motion of a test particle in a time-independent and spherically symmetric Schwarzschild space-time with total mass \(M\):
\begin{equation} 
\mathrm{d}s^{2}_{\rm{EOB}} = -A(r)\mathrm{d}t^2 + B(r)\mathrm{d}t^2 + r^2\mathrm{d}\Omega^2\,,
\label{metricEOB}
\end{equation} 
\noindent where the potentials \(A, \, B\) are known to 3PN order~\cite{Buonanno:1999,Damour:2000}. In the test-mass particle limit \(\eta\rightarrow 0\), these potentials recover the Schwarzschild results, namely:
\begin{equation}
A(u, \eta\rightarrow 0) = B^{-1}(u,  \eta\rightarrow 0)= 1-2\,u,\quad {\rm{with}} \quad u=\frac{M}{r}.
\label{limitEOB}
\end{equation}
\noindent In the EOB formalism, the orbital frequency evolution can be
derived from the Hamiltonian, \(H_{\rm{EOB}}\)~\cite{EOB:Damour},
\begin{equation} 
H_{\rm{EOB}} = M\sqrt{1+2\,\eta\left(H_{\rm{eff}} -1\right)},
\label{EOBH}
\end{equation}
\noindent using the Hamiltonian equation:
\begin{equation}
\frac{d \phi}{d \mathrm{t} } =  M\Omega = \frac{\partial H_{\rm{EOB}}}{\partial L} = \frac{u^2\,L(x)\,A(u)}{H(u)\,H_{\rm{eff}}(u)},
\label{new_phase}
\end{equation}
\noindent where 
\begin{eqnarray}
H_{\rm{eff}}(u) = \frac{A(u)}{\sqrt{\tilde{A}(u)}}\,, \qquad  \tilde{A}(u)= A(u) + \frac{1}{2}\,u\,A'(u),     \qquad H(u)=\dfrac{H_{\rm{EOB}}}{M},\\\nonumber
\label{params_for_new_phase}
\end{eqnarray}
$(')$ denotes $\pd_u$, and $L$ is the binary's orbital angular momentum.
Recent work has enabled the derivation of gravitational self-force corrections 
to the EOB potential \(A(u)\rightarrow  1-2u+\eta\, a(u) + {\cal{O}}(\eta^2)\)~\cite{barus}.
Deriving this gravitational self-force contribution,  \(a(u)\), is equivalent to 
including all PN corrections to the EOB potential \(A(u)\) at linear order in 
\(\eta\). We shall now briefly describe the construction of the 
gravitational self-force contribution \(a(u)\), emphasizing the fact that this
contribution encodes information about the strong-field regime of the gravitational field. 

As shown by Detweiler and Whiting~\cite{Detweiler:2003}, the gravitational self-force corrected worldline can be interpreted as a  geodesic in a smooth perturbed spacetime with metric
 \begin{equation}
 g_{\alpha \beta} =  g^{0}_{\alpha \beta}(m_2) + h^{R}_{\alpha \beta},
 \label{detint}
 \end{equation}
where the regularized \(R\) field is a smooth perturbation 
associated with \(m_1\). Detweiler proposed the gauge invariant 
``redshift observable'' \(z_1\) to handle the conservative effect of the gravitational
self-force in circular motion~\cite{Detweiler:2008,Detweiler:2009}.
\(z_1\) can be interpreted as the the gravitational red-shift of 
light rays emitted from the smaller compact object, and received far
away from the binary along the direction perpendicular to the
orbital plane~\cite{Detweiler:2008}.
 \begin{equation}
z_1(\Omega)= \sqrt{1-3x}\left(1- \frac{1}{2} h^{R,\,F}_{uu} + q \frac{x}{1-3x}\right),
\label{detinv}
\end{equation}
\noindent where  \(x\) is the gauge-invariant dimensionless frequency parameter 
given by \(x=\left(M \Omega\right)^{2/3}\), \(h^{R,\,G}_{uu}\) is a double
contraction of the regularized metric perturbation with the four-velocity
\(u^{\mu}\), i.e. \( h^{R,\, G}_{uu} = h^{R,\,G}_{\mu\nu}u^{\mu} u^{\nu} \), 
where the label \(G\) indicates the gauge used to evaluate the metric perturbation.
The label \(F\) in Eq.~\ref{detinv} indicates that it is valid within the class 
of asymptotically {\it F}lat gauges.
In a convenient gauge, the redshift coincides with the 
inverse time components of the four-velocities \(u^\alpha\) of the
component, namely \(z_1 = 1/u^t_1\)~\cite{Detweiler:2008}.
In~\cite{Akcay:2012}, \(z_1(\Omega)\) was calculated in the Lorenz gauge and the 
following gauge transformation can be used to link the asymptotically flat 
\(h^{R,\,F}_{uu} \) metric perturbation to its Lorenz-gauge counterpart 
\(h^{R,\,L}_{uu} \):
\begin{equation}
h^{R,\,F}_{uu} = h^{R,\,L}_{uu} + 2q\frac{x(1-2x)}{\left(1-3x\right)^{3/2}}.
\label{flgauge}
\end{equation}
Hence, inserting Eq.~\eqref{flgauge} into Eq.~\eqref{detinv} leads to
 \begin{equation}
z_1(\Omega)= \sqrt{1-3x}\left(1- \frac{1}{2} h^{R,\,L}_{uu}  - 2q\frac{x(1-2x)}{\left(1-3x\right)^{3/2}} + q \frac{x}{1-3x}\right).
\label{detinvLOR}
\end{equation}
The EOB potential \(a(x)\) can be constructed from $h_{uu}^{R,L}$ via
\begin{equation}
 a(x) = -\frac{1}{2}\left(1-3x\right)\tilde{h}^{R,\,L}_{uu} - 2x \sqrt{1-3x},
 \label{formal_a}
 \end{equation}
 \noindent with \(\tilde{h}^{R,\,L}_{uu}= q^{-1}h^{R,\,L}_{uu}\). 
 In~\cite{Akcay:2012} accurate numerical data is obtained for \(h^{R,\,L}_{uu}\)
in the Lorenz gauge. Using the above relation, Ref.~\cite{Akcay:2012} 
provides a useful fit formula for \(a(x)\) that is valid over the range 
\(0 < x < \frac{1}{3}\), 
 \begin{equation}
 a(x)= 2x^3\, \frac{(1-2x)}{\sqrt{1 - 3 x}}\,a_{E}(x),
 \label{pot}
 \end{equation}
where \(a_E(x)\) is given in Eq.~(54) of~\cite{Akcay:2012}.
% Using the above dictionary, the model for \(a(x)\) in this chapter 
% reproduces \(a_{E}(x)\) 
% to within a maximal absolute difference of \(1.2\times10^{-5}\) for 
% \(0<x<\frac{1}{3}\). 
Using the phenomenological fit for the function $a(x)$, the self-force
corrected energy and angular momentum are given by~\cite{Akcay:2012, barus}
{\allowdisplaybreaks\begin{align}\label{enofxeq}
E(u(x)) &= E_0(x) +  \eta\left(-\frac{1}{3}\frac{x}{\sqrt{1-3x}}a'(x) + \frac{1}{2}\frac{1-4x}{\left(1-3x\right)^{3/2}} a(x) - \right.\nonumber\\
& \hspace{20mm} \left. E_0(x)\left( \frac{1}{2}E_0(x) + \frac{x}{3}\frac{1-6x}{\left(1-3x\right)^{3/2}} \right)\right),\\
\label{lzofxeq}
L (u(x)) &= L_0(x) +  \eta\left( -\frac{1}{3}\frac{x}{\sqrt{x(1-3x)}}a'(x) -\frac{1}{2}\frac{1}{\sqrt{x}\left(1-3x\right)^{3/2}} a(x)  \right.\nonumber\\
& \hspace{20mm} \left. -\frac{1}{3 }\frac{1-6x}{ \sqrt{x} \left(1-3x\right)^{3/2}}\left(E_0(x)-1\right)  \right)\,,\\
\label{rofx}
{\rm{with}} \quad u(x)&= x\left( 1+ \eta\Bigg[\frac{1}{6}a'(x) + \frac{2}{3}\left(\frac{1-2x}{\sqrt{1-3x}} -1 \right) \Bigg] \right)\,,
\end{align}}
where \((')\) denotes \(\pd_x\), and \(E_0(x)\) and \(L_0(x)\) are given by
\begin{eqnarray}
E_0(x) &=&  \frac{1-2x}{\sqrt{1 - 3 x}} -1,\\
\label{enofx_0}
L_0(x)&=&   \frac{1}{\sqrt{x (1 - 3 x)}}.
\label{lzofx_0}
\end{eqnarray}
   
\begin{figure*}%[ht]
% \centerline{
\includegraphics[height=0.6\columnwidth,  clip]{figures/imrimri/eofx.eps}
\includegraphics[height=0.6\columnwidth,  clip]{figures/imrimri/lofx.eps}
% }
\caption{The panels show the energy and angular momentum given by Eqs.~\eqref{enofxeq}-\eqref{lzofxeq}, respectively. We show the functional form of these parameters for binary systems with mass-ratio values, from top to bottom, \(q \in [0,\, 1/100, \,1/20, \,1/10, \,1/6, \,1/5, \,1 ]\).  }
\label{orbitalparams}
\end{figure*}

\noindent In Figure~\ref{orbitalparams}, we  show the effect of these conservative corrections on the orbital parameters.


As discussed in~\cite{barus}, minimizing the self-force corrected energy,
given by Eq.~\eqref{enofxeq}, with respect to the orbital frequency, predicts
that binary systems with mass-ratios \(q\in \{1, \,1/2, \,1/3\}\) do not have
an ISCO. 
It was argued in~\cite{barus} that deriving self-force results in the strong
field regime may alleviate this problem. We explored this issue, and found
that using linear--in--\(\eta\) self-force corrections does not fix this
problem for comparable mass-ratio systems. 
In Figure~\ref{dedx}, we show that the existence of an ISCO is guaranteed 
for BH binaries with symmetric mass-ratio 
\(\eta\lesssim 6/49\, ({\rm{or}}\, q \lesssim 1/6)\), and its location may
be approximated by
\begin{equation}
x_{\mathrm{ISCO}}=\frac{1}{6}\left(1+ 0.83401\eta+4.59483\eta^2\right).
\label{xisco_eq}
\end{equation}
It remains to be seen whether the inclusion of \(\mathcal{O}(\eta^2)\)
conservative corrections gives an ISCO for binaries with 
mass-ratios \(q\gtrsim 1/6\). 

 
\begin{figure*}%[ht]
\centerline{
\includegraphics[height=0.6\textwidth,  clip]{figures/imrimri/dedx_for_isco.eps}
}
\caption{The location of the innermost stable circular orbit is determined by the condition \({\rm{d}}E/\rm{d}x = 0\). The panel shows \({\rm{d}}E/\rm{d}x\) as a function of the gauge invariant quantity \(x=\left(M\,\Omega\right)^{2/3}\). The various curves represent binary systems with mass-ratios, from top to bottom, \(q \in [0,\, 1/100, \,1/20, \,1/10, \,1/6, \,1/5, \,1 ]\). Note that binaries with mass-ratios \(q\gtrsim 1/6\) do not have an ISCO in this model. }
\label{dedx}
\end{figure*}

In summary, the building blocks to construct the conservative dynamics are

\begin{itemize}
\item The orbital frequency evolution is computed using Eq.~\eqref{new_phase} with the gravitational self-force contribution included in the potential \(A(u)= 1-2u + \eta\, a(u)\).
\item Eq.~\eqref{new_phase} is evaluated using the self-force-corrected expression for the angular momentum, \(L(x)\), given by Eq.~\eqref{lzofxeq}. The self-force-corrected expression for the energy, given in Eq.~\eqref{enofxeq} is only used to determine the point at which the inspiral ends and the transition region begins.
\end{itemize}

Eq.~\eqref{new_phase} accurately models the orbital frequency from early 
inspiral through the ISCO. However, the post-ISCO time evolution of this 
prescription does not render an accurate representation of the orbital 
frequency as compared to numerical relativity simulations. 
This problem was addressed in the EOB formalism by introducing the 
phenomenological non-quasi-circular coefficients~\cite{BuonannoEOBv2Main}.
The approach we follow to circumvent this problem is described
in Section~\ref{trans}.

This completes the description of the conservative part. We now describe how to couple this with the radiative part of the self-force to model the inspiral evolution. 
 
 \subsection{Dissipative dynamics}
\label{ssec:dissipative}
A consistent self-force evolution model that incorporates first-order
in mass-ratio conservative  corrections should also include second-order
radiative corrections. However, second-order self-force radiative 
corrections are not known at present. 
% Several studies have demonstrated 
% the importance of including the missing second order corrections to the
% radiative part of the self-force, both for source detection and for parameter estimation~\cite{Isoyama:2013, Burko:2012, Huerta:2012, Huerta:2010, Huerta:2009}. 
% 
We use a new prescription for the energy flux that includes PN 
corrections up to \(22^\mathrm{nd}\) PN order~\cite{Fujita:2012},
\begin{eqnarray}
\label{enfluxpn}
\left(\dot E\right)_{\rm PN}
&=& -\frac{32}{5}\frac{\mu^2}{M}x^{7/2}\Bigg[1-\frac{1247}{336}x + 4\pi x^{3/2}  - \frac{44711}{9072}x^2-\frac{8191}{672}\pi x^{5/2}  \\\nonumber 
&+&x^3 \bigg\{\frac{6\,643\,739\,519}{69\,854\,400} +\frac{16}{3}\pi^2 -\frac{1712}{105}\gamma_{\rm E} -\frac{856}{105}\ln(16x) \Big\} -\frac{16285}{504} \pi x^{7/2} \\\nonumber
&+&  x^4 \Big\{-\frac{323105549467}{3178375200}  + \frac{232597}{4410}\gamma_{\rm E} -\frac{1369}{126}\pi + \frac{39931}{294}\ln(2)  \\\nonumber
&-&\hspace{20mm}\frac{47385}{1568}\ln(3)  +\frac{232597}{4410}\ln(x) \Big\}  \\\nonumber 
&+&  x^{9/2}\Big\{ \frac{265978667519}{745113600}\pi -\frac{6848}{105}\gamma_{\rm E}\pi -\frac{13696}{105}\pi\ln(2)  -\frac{6848}{105}\pi\ln(x)\Big\} \\\nonumber
&+& {\rm{ higher \, order \, corrections \, up\, to \, 22PN \, order}} \Bigg],
\end{eqnarray}
and include the \(\mathcal{O}(\eta)\) corrections through the exponential
resummation approach of~\cite{Isoyama:2013}. In this approach, the energy flux is
\begin{eqnarray}
\left(\frac{{\mathrm{d}}E}{{\mathrm{d}}t}\right)_{\rm hybrid} &=& {\cal{L}}_{\rm 0} \exp\left( {\cal{L}}_{\eta} \right)\, ,
\label{pnfluxfit}
\end{eqnarray}
\noindent where \( {\cal{L}}_{\rm 0} \) denotes the leading-order in mass-ratio PN energy flux given in Eq.~\eqref{enfluxpn}, and \( {\cal{L}}_{\eta} \) incorporates mass-ratio corrections to the highest PN order available~\cite{Joguet:2002,Buonanno:2011_tail,Isoyama:2013}, and additional corrections characterised by a set of unknown coefficients,  \(b_i\) 
\begin{eqnarray}
\label{etacorrect}
{\cal{L}}_{\eta} 
&=& \Bigg[x\bigg[-\frac{35}{12}\eta + b_1\,\eta^2\bigg] + 4\pi x^{3/2}\bigg[ b_2\, \eta + b_3 \eta^2 \bigg]  + x^2 \bigg[\frac{9271}{504}\eta + \frac{65}{18}\eta^2\bigg]  \\\nonumber
&+& \pi x^{5/2}\bigg[ -\frac{583}{24} \eta + b_4\, \eta^2\bigg] + x^3 \bigg[\eta\left(-\frac{134\,543}{7\,776} + \frac{41}{48}\pi^2\right) -\frac{94403}{3024}\eta^2 - \frac{775}{324}\eta^3\bigg] \\\nonumber
&+&  \pi x^{7/2} \bigg[\frac{214745}{1728} \eta +  \frac{193385}{3024}\eta^2\bigg]   \Bigg].
\end{eqnarray}
\noindent  The coefficients \(b_i\) were taken to be constant 
in~\cite{Isoyama:2013}, but we found that a better match to the EOB phase
evolution could be obtained by allowing an additional dependence on mass-ratio
in these terms (see Eqs.~\eqref{B1}-\eqref{B3} below).  We constrain the 
\(b_i\) coefficients by ensuring that the resulting phase evolution  
reproduces the phase evolution predicted by the EOB model introduced 
in~\cite{BuonannoEOBv2Main, Damour:2013}, which was calibrated to NR 
simulations of comparable mass binaries. 
To do so, we implemented the EOB model~\cite{BuonannoEOBv2Main} and performed
a Monte Carlo simulation to optimize the values of the \(b_i\) coefficients 
(see Figure~\ref{bimaps}). The optimization was done in two stages. We 
started by considering the three coefficients \(b_1,\, b_2\) and \(b_4\),  
sampling a wide range of parameter space, namely \(b_i\in[-200,200]\).
We constrained the duration of the waveform from early inspiral to the 
light-ring to be similar to its EOB counterpart. Waveforms that differed from
their EOB counterparts by more than \(10^{-4}\) seconds in duration were
discarded. Once the region under consideration had been sparsely sampled, we
focused on regions of parameter space where the orbital phase evolution was
closest to the EOB evolution, and finely sampled these to obtain the optimal 
values for the coefficients. We found that this approach enabled us to 
reproduce the EOB phase evolution with a phase discrepancy of the order
\(\sim 1\) rad. After constraining \(b_1,\, b_2\) and \(b_4\), we explored
whether including additional corrections could further improve the phase 
evolution, by adding \(\eta\) corrections beyond 3PN order. Such corrections
were found to have a negligible impact on the actual phase evolution. This is
not difficult to understand, since such corrections are of order 
\(({\cal{O}}(\eta^4),\, {\cal{O}}(\eta^3))\), at (3PN, 3.5PN) respectively.
We found a similar behavior when we added leading order mass-ratios corrections
beyond 4PN order. Thus, we took a different approach: having derived the 
optimal value for \(b_1,\, b_2\) and \(b_4\), we took these results as initial
seeds for an additional MC simulation in which \(b_3\) was also included in 
Eq.~\eqref{etacorrect}, and repeated the optimization procedure.
The results of these simulations are shown in Figure~\ref{bimaps}.

We carried out several different Monte Carlo runs to find the `optimal' 
optimization interval, meaning the range of radial separations over which we 
aim to best match the phase evolution relative to the EOB model. We found that 
starting the optimization at \(r=30M\) gave results that performed moderately
well at early inspiral, but that underperformed at late inspiral, leading to 
phase discrepancies of order \(\sim 3\) rads. Starting the optimization at 
\(r=20M\) instead decreased the phase discrepancy with respect to the former 
case by a factor of 10 during early inspiral, and enabled us to reproduce the 
phase evolution in the EOB model (for all the mass-ratios considered) to within 
the accuracy of the numerical waveforms used to calibrate the EOB model 
itself~\cite{BuonannoEOBv2Main, Damour:2013}. Implementing these numerically 
optimized higher-order \(\eta\) corrections in Eq.~\eqref{etacorrect} gives
% 
\begin{eqnarray}
\label{etacorrect_new}
{\cal{L}}_{\eta} &=& \Bigg[x\bigg[-\frac{35}{12}\eta + B_1\bigg] + 4\pi x^{3/2}B_2  + x^2 \bigg[\frac{9271}{504}\eta + \frac{65}{18}\eta^2\bigg]  + \pi x^{5/2}\bigg[ -\frac{583}{24} \eta + B_3\bigg] \\\nonumber &+& x^3 \bigg[\eta\left(-\frac{134\,543}{7\,776} + \frac{41}{48}\pi^2\right) -\frac{94403}{3024}\eta^2 - \frac{775}{324}\eta^3\bigg] +  \pi x^{7/2} \bigg[\frac{214745}{1728} \eta +  \frac{193385}{3024}\eta^2\bigg]   \Bigg],
\end{eqnarray}
where
% 
\begin{eqnarray}
\label{B1}
B_1&=& \frac{1583.650 - 11760.507\, \eta}{1 + 142.389\, \eta - 981.723\, \eta^2}\,\eta^2\,,\\
\label{B2}
B_2 &=& \frac{-12.081 + 35.482\, \eta}{1 - 4.678 \eta + 13.280\, \eta^2}\,\eta +  \frac{19.045 - 240.031\, \eta}{1 - 18.461\, \eta + 74.142\, \eta^2}\,\eta^2\,,\\
\label{B3}
 B_3 &=& \frac{51.814 - 980.100\, \eta}{1 - 13.912\, \eta + 88.797\, \eta^2}\,\eta^2\,.
 \label{new_coef}
 \end{eqnarray}
% 
% This improved prescription for the energy flux, which incorporates second-order mass-ratio corrections to the PN expansion up to 3.5PN order, is sufficient to generate a model whose phase evolution reproduces with excellent accuracy the phase evolution predicted by EOB throughout inspiral and merger (see Figure~\ref{PNoptimized}).

Given the energy flux defined by Eqs~\eqref{enfluxpn}--\eqref{etacorrect}, we generate the inspiral trajectory using the chain rule
\begin{equation}
\frac{{\mathrm{d}}x}{{\mathrm{d}}t}= \frac{{\mathrm{d}} E}{{\mathrm{d}} t}\frac{{\mathrm{d}} x}{{\mathrm{d}}E}\,,
\label{radev}
\end{equation}

\noindent where we have used the mass-ratio corrected energy ---Eq.~\eqref{enofxeq}--- to compute \({\mathrm{d}}E/{\mathrm{d}}x\). Figure~\ref{PNoptimized} shows that for binaries with mass-ratio \(q=1/6\), the phase discrepancy between our self-force model and EOB is \(\lesssim 0.5\) rads at the light-ring, which is within the numerical accuracy of the simulations used to calibrate EOB. It has been shown recently that EOB remains accurate for mass-ratios up to \(q=1/8\)~\cite{Pan:2013}. In that regime the phase discrepancy between this model and EOB is $< 1$~rad, at the light-ring, as shown in Figure~\ref{PNoptimized}. For binaries with \(q=1/10\), the phase discrepancy at the light-ring is \(\lesssim 1.2\) rads, which is still within the numerical accuracy of available simulations~\cite{carlosI, carlosII}. 

\begin{figure*}%[ht]
\centerline{
\includegraphics[height=0.5\textwidth,  clip]{figures/imrimri/b1b2_mapb1b2m1m6no.eps}
\includegraphics[height=0.5\textwidth,  clip]{figures/imrimri/b3b4_mapb3b4m1m6no.eps}
}
\centerline{
\includegraphics[height=0.5\textwidth,  clip]{figures/imrimri/b1b2_mapb1b2m1m8no.eps}
\includegraphics[height=0.5\textwidth,  clip]{figures/imrimri/b3b4_mapb3b4m1m8no.eps}
}
\caption{The (top, \, bottom) panels show the results of the optimization runs that were used to constrain the values of the \(b_i\) coefficients given in Eq.~\eqref{etacorrect}. The panels show the results for binaries of mass--ratio \(q\in [1/6,\, 1/8]\), and total mass \(M= [7M_{\odot},\,   9M_{\odot}]\). The `optimal' value for the coefficients has been chosen by ensuring that the flux prescription minimizes the phase discrepancy between the EOB model and our self-force model. The color bar shows the phase difference squared between both models, which is integrated from \(r=20M\) all the way to the light-ring.} 
\label{bimaps}
\end{figure*}

It must be emphasized that even if we only use the inspiral evolution to model binaries with mass-ratios that typically describe NSBH binaries, our self-force evolution model performs better than TaylorT4, since we can reduce the phase discrepancy between TaylorT4 and EOB at the last stable circular orbit by a factor of \((\sim40, \, \sim70)\)  for binaries with \(q=(1/6,\,1/10)\)  and total mass \(M\in (7M_{\odot} ,\, 11M_{\odot} )\) (see Figure~\ref{pn_approx}).


 \begin{figure*}%[ht]
\centerline{
\includegraphics[height=0.4\textwidth,  clip]{figures/imrimri/phsiffeobsfm1m6.eps}
\includegraphics[height=0.4\textwidth,  clip]{figures/imrimri/phsiffeobsfm1m8.eps}
}
\centerline{
\includegraphics[height=0.4\textwidth,  clip]{figures/imrimri/phsiffeobsfm1m10.eps}
\includegraphics[height=0.4\textwidth,  clip]{figures/imrimri/phsiffeobsfm1m15.eps}
}
\caption{The panels show the orbital phase evolution of a self-force model making use of optimized PN energy flux given by Eq.~\eqref{pnfluxfit} and the phase evolution as predicted by the EOB model. The [top/bottom] panels exhibit this evolution for a compact binary with mass--ratio \(q=[(1/6,\,1/8), \,( 1/10,\,1/15)]\), and total mass \(M=[ (7M_{\odot} ,\, 9M_{\odot} ), \, ( 11M_{\odot},\, 16M_{\odot})  ]\), respectively. }
\label{PNoptimized}
\end{figure*}

Figure~\ref{PNoptimized} conveys an important message --- second order 
corrections to the radiative part of the self-force may provide a robust 
framework to describe in a single model not only events that are naturally 
described by BHPT, such as the mergers of stellar mass compact objects with
supermassive BHs in galactic nuclei~\cite{Huerta:2012, Huerta:2010, wargar,
cutler, gairles, SFB, GairL:2013}, but also events that are better described
by PN or numerical methods, in particular the coalescences of comparable
mass binaries~\cite{Huerta:2012, higherspin, Huerta:2011a, Huerta:2011b, smallbody}.

To finish this Section, we describe the construction of the gravitational
waveform from the inspiral trajectory. At leading post-Newtonian order, a 
general inspiral waveform can be written as
\begin{equation}
h(t) = -(h_{+} - i h_{\times}) = \sum_{\ell=2}^{\infty} \sum_{m=-\ell}^{l} h^{\ell m} {}_{-\!2}Y_{\ell m}(\iota,\Phi).
\label{inspwav}
\end{equation}

\noindent If only the leading-order modes \((\ell,m)=(2, \pm 2)\) and included, the inspiral waveform components  are given by 
\begin{eqnarray}
h_{+}(t)&=& \frac{4\, \mu\,  r^2\, \dot{\phi}^2 }{D}\left(\frac{1+\cos^2 \iota}{2}\right)\cos\left[2(\phi(t) + \Phi)\right],\label{inspp}\\
h_{\times}(t)&=& \frac{4\, \mu\, r^2\, \dot{\phi}^2}{D} \cos\iota \sin\left[2(\phi(t) + \Phi)\right],
\label{inspc}
\end{eqnarray}
\noindent where \(D\) is the luminosity distance to the source. Since the
orbital evolution will deviate from a circular trajectory during late 
inspiral (\(\dot{r}\neq0\)), we must consider more general orbits in which
both \(\dot{r}\) and \(\dot{r}\dot{\phi}\) are non-negligible. For such orbits,
at Newtonian order, the GW polarizations are given by~\cite{Gopakumar:2002}:
\begin{eqnarray}
\label{insppcor}
h_{+}(t)&=& \frac{2 \mu }{D}\Bigg\{ \left(1+\cos^2 \iota\right) \Bigg[ \cos\left[2(\phi(t) + \Phi)\right]\left(-\dot{r}^2 + r^2 \dot{\phi}^2 + \frac{1}{r}\right) \nonumber \\ &+& 2r\,\dot{r}\,\dot{\phi}\,\sin\left[2(\phi(t) + \Phi)\right]\Bigg] + \left(-\dot{r}^2 - r^2\dot{\phi}^2 + \frac{1}{r}\right)\sin^2 \iota\Bigg\}\,,\\
\label{inspccorrected}
h_{\times}(t)&=&\frac{4 \mu }{D}\cos\iota\Bigg\{  \sin\left[2(\phi(t) + \Phi)\right]\left(-\dot{r}^2 + r^2 \dot{\phi}^2 + \frac{1}{r}\right) \nonumber \\ &-& 2r\,\dot{r}\,\dot{\phi}\,\cos\left[2(\phi(t) + \Phi)\right]\Bigg\},
\end{eqnarray}
\noindent where \(\dot{r}\) can be computed using
\begin{displaymath}
\frac{{\mathrm{d}}r}{{\mathrm{d}}t} = -\frac{1}{u^2}\frac{{\mathrm{d}u}}{{\mathrm{d}}x}\frac{{\mathrm{d}x}}{{\mathrm{d}}t}\, .
\end{displaymath}
Having described the inspiral model, we now discuss the approach followed to
smoothly connect the late inspiral evolution onto the plunge phase. The
adiabatic prescription given by Eq.~\eqref{radev} breaks down when 
\(dE/dx\rightarrow 0\). Therefore, we need a scheme that enables us to match
the late inspiral phase onto the plunge phase. We will do this by modifying 
the ``transition'' phase, extending the method of Ori and Thorne~\cite{amos}.


\subsection{Transition and plunge phases}
\label{trans}
In this Section we describe an extension of the transition phase model introduced by Ori and Thorne~\cite{amos}. The basic idea behind this approach can be understood by studying the motion of an inspiralling object in terms of the effective potential, \(V(r, L)\), which takes the following simple form for  a Schwarzschild BH~\cite{mtw}:
\begin{equation}
V(r, L) = \left(1-\frac{2}{r} \right)\left(1+ \frac{L^2}{r^2}\right).
\label{effpot_fmrc}
\end{equation} 
\noindent Throughout the inspiral, the body moves along a nearly circular orbit, and hence the ratio of the energy flux to the angular momentum flux is given by:
\begin{equation}
 \frac{{\mathrm{d}}E}{{\mathrm{d}}\tau} = \Omega  \frac{{\mathrm{d}}L}{{\mathrm{d}}\tau}.
 \label{radII}
 \end{equation}
Near the ISCO, the energy and angular momentum of the body satisfy the following relations:
\begin{eqnarray}
\label{ener_emri}
E & \rightarrow & E_{\rm{ISCO}} + \Omega_{\rm{ISCO}}\, \xi,\\
\label{ang_emri}
L  & \rightarrow & L_{\rm{ISCO}}+  \xi.
\end{eqnarray}
\noindent Re-writing the effective potential, Eq.~\eqref{effpot_fmrc}, in terms of \(\xi = L-L_{\rm{ISCO}}\), one notices that during early inspiral, \(\xi \gg0\), the motion of the object is adiabatic, and the object sits at the minimum of the potential ---as shown in the top panel of Figure~\ref{effpotper}. However, as the object nears the ISCO, the minimum of the potential moves inward due to radiation reaction. At some point, the body's inertia prevents the body from staying at the minimum of the potential, and adiabatic inspiral breaks down~\cite{amos} --- illustrated in the right-hand panel of Figure~\ref{effpotper}.


\begin{figure*}%[ht]
% \centerline{
\includegraphics[height=0.6\textwidth,  clip]{figures/imrimri/effective_potential_nonpert.eps}
\includegraphics[height=0.6\textwidth,  clip]{figures/imrimri/effective_potential_pert.eps}
% }
\caption{Top panel: The object sits at the minimum of the effective potential, Eq.~\eqref{effpot_fmrc}, which corresponds to the case  \(\xi = L-L_{\rm{ISCO}} \gg 0\). Bottom panel. Blue (top) curve: radial geodesic motion, which corresponds to \(\xi = L-L_{\rm{ISCO}} \gg 0\);  Red (middle) curve: the object nears the ISCO and the orbit shrinks due to radiation reaction. Note that the minimum of the potential has moved inwards (\(\xi = 0.35\)). Yellow (bottom) curve:  body's inertia prevents it from staying at the minimum of the potential, and adiabatic inspiral breaks down  (\(\xi = 0\)). At this point the transition regime takes over the late inspiral evolution~\cite{amos}. Note: this plot is based on Figure 1 of~\cite{amos}.}
\label{effpotper}
\end{figure*}

\noindent The equation that governs the radial motion during the transition regime is found by linearising the equation 
\begin{equation}
\left(\frac{{\mathrm{d}}r}{{\mathrm{d}}\tau}\right)^2 = E(r)^2 - V(r,\,L),
\label{geoeom}
\end{equation}
\noindent using Eqs.~\eqref{ener_emri}, \eqref{ang_emri}, and
\begin{equation}
\label{xi_eq}
\frac{{\mathrm{d}}\xi}{{\mathrm{d}}\tau}= \kappa\, \eta, \quad {\rm{with}} \quad \kappa =\bigg[\frac{32}{5}\Omega^{7/3} \frac{\dot{{\cal{E}}}}{\sqrt{1-3u}}\bigg]_{\rm{ISCO}} \,,
\end{equation}
\noindent where \(\dot{{\cal{E}}}\) is the general relativistic correction to the Newtonian, quadrupole-moment formula~\cite{ori}. We now extend these Eqs. by including finite mass-ratio corrections. Eq.~\eqref{geoeom} can be replaced by 
\begin{equation}
\frac{{\mathrm{d}}x}{{\mathrm{d}}t}= \frac{u^2(1-2u)}{E(x) }\left(\frac{{\mathrm{d}}u}{{\mathrm{d}}x}\right)^{-1} \bigg[E(x)^2 - V\left(u(x),L(x)\right)\bigg]^{1/2},
\label{eomfmrc}
\end{equation}
 \noindent where we have used
 \begin{equation}
 \frac{{\mathrm{d}}\tau}{{\mathrm{d}}t}= \frac{1-2\,u(x)}{E(x)}\,,
 \label{tfact}
 \end{equation}
\noindent and the expressions for the energy and angular momentum are given by Eqs.~\eqref{enofxeq}, \eqref{lzofxeq}. In order to linearize Eq.~\eqref{eomfmrc} we replace $E(x)$ and $L(x)$ by Eqs~\eqref{ener_emri} and \eqref{ang_emri} respectively.

\begin{figure*}%[ht]
% \centerline{
\includegraphics[height=0.6\textwidth,  clip]{figures/imrimri/pot_eta_limit.eps}
\includegraphics[height=0.6\textwidth,  clip]{figures/imrimri/mod_eff_pot.eps}
% }
\caption{The top panel shows the effective potential for a Schwarzschild BH 
{\it without} including finite mass--ratio corrections. Note that the minimum
of the potential takes place at the ISCO, which can be determined using 
Eq.~\eqref{xisco_eq}.  The bottom panel exhibits the influence of finite 
mass-ratio corrections on the effective potential used to modify Ori and 
Thorne transition regime~\cite{amos}. The curves represent  binaries 
(top to bottom) with mass-ratios  
\(q \in [0,\, 1/100, \,1/20, \,1/10, \,1/6, \,1/5, \,1 ]\).}
\label{eff_pot_fig}
\end{figure*}

As discussed in~\cite{amos}, since these equations use the \(\eta\)-corrected values for \(E(x_{\rm{ISCO}} )\), \(L(x_{\rm{ISCO}}) \) and \( \Omega_{\rm{ISCO}}\), then they remain valid even for finite mass-ratio \(\eta\)~\cite{amos}. In Figure~\ref{eff_pot_fig} we show the effect that these finite mass-ratio \(\eta\) corrections have on the effective potential \(V(x, L(x))\). We determine the point at which the transition regime starts by carrying out a stability analysis near the ISCO using \({\mathrm{d}} E/ {\mathrm{d}} x\). As shown in Figure~\ref{dedx}, the ISCO is determined by the relation \({\mathrm{d}} E/{\mathrm{d}} x =0\). We have found that the relation 
\begin{equation}
\left(\frac{{\mathrm{d}}E}{{\mathrm{d}}x}\right)\Bigg |_{\rm{transition}} = -0.054 + \frac{1.757\times10^{-4}}{\eta}\,,
\label{transition_point}
\end{equation}
\noindent provides a robust criterion to mark the start of the transition regime for binaries with mass-ratios \(1/100<q<1/6\).  


In~\cite{ori}, the authors only kept terms linear in \(\xi\), but we have explored which higher order terms had a noticeable impact on the evolution by examining their impact on the length  and phasing of the waveform. We found that terms  \(\propto \xi\) and \( \propto (u- u_{\rm{ISCO}})\xi\) were important, but corrections at order \({\cal{O}}(\xi^2)\) could be ignored even for comparable mass-ratio systems.

We model the evolution of the orbital frequency during the transition regime and thereafter in a  different manner to that proposed by Ori and Thorne~\cite{amos}. In order to ensure that the late-time evolution of the orbital frequency of our self-force model is as close as possible to the orbital evolution extracted from numerical relativity simulations, we incorporate the late-time frequency evolution that was derived by Baker et al~\cite{Baker:2008} in their implicit rotating source (IRS) model, namely:
\begin{equation}
\frac{d \phi}{d \mathrm{t} }  = \Omega_{\rm{i}}+ \left(\Omega_{\rm{f}}\  -\Omega_{\rm{i}}\right)\left(\frac{1 + \tanh( \ln\sqrt\varkappa + (t-t_0)/b)}{2}\right)^{\varkappa}\, ,
\label{late_frequency}
\end{equation}
\noindent where \( \Omega_{\rm{i}}\) is the value of the orbital frequency when the transition regime begins, and  \( \Omega_{\rm{f}}\) is the value of the frequency at the light ring, which corresponds to \(\omega_{\rm{\ell m n}}/m\), where \(\omega_{\rm{\ell m n}}\) is the fundamental quasi-normal ringing frequency \((n=0)\) for the fundamental mode \((\ell, m) = (2,2)\) of the post-merger black hole (see Eq.~\eqref{finspin} below).  The constant mass-dependent coefficient \(t_0\) is computed by ensuring that \(d\Omega/dt\) peaks at a time \(t=t_0\). The parameter \(\varkappa\) is computed by enforcing continuity between the first order time derivative of the orbital frequency as predicted by the self-force evolution ---Eq.~\eqref{new_phase}--- and that given by the first order time derivative of Eq.~\eqref{late_frequency}.

% At the end of the plunge phase, we match the plunge waveforms, which are generated using Eqs.~\eqref{insppcor}, \eqref{inspccorrected}, onto the \(l=m=2\), \(n=0\) ringdown mode since this dominates the ringdown radiation. In the following Section we will describe in detail the procedure followed to attach the ringdowm waveform. 

After the transition regime, the plunge phase equations of motion are:  the second order time  derivative of Eq.~\eqref{eomfmrc} which gives the radial evolution, and Eq.~\eqref{late_frequency} which describes the orbital frequency evolution. We determine the point at which to attach the plunge phase by integrating  Eq.~\eqref{eomfmrc} backwards in time, and finding the point at which the transition and plunge equations of motion smoothly match. 


\begin{figure*}[ht]
\centerline{
\includegraphics[height=0.4\textwidth,  clip]{figures/imrimri/rvstm1m6.eps}
\includegraphics[height=0.4\textwidth,  clip]{figures/imrimri/xvsMOmegam1m6.eps}
}
\centerline{
\includegraphics[height=0.4\textwidth,  clip]{figures/imrimri/rvstm1m8.eps}
\includegraphics[height=0.4\textwidth,  clip]{figures/imrimri/xvsMOmegam1m8.eps}
}
\caption{(Top, bottom) panels: the left panel shows the inspiral, transition and plunge radial evolution for a BH binary of mass-ratio \(q=(1/6,\,1/8)\) --- and total mass \(M\in (7M_{\odot} ,\, 9M_{\odot} )\) --- using the coordinate transformation given by Eq.~\eqref{rofx}. The right panel shows the orbital frequency \(M\Omega\) from late inspiral all the way to the light ring. The evolution starts from an initial radial value \(r=30M\).}
\label{IP}
\end{figure*}

\begin{figure*}[ht]
\centerline{
\includegraphics[height=0.4\textwidth,  clip]{figures/imrimri/rvstm1m10.eps}
\includegraphics[height=0.4\textwidth,  clip]{figures/imrimri/xvsMOmegam1m10.eps}
}
\centerline{
\includegraphics[height=0.4\textwidth,  clip]{figures/imrimri/rvstm1m15.eps}
\includegraphics[height=0.4\textwidth,  clip]{figures/imrimri/xvsMOmegam1m15.eps}
}
\caption{As in Figure~\ref{IP}, but with the (top, bottom) panels showing the radial and orbital frequency evolution for binaries with mass--ratios \(q=(1/10,\,1/15)\), and total mass \(M\in( 11M_{\odot},\, 16M_{\odot})  \), respectively.  As before, the evolution starts from an initial radial value \(r=30M\).}
\label{IPI}
\end{figure*}


In Figures~\ref{IP} and \ref{IPI} we show the evolution obtained by combining Ori and Thorne's~\cite{amos} transition approach for the radial motion with the frequency evolution proposed by Baker et al~\cite{Baker:2008}.   In all the cases shown in Figures~\ref{IP} and \ref{IPI}, the orbital frequency peaks and saturates at the value given by \(\omega_{\ell m n}/m\). This can be understood if we analyze the asymptotic behavior of Eq.~\eqref{late_frequency} near the light-ring, i.e.,
\begin{equation}
\frac{d \phi}{d \mathrm{t} }  \approx \Omega_{\rm{f}} -  \left(\Omega_{\rm{f}}\  -\Omega_{\rm{i}}\right)e^{ -2(t - t_0)/b}.
\label{frequency_LR}
\end{equation}
Recasting Eq.~\eqref{late_frequency} in this form, enables us to identify the
constant coefficient \(b\) with the e-folding rate for the decay of the 
fundamental quasinormal mode (QNM). Using the IRS model for frequency evolution
therefore allows for a smooth transition from late inspiral to the plunge phase.

% As discussed previously,  
% Eq.~\eqref{late_frequency} predicts the expected orbital evolution during late
% inspiral and onward. 
% 
% To provide a unified description from late inspiral through to ringdown, we have decided to adopt the IRS approach, since this framework allows us to smoothly transition from late inspiral  onto the plunge phase, and finally describe the ringdown waveform as a natural consequence of the IRS strain-rate amplitude decay relation \(A^2(t) \propto \Omega \dot{\Omega}\)~\cite{Baker:2008}. In other words, since Eq.~\eqref{frequency_LR} has the correct behavior near the light-ring as predicted by BHPT, the IRS model provides a natural framework to attach the ringdowm waveform at the end of the plunge phase. We will discuss this feature in further detail in the following Section.

\subsection{Ringdown Waveform}
\label{RDwav}

 Numerical relativity simulations have shown that coalescing binary BHs in 
 general relativity lead to the formation of a distorted rotating remnant,
 which radiates GWs while it settles down to a stationary Kerr
 BH~\cite{Berti:2006, Berti:2006b}. The GWs emitted during this intermediate
 phase resemble a ringing bell. Hence, this type of radiation is commonly 
 known in the literature as ringdown radiation, and consists of a superposition
 of QNMs --- first discovered in numerical studies of the scattering of GWs
 in the Schwarzschild spacetime by Vishveshwara~\cite{Vish:1970}.   QNMs are
 damped oscillations whose frequencies are uniquely determined by the mass 
 and spin of the post-merger Kerr BH.  The frequency \(\hat \omega\) of each 
 QNM has two components: the real part represents the oscillation frequency, 
 and the imaginary part corresponds to the inverse of the damping time:
 \begin{equation}
 \hat \omega = \omega_{\ell m n} - i/\tau_{\ell m n}.
 \label{omega_QNM}
 \end{equation}
\noindent As discussed above, the observables 
\(\omega_{\ell m n}, \, \tau_{\ell m n}\) are uniquely determined by the final
mass, \(M_{\rm{f}}\), and final spin, \(q_{\rm{f}}\), of the post-merger Kerr BH.
The mass of the post-merger BH  \(M_{\rm{f}}\) is modeled using the 
phenomenological fit proposed in~\cite{Barausse:2012},
\begin{equation}
\frac{M_{\rm{f}}}{M} = 1- \left(1-\frac{2\sqrt{2}}{3}\right)\eta - 0.543763\, \eta^2.
\label{finalmass}
\end{equation}
% 
This expression reproduces the expected result in the test-particle 
limit, and also reproduces results from currently available NR
simulations~\cite{Barausse:2012,spif}. We determine the final spin of the 
BH remnant  \(q_{\rm{f}}\) using the fit proposed in~\cite{Bounanno:2007}:
\begin{equation}
q_{\rm{f}} =    \sqrt{12}\, \eta + s_1\,\eta^2 + s_2\,\eta^3,
\label{finspin}
\end{equation}
with
\begin{gather}
s_1=-3.454\pm0.132, \qquad  s_1=2.353\pm0.548.
\label{spin_coeff}
\end{gather}
\noindent This prescription is consistent with the numerical relativity
simulations described in~\cite{Bounanno:2007,spif}, and reproduces test-mass
limit predictions. This compact formula is also consistent with the  
prescriptions introduced in~\cite{Barus:2009,Rezzolla:2008}. The largest
discrepancy between Eq.~\eqref{finspin} and those derived in~\cite{Barus:2009,
Rezzolla:2008}  is \(\lesssim2.5\%\) for binaries with \(q\lesssim1/6\). 
The ringdown waveform is given by~\cite{Berti:2006b, Baker:2008}
\begin{eqnarray}
h(t)&=& -\left(h_{+} - i h_{\times}\right) = \frac{ M_{\rm{f}} }{D} \sum_{\ell m n} {\cal{A}}_{\ell m n}\,e^{-i \left(  \omega_{\ell m n}\, t  + \phi_{\ell m n} \right)} \, e^{-t/\tau_{\ell m n} }\,, 
\label{waverd}
\end{eqnarray}
where \(  {\cal{A}}_{\ell m n} \) and \( \phi_{\ell m n}\) are constants to be determined by smoothly matching the plunge waveform onto the subsequent ringdown. The ringdown portion of the self-force waveform model constructed in this chapter includes the mode \(\ell=m=2\) and the tones \(n=0,\, 1, \, 2\). The approach we follow to attach the leading mode and overtones is the following: 

\begin{itemize}
\item In order to ensure continuity between the plunge and ringdown waveforms, we use the end of the plunge waveform --- Eqs.~\eqref{insppcor}, \eqref{inspccorrected} --- to construct an interpolation function \(F(t)\). The interpolation method used to construct \(F(t)\) is a cubic spline.
\item We match the plunge waveform onto the leading mode  \(\ell=m=2\), \(n=0\)  of the ringdown waveform, Eq.~\eqref{waverd}, at the point where the amplitude of the plunge waveform peaks, \(t_{\rm{max}}\). Attaching the mode requires \(F(t=t_{\rm{max}})\) and \(F'(t=t_{\rm{max}})\) which are computed from the interpolation function. These conditions fix two constants per polarisation.
\item To attach the first overtone,  \(\ell=m=2,\, n=1\),   we insert into Eq.~\eqref{waverd} the constants determined by attaching the leading mode as seeds to compute the amplitude and phase coefficients for the first overtone by enforcing continuity at  \(t_{\rm{max}} + dt\).
\item Finally, we insert into  Eq.~\eqref{waverd} the value of the amplitude and phase coefficients previously determined for the leading mode and first overtone, and determine the four remaining constants by enforcing continuity at \(t_{\rm{max}} + 2\,dt\).
\end{itemize} 


Having described the methodology followed to construct complete waveforms for comparable and intermediate mass-ratio systems, we finish this Section by putting together all these various pieces to construct sample waveforms for a few systems with mass-ratio \(q\in[1/6,\,1/8,\, 1/10,\, 1/15]\), and total mass  \(M\in( 7M_{\odot},\, 9M_{\odot},\, 11M_{\odot},\, 16M_{\odot}) \) in Figure~\ref{Completewavs}. 

\begin{figure*}[ht]
\centerline{
\includegraphics[height=0.4\textwidth,  clip]{figures/imrimri/wavem1m6complete.eps}
\includegraphics[height=0.4\textwidth,  clip]{figures/imrimri/wavem1m8complete.eps}
}
\centerline{
\includegraphics[height=0.4\textwidth,  clip]{figures/imrimri/wavem1m10complete.eps}
\includegraphics[height=0.4\textwidth,  clip]{figures/imrimri/wavem1m15complete.eps}
}
\caption{The panels show sample waveforms from inspiral to ringdown for systems with mass-ratios \(q\in[1/6,\,1/8]\) ---and total mass  \(M\in( 7M_{\odot},\, 9M_{\odot}) \)--- (top panels---from left to right) and \(q\in[1/10,\, 1/15]\)  ---and total mass  \(M\in( 11M_{\odot},\, 16M_{\odot}) \)--- (bottom panels ---from left to right). The inspiral evolution for the [top,\, bottom] panels starts from  \(r=[30M,\, 25M]\). }
\label{Completewavs}
\end{figure*}

% \subsection{ Ringdown waveform construction in the context of an Implicit Rotating Source}
% 
% Having described the ringdown waveform construction as a sum of quasinormal modes, we finish this Section  by exhibiting the power of the IRS model to describe the ringdown evolution. In the IRS model, the strain-rate amplitude decay is given by~\cite{Baker:2008}:
% \begin{equation}
% A^2(t) = 16\,\pi\, \dot{E} \approx 16\,\pi\,\xi\,\Omega\,\dot{\Omega}\,.
% \label{amp_rate}
% \end{equation}
% \noindent Using~Eq.~\eqref{frequency_LR}, in the limit \(\Omega \rightarrow \Omega_{\rm{f}}\), the amplitude decay is given by
% \begin{equation}
% A_0^2\,e^{-2t/\tau} \approx \frac{32\, \pi\,\xi\, \Omega_{\rm{f}}}{b}\left(\Omega_{\rm{f}} - \Omega_{\rm{i}}\right) e^{-2(t-t_0)/\tau}\,.
% \label{rateamp}
% \end{equation}
% \noindent Hence, the late-time amplitude in the IRS model is given by
% \begin{equation}
% A^2_{\ell m}  \approx 16\,\pi\,\xi_{\ell m}\,\omega_{\ell m}\,\dot{\omega}_{\ell m}\,,
% \label{amp_rate_IRS}
% \end{equation}
% \noindent where the value of \(\xi_{\ell m}\) is set by ensuring amplitude continuity at the light-ring. In Figure~\ref{QNMvsIRS} we explicitly show the equivalence of the ringdown waveform construction  both in the IRS approach and using the sum of QNMs utilized in the previous Section. This analysis shows that the IRS is a powerful tool  to model the late time portions of the waveforms in a unified way. 
% 
% 
% \begin{figure*}[ht]
% \centerline{
% \includegraphics[height=0.35\textwidth,  clip]{figures/imrimri/QNMvsIRSm1m6.eps}
% \includegraphics[height=0.35\textwidth,  clip]{figures/imrimri/QNMvsIRSm1m8.eps}
% }
% \centerline{
% \includegraphics[height=0.35\textwidth,  clip]{figures/imrimri/QNMvsIRSm1m10.eps}
% \includegraphics[height=0.35\textwidth,  clip]{figures/imrimri/QNMvsIRSm1m15.eps}
% }
% \caption{The panels show sample the late-time evolution of waveforms whose ringdown phase is modeled using the implicit rotating source (IRS) model and a sum of quasinormal modes (QNMs). The systems shown correspond to binaries with mass-ratios \(q\in[1/6,\,1/8]\) ---and total mass  \(M\in( 7M_{\odot},\, 9M_{\odot}) \)--- (top panels---from left to right) and \(q\in[1/10,\, 1/15]\)  ---and total mass  \(M\in( 11M_{\odot},\, 16M_{\odot}) \)--- (bottom panels ---from left to right). The inspiral evolution for the [top,\, bottom] panels starts from  \(r=[30M,\, 25M]\). }
% \label{QNMvsIRS}
% \end{figure*}

\subsection{Summary}
In this Section we briefly summarize the key ingredients that were used to develop the waveform model described in this chapter:
\begin{itemize}
\item{Inspiral evolution}
\begin{itemize}
\item The building blocks of the inspiral evolution are the expressions for the energy, \(E\), and angular momentum, \(L\), --- Eqs.~\eqref{enofxeq} and~\eqref{lzofxeq} --- that include gravitational self-force corrections and are valid over the domain \(0<x<\frac{1}{3}\)~\cite{Akcay:2012}.
\item The orbital frequency is modeled using Eq.~\eqref{new_phase}. This prescription encapsulates gravitational self-force corrections that render a better phase evolution when calibrated against EOB.
\item The inspiral trajectory is modeled using the simple prescription~\eqref{radev}. This scheme is no longer valid near ISCO, where \(dE/dx =0\) for binaries with \(q\leq 6\).
\item We construct the inspiral waveform using Eqs.~\eqref{insppcor}, \eqref{inspccorrected}.
\end{itemize}
\end{itemize}
% 
In order to improve the radiative evolution, we derive the second-order
corrections to the flux of energy. This was necessary in order to construct 
a waveform model that is internally consistent, i..e, since the orbital elements
include first-order conservative corrections, then radiative corrections should
enter the flux at second order. We calibrate these corrections by enforcing a
close agreement between our self-force model and EOB. We show that our model
can reproduce the orbital phase evolution predicted by EOB within the numerical
error of the NR simulations used to calibrate the EOB model itself.
% 

When the inspiralling object nears the ISCO, we need to invoke the `transition scheme' introduced by Ori and Thorne~\cite{amos}, which enables us to smoothly attach the laIn order to improve the radiative evolution, we derive the second-order
corrections to the flux of energy. This was necessary in order to construct 
a waveform model that is internally consistent, i..e, since the orbital elements
include first-order conservative corrections, then radiative corrections should
enter the flux at second order. We calibrate these corrections by enforcing a
close agreement between our self-force model and EOB. We show that our model
can reproduce the orbital phase evolution predicted by EOB within the numerical
error of the NR simulations used to calibrate the EOB model itself.
te inspiral evolution onto the plunge phase. 

\begin{itemize}
\item{Transition phase}
\begin{itemize}
\item The transition regime starts at a point when \( \mathrm{d} E/ \mathrm{d}  x\) satisfies Eq.~\eqref{transition_point}.
\item The equations of motion that govern the transition phase are~\eqref{ener_emri}, ~\eqref{ang_emri}. These relations are valid, since the motion near the ISCO is nearly-circular. 
\item Using Eqs.~\eqref{ener_emri}, ~\eqref{ang_emri}, we linearize the second order time derivative of Eq.~\eqref{eomfmrc}.
\item In order to reproduce the orbital phase evolution predicted by numerical simulations from the ISCO to the light-ring, we modify the original transition phase by smoothly matching the inspiral orbital phase evolution, Eq.~\eqref{new_phase}, onto the IRS model, Eq.~\eqref{late_frequency} at the start of the transition phase. 
\end{itemize}
\end{itemize}

\begin{itemize}
\item{Plunge phase}
\begin{itemize}
\item The equations of motion that govern the plunge phase are given by the second order time derivative of Eq.~\eqref{eomfmrc}, and Eq.~\eqref{late_frequency}.
\item We integrate these relations backwards in time to find the point at which both the transition and plunge equations of motion smoothly match. The transition phase ends at this point.
\item Near the light-ring Eq.~\eqref{late_frequency} has the behavior predicted by BHPT, which enables us to smoothly match the plunge phase onto the ringdown. %Put in different words, we don't need to interpolate the orbital frequency evolution to attain the expected value of the orbital frequency at the light-ring. 
\item Both the transition and plunge waveforms are constructed using Eqs.~\eqref{insppcor} and \eqref{inspccorrected}.
\end{itemize}
\end{itemize}


\begin{itemize}
\item{Ringdown phase}
\begin{itemize}
\item The ringdown waveform is constructed using Eq.~\eqref{waverd}.
\item We use the plunge waveform to construct an interpolation function \(F(t)\), and then use this function to attach the leading mode \(\ell=m=2,\, n=0\) at the point where the amplitude of the plunge waveform peaks, \(t_{\rm{max}}\). We enforce continuity by ensuring that \(F( t_{\rm{max}}) = h^{n=0}_{\rm{RD}}\) and  \(F'( t_{\rm{max}}) = h'^{n=0}_{\rm{RD}}\).
\item We include the first and second overtone \(n=1,\, n=2\) in the ringdown waveform.
% \item  Using the IRS model, we have shown that having knowledge of the time evolution of the orbital frequency provides sufficient information to construct the amplitude decay during ringdowm. Hence, we can construct an alternative ringdowm waveform using only Eq.~\eqref{amp_rate_IRS}, and ensuring smooth continuity with the plunge waveform. 
% \item Finally, we have explicitly shown that the implicit rotating source approach provides a natural transition from late-time radiation to ringdown that is equivalent to ringdown waveform modeling based on a sum of QNMs.
\end{itemize}
\end{itemize}

Throughout the chapter we have emphasized the fact that our model provides a more reliable framework to model binaries whose components are non-spinning, and with mass-ratios \(q\leq 1/6\), as compared to available PN approximants. It is worth emphasizing that our model is also computationally inexpensive. All the waveforms we generated to constrain the higher-order \(\eta\) corrections in the energy flux ---Eq.~\eqref{etacorrect_new}--- can be generated in fractions of a second. A direct comparison between our code and EOB shows that, averaged over 500 realizations, our code is \(\sim20\%\) faster than the optimized version of the EOB code currently available in the LIGO Algorithms Library. It should be emphasized, though, that our code at present has not been optimized, and hence, compared to EOB our model is expected to further reduce the cost of waveform generation, making it relatively more viable for parameter estimation efforts. This is a key feature in our model that enabled 
us to sample a wide region of parameter space to constrain the \(B_i\) coefficients in Eq.~\eqref{etacorrect_new}. 
% Furthermore, these self-forced waveforms do not need to be highly sampled near the light-ring, hence decreasing the speed with which they can be generated, because the prescription we have used to model the late-time orbital frequency evolution provides the correct evolution near the light-ring. This particular feature also enables us to match the plunge waveform onto its ringdown counterpart without having to interpolate the orbital frequency evolution using a phenomenological approach.
Our model is internally consistent and the only phenomenology invoked during its construction is related to currently unknown physics, i.e., higher order radiative corrections to the energy flux. Once these corrections are formally derived in the near future, the flexibility of our model will enable us to replace the radiative corrections that we have currently computed by numerical optimization.  At that stage, we 
will be able to describe in a single unified model the dynamical evolution of binaries whose mass-ratios range from the extreme to the comparable regime. 
%\clearpage 

\section{Conclusions}
\label{conclu}

In this chapter we have developed a self-force waveform model that captures the
inspiral, merger and ringdown phases for binaries with mass-ratios 
\(q\leq 1/6\). This work suggests that a model which incorporates first-order
conservative self-force corrections in the orbital elements, and second-order
radiative corrections in the dissipative piece of the self-force, may suffice 
to describe in a unified manner the coalescence of binaries with mass-ratios 
that range from the comparable to the extreme. Using the available conservative 
self-force corrections~\cite{Akcay:2012}, we also found that binaries with
mass-ratios \(q\gtrsim1/6\) do not have an ISCO. For systems with \(q\leq1/6\),
we have derived a simple relation that provides the location of the ISCO in 
terms of the symmetric mass-ratio (see Eq.~\eqref{xisco_eq}). To describe the 
inspiral evolution, we obtain second-order corrections to the energy flux by 
minimizing the phase discrepancy between our self-force model and the EOB model~\cite{BuonannoEOBv2Main, Damour:2013} for a variety of mass-ratios. 
We show that our model reproduces the phase evolution of the EOB model within
the accuracy of available numerical simulations for a variety of mass-ratios. 

This chapter also presents an extension of the ``transition regime'' developed
by Ori and Thorne~\cite{amos} to smoothly match the late inspiral evolution
onto the plunge phase. We found that the inspiral phase expression for 
the orbital frequency does not reproduce the same accurately during the plunge
phase, as predicted by NR simulations. Therefore, we embedded the self-force 
framework in the IRS model proposed by Baker et al~\cite{Baker:2008} to 
ensure that our model is as close as possible to the orbital evolution 
predicted by NR simulations. The implementation of this prescription ensures 
that the orbital frequency saturates near the light-ring, which facilitates
matching onto the ringdown phase. 
% We have shown that the IRS model provides a natural transition onto the ringdown phase that is equivalent to a ringdown waveform construction based on a sum of QNMs. 

The motivation for constructing this model is two-fold: to exhibit the
versatility of the self-force formalism to accurately describe the evolution
of binaries beyond the extreme mass-ratio limit; and to provide a tool that 
can be used to extract information from GW observations of comparable and 
intermediate mass-ratio binaries. Current studies have only explored the
use of PN approximants to model the coalescence of NSBH  binaries, despite their
inadequacy to capture the evolution of these systems~\cite{Prayush:2013a,  
pnbuo, Nitz:2013mxa} (see Figure~\ref{pn_approx}). Comparing 
Figures~\ref{pn_approx} and~\ref{PNoptimized}, we conclude that even if second
generation GW detectors were only capable of capturing the inspiral evolution
of NSBH mergers, our self-force model would be better equipped to describe 
these events. The construction of this IRS self-force model constitutes an
important step towards the construction of more reliable waveforms to describe 
IMRCs. 
% In~\cite{Smith:2013}, it was shown that Huerta--Gair (HG) waveforms --- which are closely related to the model developed in this chapter, and which were also based on a consistent combination of BHPT and self-force corrections --- work in the relevant mass range for advanced detectors. The new model introduced in this chapter, improves upon these waveforms and is expected not only to work in the regime of interest to advanced detectors but over a wider parameter space that could be applicable to third generation detectors  or later observations.


Having developed a foundation to model binaries on circular orbits whose
components are not rotating, the next step is to incorporate more ingredients 
in order to model GW signals from binaries whose components have significant
spin~\cite{Foucart:2012, BuonannoEOBv2Main, maeda, burko, smallbody, buoerr1, 
buoII, TaylorT4Origin}, or for systems that form in core-collapsed globular 
clusters, and hence are expected to have non-negligible eccentricity at 
merger~\cite{Leary:2009, Huerta:2013a}. In order to do so, we require input
both from the self-force program --- which is making substantial progress 
towards the computation of the self-force in a Kerr background~\cite{Fan:2013b, Sam:2011,Pound:2013}--- and from Numerical Relativity
simulations~\cite{Mroue:2013xna}, especially for binaries with \(1/20<q<1/10\).
% Undoubtedly, the development and implementation of improved numerical 
% algorithms~\cite{Fan:2013a} to carry out these simulations will facilitate 
% the realization of these studies in the near future.

% Recent observational discoveries~\cite{Morscher:2013} suggest that NSBH mergers may also occur in globular clusters. Hence, in these dense stellar environments we may expect that multiple \(n\)-body interactions and binary exchange processes may lead to the formation of binaries on eccentric orbits. Detectors such as the Einstein Telescope, operating at low frequencies \(\sim 1\)Hz, may be capable of detecting the signature of eccentricity during the early inspiral. In order to assess these effects, we aim to extend the model introduced in this chapter by including eccentricity, making use of self-force corrections for generic orbits in a Schwarzschild background~\cite{wargar}. 



\Chapter{Conclusions}
\label{ch:conclusions}
%%%%%%%%%%%%%%%%%%%%%%%%%%%%%%%%%%%%%%%%%%%%%%%%%%%%%%%%%%%%%%%%%%%%%%%%%%%%%%%
%%% Describe the wondrous conclusions of the thesis.
%%%%%%%%%%%%%%%%%%%%%%%%%%%%%%%%%%%%%%%%%%%%%%%%%%%%%%%%%%%%%%%%%%%%%%%%%%%%%%%


The first observation runs of Advanced LIGO and Advanced Virgo detectors 
are scheduled for $2015$. By $2018$, these detectors will reach 
their design sensitivity. These second-generation terrestrial detectors
will be able to see up to $10$ times further out in the universe 
than their earlier counterparts. For a compact binary population
uniformly distributed in co-moving volume, this translates to 
a thousandfold increase in the expected detection rate.
% 
Gravitational wave searches make use of theoretical knowledge of
binary dynamics and employ modeled waveforms as filter templates.
With the increase in sensitivity, the resolution of the detectors 
for small errors in modeled waveforms also increases. In this dissertation,
we primarily focus on selecting and developing optimal waveform filters
for Advanced LIGO searches. We also validate gravitational-wave 
search algorithms using accurate numerically simulated signals injected 
into emulated detector noise.

Past binary black hole searches have used post-Newtonian (pN) and 
Effective-One-Body (EOB) waveforms as filters. While the pN waveforms are 
computationally inexpensive, they are restricted to the inspiral
regime of binary coalescence. EOB waveforms include the complete
coalescence process through inspiral, merger and ringdown, and also
the sub-dominant waveform harmonics. However, they are also 
computationally more expensive. For low mass binary black holes 
($m_1,m_2\leq 25M_\odot$),
we explore the region of the parameter space over which pN waveform
templates are sufficiently accurate, in the sense of being able to 
recover more than $97\%$ of the optimal signal-to-noise ratio, 
and where in the parameter space would searches need EOB
waveform templates.
% 
% For binaries with masses $m_1,m_2\leq 25M_\odot$, we compare the 
% inspiral-only post-Newtonian waveforms with the recently proposed
% Effective-One-Body (EOB) model~\cite{BuonannoEOBv2Main}. 
% As this EOB model is calibrated
% against high-accuracy numerical simulations of non-spinning binary 
% black holes, it is demonstrably accurate for {\it comparable}
% mass-ratio binaries. However, it is computationally more expensive
% than the post-Newtonian approximants. 
% We investigate the region of the parameter
% space of non-spinnning binaries where the accuracy of post-Newtonian
% approximants is sufficient and we can win with computational cost, as
% well as the region where EOB waveforms would be required. 
Here we approximate the waveforms with their dominant multipoles. Next,
we study the impact of ignoring sub-dominant waveform multipoles in 
searches. We find that including sub-dominant harmonics could increase
the reach of aLIGO and Virgo for binaries which have their orbital
angular momentum highly inclined to the line of sight connecting them
to the detector.

Numerical Relativity (NR) has seen recent breakthroughs and rapid progress
in simulating the merger of orbiting black holes. These are the most 
accurate solutions to Einstein's field equations available. Still, 
due to their computational cost, numerical relativity simulations 
span only the last stages of the binary inspiral, alongwith the merger
and ringdown. It is possible to join these short but accurate strong-field
simulations with post-Newtonian waveforms that cover the slow-motion
regime, to construct pN-NR {\it hybrids}. We demonstrate that, within 
the limits of current NR technology, it is possible and viable to use 
hybrid waveforms in gravitational wave searches. In addition, we show
that hybrid waveforms can cover the entire region of the binary black
hole parameter space where pN waveforms are insufficient for Advanced 
LIGO searches.


Apart from having applications as search templates, and in enhancing the
accuracy of waveform models, NR simulations
can be used to validate gravitational-wave search algorithms.
We do precisely this within the purview of the NINJA-2 project. 
Several numerical relativity groups contributed
post-Newtonian-hybridized simulations to the project. These were subsequently
injected in emulated advanced detector noise. We demonstrate the ability of
existing search algorithms to successfully {\it detect} these simulations
embedded within realistic noise. This is different from the NINJA-1 project
on a few counts, one of them being the nature of the emulated noise. In the 
NINJA-2 project, initial LIGO data with its non-Gaussian transient noise was
recolored to the expected sensitivity of the Advanced LIGO-Virgo detectors, as
opposed to colored Gaussian noise that was used in NINJA-1. 
Therefore this project provided a more robust test of our search methods, and 
provided a benchmark against which future search developments could be compared.


While the above concerns primarily comparable mass-ratio binaries, we 
also develop a waveform model for intermediate mass-ratio ones with 
$m_1/m_2 \in [10, 100]$. 
Intermediate mass-ratio systems, containing intermediate mass  
and stellar mass black holes will also be relatively more massive than stellar
mass binaries.
This would shift the frequency of the emitted gravitational radiation to 
lower values, and their late-inspiral and merger would occur in the most
sensitive frequency band of the Advanced detectors. This makes the modeling 
of the later portion of their waveforms crucial to their detection. 
%
First-order conservative self-force corrections have been derived for a
test-particle moving in the background of a supermassive Schwarzschild 
black hole. Using the form of these calculations, we formulate a 
prescription to model the early and late inspiral
of such binaries. Then, using the implicit rotation source picture
(due to Baker et al~\cite{Baker:2008}), we develop a model for the plunge and merger,
where the black holes are close and the orbits are no longer quasi-circular.
We then complete the description by stitching the quasi-normal modes emitted
by the black hole formed at merger. 
Therefore, we complete a model that captures the entire coalescence process
for intermediate mass-ratio binaries of non-spinning black holes.

To summarize, for {\it comparable} mass ratio binaries, we show that a combination
of post-Newtonian and post-Newtonian--Numerical-Relativity hybrid waveforms
would be sufficient for gravitational wave searches. This is true for the 
entire stellar-mass non-spinning binary black hole parameter space.
We also successfully validate gravitational wave search algorithms 
that have been used in the most recent LIGO-Virgo searches, using accurate 
numerical simulations injected in emulated detector noise. 
For {\it intermediate} mass ratios, we develop an accurate waveform 
model that captures the binary dynamics from the weak-field slow-motion
regime to the strong-field regime up to the merger of both compact objects. 
Therefore the work presented in this dissertation is an effort towards
arriving at optimal search filters for non-spinning binary black holes 
which are prospective sources detectable by the second-generation terrestrial
gravitational wave detectors; as well as towards validating existing search 
algorithms using an improved testing methodology.














% \appendix
%  \Chapter{Post-Newtonian Waveforms}
% \label{ch:pn_waveforms}
% \input{PNEqns}

\clearpage
\bibliographystyle{unsrt}
\bibliography{references}
\addcontentsline{toc}{chapter}{\numberline {Bibliography}}

\end{document}

