% $Id$

Although the upper limit that we have placed on the rate of binary black hole
MACHO inspirals in the galaxy is lower than the upper bound of the predicted
rates, the LIGO interferometers were not at design sensitivity when the S2
data was taken. At present, the sensitivities of the instruments are
significantly better than during S2, as can be seen from
figure~\ref{f:s3strain}, and progress on reducing noise in the interferometers
continues apace.  The increase in detector sensitivity makes a larger volume
of the Universe accessible to searches for binary inspirals. In addition to
this, the amount of data is also increasing as the interferometers become more
stable.

These improvements in the instruments will increase the chance of detecting
gravitational waves from binary inspirals. If the rates of binary black hole
MACHO coalescence are truly as high as predicted, then initial LIGO would
stand an excellent chance of detecting an inspiral. The first detection of
gravitational waves will be a major scientific breakthrough and will yield and
enormous amount of scientific information, particularly if the detection came
from a binary black hole MACHO. The length of binary black hole MACHO
inspirals in the sensitive band of the interferometer will allow extremely
accurate parameter estimation as well as tests of post-Newtonian theory. For
systems with total mass greater than $\sim 0.64\,\mathrm{M}_\odot$ LIGO will
be sensitive to the coalescence of the binary and will be able to study the
strong gravitational field effects when two binary black holes merge. When
this is coupled with the accurate parameter estimation available from the
earlier part of the waveform, the inspiral of a binary black hole MACHO could
be an excellent laboratory for General Relativity.  A detection would also
impact the studies of halo dark matter and early universe physics, providing a
MACHO component to the halo and suggesting that primordial black holes do
indeed form in the universe.

In the absence of detection, the improvements in detector sensitivity will
dramatically improve the upper limits placed on the rate of binary black hole
MACHO inspirals. Once these rates are below the predicted rates, we may begin
to use observations from gravitational wave interferometers to constrain the
fraction of galactic halos in the form of primordial black hole MACHOs. While
this may not be as significant as a detection, it will still be of interest to
the astrophysical community.

\newpage 

\begin{figure}[p]
\label{f:s3strain}
\vspace{5pt}
\begin{center}
\includegraphics[width=\textwidth]{figures/conclusion/s3strain}
\end{center}
\caption[Comparison of Best LIGO Interferometer Sensitivity]{%
Comparison of the best sensitivities of the LIGO interferometers between
science runs. The solid curve shows the design sensitivity for the $4$~km
interferometers: the LHO $4$~km is only a factor of $\sim 2$ away from design
at $100$~Hz during S3.
}
\end{figure}

