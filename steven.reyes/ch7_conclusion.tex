The first direct detection of gravitational waves from a binary black hole merger occurred in 2015, beginning the era of gravitational wave astronomy. 

In Chapter 3, we introduced the PyCBC offline search analysis pipeline, one of Advanced LIGO's gravitational wave detection pipelines. The PyCBC search pipeline is a matched filter search that searches for gravitational waves from compact binary coalescences by matching the data against gravitational waveform templates. Potential signals that match these waveform templates are designated as triggers and ranked according to a signal to noise ratio ranking statistic in addition to other waveform consistency checks. The statistical significance of these triggers is estimated and statistically significant candidates are considered to be gravitational wave candidates. During advanced LIGO's first observing run, two gravitational wave observations, GW150914 and GW151226 were observed as coming from merging binary black holes.

We also explored the implications of no detections of binary neutron star mergers or neutron-star black hole mergers in Advanced LIGO's first observing run. Using a Bayesian inference framework we modeled the detection of these systems as a Poisson process and estimated the sensitivity of the offline PyCBC search pipeline to mergers of systems with a neutron star through the use of simulated, software injections of these signals. We found that we could constrain the upper limit on the rate of binary neutron star mergers to be less than $12,600$ $\mathrm{Gpc}^{-3}$ $\mathrm{yr}^{-1}$ at the $90 \%$ confidence level. We found that for certain systems of neutron star-black hole systems that the merger rate could be constrained to less than $3600$ $\mathrm{Gpc}^{-3}$ $\mathrm{yr}^{-1}$. We were also able to constrain upper limits on gamma-ray burst beaming angles to be greater than $\sim 2$ degrees.

In Chapter 4, we examined further improvements on the PyCBC offline search analysis and presented 1-OGC, an open gravitational wave catalog. In the work of 1-OGC we analyzed publicly available data from Advanced LIGO's first observing run to search for gravitational waves from compact binary coalescence. Continued work on the gravitational wave ranking statistic have improved PyCBC's ability to discriminate potential gravitational wave signals against other noise transients. We confirmed the observations of gravitational wave signals GW150914, GW151226, and LVT151012. The improvement of the ranking statistic and improvements in statistical significance ranking permitted us to confidently claim GW151012 as an authentic gravitational wave detection with a $97.6 \%$ probability of being of astrophysical origin. No other statistically significant gravitational wave candidates were identified. The catalog of events was made open and public for other scientists to investigate. 

In Chapter 5, we examined the detection of GW170817, a binary neutron star merger discovered by LIGO and Virgo during their second observing run. Moreover, we conducted a Bayesian parameter estimation and hypothesis testing approach to examine whether nonlinear tides from a nonresonant, nonlinear $p$-$g$ mode instability were compatible with the observation of GW170817. Our resulting analysis showed that nonlinear tides were broadly compatible with the observation of GW170817, although we found that this occurred because the nonlinear tides did not cause a measurable change to the waveform or the nonlinear tidal parameters entered into the signal through significant degeneracies with the other intrinsic parameters in the signal. As we pursued this problem further we found that we could rule out nonlinear tides from a $p$-$g$ mode instability that matched standard waveforms with $< 99 \%$ match with large statistical significance.

As we collect more gravitational wave events and dig into lower-threshold events GW astronomy will permit us to explore new questions in astrophysics. In this thesis, we presented methods for investigating astrophysical implications for non-detections of gravitational waves from certain binary systems as well as methods for improving the sensitivity of our compact binary coalescence searches towards already detected classes of binary systems. Finally, we developed statistical techniques for testing astrophysical hypotheses on detected signals.
