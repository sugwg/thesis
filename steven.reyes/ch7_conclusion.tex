
%%%% Chapter 2 %%%%%
In chapter 2, we discussed the \pycbc{}\ search pipeline, a matched-filter
search pipeline for the detection of compact binary coalescence, and the results
that it gathered during LIGO's first observing run. In the first observing run, the LIGO detectors observed
gravitational waves from the merger of two
stellar-mass black holes, GW150914. The binary coalescence search detects GW150914 with
a significance greater than $\CBCEVENTSIGMA$$\sigma$ during this first observing
run. Detailed parameter estimation for
GW150914 is reported in Ref.~\cite{GW150914-PARAMESTIM}, the implications for
the rate of binary black hole coalescences in Ref.~\cite{GW150914-RATES}, and
tests for consistency of the signal with general relativity in
Ref.~\cite{GW150914-TESTOFGR}.  Ref.~\cite{GW150914-ASTRO} discusses the
astrophysical implications of this discovery. During the first observing run
a second binary merger, GW151226 was also discovered by \pycbc{}\. A third
gravitational wave candidate, LVT151012, was also discovered but was only found
with a false alarm rate less than \CBCSECONDEVENTFAR, and could not be
confidently claimed as a gravitational wave candidate.

%%%% Chapter 3 %%%%
In chapter 3, we reported the non-detection of \ac{BNS} and \ac{NSBH} mergers in advanced \ac{LIGO}'s first observing run.
We estimated the sensitive volume of Advanced \ac{LIGO} to such systems and were able to place 90\%
confidence upper limits on the rates of \ac{BNS} and \ac{NSBH} mergers, improving upon limits
obtained from Initial \ac{LIGO} and Initial Virgo by roughly an order of magnitude.
Specifically we constrained the merger rate of \ac{BNS} systems with component masses of $1.35\pm0.13M_{\odot}$
to be less than \MainBNSULPyCBCHighSpin~Gpc$^{-3}$~yr$^{-1}$. We also constrained
the rate of \ac{NSBH} systems with NS masses of $1.4 M_\odot$ and BH masses of at least $5 M_{\odot}$ to be less than \MainNSBHULPyCBCFiveAligned~Gpc$^{-3}$~yr$^{-1}$ if one considers a population where the component spins are (anti-)aligned with
the orbit. We constrained the rate of \ac{NSBH} systems with isotropic spin distributions in the
components of the spin direction to be less than \MainNSBHULPyCBCFiveIso~Gpc$^{-3}$~yr$^{-1}$.

We compared these upper limits with existing astrophysical rate models and found that the
current upper limits are in conflict with only the most optimistic models of the merger
rate. However, we expect that during the next two observing runs, O2 and O3, we will
either make observations of \ac{BNS} and \ac{NSBH} mergers or start placing significant constraints
on current astrophysical rates. Finally, we have explored the implications of this non-detection on
the beaming angle of short \acp{GRB}. We find that, if one assumes that all \acp{GRB}
are produced by \ac{BNS} mergers, then the opening angle of gamma-ray radiation must be larger
than \GRBBNSBeamingAngleConstraint; or larger than \GRBNSBHFiveBeamingAngleConstraint\ if
one assumes all \acp{GRB} are produced by \ac{NSBH} mergers.

%%%% Chapter 4 %%%%

In chapter 4, we presented a full catalog of gravitational-wave events and candidates from a \pycbc{}\-based, templated, matched-filter search of the LIGO O1 open data. Our analysis improved upon \cite{TheLIGOScientific:2016pea,Abbott:2016ymx} by using improved ranking of candidates via a phase, amplitude and time delay consistency check, an improved background model, and a template bank targeting a wider range of sources \citep{Nitz:2017svb, Nitz:2017lco,DalCanton:2017ala}. We verifed the discovery of GW150914 and GW151226 and report an improved significance of the candidate event LVT151012. In the analysis of \cite{TheLIGOScientific:2016pea,Abbott:2016ymx} LVT151012 was found to have a false alarm rate of approximately $1$ per $2$ years, but in the analysis of 1-OGC we found that LVT151012 could be instead found with a false alarm of $1$ per $24$ years. If the analysis had restricted itself to a search of the parameter space where binary black holes had been discovered before, the false alarm rate could have been estimated at $1$ per $446$ years. We also found that the probability of LVT151012 being of astrophysical origin is approximately $98 \%$. With these improvements of the statistical significance estimation we confidently claim LVT151012 as a gravitational wave event and designate it GW151012. Apart from these three signals, none of the other candidate events in the analysis were found with statistical significance. All of these candidates are listed in our catalog available at \release{}, along with tools for exploring and usingthe data.


%%%% Chapter 5 and 6 %%%%
In Chapter 5, we examined the detection of GW170817, a binary neutron star merger discovered by LIGO and Virgo during their second observing run. Moreover, we conducted a Bayesian parameter estimation and hypothesis testing approach to examine whether nonlinear tides from a nonresonant, nonlinear $p$-$g$ mode instability were compatible with the observation of GW170817. Our resulting analysis showed that nonlinear tides were broadly compatible with the observation of GW170817, although we found that this occurred because the nonlinear tides did not cause a measurable change to the waveform or the nonlinear tidal parameters entered into the signal through significant degeneracies with the other intrinsic parameters in the signal. As we pursued this problem further we found that we could rule out nonlinear tides from a $p$-$g$ mode instability that matched standard waveforms with $< 99 \%$ match with large statistical significance.

As we collect more gravitational wave events and dig into lower-threshold events GW astronomy will permit us to explore new questions in astrophysics. In this thesis, we presented methods for investigating astrophysical implications for non-detections of gravitational waves from certain binary systems as well as methods for improving the sensitivity of our compact binary coalescence searches towards already detected classes of binary systems. Finally, we developed statistical techniques for testing astrophysical hypotheses on detected signals.
