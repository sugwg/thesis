Advanced \ac{LIGO}'s first observing run occurred between \OoneSTART\ and \OoneEND\
and consists of data from the two \ac{LIGO} observatories in Hanford, WA and Livingston, LA.
The LIGO detectors were running stably with roughly 40\% coincident operation, and had
been commissioned to roughly a third of the design sensitivity by the time of the start of O1~\citep{Martynov:2016fzi}.
During this observing run the final offline dataset consisted of \OoneOfflineAnalysableHTimeSeconds\
of analyzable data from the Hanford observatory, and \OoneOfflineAnalysableLTimeSeconds\ of data from the
Livingston observatory. We analyze only times during which \emph{both} observatories
took analyzable data, which is \OoneOfflineAnalysableTimeSeconds. Characterization studies of the analysable
data found \OoneOfflineAnalysableCatTwoDiffSeconds\ of coincident data during which time
there was some identified instrumental problem---known to
introduce excess noise---in at least one of the interferometers~\citep{TheLIGOScientific:2016zmo}.
These times are removed before assessing the significance of events
in the remaining analysis time. Some additional time is not analysed because
of restrictions on the minimal length of data segments and because of data lost
at the start and end of those segments~\citep{TheLIGOScientific:2016qqj, TheLIGOScientific:2016pea}.
These requirements are slightly different between the two offline analyses, \pycbc\ and \gstlal\ . The \pycbc\
pipeline analysed 46.1 days of data.
