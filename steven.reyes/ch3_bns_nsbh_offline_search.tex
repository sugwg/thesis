The offline \ac{CBC} search of the \ac{O1} data set consists of two independently-implemented matched-filter
analyses: \gstlal~\citep{Messick:2016aqy} and
\pycbc~\citep{Usman:2015kfa}.
Full details of the \pycbc{}\ offline search pipeline are described in chapter~\ref{ch:GW150914_PyCBC_Offline}.

In contrast to the online search, the offline search uses data produced with
smaller calibration errors~\citep{Abbott:2016jsd}, uses complete information about the instrumental
data quality~\citep{TheLIGOScientific:2016zmo} and ensures that all available data is analysed.
The offline search in \ac{O1} forms a single search targeting \ac{BNS}, \ac{NSBH}, and \ac{BBH} systems. The
waveform filters cover systems with individual
component masses ranging from 1 to 99 $M_{\odot}$, total mass
constrained to less than 100 $M_{\odot}$ (see Figure~\ref{fig:banks}), and component dimensionless spins up to $\pm$ 0.05
for components with mass less than 2 $M_{\odot}$ and $\pm$ 0.99 otherwise~\citep{TheLIGOScientific:2016pea,Capano:2016dsf}.
Waveform filters with total mass less than 4 $M_{\odot}$ (chirp mass less than
$1.73 {M}_{\odot}$\footnote{\label{foot:note1}The
``chirp mass'' is the combination of the two component masses that
\ac{LIGO} is most sensitive to, given by $\mathcal{M} = (m_1 m_2)^{3/5} (m_1 + m_2)^{-1/5}$, where
$m_i$ denotes the two component masses})
for \pycbc\ (\gstlal) are modeled with the
inspiral-only, post-Newtonian, frequency-domain approximant
``TaylorF2''~\citep{Arun:2008kb,Bohe:2013cla,Blanchet:2013haa,Bohe:2015ana,mishra2016ready}.
At larger masses it becomes important to also include the merger and ringdown components of the waveform.
There a reduced-order model of the effective-one-body waveform calibrated against numerical
relativity is used~\citep{Taracchini:2013rva,Purrer:2015tud}. 
