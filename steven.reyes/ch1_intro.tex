On September 14, 2015 at 09:50:45 UTC the Advanced Laser Interferometer Gravitational wave Observatory (LIGO) detected a signal from the binary black hole merger GW150914~\cite{GW150914-DETECTION}. The initial detection of the event was made by low-latency searches for generic gravitational-wave transients~\cite{GW150914-BURST}. LIGO reported the results of a matched-filter search using relativistic models of compact binary coalescence waveforms that recovered GW150914 with a false alarm rate less than $5 \times 10^{-6} \, \text{yr}^{-1}$, establishing it as the first direct detection of gravitational waves from the merging of two black holes.

In LIGO's second observing run, the Livingston, LA and Hanford, WA observatories were joined by a third gravitational wave detector, Virgo. This gravitational wave network detected the gravitational wave signal from two merging binary neutron stars, GW170817~\cite{abbott2017gw170817}. The signal, GW170817, was detected with a combined signal-to-noise ratio of $32.4$ with a false alarm rate less than $1 \times 10^6$ years. The total mass of the binary system was estimated as $\sim$ 2.7 $M_\odot$ and at a luminosity distance of $\sim$ 40 Mpc. The association with the gamma-ray burst GRB 170817A, detected by Fermi-GBM 1.7 seconds after the coalescence, confirms that GW170817 involved the merging of a binary neutron star and provides the first direct evidence of a link between these mergers with a neutron star and short, hard gamma-ray bursts~\cite{Goldstein_2017}. Additional identifications of electromagnetic transient counterparts in the same location further supports the interpretation of this event as a neutron star merger~\cite{Abbott_2017_mma}. This unprecedented joint gravitational and electromagnetic observation has provided incredible opportunities and insight into astrophysics, dense matter, gravitation, and cosmology.

In chapter~\ref{ch:GW150914_PyCBC_Offline}, we introduce the \pycbc{}\ offline search analysis that was instrumental in the discoveries of GW150914 and GW170817. We describe the analysis at the time of LIGO's first observing run. The \pycbc{} search is a compact binary coalescence search~\cite{thorne.k:1987,Sathyaprakash:1991mt,Cutler:1992tc,Finn:1992wt,Finn:1992xs,Dhurandhar:1992mw,Balasubramanian:1995bm,Flanagan:1997sx} that targets gravitational waves from binary neutron stars, binary black holes, and neutron star--black hole binaries, using matched filtering~\cite{wainstein:1962} with waveforms predicted by general relativity~\cite{Peters:1963ux,Blanchet2006,Buonanno:1998gg,Buonanno:2000ef,Damour:2000we,Damour:2001tu,Ajith:2007qp,Ajith:2009bn,Santamaria:2010yb}.  The \pycbc{} analysis correlates the detector data with template waveforms that model the expected signal. The analysis identifies candidate events that are detected at both observatories consistent with the $10$~ms inter-site propagation time. Events are assigned a detection-statistic value that ranks their likelihood of being a gravitational-wave signal. This detection statistic is compared to the estimated detector noise background to determine the probability that a candidate event is due to detector noise. The probability that a gravitational wave candidates is due to detector noise is evaluated for the candidate event with the largest detection statistic and in the case that this probability is lower than $5 \sigma$ confidence, we remove the event from the background analysis and recalculate the probability that the other gravitational wave candidates are due to detector noise.

During the first observing run there were no discoveries of gravitational waves from compact binaries that contained a neutron star~\cite{O1BBH}. Conditional on the non-detection of these signals, the LIGO and Virgo collaboration searches established upper limits on the rate of mergers of these signals. The non-detection of mergers from binary neutron stars and neutron star-black hole binaries during LIGO's first observing run had important implications on plausible astrophysical formation channels for these binaries, and on whether mergers of binaries containing neutron stars could still be considered plausible mechanisms for unexplained astrophysical phenomenon such as short, hard gamma-ray bursts. We describe the analysis techniques used to set the estimated upper limit merger rates for binary systems that contain a neutron star. We also presented estimates for future rate estimations for the subsequent second and third observing runs.

Since the publication of the results from LIGO's first observing run there was considerable development of gravitational wave astrophysical analysis techniques that permitted increased sensitivity in the \pycbc{}\ search analysis \citep{Nitz:2017svb,Nitz:2017lco,DalCanton:2017ala}. LIGO's second observing run which ran between November 30, 2016 and ended on August 25, 2017 and also involved the Advanced Virgo (Virgo) from August 1, 2017, onward, presented a useful testbed for these techniques. At the same time, LIGO made the gravitational wave strain data needed for analysis publicly available for the entire first-observing run the GW Open Science Center~\citep{Vallisneri:2014vxa}. In Chapter~\ref{ch:1_OGC}, we present the results of a re-analysis of the publicly available data using the \pycbc{}\ search and we publish a full catalog of candidate gravitational wave events. We call this open catalog 1-OGC. The search was successful in re-estimating the statistical significance of LVT151012, which went from a marginal event to having a 97.6$\%$ probability of being of astrophysical origin. Thus we designate LVT151012 as GW151012. 


In Chapter~\ref{ch:Bayesian_data_analysis} of this dissertation we introduce advanced methods and tools for conducting Bayesian statistical analyses on gravitational wave data. In particular, we focus on Bayesian hypothesis testing and the advancements in many Markov-Chain Monte Carlo techniques for conducting Bayesian hypothesis testing. We introduce three distinct approaches for hypothesis testing, two based on a parallel tempering technique~\cite{lartillot2006computing,friel2008marginal,xie2010improving}, and one based on testing nested models~\cite{edwards1963bayesian, dickey1971weighted}. We discuss how these techniques apply to gravitational wave astronomy. 

In Chapter~\ref{ch:pg_modes} we apply these Bayesian hypothesis testing tools to explore astroseismology using the binary neutron star merger GW170817. Recent studies have estimated the star's tidal deformability and placed constraints on the equation of state of the neutron stars~\citep{TheLIGOScientific:2017qsa,Tews:2018iwm,Most:2018eaw,Raithel:2018ncd,de2018tidal,Abbott:2018exr,Abbott:2018wiz,Radice:2018ozg,LIGOScientific:2019eut,Capano:2019eae}. We explore a suggestion of \cite{Weinberg:2013pbi} that the star's tidal deformation can induce nonresonant and nonlinear daughter wave excitations in $p$- and $g$-modes of the neutron stars via a quasi-static instability. This instability would remove energy from a binary system and possibly affect the phase evolution of the gravitational waves radiated during the inspiral. Ref. \cite{Weinberg:2015pxa} claimed that the instability can rapidly drive modes to significant energies before the merger of the binary. The details of the instability saturation are unknown and so the size of the effect of the $p$-$g$ mode coupling on the gravitational-waveform is not known~\citep{Weinberg:2015pxa}. We conduct parameter estimation on the GW170817 signal using parameters modeling the $p$-$g$ mode instability. We report a Bayes factor of unity indicating a nonsignificant result. We find that modeling GW170817 with nonlinear tidal parameters create degeracies in the other intrinsic parameters of the binary. 
