On September 14, 2015 at 09:50:45 UTC the Advanced Laser Interferometer Gravitational wave Observatory (LIGO) Hanford, WA, and Livingston, LA, observatories detected a signal from the binary black hole merger GW150914~\cite{GW150914-DETECTION}. The initial detection of the event was made by low-latency searches for generic gravitational-wave transients~\cite{GW150914-BURST}. LIGO reported the results of a matched-filter search using relativistic models of compact binary coalescence waveforms that recovered GW150914 at a greater than $5 \, \sigma$ confidence, firmly establishing it as the first direct detection of gravitational waves from the merging of two black holes.
MERGER DETAILS


A subsequent detection of another binary black hole merger was made within the LIGO's first observing run on December 26, 2015~\cite{}. The signal, GW151226, was detected in the LIGO Livingston detector at 03:38:53.647 UTC and $\sim 1.1$ milliseconds later at LIGO Hanford. This signal was also detected with a statistical significance of $5 \sigma$. These two events marked the beginning of gravitational wave astronomy and marked the success of LIGO's first observing run which spanned from September 12, 2015 to January 12, 2016. MERGER DETAILS

This dissertation spans analyses pertaining to data from the LIGO's first and second observing runs. It contains analyses that were done within the LIGO collaboration and analyses that were conducted independently of the LIGO collaboration through the use of publicly available data.

In the second chapter, we introduce the PyCBC offline search analysis that was been instrumental in discovering and validating gravitational wave signals from LIGO's first observing run. The analysis is described as it was used in LIGO's first observing run. The \pycbc{} search analysis is a compact binary coalescence search~\cite{thorne.k:1987,Sathyaprakash:1991mt,Cutler:1992tc,Finn:1992wt,Finn:1992xs,Dhurandhar:1992mw,Balasubramanian:1995bm,Flanagan:1997sx} that targets gravitational waves from binary neutron stars, binary black holes, and neutron star--black hole binaries, using matched filtering~\cite{wainstein:1962} with waveforms predicted by general relativity.  The \pycbc{} analysis correlates the detector data with template waveforms that model the expected signal. The analysis identifies candidate events that are detected at both observatories consistent with the $10$~ms inter-site propagation time. Events are assigned a detection-statistic value that ranks their likelihood of being a gravitational-wave signal. This detection statistic is compared to the estimated detector noise background to determine the probability that a candidate event is due to detector noise. The probability that a gravitational wave candidates is due to detector noise is evaluated for the loudest candidate events and in the case that this probability is lower than $5 \sigma$ confidence, we remove the event from the background analysis and recalculate the probability that the other gravitational wave candidates are due to detector noise. This hierarchical removal procedure was automated in the second observing run and is addressed again in chapter 4 within a re-analysis of the first observing run.

During the first observing run there were no discoveries of gravitational waves from compact binaries that contained a neutron star \cite{O1BBH}. Conditional on the non-detection of these signals the LIGO collaboration investigated and established upper limits on the estimated rate of mergers of these signals. The non-detection of mergers from binary neutron stars and neutron star-black hole binaries during LIGO's first observing run had important implications on plausible astrophysical formation channels for these binaries, on whether mergers of binaries containing neutron stars could still be considered plausible mechanisms for unexplained astrophysical phenomenon such as short, hard gamma-ray bursts, kilonova, and the generation of many of the chemical elements within the universe. Chapter 3 of this dissertation describes the analysis techniques used to set the estimated upper limit merger rates for binary systems that contain a neutron star. It also presented estimates for future rate estimations for the subsequent second and third observing runs.

Since the publication of the results from LIGO's first observing run there was considerable development of gravitational wave astrophysical analysis techniques that permitted increased sensitivity in the PyCBC Offline search analysis (CITE, CITE). LIGO's second observing run which ran between November 30, 2016 and ended on August 25, 2017 and also involved the Advanced Virgo (Virgo) from August 1, 2017, onward, presented a useful testbed for these techniques. At the same time, LIGO made the gravitational wave strain data needed for analysis publicly available for the entire first-observing run the GW Open Science Center~\citep{Vallisneri:2014vxa,gw170817-losc}. In Chapter 4, we present the outcome of a concerted effort to reanalyze the publicly available data and publish a full catalog of candidate gravitational wave events from a matched filter search for compact binary coalescences, which we call 1-OGC. The search was successful in re-appraising the statistical significance of LVT151012, which went from a marginal event to having a 97.6$\%$ probability of being of astrophysical origin. In so doing, we decided that LVT151012 had more in common with a gravitational wave from a merging black hole than a noise transient and designated it as GW151012.

LIGO and Virgo's second observing run, in addition to a plethora of new binary black hole detections, offered a new insight into the world of astrophysics. The three-detector network was able to detect the gravitational wave signal from two merging binary neutron stars. . The signal, GW170817, was detected with a combined signal-to-noise ratio of 32.4 at a confidence level greater than $5 \sigma$ (CITE). The total mass of the binary system was estimated as $\sim$ 2.7 $M_\odot$, with a luminosity distance of $\sim$ 40 Mpc. It marks the closest and most precisely localized gravitational-wave signal presented at the writing of this dissertation. The association with the gamma-ray burst GRB 170817A, detected by Fermi-GBM 1.7 seconds after the coalescence, corroborates the hypothesis of a neutron star merger and provides the first direct evidence of a link between these mergers and short gamma-ray bursts. Subsequent identification of transient counterparts across the electromagnetic spectrum in the same location further supports the interpretation of this event as a neutron star merger. This unprecedented joint gravitational and electromagnetic observation provides insight into astrophysics, dense matter, gravitation, and cosmology. GW150914 gave birth to the field of gravitational wave astronomy, and GW170817 opened the door to multi-messenger astronomy, the co-operation of astronomers from the electromagnetic spectrum with gravitational wave astronomers. The impact of GW170817 on the field of astronomy and astrophysics cannot be overstated. Multimessenger astronomy offers opportunities to peer into the intersection of high energy matter and general relativity, and it offers incredible insights into cosmology.

In the fifth chapter of this dissertation we introduce advanced methods and tools for conducting Bayesian statistical analyses on gravitational wave data. They are tools to help determine and estimate the parameters that may characterize gravitational wave signals. They poke and prod at the ambiguities in parametrizing these signals. The tools presented in this fifth chapter also pursue the other side of the equation of statistical inference, and that is hypothesis testing. It is not enough to know how the parameters of the signal change under a different set of parameters or hypotheses, we desire to know whether our prior beliefs about the signal are efficient or useful in explaining the data present in the gravitational wave strain. We also lay down some of the basics of how this Bayesian inference is applied to gravitational wave astronomy.

In the sixth chapter of this dissertation we apply the development of these tools for exploring the possibility of astroseismology on the binary neutron star GW170817. GW170817 offers the possibility to infer nuclear equations of state that describe the matter within GW170817, but it also permits us to explore the oscillatory modes or interstellar star-quakes within the neutron star interior. Chapter six pursues the question of the presence of nonlinear tides caused by a hypothetical nonresonant, nonlinear, and unstable pairing of oscillation modes within neutron stars. This instability is termed the $p$-$g$ mode instability since involves a nonlinear and nonresonant coupling between pressure and gravity modes within the neutron star interior (CITE). The instability is hypothesized to be capable of drawing energy from the inspiral of neutron stars, thus impacting the gravitational wave signal itself. It was hypothesized in (CITE) that this effect could have a potential for causing gravitational wave detectors to miss greater than $70 \%$ of gravitational wave signals from binary neutron stars. The discovery of GW170817 would permit us to explore this instability and infer whether the parametrized model of the instability could provide a better fit to the data than a standard model that excludes these nonlinear tidal effects. We found that the nonlinear tidal model had a goodness of fit to the data that were non-significant, meaning that the data were not informative towards accepting or rejecting the hypothesis. Furthermore, we found that the nonlinear tidal model produced severe parameter degeneracies in the inferred intrinsic parameters of GW170817. Breaking these parameter degeneracies would require measurements of the parameters of GW170817 that are independent of gravitational wave observation at accuracies that are not currently thought possible. We also found that when we excluded portions of the parameter space from the nonlinear tidal model that had a high match to standard waveforms, that we could effectively rule out certain portions of the $p$-$g$ mode instability parameter space. Bayesian hypothesis testing in this framework does not usually invalidate entire theories, but rather permits us to further our inference on the compatible parameters of the theory (CITE). It is possible that future detections of GW170817 will permit new knowledge about how compatible the $p$-$g$ mode instability is with neutron star astrophysics. The future may also hold new insights into the broader study of astroseismology in neutron stars.

LIGO and Virgo now continue their third observing run which began on April 1, 2019 and has a planned end date of April 30, 2020 (CITE). In the coming years, there will be no shortage of gravitational wave signals to discover and learn from. Due to the incredible work conducted by gravitational wave scientists, past and present, it is in fact, a very good time to do gravitational wave astronomy.
