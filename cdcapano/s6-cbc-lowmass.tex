\documentclass[prd,superscriptaddress,showpacs,amssymb,amsmath,amsfonts,aps,altaffilletter,nofootinbib,letterpaper,twocolumn]{revtex4}

\usepackage{color}
\usepackage{graphicx}
\usepackage{fancyhdr}
\usepackage{float}
\usepackage{ulem}
\usepackage[raggedright]{subfigure}
\usepackage{acronym}
\usepackage{multirow}
\usepackage{array}
%\usepackage[style=altlist]{glossary} 

\normalem

\makeatletter
\makeatletter
\def\@fnsymbol#1{\ifcase#1\or * \or  $+$ \or  \$ \or \#  \or \dag \or \ddag \or
$\mathsection$ \or $ \mathparagraph$ \or $\|$  \or \textordfeminine \or \textbul
let   
\or ** \or $++$ \or  \$\$ \or \#\#  \or \dag\dag \or \ddag\ddag \or
$\mathsection\mathsection$ \or $ \mathparagraph\mathparagraph$ \or $\|\|$  \or 
\textordfeminine\textordfeminine \or \textbullet \textbullet \or *** \or $+++$ 
\or  \$\$\$ \or \#\#  \or \dag\dag \or \ddag\ddag \or
$\mathsection \mathsection\mathsection$ \or $ \mathparagraph 
\mathparagraph\mathparagraph$ \or $\|\|\|$  \or 
\textordfeminine\textordfeminine\textordfeminine \or 
\textbullet\textbullet\textbullet \or \else \@ctrerr\fi}
%\renewcommand{\thefootnote}{\fnsymbol{footnote}}
\makeatother

%% Macros comments


\newcommand\fake[1]{\textcolor{red}{#1}}
\newcommand\assign[1]{\begin{center}\textcolor{red}{#1}\end{center}}

\newcommand\checkme[1]{\textcolor{blue}{\textbf{#1}}}
\newcommand\checked[1]{#1}
\newcommand\marginnote[2]{%
  \mbox{}\marginpar{\raggedleft\hspace{0pt}\scriptsize
    \textcolor{blue}{#1: #2}}}
\newcommand\scapro[2]{< #1 , #2 >}
%% RCS Id macro \rcsid
\def\thercsid{\relax}
\def\rcsid#1{\def\next##1#1{\def\thercsid{##1}}\next}



\rcsid$Id: s6-cbc-lowmass.tex,v 1.28 2011/06/01 20:11:33 cdcapano Exp $



%% Today macro \today
\renewcommand{\today}{\number\day\space\ifcase\month\or
  January\or February\or March\or April\or May\or June\or
  July\or August\or September\or October\or November\or December\fi
  \space\number\year}

\def\Msun{\ensuremath{\mathrm{M_{\odot}}}}
\def\Mtotal{\ensuremath{\mathrm{M}}}

% loudest FARs
\def\firstFAR{\ensuremath{\mathrm{2.2~yr^{-1}}}}
\def\secondFAR{\ensuremath{\mathrm{5.6~yr^{-1}}}}
\def\thirdFAR{\ensuremath{\mathrm{9.4~yr^{-1}}}}
\def\expectedLoudestFAR{\ensuremath{\mathrm{\sim2~yr^{-1}}}}

% livetime numbers
\def\HLltpreVeto{\ensuremath{0.21} yr}
\def\HVltpreVeto{\ensuremath{0.13} yr}
\def\LVltpreVeto{\ensuremath{0.08} yr}
\def\HLVltpreVeto{\ensuremath{0.14} yr}
\def\totalpreVeto{\ensuremath{0.56} yr}
\def\HLlt{\ensuremath{0.17} yr}
\def\HVlt{\ensuremath{0.10} yr}
\def\LVlt{\ensuremath{0.07} yr}
\def\HLVlt{\ensuremath{0.09} yr}
\def\totalTime{\ensuremath{0.43} yr}

% citation shortcuts
\def\12to18{Abbott:2009qj}
\def\sfive1yr{Collaboration:2009tt}
\def\sfivelvc{S5LowMassLV}


% BNS NUMBERS
%%%%%%%%%%%%%%%%%%%%%%%%%%%%%%%%%%%%%%%
\def\BNShd{\ensuremath{50}}
\def\BNSul{\ensuremath{7 \times 10^{-5}}}
\def\LVBNS{\ensuremath{1.7 \times 10^{-4}}}

%%%%%%%%%%%%%%%%%%%%%%%%%%%%%%%%%%%%%%%

% NSBH NUMBERS
%%%%%%%%%%%%%%%%%%%%%%%%%%%%%%%%%%%%%%%
\def\NSBHhd{\ensuremath{80}}
\def\NSBHul{\ensuremath{2 \times 10^{-5}}}
\def\sNSBHul{\ensuremath{3 \times 10^{-5}}}
\def\LVNSBH{\ensuremath{4.4 \times 10^{-5}}}

% BBH NUMBERS
%%%%%%%%%%%%%%%%%%%%%%%%%%%%%%%%%%%%%%%
\def\BBHhd{\ensuremath{90}}
\def\BBHul{\ensuremath{6 \times 10^{-6}}}
\def\sBBHul{\ensuremath{10 \times 10^{-6}}}
\def\LVBBH{\ensuremath{8.8 \times 10^{-6}}}

%%%%%%%%%%%%%%%%%%%%%%%%%%%%%%%%%%%%%%%

%%%%%%%%%%%%%%%%%%%%%%%%%%%%%%%%%%%%%%%

% Dog Numbers
\newcommand\perMpcyr{\ensuremath{\mathrm{Mpc}^{-3} \mathrm{yr}^{-1}}}
\def\dogDate{16 September 2010}
\def\injectedDogTime{06:42:23 UTC}
\def\recoveredDogTime{06:42:23 UTC}
\def\dogTrialFAR{1 in $4 \times 10^{4}$ years}
\def\dogFAR{1 in $7,000$ years}
\def\dogTrialFAP{\ensuremath{3 \times 10^{-6}}}
\def\dogFAP{\ensuremath{7 \times 10^{-5}}}
\def\dogChirp{between $4.4$ and $5.2 \Msun$}
\def\dogEta{$0.16 \le \eta \le 0.25$}
\def\dogDist{between $3$ and $60$ Mpc}
\def\dogSNR{\ensuremath{\rho_c = 12.5}}

\begin{document}

\title{Search for Gravitational Waves from Low Mass Compact Binary Coalescence
in LIGO's Sixth Science Run and Virgo's Science Runs 2 and 3\\ }


\author{The LSC and Virgo (will add full author list)}
%
 \collaboration{The LIGO Scientific Collaboration, http://www.ligo.org}
 \noaffiliation
%
%


%\date[\relax]{ RCS \thercsid; compiled \today }
%\fake{\pacs{95.85.Sz, 04.80.Nn, 07.05.Kf, 97.60.Jd, 97.60.Lf, 97.80.-d}}

\begin{abstract}

We report on a search for gravitational waves from coalescing compact
binaries using LIGO and Virgo observations between July 7, 2009 and
October 20, 2010.  Binaries with total mass between $2$ and $25~\Msun$
were searched for; this includes binary neutron stars, binary black
holes, and binaries consiting of a black hole and neutron star. No
gravitational-wave signals were detected. We report upper limits on the
rate of compact binary coalescence as a function of total mass. The
cumulative 90\%-confidence rate upper limits of the binary coalescence
of binary neutron star, neutron star--black hole and binary black holes
systems are \BNSul, \NSBHul and \BBHul $\mathrm{Mpc^{-3}yr^{-1}}$,
respectively.

\end{abstract}

\maketitle

\acrodef{BBH}{binary black holes}
\acrodef{BNS}{binary neutron stars}
\acrodef{NSBH}{neutron star--black hole binaries}
\acrodef{SNR}{signal-to-noise ratio}
\acrodef{SPA}{stationary-phase approximation}
\acrodef{LIGO}{Laser Interferometer Gravitational-wave Observatory}
\acrodef{LHO}{LIGO Hanford Observatory}
\acrodef{LLO}{LIGO Livingston Observatory}
\acrodef{LSC}{LIGO Scientific Collaboration}
\acrodef{CBC}{compact binary coalescence}
\acrodef{GW}{gravitational wave}
\acrodef{ISCO}{innermost stable circular orbit}
\acrodef{FAR}{false alarm rate}
\acrodef{cFAR}{combined FAR}
\acrodef{IFAR}{inverse false alarm rate}
\acrodef{CL}{confidence level}
\acrodef{pN}{post-Newtonian}
\acrodef{DQ}{data quality}
\acrodef{VSR1}{first Virgo science run}
\acrodef{VSR2}{Virgo's second science run}
\acrodef{VSR3}{its third science run}
\acrodef{S5}{LIGO's fifth science run}
\acrodef{S6}{LIGO's sixth science run}
\acrodef{DQ}{data quality}

%%%%%%%%%%%%%%%%%%%%%%%%%%%%%%%%%%%%%%
\section{Introduction}\label{sec:overview}

During 2009 and 2010, both the \ac{LIGO}~\cite{Abbott:2007kv} and
Virgo~\cite{Acernese:2008b} gravitational wave detectors undertook
science runs with better sensitivity across a broader range of frequencies
than previously achieved.  Among the most promising sources of
gravitational waves for these detectors are those emitted by compact
stellar mass binaries as they spiral in toward each other and eventually
merge.    For such systems, which include \ac{BNS}, \ac{BBH}, and
\ac{NSBH}, the late stages of inspiral and merger occur in the most
sensitive band (between 40 and 1000 Hz) of the \ac{LIGO} and Virgo
detectors.  In this paper, we report on a search for gravitational waves
from binary systems with a maximum total mass of $25\Msun$, and a
minimum component mass of $1\Msun$.  

A hardware injection was performed during the analysis without the knowledge of
the data analysis teams as apart of a ``blind injection challenge." This
challenge was intended to test the data analysis procedures and processes for
evaluating candidate events. The injection was performed by actuating the
mirrors on the \ac{LIGO} and Virgo detectors to mimic a gravitational-wave
signal. 

The blind injection occurred on \dogDate~and was identified by multiple
searches. It was initially observed by a low-latency search for unmodeled
transients.  This search identified the injection as a gravitatioinal-wave
candidate with high significance in both LIGO detectors, chirping upward in
frequency from 40 to 400\,Hz. Virgo was observing with lower sensitivity and
did not show a significant signal. Coherent analyses showed that the lack of a
significant event in Virgo was consistent with the signal seen by the two LIGO
detectors. The observed chirp waveform was consistent with gravitational waves
radiated in the final moments of a coalescing compact binary system. The
detectors were in normal operation at the time of detection and no evidence
that the signal was of instrumental or environmental origin was found. The
event had a false alarm rate of less than \dogFAR in the search
reported here.

After the analysis of the event was finished it was revealed to be a blind
injection and removed from the data. With the injection removed, there were no
gravitataional waves observed above the noise background. As a result we place
upper limits on rates of \ac{CBC}, using upper limits from previous
gravitational-wave searches \cite{\sfivelvc} as prior information. The upper
limits presented here are a factor of two lower than previously
derived limits but still two orders of magnitude above expected \ac{CBC} rates. 

%The search revealed a single gravitational-wave candidate, observed on
%\dogDate\ in coincident data taken from the \ac{LIGO}, Virgo and
%GEO\,600~\cite{Grote:2008} gravitational-wave detectors. The event was
%initially identified by a low-latency search for unmodeled transients.  This
%search identified the event with high significance in both LIGO detectors,
%chirping upward in frequency from 40 to 400\,Hz. Virgo and GEO\,600 were
%observing with lower sensitivity and did not show a significant signal, but
%coherent analyses showed their data to be consistent with the signal seen by
%the two LIGO detectors, and there was some evidence (described below) for a
%signal in Virgo. The observed chirp waveform was consistent with gravitational
%waves radiated in the final moments of a coalescing compact binary system. The
%detectors were in normal operation at the time of detection and no evidence
%that the signal was of instrumental or environmental origin was found. The
%event had a false alarm rate of less than \dogFAR in the search
%reported here.

%This event was later revealed to be a hardware injection performed by
%actuating the mirrors on the \ac{LIGO} and Virgo detectors to mimic a
%gravitational wave signal.  It was performed as part of a ``blind
%injection challenge'' in which a small number (possibly zero) of
%coherent hardware injections were performed without the knowledge of the
%data analysis teams, and revealed only afterward.  The challenge was
%intended to test the data analysis procedures and processes for
%evaluating candidate events.

%With the ``blind injection'' removed from the analysis, there were no
%gravitataional waves observed above the noise background.  As a result
%we place upper limits on rates of \ac{CBC}, using upper limits from
%previous gravitational wave searches \cite{\sfivelvc} as prior
%distributions.  
  
The paper is laid out as follows.  In Section \ref{sec:dets}, we provide
a brief description of the detectors and their sensitivities during
\ac{S6} and Virgo's second and third science runs (VSR2 and vSR3). In
Section \ref{sec:search} we present a brief overview of the analysis
methods used in performing the search.  In Section \ref{sec:results} we
present the results of the search.  We discuss the results both with the
blind injection present, and after it has been removed.  In Section
\ref{sec:ul} we give the upper limits obtained from the search and close
with a brief discussion in Section \ref{sec:discussion}.

%%%%%%%%%%%%%%%%%%%%%%%%%%%%%%%%%%%%%%%
\section{Detectors}
\label{sec:dets}

The \ac{LIGO} observatory comprises two sites, one in Hanford, WA and the
second in Livingston, LA.  The data used in this search were taken during
\ac{S6}, which took place between 7 July 2009 and 20 October 2010.  During
\ac{S6} both of these sites operated a single 4km Laser interferometer (denoted
H1 and L1 respectively).  The 2km H2 instrument at the Hanford site which
operated in earlier science runs was not operational in \ac{S6}.  Following
\ac{S5} \cite{Abbott:2007kv}, several hardware changes were made to the
\ac{LIGO} detectors so that prototypes of advanced LIGO technology could be
installed and tested. This included the installation of a more powerful,
$35~\mathrm{W}$ laser, and the implementation of a DC readout system that
included a new Output Mode Cleaner on an advanced LIGO seismic isolation
table~\cite{Adhikari:2006}. In addition, the hydraulic seismic isolation system
was improved by fine-tuning its feed-forward path.  

The Virgo detector (denoted V1) is a single, 3km Laser Interferometer
located in Cascina, Italy.  The data used in this search were taken from
both \ac{VSR2},  which ran from 7 July 2009 to 11 January 2010, and
\ac{VSR3},  which ran from 11 August 2010 to 20 October 2010.  In the
period between \ac{VSR1} and \ac{VSR2}, several enhancements were made
to the Virgo detector.  Specifically, a more powerful laser was
installed in Virgo, along with a thermal compensation system and
improved scattered light mitigation.  During early 2010, monolithic
suspension was installed, which involved replacing Virgo's test masses
with new mirrors hung from fused-silica fibers.  Following this upgrade
Virgo undertook \ac{VSR3}. 

The sensitivity of the detectors to binary coalescence signals is shown
in Figure \ref{fig:sensitivity}.   This figure shows the distance at
which an optimally oriented and located binary would produce a \ac{SNR}
of 8 in a given detector.  The figure clearly demonstrates an
improvement in sensitivity for the \ac{LIGO} detectors between \ac{S5}
and \ac{S6} and for Virgo between \ac{VSR1} and \ac{VSR2}.  The
reduction in the horizon distance of the Virgo detector in \ac{VSR3} is
due to a mirror with an incorrect radius of curvature being installed
during the conversion to monolithic suspension.

\begin{figure}[ht]
\includegraphics[width=\columnwidth]{Images/sensitivity.pdf}
\caption{
Inspiral horizon distance versus total mass from S5/VSR1 (gray lines)
and S6/VSR2/VSR3 (colored lines). The horizon distance is the distance
at which an optimally located and oriented binary would produce an
expected signal-to-noise ratio of 8.  The figure shows the the best
sensitivity achieved by each detector during the runs.} 
\label{fig:sensitivity}
\end{figure}


%%%%%%%%%%%%%%%%%%%%%%%%%%%%%%%%%%%%%%%
\section{Binary Coalescence Search}
\label{sec:search}

To search for gravitational waves from compact binary
coalescence~\cite{Collaboration:2009tt,Abbott:2009qj, S5LowMassLV}, we
use matched filtering to correlate the detector's strain output with a
theoretical model of the gravitational waveform~\cite{Allen:2005fk}.
Each detector's output is separately correlated against a
bank~\cite{BBCCS:2006} of template waveforms generated at 3.5
post-Newtonian order in the frequency domain~\cite{Blanchet:1995ez,
Blanchet:2004ek}.  Templates were laid out across the mass range such
that no more than $3\%$ of the \ac{SNR} was lost due to the discreteness
of the bank.  In the early stages of the run, as in previous searches
\cite{\sfive1yr,\12to18,\sfivelvc}, the template bank included waveforms
from binaries with a total mass $\leq 35\Msun$.  However, the search
results indicated that the higher mass templates ($M > 25 \Msun$) were
more susceptible to non-stationary noise in the data.  Furthermore, it
is at these higher masses where the merger and ringdown phases of the
signal come into the detector's sensitive band.  Consequently, the upper
mass threshold of the search was reduced to $25 \Msun$ for this search.
Results of a search for higher mass black holes using templates which
incorporate a full coalescence (inspiral-merger-ringdown) template
waveform will be presented in a future publication.  Although the
template waveforms neglect the spin of the binary components, the search
is still capable of detecting binaries whose waveforms are modulated by
the effect of spin~\cite{VanDenBroeck:2009gd}. 

We require signals to have a matched filter \ac{SNR} greater than 5.5
and be coincident in template mass and time-of-arrival (allowing for
travel-time difference) in at least two detectors~\cite{Robinson:2008}.
We use a chi-squared test~\cite{Allen:2004} to suppress non-Gaussian
noise transients, which have a high \ac{SNR} but whose time-frequency
evolution is inconsistent with the template waveform. If the reduced
chi-squared $\chi^2_r$ of a signal is greater than unity, we re-weight
the \ac{SNR} $\rho$ as 
%
\begin{equation}\label{eq:new_snr}
\hat{\rho} = \rho/[(1+(\chi^2_r)^3)/2]^{1/6}
\end{equation}
%
to suppress the significance of false
signals~\cite{Collaboration:S6CBClowmass}. Our analysis reports the time
and the quadrature sum, $\rho_c$, of re-weighted SNRs for events
coincident between the detectors. The statistic $\rho_c$ is then used to
rank events by their significance above the expected background.  To
measure the background rate of coincident events in the search, we
time-shift data from the detectors by an amount greater than the
gravitational-wave travel time difference between detector sites and
re-analyze the data. Many independent time-shifts are performed to
obtain a good estimate of the probability of single-site noise
transients coinciding.  The analysis procedure described above is
similar to the one used in previous searches of \ac{LIGO} and Virgo data ---
such as \cite{\12to18} and \cite{\sfive1yr} --- and is described in greater
detail in \cite{pipeline}.

The background rates of coincident events were initially estimated using
100 time-shifted analyses. These background rates vary depending on the
binary's mass, via the waveform duration and frequency band, and also on
the detectors involved in the coincidence (the {\it coincidence type}).
Thus, we sort coincident events into three bins by chirp mass
$\mathcal{M}$, and also sort by the coincidence type
\cite{Collaboration:2009tt}.  During times when all three detectors were
operational, there are four distinct {\it coincidence types}: H1L1V1,
H1L1, H1V1, and L1V1. In S6/VSR2, all four coincidence types were kept.
In S6/VSR3, H1V1 and L1V1 triggers in triple-coincident time were
discarded due to the heightened rate of transient noise in Virgo and its
decreased sensitivity.  For each candidate, a \ac{FAR} is computed by
comparing its $\rho_c$ value to background events in the same mass bin
and coincidence type. Candidates' \ac{FAR} values are then compared to
background events in {\em all}\/ bins and coincidence types around the
candidate time to calculate a combined \ac{FAR}. This is the detection
statistic which is used to assess the significance of events over the
entire analysis time.

Due to the finite number of slides performed, the
smallest non-zero FAR that can be calculated is $1/T_{\rm bg}$, where
$T_{\rm bg}$ is the total background time obtained by summing the
coincident live time in each time slide. If a trigger was found to be
louder than all background triggers within its analysis period,
additional background estimation was performed to calculate a \ac{FAR}
for the event.

The requirement of a coincident signal between at least two sites
restricts the times that can be analyzed.  Between July 2009 and October
2010, a total of \totalpreVeto~of two or more site coincident data was
collected.  This comprised \HLVltpreVeto~of H1L1V1 coincident data,
\HLltpreVeto~of H1L1 data, \HVltpreVeto~of H1V1 data, and
\LVltpreVeto~of L1V1 data.  Although the detectors are enclosed in
vacuum systems and isolated from vibrational, acoustic and
electromagnetic disturbances, their typical output data contains a
larger number of transient noise events (glitches) with higher amplitude
than expected from Gaussian processes alone.  Each observatory is
equipped with a system of environmental and instrumental monitors that
are sensitive to glitch sources but have a negligible sensitivity to
gravitational waves. These sensors were used to identify times when the
detector output was corrupted~\cite{Christensen:2010}; any coincident
events from these times were vetoed.  We constructed two categories of veto
for use in the search~\cite{s6detchar}.  The primary search, both for
the identification of detection candidates and calculation of upper
limits, was performed with both categories of veto applied.  We also
examined the data after only the most severe category of vetoes had been
applied, in case a loud event occurred during a time that would have
otherwise been vetoed.  Additionally, approximately $10\%$ of the data,
designated {\it playground}, was used for tuning and data quality
investigations.  This data was searched for gravitational waves, but not
used in calculating upper limits.  After all vetoes were applied and
playground time excluded, there was \HLVlt~of H1L1V1 time, \HLlt~of H1L1
time, \HVlt~of H1V1 time, and \LVlt~of L1V1 time, giving a total
analysis time of \totalTime.


% \begin{equation} \label{eq:combined_newsnr}
% \rho_{{\rm new}}^2 = \begin{cases}
% \rho^2, & \chi^2_{r} \leq 1, \\ 
% \frac{\displaystyle \rho^2}{\displaystyle[(1+(\chi^2_{r})^3)/2]^{1/3}}, \ & \chi^2_{r} > 1,  \end{cases}  
% \end{equation}
%where $\rho$ and $\chi^2_{r}$ are respectively the SNR and the signal-based
%$\chi^{2}$ test \cite{Allen:2004} normalized to the number of degrees of
%freedom. This functional form was an incremental improvement on the previously
%used statistic.\cite{LIGOS3S4all}


% One of the most substantial changes from \cite{\12to18} was that data was
% analyzed in two-week blocks to reduce analysis latency and allow for feedback
% to improve \ac{DQ}. Once preliminary results were obtained and \ac{DQ}
% finalized, data was grouped into $\sim6$ week chunks for background estimation
% and calculation of \acp{FAR}.

%%%%%%%%%%%%%%%%%%%%%%%%%%%%%%%%%%%%%%%
\section{Search results}
\label{sec:results}

The \ac{CBC} pipeline determined that a gravitational-wave candidate occurred
on \dogDate~at \recoveredDogTime, with \dogSNR in
coincidence between the two LIGO detectors in the {\it medium mass bin} ($3.48
\le \mathcal{M}/\Msun < 7.40$).  Virgo was also operating at the time of the
event, but its sensitivity was a factor of four lower than the LIGO detectors;
the absence of a signal in Virgo above the single-detector SNR threshold of 5.5
was consistent with this fact.  In the LIGO detectors, the signal was louder
than all time-shifted H1L1 coincident events in the same mass bin throughout
\ac{S6}. However, with only 100 time shifts, we could only bound the FAR to $<
1/23$ years, even when folding in all data from the entire analysis. To obtain
a better estimate of the event's FAR we performed all possible multiples of
5-second time shifts on four calendar months of data around the event,
corresponding to an effective analysis time of $2.0\times 10^5$
years, and considered only events with values of $\rho_{c} \ge 11$.  We found
five events with a value of $\rho_c$ equal to or larger than the
candidate's, as shown in Figure~\ref{fig:far}.  These five events were all
coincidences between the candidate's signal in H1 and time-shifted transient
noise in L1.  When we excluded 8 seconds from around the event's time in the
background estimation, we found {\em no}\/ background events with $\rho_c$
greater than the candidate and we obtained a significantly different background
distribution, also shown in Figure~\ref{fig:far}. 

\begin{figure}[tp]
\includegraphics[width=\columnwidth]{../s6-bigdog/Images/rate_vs_newsnr.png}
\caption{ The cumulative rate of events with chirp mass $3.48 \le
\mathcal{M}/\Msun < 7.40$ coincident in the H1 and L1, seen in four
months of data around the 16 September candidate, as a function of the
threshold ranking statistic $\rho_c$.  The blue triangles show
coincident events.  Black dots show the background estimated from 100
time shifts and black crosses show the extended background estimation
from all possible 5-second shifts on this data.  The gray dots and
crosses show the corresponding background estimates when a few seconds
of data around the time of the candidate are excluded.  Gray shaded
contours show the $1 - 5\sigma$ (dark to light) consistency of
coincident events with the estimated background including the extended
background estimate, for the events and analysis time shown, including
the candidate time.  This event was late revealed to have been a blind
injection. }
\label{fig:far}
\end{figure}


Including the events at the time of the candidate in the background estimate,
the FAR of the event in the $3.48 \le \mathcal{M}/\Msun < 7.40$ mass bin,
coincident in the LIGO detectors, was estimated to be \dogTrialFAR.
Since this event occurred in H1L1V1 time during VSR3, only two coincidence
types were considered: H1L1 double-coincident events and H1L1V1
triple-coincident events.  This resulted in a trials factor of 6 (accounting
for the three mass bins and two coincidence types) and a combined FAR of
\dogFAR.  The probability of a false alarm with this combined FAR
value or lower, in this analysis, was \dogFAP.

\begin{figure*}[htb]
\begin{center}
% 'hot' colorscheme plots
\includegraphics[width=2.42in]{../s6-bigdog/Images/omegagram_h1h.png}
\includegraphics[width=2.09in]{../s6-bigdog/Images/omegagram_l1h.png}
\includegraphics[width=2.09in]{../s6-bigdog/Images/omegagram_v1h.png}
\caption{A time-frequency decomposition of the signal power associated with the
detection event observed in the \ac{LHO} (left), \ac{LLO} (middle), and Virgo
(right) detectors~\cite{ref:omegagrams}.  The zero of time in these plots is
the event time reported above.  The ``normalized tile energy'' is roughly
equivalent to SNR$^2$/2 in a time-frequency tile. For Gaussian noise in the
absence of signal, the normalized tile energy rarely exceeds 8.  The images
from the LIGO detectors both show a significant signal with frequency
increasing over time, characteristic of compact binary coalescence.
This event was later revealed to have been a blind injection.}
\label{fig:omegagrams}
\end{center}
\end{figure*}

\begin{figure}[t]
\includegraphics[width=\columnwidth]{../s6-bigdog/Images/lal_strain_16sample_average.pdf}
\caption{Detector strain amplitude spectrum around the time of the injection:
\ac{LHO} in red, \ac{LLO} in green and Virgo in magenta.  At these noise
levels, an optimally located and oriented (5,5)$\Msun$ binary would give a
matched-filter signal-to-noise ratio (SNR) of 8 at distances of 120, 130 and 30
Mpc in LHO, LLO and Virgo respectively.  The diagonal lines show the strength
of binary coalescence signals observed in the LHO (solid) and LLO (dashed)
detectors with SNRs of 15 and 10, respectively, as explained in the text.
\label{f:noise_curve}}
\end{figure}

Figure~\ref{fig:omegagrams} shows time-frequency plots of the detector data
around the time of the injection. A ``chirp'' waveform was clearly visible in
the H1 and L1 data. This signal was also audible in the whitened detector
strain data. The highest matched-filter SNR obtained in the search was $15$ at
$\mathcal{M} = 4.7\,\Msun$ in H1 and $10$ at $\mathcal{M} = 4.4\,\Msun$ in L1.
This difference in SNRs is consistent with typical differences in antenna
response factors for these differently-oriented detectors.  The quantity
$|2\tilde{s}(f)\sqrt{f}|$, where $\tilde{s}(f)$ is the Fourier transform of the
best-matched filter waveform, can be directly compared with the strain noise
amplitude spectral density in each detector; this is shown in
Figure~\ref{f:noise_curve} to illustrate the strength of the signal seen in the
two LIGO detectors.

We performed an extensive investigation to check for any
non-gravitational-wave causes for the observed event. The injection
passed all checks performed. We performed a thorough study of the
coupling of environmental noise to the gravitational-wave channel. None
of the environmental monitors recorded a transient during the time of
the injection.  The LIGO observatories were in quiet night-time
operation with near maximum astrophysical range; the Virgo detector
had quiet environmental conditions, except for elevated micro-seismic
levels. Most mechanisms that could cause coincident signals among widely
separated detectors were ruled out. It is very unlikely that such mechanisms could
cause a signal that monotonically increases in frequency and is visible
in the LIGO data --- as seen in time-frequency spectrograms shown in
Figure~\ref{fig:omegagrams} --- without leaving a clear signature in the
environmental monitors. There were no significant earthquakes at the
time, and seismic up-conversion through the Barkhausen effect, harmonic
up-conversion or bilinear coupling were all ruled out as potential
causes. Acoustic coupling was also ruled out. No extraordinary
electromagnetic transients were seen at the observatories at the time,
and the worldwide electromagnetic ``weather'' was calm, ruling out, e.g.,
very low-frequency radio, whistlers, chorus and riser
signals~\cite{Singh:2004}.  Observatory cosmic ray detectors showed no
unusual activity.

A loud transient occurred in L1 9 seconds before the coalescence time of the
signal. The transient belonged to a known family of sharp ($\sim$ 10ms) and
loud (SNR $\approx 200$-$80000$) glitches that appear 10-30 times per day in
the output optical sensing system of this detector. Since the candidate signal
swept through the sensitive band of the detector, from 40 Hz to coalescence,
in less than 4 seconds, it did not overlap the loud transient.  Studies,
including re-analysis of the data with the glitch removed, indicated that the
signal was not related to the earlier instrumental glitch.  No evidence was
found that the observed signal was associated with, or corrupted by, any
instrumental effect.  

Following the completion of this analysis, the event was revealed to be
a blind injection.  The recovered parameters of the event identified by
the search were broadly consistent with the injected values, and in
particular the coalescence time and chirp mass reported above were in
good agreement with the hardware injection.

In order to more accurately determine the parameters of the event, we performed
coherent Bayesian analyses of the data using models of both spinning and
non-spinning compact binary objects prior to the unblinding. Parameter
estimates varied significantly depending on the exact model used for the
gravitational waveform, particularly when we included spin effects. However,
conservative unions of the confidence intervals from the different waveform
models were consistent with most injected parameters, including chirp mass,
time of coalescence, and sky location. In addition the signal was correctly
identified as having at least one mis-aligned spin. We will describe the
details of parameter estimation on signals from \acp{CBC} in a future paper.

After the event was revealed to be a blind injection the data containing it was
removed from the analysis. With the injection excluded, there were no
gravitational-wave candidates observed in the data. Indeed the search result
was consistent with the background estimated from time-shifting the data.  The
most significant trigger was an H1L1 trigger in H1L1V1 time with a
combined FAR of \firstFAR. The second and third most significant triggers had
combined FARs of \secondFAR~and \thirdFAR, respectively. All of these triggers
were consistent with background: having analyzed
$\mathrm{\sim0.5~yr}$~of data, we would expect the loudest event to
have a FAR of \expectedLoudestFAR. Although no detection candidates were found,
a detailed investigation of the loudest triggers in each analysis period was
performed, to improve our understanding of instrumental data quality.

\section{Binary Coalescence Rate Limits}
\label{sec:ul}

Given the absence of gravitational-wave signals, we used our observations
to set upper limits on coalescence rates of \ac{BNS}, \ac{BBH}, and
\ac{NSBH} systems.  We used the procedure described in
\cite{Fairhurst:2007qj,loudestGWDAW03,Biswas:2007ni} to compute 90\%
rate upper limits for the various systems, making use of previous
results \cite{\sfive1yr,\12to18,\sfivelvc} as prior information on the
rates.

The rate of binary coalescences in a spiral galaxy is expected to be
proportional to the star formation rate, and hence blue light luminosity, of
the galaxy \cite{LIGOS3S4Galaxies}. There are, however, numerous challenges to
performing this calculation accurately, not least due to the large
uncertainties in both the luminosity of and distance to nearby galaxies, as well
as the lack of a complete galaxy catalogue at larger distances.  On large
scales (greater than $\sim 20\mathrm{Mpc}$), the luminosity per unit volume is
approximately constant (see Figure 6 of \cite{LIGOS3S4Galaxies}); consequently
the analysis can be simplified by reporting upper limits per unit volume per
unit time.  During the current analysis, the sensitivity of the detectors to
the systems of interest (as shown in Figure \ref{fig:sensitivity}) was
sufficiently large that we could assume signals were uniformly distributed in
volume. We therefore quote upper limits in units of $\mathrm{Mpc^{-3}yr^{-1}}$.

Previous searches \cite{\sfive1yr,\12to18,\sfivelvc} presented upper limits in
terms of blue light luminosity, using units of $\mathrm{L_{10}^{-1}yr^{-1}}$,
where one $\mathrm{L_{10}}$ is $10^{10}$ times the solar blue light luminosity.  To
incorporate the previous results as prior distributions, we converted from
$\mathrm{L_{10}}$ to $\mathrm{Mpc^3}$ using a conversion factor of $1 \,
\mathrm{L_{10}} = 0.02 \, \mathrm{Mpc^3}$ \cite{LIGOS3S4Galaxies}.

The volume to which the search is sensitive was determined by reanalyzing the
data with signals injected into it that simulated the source population. Our
ability to detect a signal depends upon the parameters of the source, including
the component masses, the distance to the binary, its sky location, and its
orientation with respect to the detectors. Numerous signals with randomly
chosen parameters were therefore injected into the data. To compute the
sensitive volume for a given binary mass, we performed a Monte Carlo
integration over the other parameters to obtain the efficiency of the search
--- determined by the fraction of simulated signals found louder than the
loudest foreground event --- as a function of distance. Integrating the
efficiency as a function of distance gave the sensitive volume.

There are numerous systematic uncertainties, such as detector
calibration errors, waveform errors, and Monte Carlo errors
\cite{\sfive1yr}, that limit the accuracy of the measured search volume
and therefore the upper limits. We propagate these uncertainties into an
uncertainty in the search volume by a bootstrapping technique.  We then
marginalize over the uncertainty in volume to obtain an upper limit
which takes into account these systematic uncertainties.

%\begin{figure}[ht]
%\includegraphics[width=\columnwidth]{Images/BNS_rate_posterior.png}
%\caption{The probability distribution of coalescence rate of binary
%neutron star systems. The blue line shows the rate posterior from
%\ac{S5}-\ac{VSR1}, which is used as a prior to this search.  The red
%line shows the rate posterior from \ac{S6}-VSR2/3. Vertical lines
%indicate the upper limit at the $90\%$ confidence level for the
%distributions shown.}
%\label{fig:bns_posterior}
%\end{figure}


%In Figure \ref{fig:bns_posterior}, we show the posterior rate
%distribution for \ac{BNS} systems, assuming non-spinning components and
%a canonical mass distributions for non-spinning \ac{BNS} ($m_1 = m_2 =
%1.35 \pm 0.04 \Msun$), using the \ac{S5}-\ac{VSR1} results as a prior.
%The distribution is peaked at zero, as expected since observations were
%consistent with the estimated background.  The corresponding posterior
%from \cite{\sfivelvc} is also shown; the improvement over
%\ac{S5}-\ac{VSR1} is evident.  Even though the \ac{S5} run lasted longer
%than \ac{S6} and VSR2/3, the improved sensitivity of the current runs
%more than compensates for the shorter duration.  The vertical lines show
%the 90\% upper limit values, and demonstrate an improvement of
%approximately a factor of two over previous results. 

In Table \ref{tab:ul} we present the marginalized upper limits at the $90\%$
confidence level assuming canonical mass distributions for non-spinning
\ac{BNS} ($m_1 = m_2 = 1.35 \pm 0.04 \Msun$), \ac{BBH} ($m_1 = m_2 = 5 \pm 1
\Msun$), and \ac{NSBH} ($m_1 = 1.35 \pm 0.04 \Msun$, $m_2 = 5 \pm 1 \Msun$)
systems. We also compute upper limits as a function of total mass using a
uniformly distributed mass ratio and, for \ac{NSBH} systems, as a function of
the black hole mass, keeping the neutron-star mass fixed in the range
$1-3\Msun$. These are presented in Figure \ref{fig:ULplots}. Figure
\ref{fig:rate_comp} compares the upper limits presented in this analysis
(shaded gray regions) to limits obtained in S5-VSR1 (hatched gray region) and
to astrophysically-predicted rates (blue regions) for \ac{BNS}, \ac{NSBH}, and
\ac{BBH} systems. The improvement over S5-VSR1 is about a factor of
two.


\begin{table}
\center
\begin{tabular}{p{4cm}  c  c  c}
\hline \hline
\raggedright{System}                                                    &   BNS       &    NSBH    &    BBH      \\[2pt]
\hline                                                                                             
\raggedright{Component masses ($\Msun$)}                                & 1.35 / 1.35 & 1.35 / 5.0 & 5.0 / 5.0   \\[2pt]
\raggedright{$D_{\rm horizon}$ ($\mathrm{Mpc}$)}                        &   \BNShd    &   \NSBHhd  &   \BBHhd    \\[2pt]
\raggedright{Non-spinning upper limit ($\mathrm{Mpc^{-3} yr^{-1}}$)}    &   \BNSul    &   \NSBHul  &   \BBHul    \\[2pt]
\raggedright{Spinning upper limit ($\mathrm{Mpc^{-3} yr^{-1}}$)}        &    ...      &   \sNSBHul &   \sBBHul   \\[2pt]
\hline
\hline
\end{tabular}
\caption{Rate upper limits of \ac{BNS}, \ac{BBH} and \ac{NSBH} coalescence,
assuming canonical mass distributions.  $D_{\rm horizon}$ is the horizon
distance averaged over the time of the search.  The first set of upper limits
are those obtained for binaries with non-spinning components.  The second set
of upper limits are produced using black holes with a spin uniformly
distributed between zero and the maximal value of $Gm^{2}/c$.}
\label{tab:ul}
\end{table}


\begin{figure}[htb]
\begin{center}
\includegraphics[width=\columnwidth]{Images/ULvTotalMass.png}
\includegraphics[width=\columnwidth]{Images/ULvComponentMass.png}
\caption{The marginalized upper limits as a function of mass. The top plot
shows the limit as a function of total mass, using a uniformly distributed
mass-ratio. The lower plot shows the limit as a function of the black hole
mass, keeping the neutron start mass fixed in the range $1-3\Msun$. The
light bars indicate upper limits from previous searches. The dark bars
indicate the combined upper limits including the results of this search.}
\label{fig:ULplots}
\end{center}
\end{figure}

\begin{figure}[ht]
\includegraphics[width=\columnwidth]{Images/rate_comparisons.pdf}
\caption{Comparison of CBC upper limit rates for \ac{BNS}, \ac{NSBH} and \ac{BBH}
systems.  The hatched gray regions display the upper limits obtained in the
S5-VSR1 analysis; non-hatched gray regions show the upper limits obtained in
this analysis, using the S5-VSR1 limits as priors.  The new limits represent a
factor of $\sim 2$ improvement over the previous results.  The blue
regions show the spread in the astrophysically predicted rates, with the
dashed-black lines showing the best estimate.  \cite{ratesdoc} {\it Note}: To
be consistent with \cite{ratesdoc}, the component masses used for the \ac{NSBH}
and \ac{BBH} rates shown here are ($m_1 = 1.35\Msun,~m_2 = 10\Msun$) and ($m_1
= m_2 = 10\Msun$), respectively.}
\label{fig:rate_comp} 
\end{figure}


Although we searched with a bank of non-spinning templates, we also compute
upper limits for \ac{NSBH} and \ac{BBH} systems in which one or both of the
component masses are spinning. These results are also presented in Table
\ref{tab:ul}. We did not compute upper limits for spinning \ac{BNS} systems
because astrophysical observations indicate that neutron stars cannot have
large enough spin to significantly affect waveforms observable in the \ac{LIGO}
frequency band. \cite{ATNF:psrcat,Apostolatos:1994} Black hole spins were
uniformly distributed in both orientation and magnitude, $S$, with $S$
constrained to the range $0 \leq S \leq Gm^2/c$. As can be seen in Table
\ref{tab:ul}, the spinning upper limits are $\sim20\%$ larger than
non-spinning. This is due to poorer recovery efficiency inherent in using a
non-spinning template bank.

While the rates presented here represent an improvement over the
previously published results from earlier \ac{LIGO} and Virgo science
runs, they are still above the astrophysically predicted rates
of binary coalescence.  There are numerous uncertainties involved in
estimating astrophysical rates, including limited numbers of
observations and unknown model parameters; consequently the rate
estimates are rather uncertain.  For \ac{BNS} the estimated rates vary
between $1 \times 10^{-8}$ and $1 \times 10^{-5}$ Mpc$^{-3}$ Myr$^{-1}$,
with a best estimate of $1 \times 10^{-6}$ Mpc$^{-3}$ Myr$^{-1}$.  For
\ac{BBH} and \ac{NSBH}, a realistic estimates of the rate are
$5 \times 10^{-9}$ Mpc$^{-3}$ Myr$^{-1}$ and 
$3 \times 10^{-8}$ Mpc$^{-3}$ Myr$^{-1}$ with at least an order of
magnitude uncertainty in either direction \cite{ratesdoc}.  In all cases, the upper
limits derived here are around two orders of magnitude above the ``best
estimate'' rates, and about a factor of ten above the most optimistic
predictions.  These results are summarized in Figure
\ref{fig:rate_comp}.

%%%%%%%%%%%%%%%%%%%%%%%%%%%%%%%%%%%%%%%
\section{Discussion}
\label{sec:discussion}


We performed a search for gravitational waves from compact binary
coalescences with total mass between $2$ and $25\Msun$ with the
\ac{LIGO} and Virgo detectors using data taken between July 7, 2009 and
October 20, 2010.  No gravitational waves candidates were detected, so
we placed new upper limits on \ac{CBC} rates.  These new limits give a
factor of two improvement over those achieved using the
\ac{S5} and \ac{VSR1} data \cite{\sfivelvc}, but remain an order of
magnitude or more above the astrophysically predicted rates.  The
installation of the advanced \ac{LIGO} and Virgo detectors has begun.
When operational, these detectors will provide a factor of ten increase
in sensitivity over the initial detectors, providing a factor of
$\sim1000$ increase in the sensitive volume.  At this time, we expect to
observe tens of binary coalescences per year. \cite{ratesdoc}

In order to detect this population of gravitational wave signals, we
will have to be able to confidently discriminate it from backgrounds
caused by both stationary and transient detector noise.  It is customary
\cite{ratesdoc} to assume that a signal with \ac{SNR} 8 would stand far
enough above background that we would consider it to be a detection
candidate.  The blind injection had somewhat larger \ac{SNR} than 8 in
each detector, and we were able estimate a \ac{FAR} of 1 in 7000 years
for that event.  Alternatively, consider a coincident signal with
\ac{SNR} 8 in two detectors.  Provided the signal is a good match to the
template waveform ($\chi^{2}_{r} \approx 1$ in equation
\ref{eq:new_snr}) this corresponds to $\rho_c = 11.3$. As can be seen
from the extended background triggers with the blind injection removed
in Figure \ref{fig:far} (light-gray crosses), this gives a \ac{FAR} of
$\sim 1$ in $2 \times 10^{4}$ years in a single trial, or $1$ in 3000
years over all trials.  Achieving similar-or-better background
distributions in Advanced LIGO and Virgo will require detailed \ac{DQ}
studies of the detectors and feedback from the \ac{CBC} pipeline, along
with well-tuned signal-based vetoes. We have continued to develop the
pipeline with these goals in mind. For this analysis we significantly
decreased the latency between taking data and producing results, which
allowed \ac{DQ} vetoes to be finely tuned for the \ac{CBC} search. These
successes, along with the successful recovery of the blind injection,
give us confidence that we will be able to detect gravitational waves
from \ac{CBC}s at the expected rates in Advanced LIGO and Virgo.


%%%%%%%%%%%%%%%%
\bibliography{../bibtex/iulpapers}

\end{document}
