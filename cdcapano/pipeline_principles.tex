Using General Relativity we were able to write down an expression for the strain induced in an interferometer from a passing gravitational wave. In this chapter we detail how to detect gravitational-wave signals from \acp{CBC} in the interferometer data in the presence of noise. We begin by deriving the matched filter for a single template, which is the optimal filter is the detector noise is stationary and Gaussian. Next, we show how matched filtering can be done for a bank of templates in order to recover the parameters of a signal. This is followed by a discussion of the $\chi^2$ test, which can be used to suppress triggers from non-Gaussian transients that are present in real detector data. Finally, we detail how a coincidence test can be done across multiple detectors to further decrease the number of false noise triggers produced by a search.

\section{Detecting a Gravitational Wave Using a Matched Filter}
\label{sec:matched_filter}

Equation \ref{eqn:non_spin_template} gave the strain, $h(t)$, induced in an \ac{IFO} when a gravitational-wave from an inspiraling compact binary with non-spinning component masses passes through. The strain was a function of chirp mass, $\mchirp$, and symmetric-mass ratio, $\eta$. For varying pairs of $\mchirp$ and $\eta$ we can generate a waveform, or template, that characterizes the \ac{GW}. However, in this section we will assume that $\mchirp$ and $\eta$ of our template is the same as the gravitational wave we are looking for. We now address the question: if a gravitational wave exists in the \ac{IFO} data, how do we find it?

Let the output of the detector be the time series $s(t)$, which is a measure of the displacement of the \ac{IFO}'s mirrors as a function of time. If a gravitational-wave with waveform $h(t)$ exists in the data, then:

\begin{equation}
\label{eqn:ifo_data}
s(t) = h(t) + n(t)
\end{equation}
where $n(t)$ is the strain induced by all other, non gravitational-wave,  sources, or ``noise," as a function of time. If no signal exists in the data, then $s(t) = n(t)$. We do not know \emph{a priori} if $h$ exists in the data; further, even if $h$ does exist, we do not know \emph{when} it occurs --- that is, we do not know its coalescence time.\footnote{Note that the exact time when $h$ occurs is somewhat arbitrary. We could choose any point in the evolution of the waveform and label that as the point that $h$ occurs. For \ac{CBC} templates, we chose to use the time of coleascence since it is easily identifiable: it is the point that the \ac{PN} approximation goes to infinite frequency.} Our goal, then, is to find a filter that takes $s$ as input and returns a time series that is proportional to the probability that $h$ is in $s$, $P(h|s)$. We additionally require that, if $h$ is in $s$, this filter is maximized at the point that $h$ occurs. The problem of finding an optimal filter to extract signals from (Gaussian) noise is well studied, and has been applied to radar analysis for decades. Here, we follow the method outlined in \cite{ref:Finn,ref:Finn_Chernoff,ref:Brown} --- all of which use methods detailed in Wanstein and Zubakov \cite{ref:Wanstein_Zubakov} --- to derive the filter.

Using Bayes' theorem \cite{ref:Sivia} we can write $P(h|s)$ as:
\begin{equation}
\label{eqn:P_of_h1}
P(h|s) = \frac{P(s|h) P(h)}{P(s)}
\end{equation}
where:
\begin{eqnarray}
P(s|h) & \equiv & \mbox{the probability of getting $s$ if $h$ exists in it} \nonumber \\
P(h)   & \equiv & \mbox{the probability of the signal, $h$, occurring} \nonumber \\
P(s)   & \equiv & \mbox{the probability of getting $s$} \nonumber
\end{eqnarray}
Since the signal either does or does not exist in the data, the probability of getting a particular instance of $s$ is:
\begin{equation}
P(s) = P(s|0)P(0) + P(s|h)P(h)
\end{equation}
where:
\begin{eqnarray}
P(s|0) & \equiv & \mbox{the probability of getting $s$ if no signal exists} \nonumber \\
P(0) & \equiv & \mbox{the probability of getting no signal} \nonumber
\end{eqnarray}
Plugging this into \ref{eqn:P_of_h1} we have:
\begin{eqnarray}
\label{eqn:P_of_h}
P(s|h) & = & \frac{ P(s|h) P(h) }{ P(s|0)P(0) + P(s|h)P(h) } \nonumber \\
 & = & \frac{ P(s|h) } { P(s|0) \left(\, P(0)/P(h) + P(s|h)/P(s|0) \,\right) } \nonumber \\
 & = & \frac{ \Lambda }{ P(0)/P(h) + \Lambda }
\end{eqnarray}
where
\begin{equation}
\label{eqn:likelihood_ration}
\Lambda \equiv \frac{ P(s|h) }{ P(s|0) }
\end{equation}
is the \emph{likelihood ratio}. $P(0)$ and $P(h)$ are known as \emph{priors}: they represent our \emph{a priori} belief that a signal does or does not exist, irrespective of the detector's ability to detect it. We do not need to concern oursevles with assigning values to them, however. Instead we note that $P(s|h)$ is a monotonically increasing function of $\Lambda$. Since we are only interested in a filter that maximizes $P(s|h)$, and not the exact value of $P(s|h)$, we can therefor limit our focus to evaluating $\Lambda$, and threshold on the point that it reaches a maximum.

To calculate the likelihood ratio, we need the \acp{PDF} $P(s|h)$ and $P(s|0)$. We focus first on $P(s|0)$. Let us assume that the noise is a stationary Gaussian process with zero mean. For our purposes it will be useful to characterize the noise by its \ac{PSD}, $S_n(|f|)$, instead of the variance. The \ac{PSD} is defined as the Fourier Transform of the autocorrelation of the noise \cite{ref:Wainstein_Zubakov}:
\begin{equation}
S_n(|f|) \equiv \int_{-\infty}^{\infty} R(\tau)e^{-i 2\pi ft} \d t
\end{equation}
where,
\begin{equation}
R(\tau) = \overline{n(t)n(t-\tau)}
\end{equation}
is the autocorrelation function. Note that $R(0) = \overline{n(t)^2}$, which is the variance since $\overline{n(t)} = 0$. We write $S_n(|f|)$ instead of $S_n(f)$ because the frequency must be positive for real noise; thus $S_n(|f|)$ is a one-sided \ac{PSD}. It is shown in \cite{ref:Finn} that with these assumptions, the probability of getting $s$ without a signal is:
\begin{equation}
\label{eqn:p_no_sig}
P(s|0) = \alpha \exp[ - \frac{1}{2} (s|s) ]
\end{equation}
where $\alpha$ is an arbitrary constant and the inner-product $(\cdot|\cdot)$ is defined as:
\begin{eqnarray}
\label{eqn:inner_product1}
(a|b) & \equiv & \int_{-\infty}^{\infty} \frac{ \widetilde{a}^{*}(f)\widetilde{b}(f) + \widetilde{a}(f)\widetilde{b}^{*}(f) }{S_n(|f|)} \d f \\
\label{eqn:inner_product2}
 & = & 2 \int_{-\infty}^{\infty} \frac{ \widetilde{a}^{*}(f)\widetilde{b}(f) }{S_n(|f|)} \d f
\end{eqnarray}
In going from \ref{eqn:inner_product1} to \ref{eqn:inner_product2} we have used the fact that for real functions of time (which both $h$ and $s$ are), $\widetilde{g}^{*}(f) = \widetilde{g}(-f)$.

Using this result we can also find $P(s|h)$. If $h$ is in the data, then $n(t) = s(t) - h(t)$. Thus,
\begin{eqnarray}
\label{eqn:p_sig}
P(s|h) & = & \alpha \exp\{ -\frac{1}{2} (s - h | s - h ) \} \nonumber \\
 & = & \alpha \exp\{ -\frac{1}{2} [ (s|s) - 2(s|h) + (h|h) ] \} \nonumber \\
 & = & P(s|0) \exp\{ (s|h) - (h|h)/2 \}
\end{eqnarray}
Plugging \ref{eqn:p_no_sig} and \ref{eqn:p_sig} into \ref{eqn:likelihood_ratio}, the log-likelihood becomes:
\begin{equation}
\ln(\Lambda) = 2 \frac{(s|h)}{(h|h)}
\end{equation}

