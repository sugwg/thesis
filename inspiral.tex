% $Id$

\section{Gravitational Radiation}

Linarized GR, review of gravitational waves.

\section{Generation Of Gravitational Waves By Binary Inspiral}

\subsection{Quadrupole Emission of Gravitational Waves}

Peter's and Matthew's calculation

\subsection{Post-Newtonian Corrections to the Waveform}

Higher order corrections in $v/c$.

\subsection{The Stationary Phase Inspiral Waveform}

To detect the waveforms decribed above we use matched filtering in the
frequency domain. The Fourier transform of chirp signal $h(t)$, as defined in
the introduction, is
\begin{equation}
\tilde{h}(f) = \int_{-\infty}^{\infty} h(t) e^{-2\pi i f t}.
\end{equation}
There are two ways of obtaining $\tilde{h}(f)$. The first is to compute $h(t)$
and then take the Fourier transform. This is compulationally expensive,
however, as it requires an additional FFT for each template that we wish to
filter. The second method is to use the stationary phase
approximation\cite{xxx} to express the chirp directly in the frequency
domain\cite{xxx}. Given a function
\begin{equation}
B(t) = A(t) \cos \phi(t)
\end{equation}
where
\begin{equation}
\frac{d\ln A}{dt} \ll \frac{d\phi}{dt}
\end{equation}
and
\begin{equation}
\frac{d^2\ln A}{dt^2} \ll \left(\frac{d\phi}{dt}\right)^2
\end{equation}
then the stationary phase approximation to the Fourier transform of $B(t)$ is
given by
\begin{equation}
\tilde{B}(f) = \frac{1}{2} A(t) \left(\frac{df}{dt}\right)^{-\frac{1}{2}}
\exp\left[ -i \left(2\pi f t - \phi(f) - \frac{\pi}{4} \right)\right]
\end{equation}
where $t$ is the time at which
\begin{equation}
\frac{d\phi(t)}{dt} = 2\pi f
\end{equation}
and
\begin{equation}
\phi(f) = \phi\left[t(f)\right].
\end{equation}

According to the Newtonian quadrupole formula, the orbital frequency at any
instant is given by
\begin{equation}
\Omega = \frac{M^{\frac{1}{2}}}{r^{\frac{3}{2}}}.
\end{equation}
The inspiral rate for circular orbits is given by
\begin{equation}
\frac{dr}{dt} = - \frac{r}{E} \frac{dE}{dt} = 
- \frac{64}{5} \frac{\mu M^2}{r^3}.
\end{equation}
Thus
\begin{align}
r^3 \, dr &= - \frac{64}{5} \mu M^2 \, dt \\
\frac{r^4}{4} &= \frac{64}{5} \mu M^2 (t - t_c)
\end{align}
and so
\begin{equation}
r(t) = \left(\frac{256}{5} \mu M^2 \right)^{\frac{1}{4}}
       \left(t_c - t\right)^{\frac{1}{4}}.
\end{equation}
Since the emitted radiation is quadrupolar
\begin{equation}
f = \frac{\Omega}{\pi}.
\end{equation}
The gravitational wave strain is
\begin{equation}
h(t) = \frac{Q}{D} \left(\frac{384}{5}\right)^\frac{1}{2} \pi^\frac{2}{3} \mu M
r^{-1}(t) \cos \left( \int 2\pi f(t) \, dt \right).
\end{equation}
Now
\begin{align}
\frac{df}{dt} 
     &= \frac{d}{dt} \left(\frac{\Omega}{\pi}\right) 
      = \frac{d}{dt} \left(\frac{M^\frac{1}{2}}{\pi} r^{-\frac{3}{2}}\right) \\
     &= \frac{M^\frac{1}{2}}{\pi} \frac{dr}{dt} \left(-\frac{3}{2} r^{-frac{5}{2}}\right) \\
     &= \frac{M^\frac{1}{2}}{\pi} \left(-\frac{64}{5} \frac{\mu M^2}{r^3}\right)
        \left(-\frac{3}{2} r^{-frac{5}{2}}\right) \\
     &= \frac{96}{5} \frac{M^\frac{5}{2} \mu}{\pi} r^{-\frac{11}{2}}. 
     \label{eq:sp:dfdt}
\end{align}
Now
\begin{equation}
f = \frac{M^\frac{1}{2}}{\pi} r^{-\frac{3}{2}}.
\end{equation}
Solving this for $r$, we obtain
\begin{equation}
r = \frac{M^\frac{1}{3}}{\pi^\frac{2}{3} f^\frac{2}{3}}
\end{equation}
which we can substitute into equation (\ref{eq:sp:dfdt}) and obtain
\begin{align}
\frac{df}{dt} &= \frac{96}{5} \pi^\frac{8}{3} \mu M^\frac{2}{3} f^\frac{11}{3} \\
&= \frac{96}{5} \pi^\frac{8}{3} \mathcal{M}^\frac{5}{3} f^\frac{11}{3}
\end{align}
where we have defined the \emph{chirp mass} by
\begin{equation}
\mathcal{M} = \mu^\frac{3}{5} M^\frac{2}{5}.
\end{equation}
Now
\begin{equation}
\phi(t) = \int^t 2\pi f(t') \, dt'
\end{equation}
we won't worry about this yet.

Therefore using the stationary phase approximation, we obtain
\begin{align}
\tilde{h}(f) &= \frac{1}{2} \frac{Q}{D} \left(\frac{384}{5}\right)^\frac{1}{2} \pi^{2}{3} \mu M
                r^{-1}(t) \left(\frac{df}{dt}\right)^{-\frac{1}{2}} \exp\left[-i \Psi\right] \\
             &= \frac{1}{2} \frac{Q}{D} \left(\frac{384}{5}\right)^\frac{1}{2} \pi^{2}{3} \mu M
                \frac{\pi^\frac{2}{3} f^\frac{2}{3}}{M^\frac{1}{3}}
                \left(\frac{df}{dt}\right)^{-\frac{1}{2}} \exp\left[-i \Psi\right] \\
             &= \frac{1}{2} \frac{Q}{D} \left(\frac{384}{5}\right)^\frac{1}{2} \pi^{4}{3} \mu
                M^\frac{2}{3} r^\frac{2}{3}
                \left(\frac{df}{dt}\right)^{-\frac{1}{2}} \exp\left[-i \Psi\right].
\end{align}
Now
\begin{equation}
\left(\frac{df}{dt}\right)^{-\frac{1}{2}} =
\left(\frac{5}{96}\right)^\frac{1}{2} \pi^{-\frac{4}{3}} 
\mu^{-\frac{1}{2}} M^{-\frac{1}{3}} f^{-\frac{11}{6}}
\end{equation}
and so 
\begin{align}
\tilde{h}(f) &= \frac{1}{2} \frac{Q}{D} \left(\frac{384}{5} \frac{5}{96}\right)^\frac{1}{2}
                \mu^\frac{1}{2} M^\frac{1}{3} f^{-\frac{7}{6}} \exp\left[-i \psi\right] \\
             &= \frac{Q}{D} \mathcal{M}^\frac{5}{6} f^{-\frac{7}{6}} \exp\left[-i \psi\right].
             \label{eq:sp:template}
\end{align}
Equation (\ref{eq:sp:template}) is the stationary phase binary inspiral template that we will use in
the matched filter.

\section{Interferometric Gravitational Wave Detectors}


%
\section{Previous Searches for Compact Binary Inspiral}

40m, TAMA, S1. Plot of rate vs time.
