% $Id$

\section{Gravitational Radiation}

Linarized GR, review of gravitational waves.

\section{Waveforms from inspiraling compact binaries}
\label{s:waveforms}

\subsection{Quadrupole Emission of Gravitational Waves}

Peter's and Matthew's calculation

\subsection{Post-Newtonian Corrections to the Waveform}



Although the binary systems are expected to be highly eccentric and widely
separated when they form\cite{bkb}, the emission of gravitational radiation at
twice the orbital frequency of the binary system causes the orbit to
circularize much faster than it shrinks\cite{peters}. By the time the
frequency of the gravitational radiation enters the sensitive band of the 
detector the orbit is nearly circular, so we neglect eccentricity in the
template waveforms. The effect of spin on the detectability of the waveforms
is not expected to be a significant for low mass binaries\cite{apos} and we
neglect it in the templates used here.

We use restricted post-Newtonian (PN) waveforms to construct the filters. The
amplitude evolution is computed from the quadrupole-formula\cite{petersmath}
and the phase evolution is computed to second PN order ($\mathcal{O}(v/c)^4$
where $v$ is the orbital velocity).  For a binary system the gravitational 
wave strain is
\begin{equation}
h(t) = \frac{1\,\mathrm{Mpc}}{\mathcal{D}}\left(F_{+}h_{+}(t) +
F_{\times}h_{\times}(t)\right),
\end{equation}
where $\mathcal{D}$ is the actual distance to the chirp in Mpc. $h_{+}(t)$
and $h_{\times}(t)$ are the two polarizations of the gravitational wave signal
and $F_{+}$ and $F_{\times}$ are the detector beam pattern functions. The
``$+$'' (plus) polarization is given by
\begin{equation}
h_{+}(t) = -\frac{1 + \cos^2 i}{2}\,h_c(t)
\end{equation}
and the ``$\times$'' (cross) polarization is given by
\begin{equation}
h_{\times}(t) = -\cos i\,h_s(t).
\end{equation}
where $i$ is the inclination angle (radians) of the angular momentum axis
relative to the line-of-sight. The waveforms $h_c(t)$ and $h_s(t)$ are referred
to as the cosine and sine polarizations of the inspiral signal~\cite{biww}.

The template waveforms $h_c(t)$ and $h_s(t)$ are computed for a binary
system that is {\it optimally oriented}. An optimally oriented binary system
lies on the $z$-axis with its orbital plane parallel to the $x$-$y$ plane,
which is defined by the arms of the interferometer. The detector has a
non-uniform response over the sky due to its quadrupolar antenna pattern,
given by $F_{+,\times}$. If the source is not optimally oriented (i.e., not on the
$z$-axis or the orbital plane is tipped), then the effective distance to the
binary will be greater than the source-detector distance. Since the search
records the effective distance to a candidate signal, when we discuss
distances to candidate events found by the search code we always quote
effective distance, that is, distance to an optimally oriented event.

Higher order corrections in $v/c$.

\subsection{The Stationary Phase Inspiral Waveform}

Since the matched filter is performed in the frequency domain, we use the
stationary phase approximation to the Fourier transform of the 2~PN
waveform\cite{poissonwill}. The stationary phase approximations to the Fourier
transforms of $h_c(t)$ and $h_s(t)$ are given by


To detect the waveforms decribed above we use matched filtering in the
frequency domain. The Fourier transform of chirp signal $h(t)$, as defined in
the introduction, is
\begin{equation}
\tilde{h}(f) = \int_{-\infty}^{\infty} h(t) e^{-2\pi i f t}.
\end{equation}


The 2~PN waveform is a sweeping sinusoid which is computed from a specified 
low frequency cutoff which is chosen based on the characteristic noise in the
detector. It sweeps through the sensitive band of the detector with increasing
frequency and amplitude.  The upper frequency cutoff of the waveform is
determined by the frequency at which the post-Newtonian approximation breaks
down and we no longer trust the waveforms. For post-Newtonian chirps computed
in the time domain, the frequency at which the frequency evolution becomes
non-monotonic is a good indication of this.  In the stationary phase
approximation there is no equivalent measure of the breakdown.  Therefore to
set a high frequency cutoff for the inspiral waveforms, we choose the
frequency of the Schwartzchild innermost stable orbit (ISCO). This can be
computed exactly and is given by
\begin{equation}
f_{\mathrm{max}} = \left( 6M\pi\sqrt{6}T_\odot \right)^{-1}\,\mathrm{Hz}
\label{eq:isco}
\end{equation} where $M$ is the total mass of the binary system in solar
masses. Since the ISCO frequency is currently not known for a pair of
comparably massive objects, it is something of a kludge to use this, however
we find in practice that this yields a maximum frequency which is close to that
given by the frequency at which the time domain waveforms are cut off.  The
upper frequency cutoff for the waveform of a $2\times 1.4 \> M_\odot$ binary
system is thus $1570$~Hz.

\subsection{The Stationary Phase Waveforms}
\label{ss:stationaryphase}

The inspiral waveforms that we need in the matched filter are $\tilde{h}_c(f)$
and $\tilde{h}_s(f)$, rather than the waveforms $h_c(t)$ and $h_t(t)$ computed
in chapter \ref{ch:inspiral}. There are two ways of obtaining $\tilde{h_c}(f)$.
The first is to compute $h_c(t)$ then Fourier transform the signal. This is
compulationally expensive, however, as it requires an additional FFT for each
template that we wish to filter. The second method is to use the stationary phase
approximation\cite{Mathews:1992} to express the chirp directly in the frequency
domain\cite{WillWiseman:1996,Cutler:1994}. Given a function
\begin{equation}
B(t) = A(t) \cos \phi(t)
\end{equation}
where
\begin{equation}
\frac{d\ln A}{dt} \ll \frac{d\phi}{dt}
\end{equation}
and
\begin{equation}
\frac{d^2\ln A}{dt^2} \ll \left(\frac{d\phi}{dt}\right)^2
\end{equation}
then the stationary phase approximation to the Fourier transform of $B(t)$ is
given by
\begin{equation}
\tilde{B}(f) = \frac{1}{2} A(t) \left(\frac{df}{dt}\right)^{-\frac{1}{2}}
\exp\left[ -i \left(2\pi f t' - \phi(f) - \frac{\pi}{4} \right)\right]
\end{equation}
where $t'$ is the time at which
\begin{equation}
\frac{d\phi(t)}{dt} = 2\pi f
\end{equation}
and
\begin{equation}
\phi(f) = \phi\left[t(f)\right].
\end{equation}

For the restricted post$^2$-Newtonian chirps discussed in chapter
\ref{ch:inspiral}, the orbital frequency at any instant is given by
\begin{equation}
\Omega = \frac{M^{\frac{1}{2}}}{r^{\frac{3}{2}}}.
\end{equation}
The inspiral rate for circular orbits is given by
\begin{equation}
\frac{dr}{dt} = - \frac{r}{E} \frac{dE}{dt} = 
- \frac{64}{5} \frac{\mu M^2}{r^3}.
\end{equation}
Thus
\begin{align}
r^3 \, dr &= - \frac{64}{5} \mu M^2 \, dt \\
\frac{r^4}{4} &= \frac{64}{5} \mu M^2 (t - t_c)
\end{align}
and so
\begin{equation}
r(t) = \left(\frac{256}{5} \mu M^2 \right)^{\frac{1}{4}}
       \left(t_c - t\right)^{\frac{1}{4}}.
\end{equation}
Since the emitted radiation is quadrupolar
\begin{equation}
f = \frac{\Omega}{\pi}.
\end{equation}
The gravitational wave strain is
\begin{equation}
h(t) = \frac{Q}{D} \left(\frac{384}{5}\right)^\frac{1}{2} \pi^\frac{2}{3} \mu M
r^{-1}(t) \cos \left( \int 2\pi f(t) \, dt \right).
\end{equation}
Now
\begin{align}
\frac{df}{dt} 
     &= \frac{d}{dt} \left(\frac{\Omega}{\pi}\right) 
      = \frac{d}{dt} \left(\frac{M^\frac{1}{2}}{\pi} r^{-\frac{3}{2}}\right) \\
     &= \frac{M^\frac{1}{2}}{\pi} \frac{dr}{dt} \left(-\frac{3}{2} r^{-frac{5}{2}}\right) \\
     &= \frac{M^\frac{1}{2}}{\pi} \left(-\frac{64}{5} \frac{\mu M^2}{r^3}\right)
        \left(-\frac{3}{2} r^{-frac{5}{2}}\right) \\
     &= \frac{96}{5} \frac{M^\frac{5}{2} \mu}{\pi} r^{-\frac{11}{2}}. 
     \label{eq:sp:dfdt}
\end{align}
Now
\begin{equation}
f = \frac{M^\frac{1}{2}}{\pi} r^{-\frac{3}{2}}.
\end{equation}
Solving this for $r$, we obtain
\begin{equation}
r = \frac{M^\frac{1}{3}}{\pi^\frac{2}{3} f^\frac{2}{3}}
\end{equation}
which we can substitute into equation (\ref{eq:sp:dfdt}) and obtain
\begin{align}
\frac{df}{dt} &= \frac{96}{5} \pi^\frac{8}{3} \mu M^\frac{2}{3} f^\frac{11}{3} \\
&= \frac{96}{5} \pi^\frac{8}{3} \mathcal{M}^\frac{5}{3} f^\frac{11}{3}
\end{align}
where we have defined the \emph{chirp mass} by
\begin{equation}
\mathcal{M} = \mu^\frac{3}{5} M^\frac{2}{5}.
\end{equation}
Now the phase is given to post$^2$-Newtonian order by\cite{Blanchet:1996pi}
\begin{equation}
\phi(t) = \int^t 2\pi f(t') \, dt'
\end{equation}
we won't worry about this yet.

Therefore using the stationary phase approximation, we obtain
\begin{align}
\tilde{h}(f) &= \frac{1}{2} \frac{Q}{D} \left(\frac{384}{5}\right)^\frac{1}{2} \pi^{2}{3} \mu M
                r^{-1}(t) \left(\frac{df}{dt}\right)^{-\frac{1}{2}} \exp\left[-i \Psi\right] \\
             &= \frac{1}{2} \frac{Q}{D} \left(\frac{384}{5}\right)^\frac{1}{2} \pi^{2}{3} \mu M
                \frac{\pi^\frac{2}{3} f^\frac{2}{3}}{M^\frac{1}{3}}
                \left(\frac{df}{dt}\right)^{-\frac{1}{2}} \exp\left[-i \Psi\right] \\
             &= \frac{1}{2} \frac{Q}{D} \left(\frac{384}{5}\right)^\frac{1}{2} \pi^{4}{3} \mu
                M^\frac{2}{3} r^\frac{2}{3}
                \left(\frac{df}{dt}\right)^{-\frac{1}{2}} \exp\left[-i \Psi\right].
\end{align}
Now
\begin{equation}
\left(\frac{df}{dt}\right)^{-\frac{1}{2}} =
\left(\frac{5}{96}\right)^\frac{1}{2} \pi^{-\frac{4}{3}} 
\mu^{-\frac{1}{2}} M^{-\frac{1}{3}} f^{-\frac{11}{6}}
\end{equation}
and so 
\begin{align}
\tilde{h}(f) &= \frac{1}{2} \frac{Q}{D} \left(\frac{384}{5} \frac{5}{96}\right)^\frac{1}{2}
                \mu^\frac{1}{2} M^\frac{1}{3} f^{-\frac{7}{6}} \exp\left[-i \psi\right] \\
             &= \frac{Q}{D} \mathcal{M}^\frac{5}{6} f^{-\frac{7}{6}} \exp\left[-i \psi\right].
             \label{eq:sp:template}
\end{align}
Equation (\ref{eq:sp:template}) is the stationary phase binary inspiral
template that we will use in the matched filter.

Discussion of validity of SP waveform.

\begin{equation}
t = \frac{5m}{256\eta} 
  x^{-8}\left(1 + \alpha x^2 + \beta x^3 + \gamma x^4 \right)
 \label{eq:chirplength}
\end{equation}
where
\begin{eqnarray}
x & = & \left(\pi m T_\odot f_{\mathrm{low}}\right)^{\frac{1}{3}}, \\
\alpha & = & \frac{743}{252} + \frac{11}{3}\eta, \\
\beta & = & -\frac{32\pi}{3}, \\
\gamma & = & \frac{3058673}{508032}+\frac{5429}{504}\eta+\frac{617}{71}\eta^2,
\end{eqnarray}
$m$ is the mass of the more massive object and $f_{\mathrm{low}}$ is the low
frequency cutoff of the inspiral template.


\section{Interferometric Gravitational Wave Detectors}
\label{s:ifo}

\subsection{Calibration of the Data}
\label{ss:calibration}

\section{Previous Searches for Compact Binary Inspiral}

40m, TAMA, S1. Plot of rate vs time.


The Laser Interferometric Gravitational Wave Observatory (LIGO)\cite{barish}
has completed three ``science runs'' during which all three interferometers
were collecting data simultaneously under stable operation. Analysis of the
data for gravitational waves from coalescing compact binaries has been
completed for the first two runs\cite{bns1,bns2,macho} and is in progress for
the third run. Two of the target populations in these searches are binary
neutron stars\cite{300yrs} and binary black hole
MACHOs\cite{sammacho,nakamura}. For these low mass systems, which have
component masses below $3 M_\odot$, the waveforms of the gravitational
radiation emitted are well known\cite{bdiww,biww}.  Matched filtering is a
common and effective technique for extracting known signals from
noise\cite{wz}. We have implemented a matched filter to extract inspiral
signals from interferometer noise in the package \emph{findchirp} which can be
found in the LIGO Algorithm Library (LAL)\cite{lal}.

