The Advanced LIGO~\cite{TheLIGOScientific:2014jea} and
Virgo~\cite{TheVirgo:2014hva} observatories have completed three observing runs to date, searching for gravitational waves emitted during the inspirals of compact object
binaries, composed of stellar-mass black holes (BHs) or neutron stars (NSs). During the first observing run, the LIGO observatories reported the first direct observations of gravitational waves from a binary black hole merger, GW150914~\cite{Abbott:2016blz}. This observation was followed by two more binary black hole detections in the same observing run~\cite{TheLIGOScientific:2016pea}. During the second observing run, the Virgo observatory joined the LIGO observatories, and reported for the first time, direct detection of gravitational waves from a binary neutron star inspiral~\cite{TheLIGOScientific:2017qsa}. Along with gravitational waves, the same source, referred to as GW170817, was observed across the full electromagnetic spectrum~\cite{GBM:2017lvd}, providing opportunities to answer a whole host of long-standing open questions in physics. In addition to the binary neutron star detection, the second observing run also reported observations of seven binary black hole mergers~\cite{TheLIGOScientific:2014jea,Abbott:2017vtc,Abbott:2017gyy,LIGOScientific:2018mvr,LIGOScientific:2018mvr,LIGOScientific:2018mvr,Abbott:2017oio,LIGOScientific:2018mvr}. From the third observing run, the community has been alerted of 33 merger candidates to date~\url{https://gracedb.ligo.org/superevents/public/O3/}, with one confirmed neutron star merger~\cite{abbott_gw190425_2020} and one confirmed binary black hole merger~\cite{LIGOScientific:2020stg}. With the observatories starting to make routine detections, we now have incredible opportunities to probe the properties of neutron stars and black holes, and understand the physics of binary mergers. The plethora of exciting questions relating to compact object mergers can be divided into three broad categories: (i) What are the characteristics of neutron stars and stellar-mass black holes in our universe and how to accurately extract this information from the gravitational waves they emit? (ii) how do these close compact object binaries form? (iii) what are the outcomes of the mergers---what are the astrophysical processes that occur and the remnant objects that form after mergers? In this thesis we study some of the questions under these broad categories. Below we summarize the background of the topics we study and the directions we undertake to pursue these problems.

\section{Information extraction from gravitational-wave signals}
Gravitational waves detected by the LIGO and Virgo observatories carry imprints of properties of the compact objects in their astrophysical source systems. The systems that LIGO-Virgo is searching for, comprises of black hole - black hole binaries, neutron star - neutron star binaries, and neutron star - black hole binaries. Data streams from the observatories that contain gravitational-wave signals can be analyzed to extract the measurable properties of the source binaries---pointing to the characteristics of stellar-mass black holes and neutron stars in our universe. In practice, accurate measurements of signal properties are performed using Bayesian inference~\cite{Bayes:1763,Jaynes:2003jaq}. Bayesian inference allows us to select the signal model that is best supported by observations, and to obtain probability distributions for a model's parameters---serving as measurements of the parameter values. The main source parameters of interest are masses and spins of the component objects, distance to the source, viewing angle of the binary (angle between the binary's angular momentum and line of sight) and sky location of the binary. If the detected source comprises of neutron stars, there can be additional parameters, such as tidal deformabilities of the objects---we discuss this parameter in detail later in this thesis. 

In Chapter \ref{ch:o2_bbh_pe}, we present Bayesian inference analyses of the seven binary black hole mergers from LIGO-Virgo's second observing run, using the \texttt{PyCBC Inference}~\cite{Biwer:2018osg} software. We describe the methodology used in such analyses to extract information about the parameters of interest from compact object binaries, and present measurements of source properties of the binary black hole mergers.

In Chapter \ref{ch:common_eos}, we use Bayesian parameter estimation to measure parameters of interest from the first observations of gravitational waves from a binary neutron star inspiral, from LIGO-Virgo's second observing run, GW170817. This study was focused on extracting the tidal deformability and radius parameters of the component objects, and the physics questions that the measurements addressed. 

Neutron stars are laboratories for studying matter at highest densities in the observable universe. The behavior of such incredibly dense matter is described by the nuclear equation of state. Gravitational waves from neutron star mergers can be used to measure the nuclear equation of state. In a neutron star binary system, as the two companions come in close vicinity to each other at the end of their inspiral phase, the gravitational field of each object induces a deformation in the structure of its companion. This deformation is measurable as a parameter, referred to as tidal deformability. The tidal deformability parameter enters into the phase of the gravitational-wave signal emitted from the binary. In addition to the tidal deformability parameter, it is also possible to measure the radius of the component neutron stars using gravitational waves. Both the tidal deformabilities and radii tell us how compact the neutron stars are, and their measurements are critical to determining the nuclear equation of state, as well as for interpreting multimessenger observations of neutron star mergers---observations of the same source with different types of signals or ``messengers'').

We implement a physical constraint on the nuclear equation of state, and information from the electromagnetic observations of GW170817, directly into our Bayesian parameter estimation analysis of the gravitational-wave data, to constrain the tidal deformabilities of the neutron stars of the binary. The constraint we use includes the undeniable correlations relating tidal deformabilities and masses of neutron stars. It is computed using parameterized hadronic equations of state, simulated using a fixed neutron star crust coupled with three polytropic segments. The relation also takes into account causality and the observed minimum value of the maximum neutron star mass. We use the tidal deformability constraints and mass estimates of the binary extracted from the gravitational-wave data to measure the radii of the neutron stars in the detected binary. It is also possible to directly measure the radii from the gravitational-wave data, and this approach is adopted in Ref.~\cite{capano_stringent_2020}.

\section{Formation of compact object binary sources---the common envelope phase}
The compact object binaries observed by LIGO-Virgo are end products of the evolution of binary systems comprising of massive stars. However, the existing problem in this scenario is that these progenitor binary stars are characterized by an orbital separation comparable to an astronomical unit. As gravitational-wave luminosity is inversely proportional to the fifth power of the binary separation~\cite{PhysRev.136.B1224}, widely separated binaries lose energy incredibly slowly and spiral-in negligibly over billions of years. On the other hand, the colliding compact object binary sources observed by LIGO-Virgo necessitate these binaries to have an initial orbital separations several orders of magnitude smaller than those in the massive star binaries; the stars would need to have an orbital separation comparable to a solar radius, for them to be driven to merger through gravitational-wave emission, within the age of the universe. Therefore, for massive star binaries to produce LIGO-Virgo sources, there needs to be a transformation in the orbit of the parent binary, such that the components are brought in much closer proximity to each together. The standard framework by which this transformation is believed to take place involves the parent binary evolving into a short phase, called ``common envelope'' during its lifetime. This is a critical stage in the binary star system's evolution, when it is tightened by a factor of two or more orders of magnitude through dynamically unstable mass transfer.

The evolution of binary stars through the common envelope phase can be outlined as follows (See \cite{Postnov:2014tza,2013A&ARv..21...59I,Mandel:2018hfr} for details). The parent binary comprises of a pair of massive stars widely separated by a few astronomical units from each other. The more massive star (primary) leaves the main-sequence phase, and expands rapidly. When its radius crosses the Roche lobe radius~\cite{1983ApJ...268..368E}, it starts transferring mass on to the less massive star (secondary), which is still in the main-sequence phase. Mass transfer at this step may be non-conservative but is stable. The transfer takes place on the thermal timescale of the primary, with the secondary being unable to assimilate the incoming mass at a thermal equilibrium state, as it is less evolved, and has a longer thermal timescale. At the end of the mass transfer, the primary turns into a Wolf-Rayet star, and eventually undergoes a core-collapse supernova explosion, to form a compact object (such as black hole or neutron star). %The secondary star can be spun up by the angular momentum endowed by the transferred matter onto it. 
The secondary eventually grows out of its main sequence phase and the two companions switch roles. The secondary now grows and impinges upon the orbit of the primary (now a compact object). Note that the primary at this point is typically much smaller in size and mass than the secondary, as a result of the processes it has passed through in the course of evolution into a compact object. The mass transfer in this step is non-conservative as well as unstable. The asymmetric mass ratio results in the compact object to be dragged into the envelope of the more massive star. As the object spirals towards denser stellar atmospheres, it encounters drag forces, that cause a rapid decay in its orbit, resulting in the tightening of the orbit of the binary. Additionally, it may get modified by accretion of mass from the envelope. At the end of the dynamical inspiral phase, there can be two typical outcomes. The orbital energy deposited into the envelope ejects the envelope, resulting in the formation of a Wolf Rayet star - compact object binary. The Wolf-Rayet star then undergoes a supernova explosion and collapses into a black hole. If the system survives the explosion, a close compact object binary is formed, which emits gravitational radiation, becomes a LIGO-Virgo source, and merges on timescales less than a Hubble time. Alternatively, if the envelope is retained, the compact object and the core of the companion may merge into a single compact object. One possible outcome in this case, based on the kind of compact object involved in the system, is formation of a Thorne-Zytkow object~\cite{1977ApJ...212..832T,2014MNRAS.443L..94L}.

In Chapter \ref{ch:common_envelope}, we explore the dynamical inspiral phase of common envelope episodes, during which the crucial orbital transformation of the binary takes place. The major challenge in modeling this scenario is that there are huge ranges of spatial and temporal scales involved, that should be simultaneously tackled. Time scales may range between order of seconds to order of a thousand years. Spatial scales may vary between order of a few kilometers to order of a few thousand solar radius~\cite{2013A&ARv..21...59I}. %Time scales may range between the dynamical time scales of a neutron star, for example---order of seconds, to thermal time scales of the envelope---order of a thousand years. This gives $\approx 10^{10}$ order of magnitude difference. Spatial scales may vary between the size of the compact object---order of a few kilometers to size of the envelope---order of a few thousand solar radius~\cite{2013A&ARv..21...59I}. 
Due to these reasons, modeling the full common envelope evolution in a single simulation is a challenging task. Hereby, we approach this problem by breaking down the complex physics of common envelope interactions, and look at individual aspects of the problem with simplified calculations. We isolate the flow around the embedded object from the rest of the envelope, and study its behavior in response to changing surrounding conditions. We repeat these calculations varying the surrounding conditions, to model flow morphologies in various regions along the envelope's radial extent, as well as across the range of typical common envelope encounters. A synthesis of the suite of simulations collectively provides a modeling of the full common envelope dynamical inspiral phase. We use our simulation results to study the evolution of component objects and the orbit of the binary during these episodes. %understand the effect of this phase on the observable properties---such as mass and spin---of the LIGO-Virgo stellar-mass black hole populations.

\section{Outcomes of neutron star mergers}
Unlike black hole mergers, which are purely gravitational events, the merging of a binary that involves a neutron star, such as a neutron star - neutron star or a neutron star - black hole merger, involves matter, which plays a significant role during and after the collision of the two objects. The work of astrophysicists in the past few decades predicted that the energetic processes taking place during the merger and interactions of matter released by the collision with the surrounding medium, give rise to a series a non-thermal and thermal emissions across the electromagnetic spectrum (for example, \cite{Bloom:2008ua,Metzger:2011bv,Piran:2012wd,Rosswog:2015nja,Fernandez:2015use,1986ApJ...308L..43P,Eichler:1989ve,Narayan:1992iy,Nakar:2011,Hotokezaka:2015eja,Li:1998bw,2010MNRAS.406.2650M,2011ApJ...736L..21R}). Furthermore, the work of Ref.~\cite{1974ApJ...192L.145L} predicted that neutron star - black hole mergers would eject significant amounts of neutron rich material into the interstellar medium, which would be promising sources of a phenomenon called rapid neutron capture, or $r$-process nucleosynthesis. Later, Refs.~\cite{1982ApL....22..143S, Eichler:1989ve} suggested that such a process would also take place in case of neutron star - neutron star mergers.  During the $r$-process, heavy seed nuclei in locations of high density neutron rich material undergo a rapid capture of free neutrons, at a high rate, with no time for radioactive decay between captures. These processes are responsible for forming approximately half of the elements in the periodic table that are heavier than iron.

The first observations of a binary neutron star merger by the LIGO-Virgo detectors, GW170817, gave us the opportunity to examine the theoretical predictions of electromagnetic emissions and nucleosynthetic yields associated with the merger. The gravitational-wave signal was followed up exhaustively by a global array of telescopes in search of electromagnetic counterparts~\cite{GBM:2017lvd}. Less than $\sim$ 2 seconds after the merger time extracted from the gravitational-wave signal, a short gamma-ray burst (GRB 170817A) was produced at the same source location. This was followed by an optical transient SS17a/AT2017gfo. The source was eventually observed in X-ray, ultraviolet, infrared, and radio bands over hours, days, and weeks. The optical and infrared transients from this events could be explained to have been triggered by $r$-process events associated with the merger. In the $r$-process nucleosynthesis, after all the free neutrons in the reservoir are consumed by seed nuclei, heavy unstable elements are formed. The unstable elements then emit radioactive energy and decay to form final stable nuclei. Therefore, GW170817 provided evidence of neutron star mergers being the origin of short gamma-ray bursts and kilonovae, and being astrophysical sites for production of heavy elements in the universe.  

There are several ways by which $r$-process material can be ejected in a neutron star merger. One of the mechanisms is via tidal tail ejecta. Close to the time of merger, as two neutron stars or a neutron star and a black hole in the binary come in close vicinity to each other, material can be tidally shredded off each neutron star component by its companion~\cite{Davies:1993zn,Ruffert:1996by,Rosswog:1998hy}. The material typically expands outwards from the merger location in the equatorial plane of the binary. Another mechanism of mass ejection would be via shock heated ejecta. As the two components come into contact with each other, shock heating can give rise to ejecta, that is released out of the polar regions~\cite{Oechslin:2006,PhysRevD.87.024001,Bauswein:2013jpa}. A third type of ejecta can be from postmerger accretion disk outflows. Merger of the binary components leads to the formation of a compact object (black hole or neutron star). Matter released in this process from the neutron star components in the binary, can have sufficient angular momentum to circularize around the remnant compact object in the form of an accretion disk. Eventually, these accretion disks can give rise to strong outflows, that are sites for $r$-process nucleosynthesis~\cite{Metzger:2008av,Dessart:2008zd}. Neutron star mergers involve strong gravitational and magnetic fields, due to which it is appropriate to model them using general-relativistic magnetohydrodynamic simulations. In Chapter \ref{ch:kilonova}, we use such simulations with weak interactions to model postmerger accretion disks, outflows, and nucleosynthetic yields applicable to a variety of neutron star - neutron star and neutron star - black hole merger scenarios. The physics associated with such disks, their $r$-process outcomes, and the kilonova transients they trigger are expected to vary across mergers. These outcomes depend on the initial conditions, which comprise of a complex combination of the masses and spins of the components, the type of components (neutron star - neutron star and neutron star - black hole pairs), as well as the nuclear equation of state. We explore the properties of distinct states of accretion disks and their outcomes across the binary parameter space. We provide theoretical models and predictions that could be tested against a variety of future neutron star merger observations.