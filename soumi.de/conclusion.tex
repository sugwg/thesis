We are currently in an exciting era of astrophysics, where we have begun to solve many mysteries of astrophysical compact objects in our universe with the help of both gravitational and electromagnetic waves. Binary black holes are now frequently observed by the LIGO-Virgo detectors. These binary black hole detections to date span a broad range of source component masses between $\sim 8~M_\odot$ to $\sim 50~M_\odot$. The other type of binary systems from which gravitational waves have been observed are binary neutron stars. LIGO-Virgo has observed two such incredibly exciting systems to date, both very different from each other. Observations of these sources told us that neutron star mergers can indeed be used to probe the nuclear equation of state, that they provide astrophysical sites for formation of heavy elements in our universe, and that they give rise to gamma-ray bursts. The observations have also unveiled a new population of neutron stars that are significantly heavier than the previously known populations of galactic neutron stars. As the LIGO-Virgo detectors proceed towards design sensitivity, black hole and neutron star mergers will be regularly observed. We anticipate that these new detections will tremendously help in gaining new insights about the physics of compact objects.

In this thesis, we have studied compact object binaries made of neutron stars and black holes, using a synthesis of observations and computational modeling. In particular, the work presented in this thesis connects three different phases in the lifetimes of these binaries---formation, evolution, and fate. We have used observations made by LIGO and Virgo to extract information from the phase when they ``evolve'' in a binary system, and we have used computational modeling to simulate aspects of their ``formation'' and ``fate''.

We have presented parameter estimation analyses of the seven binary black hole mergers observed during LIGO-Virgo's second observing run, and have presented samples of the posterior probability density function for the masses, spins, distances, inclination angles, and sky locations of the binaries. These estimates help in understanding the formation and evolution of populations of black hole sources. The parameter estimates indicate that all the detected events came from binary black hole systems that have a significant probability of being nearly equal mass systems. The effective binary spins measured are fairly low, indicating that the black hole spins have low magnitudes or are largely misaligned with the orbital angular momentum.

We have provided estimates of tidal deformabilities and radii of neutron stars in LIGO-Virgo's first observed binary neutron star merger, by combining gravitational wave observations with a physical constraint on the stars' equation of state, and information from the electromagnetic observations of the event. Our analysis extracted parameters of the binary by exploring a prior parameter space that---took into account existing correlations between the tidal deformability and mass parameters of the stars, was constructed in a manner that included a complete representation of physical equations of state, with no bias towards selected models, and was restricted by causality and the observed minimum value of the maximum neutron star mass. The methodology presented in our analysis could also be used in the analysis of future binary neutron star events with modifications based on any event-specific characteristics. Several other studies~\cite{Abbott:2018exr,Radice:2018ozg,Coughlin:2018fis,capano_stringent_2020} have performed their own analyses of the gravitational-wave data for GW170817, with different approaches, to measure tidal deformabilities and radii. Results from these analyses are broadly in agreement with our constraints.

As LIGO-Virgo continue to detect more gravitational waves from binary neutron star mergers, it should be possible to update the GW170817-specific correlations presented here, with those that represent the broader populations of neutron stars. New events would also improve the neutron star tidal deformability, radii, and hence equation of state constraints obtained from GW170817. We expect these improvements to increase with higher signal-to-noise ratio events. Furthermore, for GW170817, gravitational waves alone could not distinguish between a binary neutron star model (allowing non-zero tidal deformability) and a binary black hole model (zero tidal deformability), or between different tidal deformability prior models or between different radius prior models; several investigations found that the posterior inferences of tidal deformabilities and radii could be significantly influenced by prior assumptions, which made the choice of prior for GW170817---with a reasonably high signal-to-noise ratio $\sim 32$---an important factor contributing to accuracy of results for this event. As the detectors approach design sensitivity, we predict that it will be possible to achieve signal-to-noise ratios as high as $\sim 100$ with an event similar to GW170817. With such high signal-to-noise ratio events in the near future, it will be possible to rely on gravitational waves alone to distinguish between a binary neutron star from a binary black hole signal, in the absence of electromagnetic counterparts. Additionally, constraints from NICER~\cite{2012SPIE.8443E..13G} observations would complement those from LIGO-Virgo to jointly probe the nuclear equation of state.

We have presented a framework for modeling the common envelope inspiral phase with three-dimensional hydrodynamical simulations of flows around compact objects orbiting through stellar atmospheres. By isolating the flow around the compact objects from the full complex interaction, we have been able to explore a broad range of the common envelope flow parameter space, and have systematically investigated the effect of each parameter on the flow morphology. Findings from these local simulations may be used in understanding outcomes of global simulations, and a synthesis of the local and global approaches paves the way for further advancements in understanding the intricacies of the common envelope phase. We have shown the dependence of accretion and drag forces in these episodes on the flow and binary parameters, and have pointed to the key role of the coupled effect of these quantities in modulating the rate of transformation of objects during the common envelope phase. Our inferences of percent order mass and spin accumulation by black holes during the common envelope inspiral phase predict that black holes remain mostly unmodified on passing through such phases during their assembly into a binary system. Therefore, the black hole properties observed by LIGO-Virgo would be attributable to previous evolutionary phases through which these objects have evolved. This has direct implications in interpreting the properties and evolutionary history of LIGO-Virgo's growing number of black hole binaries. More observations of compact objects in the future would help in bridging the gap between the measurements from observations and the predictions from models. The growing catalog of detections would help in identifying the contributions of different formation channels to compact binary formation, and will provide prospects of shedding light on the binary stellar evolution problem.

We have presented self-consistent general-relativistic magnetohydrodynamic modeling of accretion disks formed as an end-product of neutron star mergers---applicable to both neutron star - neutron star and neutron star - black hole systems. We have demonstrated an exploration of the relations between binary parameters and disk properties. Our models have spanned a range of parameter space that encompasses an ``ignition threshold'' on the accretion rate, controlling weak interactions in the disks. We have shown how the properties of the disks qualitatively differ as the threshold is crossed. Our simulations provide a suite of theoretical models that can be used as templates to investigate future neutron star merger observations. Connecting the simulation results to future detections, we predict production of a blue kilonova would be possible from large accretion disk masses, which correspond to lighter neutron star binaries. With more detections of neutron star binary mergers, we expect to observe diversity in the types of electromagnetic emissions and nucleosynthetic yields. Different binary detections will have different binary parameters, viewing angles, and distances. Each configuration would influence the properties of the merger ejecta structure, and well as the detectability of the counterparts in a different way. Hereby, we anticipate fantastic opportunities to use well-studied models in understanding properties of observations, as well as use the observations to validate the models.