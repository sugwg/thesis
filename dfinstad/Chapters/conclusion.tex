In the current era of gravitational-wave astrophysics we are moving beyond first direct detections and first multimessenger observations, to now making routine discoveries that deepen our understanding of the compact objects in our cosmic neighborhood. The LIGO-Virgo gravitational-wave detector network has detected 52 confirmed binary merger observations so far, and the detection rate has only accelerated as improved detector sensitivity extends our reach deeper into the universe. From the two observed binary neutron star mergers, our knowledge of the dynamics of these events and of neutron star physics has grown dramatically. They have provided confirmation of binary neutron star mergers as a source of short gamma-ray bursts, and also as important sites of heavy element production through $r$-process nucleosynthesis that can help explain observed chemical abundances. They have also shown that it is possible to measure the tidal information in a gravitational-wave signal to meaningfully update our constraints on the nuclear equation of state. As the LIGO-Virgo detectors approach their design sensitivity, and as third-generation detectors begin to come online, we expect to see many more binary neutron star mergers in the coming years. We anticipate that these new detections will provide even further insights into the physics of neutron stars.

In this thesis we have studied binary neutron star mergers, through a combination of observations and computational modeling. Specifically we explore the ability of a gravitational-wave analysis to extract physical parameters of the binary system, and of the neutron stars involved in the merger. We investigate the impact of multimessenger information on a gravitational-wave analysis, and we study the measurability of the nuclear equation of state, both now and in the future.

We have presented an analysis of the binary neutron star merger GW170817 informed by electromagnetic distance measurements of its identified host galaxy, and we demonstrated that using an independent distance measurement in a gravitational-wave analysis can break the distance-inclination degeneracy to allow for much tighter constraints on the inclination angle of the binary. We find our improved measurement of the inclination supports models for a structured relativistic jet and its afterglow emission being viewed off-axis.

We have presented measurements of the tidal deformabilities and radii of the neutron stars in GW170817. Our analysis imposed a physical constraint to require that both neutron stars obey the same equation of state, and we used a prior on the leading order tidal parameter constructed to contain all physical models of the equation of state without biasing the measurement toward any particular model. We note that the methodology we employed could be adapted for the analysis of future binary neutron star merger events with similar masses. We find our results are broadly consistent with several other studies~\cite{Abbott:2018exr,Radice:2018ozg,Coughlin:2018fis,Capano:2019eae} which employed various methods to measure the tidal deformabilities and radii in their own analyses of GW170817.

We have presented a likelihood model developed for \textit{PyCBC Inference} that uses the relative binning parameter estimation technique to reduce computational cost for potential multimessenger gravitational-wave sources. We extended the work of previous implementations to make our relative likelihood model a coherent network statistic so that it can additionally measure sky locations. We validated the relative model on populations of simulated binary neutron star and simulated neutron star--black hole merger signals, and we showed that it is possible to seed the relative analysis with the best-fit template parameters from a low-latency search pipeline. We found that the parameter estimation for all signals in our simulated populations completed in less than 20 minutes, with sky localization and intrinsic parameter estimates that are comparable to analyses done with a standard non-relative likelihood.

We have presented a comprehensive study of the future prospects for a precise equation of state measurement from Advanced LIGO and Cosmic Explorer. We explored the measurability of the equation of state across the full parameter space allowed by combined constraints from astrophysical observations and nuclear experiments. We showed that a precision threshold for measurements to distinguish between substantially similar theoretical models for the equation of state is equivalent to measuring the radius of a 1.4\msun\ neutron star to better than $2\%$, and we presented a framework for combining individual equation of state measurements across entire populations to produce a combined, high-precision measurement. We found it is unlikely that Advanced LIGO will achieve $2\%$ precision in the next observing runs given current estimates of the merger rate for binary neutron stars, however Cosmic Explorer will measure the equation of state to better than $1\%$ within one year of operation. Our framework can be directly applied to any future signals from binary neutron star mergers, and we anticipate that the resulting precise knowledge of the true equation of state will be invaluable for efforts to model these merger events and their associated kilonovae.